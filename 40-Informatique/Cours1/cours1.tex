\documentclass[french,11pt,twoside]{VcCours}

\begin{document}

\Titre{PSI}{Promotion 2021--2022}{Informatique}{Algorithmes incontournables.}

\subsection{Calcul d'une somme.}
\subsubsection{L'algorithme.}

On a une liste $L$, on veut calculer la somme des termes de la liste.

L'idée principale est d'avoir une variable $S$ (l'accumulateur) contenant la
somme des termes déjà lus.

Au départ $S$ vaut $0$, puis on rajoute chacun des éléments de la liste à cet
accumulateur $S$. Quand on a fini de passer en revue tous les éléments de la
liste, $S$ contient le total souhaité.

\begin{Python}
def somme(L) :
    """Calcule la somme des termes de la liste L."""
    n = len(L) # nombre de terme de la liste
    S = 0
    for k in range(n) :
        S = S + L[k]
    return(S)
\end{Python}

Un exemple d'exécution :
\begin{Python*}
somme([5,9,-2,5])
\end{Python*}

Le tableau suivant retrace l'exécution de la fonction \textalltt{somme}.

Il donne l'état des variables après l'exécution de chacune des lignes (donnée
par leur numéro).

\begin{center}
\begin{tabular}{|c|c|c|c|c|}
\hline
  N°
  ligne&\makebox[2cm]{S}&\makebox[1cm]{k}&\makebox[1cm]{n}&\makebox[8cm]{commentaires}\\
  \hline
  1 &&&&\\
  \hline
  3 &&&&\\
  \hline
  4 &&&&\\
  \hline
  5 &&&&\\
  \hline
  6 &&&&\\
  \hline
  5 &&&&\\
  \hline
  6 &&&&\\
  \hline
  5 &&&&\\
  \hline
  6 &&&&\\
  \hline
  5 &&&&\\
  \hline
  6 &&&&\\
  \hline
  5 &&&&\\
  \hline
  7 &&\multicolumn{3}{c|}{}\\
  \hline
\end{tabular}
\end{center}

\pagebreak
\subsubsection{Variantes.}
\begin{Exercice}{}
Écrire une fonction \textalltt{somme(n)} qui, pour $n\in\N$, donne
$\sum_{k=1}^n\frac{1}{k^2}$.
\end{Exercice}

\begin{Exercice}{}
Écrire une fonction \textalltt{produit(L)} qui, pour une liste $L$, donne
le produit de tous ses éléments.
\end{Exercice}

\begin{Exercice}{}
Écrire une fonction \textalltt{sommeTermesPositifs(L)} qui, pour une
liste $L$, donne le somme des éléments positifs de la liste.
\end{Exercice}

\begin{Exercice}{}
Écrire une fonction \textalltt{somme(T)} qui, pour un
tableau $T$ à deux dimensions, donne le somme des éléments du tableau.
\end{Exercice}

\begin{Exercice}{}
Écrire une fonction \textalltt{sommesParLignes(T)} qui, pour un
tableau $T$ à deux dimensions, donne la liste contenant la somme de
chacune des lignes du tableau.
\end{Exercice}

\newpage
\subsection{Calcul du maximum.}
\subsubsection{L'algorithme.}
On a une liste $L$, on veut déterminer le plus grand des termes
de la liste.

L'idée principale est d'avoir une variable $M$ (la mémoire)
contenant le plus grand terme rencontré jusque là.

Au départ $M$ prend la valeur du premier terme de la liste, puis on compare
chacun des autres termes à $M$ et, s'il est plus grand, on met à jour $M$ avec
cette valeur.
Quand on a fini de passer en revue tous les éléments de la liste, $M$ contient
le plus grand de tous.

\begin{Python}
def maximum(L) :
    """Donne le plus grand des termes de la liste L.
    """
    n = len(L) # nombre de terme de la liste
    M = L[0]
    for k in range(1,n) :
        if L[k]>M :
            M = L[k]
    return(M)
\end{Python}

Un exemple d'exécution :
\begin{Python*}
maximum([5,2,9,-2,11,5])
\end{Python*}

Le tableau suivant retrace l'exécution de la fonction \textalltt{maximum}.

Il donne l'état des variables après l'exécution de chacune des lignes (donnée
par leur numéro).

\begin{center}
\begin{tabular}{|c|c|c|c|c|}
\hline
  N° ligne&\makebox[2cm]{M}&\makebox[2cm]{k}&\makebox[2cm]{n}&\makebox[5cm]{L}\\
  \hline
  1 &&&&\\
  \hline
  4 &&&&\\
  \hline
  5 &&&&\\
  \hline
  6 &&&&\\
  \hline
  7 &&\multicolumn{3}{l|}{Le test \textalltt{L[k]>M} est}\\
  \hline
  6 &&&&\\
  \hline
  7 &&\multicolumn{3}{l|}{Le test \textalltt{L[k]>M} est}\\
  \hline
  8 &&&&\\
  \hline
  6 &&&&\\
  \hline
  7 &&\multicolumn{3}{l|}{Le test \textalltt{L[k]>M} est}\\
  \hline
  6 &&&&\\
  \hline
  7 &&\multicolumn{3}{l|}{Le test \textalltt{L[k]>M} est}\\
  \hline
  8 &&&&\\
  \hline
  6 &&&&\\
  \hline
  7 &&\multicolumn{3}{l|}{Le test \textalltt{L[k]>M} est}\\
  \hline
  6 &&&&\\
  \hline
  9 &&\multicolumn{3}{c|}{}\\
  \hline
\end{tabular}
\end{center}

%\pagebreak
\subsubsection{Variantes.}
\begin{Exercice}{}
Écrire une fonction \textalltt{positionDuMax(L)} qui, pour une liste $L$, donne
la/une position du plus grand des éléments de la liste.
\end{Exercice}

\begin{Exercice}{}
Écrire une fonction \textalltt{positionsDuMax(L)} qui, pour une liste $L$,
donne la liste des positions du plus grand des éléments de la liste.
\end{Exercice}

\begin{Exercice}{}
Écrire une fonction \textalltt{maximum(T)} qui, pour un
tableau $T$ à deux dimensions, donne le plus grand des éléments du tableau.
\end{Exercice}

\begin{Exercice}{}
Écrire une fonction \textalltt{maximumParLigne(T)} qui, pour un
tableau $T$ à deux dimensions, donne la liste contenant le maximum de chacune
des lignes du tableau.
\end{Exercice}

\newpage
\subsection{Termes d'une suite.}
\subsubsection{L'algorithme.}
On a une suite $(u_n)_{n\in\N}$, définie par :
\[\sys{u_0=5\\\forall n\in\N,\ u_{n+1}=2u_n+\frac{6}{n+1}}\]

On veut calculer le terme $u_n$ pour un $n$ donné.

L'idée principale est d'avoir deux variables $U$ et $k$ qui à elles deux
représente $u_k$.

Au départ $U$ prend la valeur du premier terme de la suite et $k$ la valeur de
l'indice correspondant. Puis, on remplace $U$ et $k$ par la valeur et l'indice
de $u_{k+1}$ et on répète l'opération jusqu'à obtenir le terme souhaité $u_n$.
À la fin, $U$ contient la valeur demandée.

\begin{Python}
def valeurDeU(n) :
    """Donne la valeur de Un.
    """
    U = 5
    k = 0
    while k<n :
        U = 2*U+6/(k+1) # Calcul Uk+1 à partir de Uk
        k = k+1
    return(U)
\end{Python}

Un exemple d'exécution :
\begin{Python*}
valeurDeU(4)
\end{Python*}

Le tableau suivant retrace l'exécution de la fonction \textalltt{valeurSuiteU}.

Il donne l'état des variables après l'exécution de chacune des lignes (donnée
par leur numéro).

\begin{center}
\begin{tabular}{|c|c|c|c|c|}
\hline
  N°
  ligne&\makebox[2cm]{U}&\makebox[2cm]{k}&\makebox[2cm]{n}&\makebox[5cm]{Commentaire}\\
  \hline
  1 &&&&\\
  \hline
  4 &&&&\\
  \hline
  5 &&&&\\
  \hline
  6 &&&&\multicolumn{1}{l|}{Le test \textalltt{k<n} est}\\
  \hline
  7 &&&&\\
  \hline
  8 &&&&\\
  \hline
  6 &&&&\multicolumn{1}{l|}{Le test \textalltt{k<n} est}\\
  \hline
  7 &&&&\\
  \hline
  8 &&&&\\
  \hline
  6 &&&&\multicolumn{1}{l|}{Le test \textalltt{k<n} est}\\
  \hline
  7 &&&&\\
  \hline
  8 &&&&\\
  \hline
  6 &&&&\multicolumn{1}{l|}{Le test \textalltt{k<n} est}\\
  \hline
  7 &&&&\\
  \hline
  8 &&&&\\
  \hline
  6 &&&&\multicolumn{1}{l|}{Le test \textalltt{k<n} est}\\
  \hline
  9 &&\multicolumn{3}{c|}{}\\
  \hline
\end{tabular}
\end{center}

\pagebreak
Voici la même fonction avec une boucle \textalltt{for} qui gère la variable
\textalltt{k}.
\begin{Python}
def valeurDeU(n) :
    """Donne la valeur de Un.
    """
    U = 5
    for k in range(0,n) :
        U = 2*U+6/(k+1) # Calcul Uk+1 à partir de Uk
    return(U)
\end{Python}



\subsubsection{Variantes.}
\begin{Exercice}{}% cspell:ignore depasse
Écrire une fonction \textalltt{depasse(M)} qui donne le plus petit indice $n$
pour lequel $u_n\geqslant M$, $M$ étant un réel donné. (La suite tendant vers
$+\infty$, on peut toujours trouver un tel $n$).
\end{Exercice}

\begin{Exercice}{}
Écrire une fonction \textalltt{ListeDesTermesDeU(n)} qui, pour un
$n\in\N$, donne la liste des $[u_0,\ldots,u_n]$.
\end{Exercice}

\begin{Exercice}{}
Écrire une fonction \textalltt{sommeU(n)} qui, pour un
$n\in\N$, donne la liste des $u_0+\ldots+u_n$.
\end{Exercice}

\end{document}
