\documentclass[french,11pt,twoside]{VcCours}
\usepackage{hyperref}
    \hypersetup{
      breaklinks=true,  % so long urls are correctly broken across lines
      colorlinks=true,
      urlcolor=black,
      linkcolor=black,
      citecolor=black,
}
\lstnewenvironment{PYout}{\lstset{language=Python,numbers=none,backgroundcolor=\color{white}}}{}

\begin{document}

\Titre{PSI}{Promotion 2021--2022}{Informatique}{TP d'info n°3 -- Les piles}

\tableofcontents
\separationTitre

\section{Cours}

\subsection{Introduction}
Le but de ce TP est définir une nouveau type de structure de données : les piles. 
Donnons un exemple très simple en considérant une pile d'assiettes :

\medskip

\begin{center}
\includegraphics[scale=0.2]{assiette}
\end{center}
Si l'on considère que les assiettes sont très lourdes, les actions 
possibles concernant cette pile d'assiettes sont :
\begin{itemize}
\item Enlever l'assiette située au sommet de la Pile (si celle-ci n'est pas vide).
\item Ajouter une assiette au sommet de la Pile.
\end{itemize}

\medskip

A priori, aucune autre action n'est disponible donc ne peut pas accéder 
directement à une assiette qui n'est pas au sommet de la pile : 
il faut dépiler toutes les assiettes situées au dessus de celles 
qui nous intéressent ...

\medskip

Dans ce TP, on considérera une pile comme une liste sur laquelle on limite les accès : 
\begin{itemize}
\item On ne peut insérer des éléments qu'à une seule extrémité que l'on appelle 
le \textit{sommet} de la pile.
\item On ne peut retirer un élément de la pile que s'il se situe 
au sommet de celle-ci.
\item Si on souhaite accéder à un élément qui n'est pas au sommet, on doit 
retirer un par un les éléments qui sont au dessus de celui-ci, en partant du sommet.
\end{itemize}

\medskip
%cSpell:ignore LIFO
On appelle ce type de structure de données LIFO : \og Last In, First Out \fg.

\medskip

Voici quelques exemples courant utilisant des piles :
\begin{itemize}
\item L'historique d'un navigateur internet : on empile chaque nouvelle page visitée et on dépile quand on clique \og page précédent \fg.
\item La fonction \og{}Annuler\fg{} dans un logiciel de traitement de texte par exemple : on empile chaque nouvelle action et on dépile quand on annule une action.
\end{itemize}

\subsection{Structure de Pile}
La notion de Pile étant abstraite, une des questions de base est de savoir quelles fonctions peuvent être utilisées sur celles-ci, indépendamment du langage choisi et de la manière dont on l'implémente. Voici quelques fonctions usuelles :
\begin{itemize}
\item Créer une pile vide.
\item Tester si une pile est vide.
\item Empiler un élément à une pile.
\item Dépiler une pile (enlever le sommet).
\end{itemize}
À ces fonctions indispensables peuvent s'ajouter :
\begin{itemize}
\item Donner la taille de la pile.
\item Renvoyer le sommet d'une pile (sans le retirer).
\item Vider la pile.
\end{itemize}

\medskip

Une manière de créer des piles avec Python est de considérer des listes 
$L=[a_n, a_{n-1}, \ldots, a_1]$ où $a_1$ est le sommet de la liste. 
C'est que l'on fera dans la suite.

% \begin{Remarque}{} 
%     On peut aussi considérer que le sommet d'une pile représentée par une 
%     liste est le premier élément de la liste.
% \end{Remarque} 

\begin{Exemple}%cSpell:ignore depile
    Donnons les fonctions de bases avec le point de vue précédent. 
    On les appellera : \code{creer\_pile}, \code{est\_vide}, \code{depiler}, 
    \code{empiler}.
\end{Exemple}

%\vspace{9.5cm}

\newpage
\section{TP}
Dans toute la suite, les seules fonctions de bases utilisables sont celles 
données dans le cours. Les autres liées aux listes sont interdites. 

\subsection{Pour se chauffer}
\begin{enumerate}
\item
\begin{enumerate}
\item Écrire une fonction \code{second} prenant en argument une pile et 
renvoyant le deuxième élément de la Pile si celui-ci existe.
\item Écrire une fonction \code{second2} prenant en argument une pile et 
renvoyant le deuxième élément de la Pile si celui-ci existe, et tout ceci en
laissant la pile dans son état de départ.
\end{enumerate}
\item 
\begin{enumerate}
\item Écrire une fonction \code{retourne} prenant en argument une pile et renvoyant 
la pile inverse de celle-ci (le sommet de la Pile devient la queue de celle-ci, 
le deuxième élément de la pile devient l'avant-dernier...).
\item  Écrire une fonction \code{copie} prenant en argument une pile et 
renvoyant une copie de celle-ci sans modifier la pile de départ.
\item Écrire une fonction \code{inverse2} prenant en argument une pile et 
renvoyant la pile inverse de celle-ci, et tout ceci en
laissant la pile dans son état de départ.
\end{enumerate}
\end{enumerate}
\subsection{Expressions bien parenthésées}
Une chaîne de caractère est dite \textit{bien parenthésée} si elle vérifie
l'une des conditions suivantes :
\medskip
\begin{enumerate}
\item Elle ne contient aucune parenthèse ( ou ).
\item Elle est de la forme (expr) ou expr est une expression bien parenthésée.
\item Elle est la concaténation de deux expressions bien parenthésées.
\end{enumerate}

\medskip

Quelques exemples :
\begin{itemize}
\item (), (1), (()()), ()((a)((b()))) sont bien parenthésées.
\item (, )(, ((truc) ne sont pas bien parenthésées.
\end{itemize}

\medskip

Écrire une fonction \code{Par} prenant en argument une chaîne de caractères et 
renvoyant True si celle-ci est bien parenthésée et False sinon. On pourra 
parcourir la chaîne de caractères de gauche à droite et utiliser une pile liée 
aux parenthèses ouvrantes présentes dans celle-ci.


\subsection{Permutation circulaire d'une pile}
Écrire une fonction \code{Perm}, prenant en argument une pile $P$ et un entier 
naturel $k$, renvoyant la permutation circulaire de la Pile $P$ de $k$ rangs. 
Par exemple si la pile est $P=[1,2,3,4]$ avec $4$ le sommet :

\begin{itemize}
\item Perm($P,0$)=Perm($P,4$)=$P$.
\item Perm($P,1$)=Perm$(P,5$)=[4,1,2,3].
\item Perm($P,2$)=Perm$(P,6$)=[3,4,1,2].
\item Perm($P,3$)=Perm$(P,7$)=[2,3,4,1].
\end{itemize} 

\subsection{Notation polonaise}
Pour les plus rapides : \url{https://lepython.com/files-et-piles/#exercice9}
\end{document}