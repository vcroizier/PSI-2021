\documentclass[french,11pt,twoside]{VcCours}
\usepackage{hyperref}
    \hypersetup{
      breaklinks=true,  % so long urls are correctly broken across lines
      colorlinks=true,
      urlcolor=black,
      linkcolor=black,
      citecolor=black,
}
\lstnewenvironment{PYout}{\lstset{language=Python,numbers=none,backgroundcolor=\color{white}}}{}

\begin{document}

\Titre{PSI}{Promotion 2021--2022}{Informatique}{TP d'info n°2 -- Manipulation des images}

\tableofcontents
\separationTitre

\section*{Mise en place}
Récupérer l'archive du TP2 dans le cours d'informatique sur moodle.

Décompresser cette archive avec \og{} extraire ici \fg{} dans votre dossier de travail
(un dossier \code{TP2-2021} est automatiquement créé). 

On va utiliser deux fichiers ".py" et le document PDF:
  \begin{enumerate}
    \item \code{exemples-array.py} pour la première partie.
    \item \code{exercices-images.py} pour la seconde partie.
  \end{enumerate}
  Dans ce cas, pour exécuter les blocs de lignes délimités par \code{\#\#}
  (ou \code{\#\%\%} pour Spyder), 
  il suffit de faire \code{Ctrl+Entrée} ou \code{Ctrl+Shift+Entrée},
  le premier reste sur place, le second avance au bloc suivant (ou
  utiliser les boutons dédiés pour Spyder). 



\section{Très court aperçu du type ARRAY de Numpy.}

Ici, nous aurons un usage très limité du type ARRAY du module Numpy,
mais incontournable. En effet, les images seront stockées sous cette forme.
Le but est donc ici de découvrir les quelques commandes simples dont nous aurons besoin.

\pagebreak
\subsection{Construction d'un ARRAY.}
Il faut naturellement charger Numpy.
\begin{PY}
import numpy as np
\end{PY}
np.array permet de créer un ARRAY à partir d'une liste (de listes de listes...) habituelle.
\begin{PY}
a = np.array([[1,2],[3,4],[5,6]])
a
\end{PY}
Cela donne :
\begin{PYout}
array([[1, 2],
       [3, 4],
       [5, 6]])
\end{PYout}

Pour obtenir les dimensions du tableau on utilise shape.
\begin{PY}
print(a.shape)
n,p = a.shape
print(n)
print(p)
\end{PY}
\begin{PYout}
(3, 2)
3
2
\end{PYout}
Si besoin est, on peut créer un tableau rempli de zéros de la dimension que l'on veut. (Remarquez les deux jeux de parenthèses, car l'argument est un TUPLE.)
\begin{PY}
b = np.zeros((2,5))
b
\end{PY}
\begin{PYout}
array([[ 0.,  0.,  0.,  0.,  0.],
       [ 0.,  0.,  0.,  0.,  0.]])
\end{PYout}
Nous n'avons utilisé ici que des tableau à deux dimensions, mais on peut étendre les exemples à la dimension nn.

\subsection{Accéder et modifier les éléments d'un ARRAY.}
\begin{PY}
print(a)
print('a[1]    donne',a[1])
print('a[1][0] donne',a[1][0])
print('a[1,0]  donne',a[1,0])
print('a[1,:]  donne',a[1,:])
print('a[:,1]  donne',a[:,1])
\end{PY}
\begin{PYout}
[[1 2]
 [3 4]
 [5 6]]
a[1]    donne [3 4]
a[1][0] donne 3
a[1,0]  donne 3
a[1,:]  donne [3 4]
a[:,1]  donne [2 4 6]
\end{PYout}
\begin{PY}
a[1,0]=30
a
\end{PY}
\begin{PYout}
array([[ 1,  2],
       [30,  4],
       [ 5,  6]])
\end{PYout}
\begin{PY}
a[1,:]=[8,9]
a
\end{PY}
\begin{PYout}
array([[1, 2],
       [8, 9],
       [5, 6]])
\end{PYout}
\begin{PY}
a[:,1]=[-1,-2,-3]
a
\end{PY}
\begin{PYout}
array([[ 1, -1],
       [ 8, -2],
       [ 5, -3]])
\end{PYout}
Comme notre tableau est à deux dimensions, il n'y a que deux coordonnées. Biens sûr, cela s'adapte facilement à la dimension $n$.

Dans la suite nous manipulerons essentiellement des ARRAY de dimension $3$.

\section{Traitement des images.}
\subsection{Les bases de la manipulation d'images.}
Nous aurons besoin du module Numpy et des sous-modules Image et Pyplot de Matplotlib. 

(ils seront tous rebaptisés pour plus de facilité.)
\begin{PY}
import numpy as np
import matplotlib.image as img
import matplotlib.pyplot as plt
\end{PY}

La commande \code{imread} ne sait lire nativement que le format d'image PNG.

(Cependant, l'installation de la librairie PIL permet d'obtenir le support de tous les formats d'images habituels.)
\begin{PY}
# Chargement du fichier image :
A = img.imread("Chloroplastes.png")
# Enregistrement de l'image dans un fichier :
img.imsave("Chloroplastes_bis.png",A)
# Affichage de l'image :
plt.imshow(A, interpolation='none')
plt.show()
\end{PY}

Le fichier est pris dans le répertoire \code{TP\_Python}, mais on peut faire toute sorte de chemins relatifs ou absolus.

Remarquez que, maintenant, il y a un fichier \code{Chloroplastes\_bis.png} dans le répertoire \code{TP\_Python}.

\begin{PY}
print(type(A))
n,p,c=A.shape
print((n,p,c))
print(A[10,40])
\end{PY}
\begin{PYout}
<class 'numpy.ndarray'>
(200, 300, 3)
[ 0.44313726  0.45882353  0.35294119]
\end{PYout}

L'image est stockée à l'aide d'un objet ARRAY. On voit qu'il compte 200 lignes, 300 colonnes et que ses couleurs sont des listes de 3 valeurs entre 0 et 1.

Ces trois nombres sont les composantes RGB (rouge, vert, bleu). [0,0,0] représente le noir et [1,1,1] le blanc.

Attention : il s'agit ici des couleur primaires additives.
\begin{PY}
A=np.array([[ [0,0,0] , [0.5,0.5,0.5] , [1,1,1] ]
           ,[ [0,0,0.5] , [0,1,1] , [0,0.5,0]]])
​
plt.imshow(A,interpolation='none')
plt.show()
\end{PY}

\subsection{Convertir en noir et blanc, en niveaux de gris.}
\begin{Exercice}
	Écrire une fonction \code{nbconvert(image,seuil)} qui convertit l'image en noir et blanc.

	\code{image} est un array et \code{seuil} est un réel entre $0$ et $3$. Le résultat est un array,
	dans lequel le pixel est noir si la somme des composantes RGB est inférieure au \code{seuil} et blanc sinon.
\end{Exercice}
\begin{PY}
def nbconvert(image,seuil) :
    """ Convertit l'image en noir et blanc selon le seuil.
    """
    n,p,c = image.shape
    resultat = np.zeros((n,p,c))

    ### A compléter ###
        
    return(resultat)
\end{PY}
\begin{PY}
A = img.imread("Chloroplastes.png")
B = nbconvert(A , 1.5)
plt.imshow(B, interpolation='none')
plt.show()
img.imsave("Chloroplastes_NB.png",B)
\end{PY}

\begin{Exercice}
Écrire une fonction \code{ngconvert(image)} qui convertit l'image en niveau de gris.

Il suffit pour cela d'égaliser les composantes RGB de chaque pixel, en essayant de
conserver la même luminosité (le même total des composantes couleurs).
\end{Exercice}

\begin{PY}
A = img.imread("Chloroplastes.png")
B = ngconvert(A)
plt.imshow(B, interpolation='none')
plt.show()
img.imsave("Chloroplastes_gris.png",B)
\end{PY}

\subsection{Faire un filtre de couleur.}
\begin{Exercice}
	Écrire une fonction \code{filtre\_bleu(image)} qui élimine les composantes rouges et
	vertes de chaque pixel (en les forçant à 0).

	On pourrait faire de même avec \code{filtre\_rouge(image)} et \code{filtre\_vert(image)}.
\end{Exercice}

\begin{PY}
def filtre_bleu(image) :
    """ Redonne l'image après lui avoir appliqué un filtre bleu.
    """
    n,p,c = image.shape
    resultat = np.zeros((n,p,c))

    ### A compléter ###
        
    return(resultat)
\end{PY}
\begin{PY}
A = img.imread("Chloroplastes.png")
B = filtre_bleu(A)
plt.imshow(B, interpolation='none')
plt.show()
\end{PY}

\subsection{Faire un négatif.}
\begin{Exercice}
	Écrire une fonction \code{negatif(image)} qui donne le négatif de l'image.
	
	Si on additionnait le positif et le négatif, on devrait obtenir une image totalement blanche.
\end{Exercice}

\begin{PY}
A = img.imread("Chloroplastes.png")
# négatif
B = negatif(A)
plt.imshow(B, interpolation='none')
plt.show()
# négatif du négatif
C = negatif(B)
plt.imshow(C, interpolation='none')
plt.show()
\end{PY}

\subsection{Rotations et symétries.}
\begin{Exercice}
	Écrire une fonction \code{miroir(image)} qui donne l'image retournée gauche/droite.
	
	Puis écrire une fonction \code{rotation(image)} qui tourne l'image de 90° dans le sens trigonométrique.
\end{Exercice}

\begin{PY}
A = img.imread("Chloroplastes.png")
# effet miroir
B = miroir(A)
plt.imshow(B, interpolation='none')
plt.show()
\end{PY}

\begin{PY}
       A = img.imread("Chloroplastes.png")
       plt.imshow(A, interpolation='none')
       plt.show()
       # une rotation
       B = rotation(A)
       plt.imshow(B, interpolation='none')
       plt.show()
       # deuxième rotation
       C = rotation(B)
       plt.imshow(C, interpolation='none')
       plt.show()
       ​\end{PY}

\subsection{Contraste et luminosité.}
\begin{Exercice}
	Écrire une fonction \code{luminosite(image,coeff)} qui multiplie chaque composante par coeff.
	
Si coeff<1 on baisse la luminosité et si coeff>1 on l'augmente.

(Vous aurez besoin des fonctions min et max pour éviter les dépassements.)
\end{Exercice}

\begin{PY}
A = img.imread("Chloroplastes.png")
# Foncer
B = luminosite(A,0.5)
plt.imshow(B, interpolation='none')
plt.show()
#Eclaircir
C = luminosite(A,2)
plt.imshow(C, interpolation='none')
plt.show()
\end{PY}

\setcounter{exercices}{\theexercices-1}
\begin{Exercice}[\textbf{(suite)}]
Écrire une fonction \code{contraste(image,coeff)} qui multiplie l'écart de chaque composante (par rapport à la moyenne des composantes sur l'image) par coeff.
Si coeff<1 on diminue le contraste et si coeff>1 on l'augmente.

(Vous aurez besoin des fonctions min et max pour éviter les dépassements.)
\end{Exercice}

\begin{PY}
A = img.imread("Chloroplastes.png")
# Diminuer le contraste
B = contraste(A,0.5)
plt.imshow(B, interpolation='none')
plt.show()
#E Augmenter le contraste
C = contraste(A,2)
plt.imshow(C, interpolation='none')
plt.show()
\end{PY}


\subsection{Changer la résolution.}
\begin{Exercice}
	Écrire une fonction \code{reduire(image,coeff)} divise la taille par le coefficient entier donné.
	
	Par exemple, si le coefficient est 3, un carré de 9 pixels, en deviendra 1 seul.
	
	Chaque nouveau pixel étant la moyenne du groupe de pixels qu'il remplace. Si la
	taille ne tombe pas juste, on arrondira à l'entie inférieur à l'aide de la
	fonction \code{floor}.
\end{Exercice}

\begin{PY}
from math import floor
​
#exemple d'usage de floor :
floor(2.72)
\end{PY}
\begin{PYout}
2
\end{PYout}

\begin{PY}
A = img.imread("Chloroplastes.png")
plt.imshow(A, interpolation='none')
plt.show()
# image / 3
B = reduire(A,3)
plt.imshow(B, interpolation='none')
plt.show()
# Remarquer la graduation des images.
\end{PY}


\begin{Exercice}
Écrire une fonction \code{agrandir(image,coeff)} qui multiplie la taille par le coefficient entier donné.

Par exemple, si le coefficient est 3, un pixel sera remplacé par un carré de 9
pixels, tous affectés de la couleur du pixel de départ.

 (on pourrait le faire pour un coefficient quelconque, il faudrait alors faire une moyenne pondérée des couleurs des pixels recouverts par chaque nouveau pixel, la pondération étant l'aire de recouvrement.)
\end{Exercice}

\begin{PY}
A = img.imread("Chloroplastes.png")
plt.imshow(A, interpolation='none')
plt.show()
# image * 3
B = agrandir(A,3)
plt.imshow(B, interpolation='none')
plt.show()
# Remarquer la graduation des images.
\end{PY}

\subsection{Faire un flou.}
\begin{Exercice}
Écrire une fonction \code{flou(image)} qui réalise un flou : 
la couleur de chaque pixel est la moyenne de sa prope couleur (coeff 4)
et des couleurs des pixel voisins (coeff 1) nord-sud-est-ouest. 
On ne fera aucun traitement sur le bord pour simplifier (les pixels 
seront recopiés).
\end{Exercice}

\begin{PY}
A = img.imread("Chloroplastes.png")
plt.imshow(A, interpolation='none')
plt.show()
# Fait un flou
B = flou(A)
plt.imshow(B, interpolation='none')
plt.show()
# Fait 10 fois le flou
for k in range(9) :
    B=flou(B)
plt.imshow(B, interpolation='none')
plt.show()
\end{PY}

\subsection{Detection de contours.}
\begin{Exercice}
Écrire une fonction \code{contour(image,seuil)} qui créer une image noir et blanc :

On choisit de colorer un pixel en noir si la variation de couleur y est
importante. Pour ce faire, on calcul la somme sur toutes les composantes couleurs de  
$|E−O|+|S−N|$  ( $E$  pour est,  $O$  pour ouest,  $S$  pour sud et 
$N$ pour nord : les pixels voisins du pixel observé). Si cette grandeur dépasse
le seuil, on mettra un pixel noir et sinon un blanc.

On ne fera aucun traitement sur le bord pour simplifier, il sera supprimé.
\end{Exercice}

\begin{PY}
A = img.imread("Chloroplastes.png")
plt.imshow(A, interpolation='none')
plt.show()
# Détection de contours
B = contour(A,1)
plt.imshow(B, interpolation='none')
plt.show()
\end{PY}




\end{document}
