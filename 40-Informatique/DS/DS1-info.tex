\documentclass[twoside,french,11pt]{VcCours}

% cSpell:ignore ndarray Drapide

\begin{document}

\Titre{PSI}{Promotion 2021--2022}{Informatique}{Devoir surveillé n°1}

\begin{center}
\large 
Le samedi 2 octobre 2021

\bigskip
\textbf{Durée: 1h}

\bigskip
\large\underline{\textbf{Calculatrice interdite}}
\end{center}

\bigskip
\begin{itemize}
  \item Le candidat attachera la plus grande importance à la clarté, à la précision et à la concision de la rédaction. Si un candidat est amené à repérer ce qui peut lui sembler être une erreur d'énoncé, il le signalera sur sa copie et devra poursuivre sa composition en expliquant les raisons des initiatives qu'il a été amené à prendre.
  \item Il est conseillé au candidat de lire l'intégralité du sujet et de repérer les parties qui lui semblent plus abordables.
  %\item Les questions de cours sont à traiter \underline{sur la feuille d'énoncé} et les exercices sont à traiter sur une autre feuille.
  \item Le candidat \fbox{encadrera} ou \underline{soulignera} les résultats. 
  %\item Les étudiants visant un concours plus dur que CCP traiteront \textbf{obligatoirement} l'exercice $2$.
  \end{itemize}
\separationTitre

\section*{Simulation de la cinétique d'un gaz parfait}
La théorie cinétique des gaz vise à expliquer le comportement macroscopique d'un gaz à partir des mouvements
des particules qui le composent. Depuis la naissance de l'informatique, de nombreuses simulations numériques
ont permis de retrouver les lois de comportement de différents modèles de gaz comme celui du gaz parfait.

Ce sujet s'intéresse à un gaz parfait monoatomique. Nous considérerons que le gaz étudié est constitué de $N$
particules sphériques, toutes identiques, de masse $m$ et de rayon $R$, confinées dans un récipient rigide. La
simulation sera réalisée dans un espace à une dimension; le récipient contenant le gaz sera
un segment de longueur $L$.

Dans le modèle du gaz parfait, les particules ne subissent aucune force (leur poids est négligé) ni aucune autre
action à distance. Elle n'interagissent que par l'intermédiaire de chocs, avec une autre particule ou avec la paroi
du récipient. Ces chocs sont toujours élastiques, c'est-à-dire que l'énergie cinétique totale est conservée.
Le seul langage de programmation autorisé dans cette épreuve est Python. Pour répondre à une
question il est possible de faire appel aux fonctions définies dans les questions précédentes. Dans tout le sujet
on suppose que les bibliothèques \code{math}, \code{numpy} et \code{random} ont été importées grâce aux instructions
\begin{Python*}  
import math
import numpy as np
import random
\end{Python*}  
Si les candidats font appel à des fonctions d'autres bibliothèques ils doivent préciser 
les instructions d'importation correspondantes.
Ce sujet utilise la syntaxe des annotations pour préciser le types des arguments et du résultat des fonctions à
écrire. Ainsi
\begin{Python*}
def maFonction(n:int, x:float, l:[str]) -> (int, np.ndarray):
\end{Python*} 
signifie que la fonction maFonction prend trois arguments, le premier est un entier, le deuxième un nombre à
virgule flottante et le troisième une liste de chaînes de caractères et qu'elle renvoie un couple dont le premier
élément est un entier et le deuxième un tableau numpy. Il n'est pas demandé aux candidats de recopier les
entêtes avec annotations telles qu'elles sont fournies dans ce sujet, ils peuvent utiliser des entêtes classiques. Ils
veilleront cependant à décrire précisément le rôle des fonctions qu'ils définiraient eux-mêmes.
Une liste de fonctions utiles est donnée à la fin du sujet.

\medskip
\textbf{Représentation en Python}

\smallskip
Chaque particule est représentée par une liste de deux éléments, le premier correspond à la position de son
centre, la deuxième à sa vitesse. Chacun de ces éléments (position et vitesse) est représenté par un vecteur
(np.ndarray) dont le nombre de composantes correspond à la dimension de l'espace de simulation.
Les positions et vitesses sont exprimées sous forme de coordonnées cartésiennes dans un repère orthonormé
dont l'origine est placée dans un coin du récipient contenant le gaz et dont les axes sont parallèles aux côtés
du récipient issus de ce coin de façon à ce que tout point situé à l'intérieur du récipient ait ses coordonnées
comprises entre $0$ et $L$.

Les positions sont exprimées en mètres et les vitesses en $m.s^{−1}$. La figure 1 propose des exemples de particules
dans des espaces de diverses dimensions.
\begin{Python*} 
p1 = [np.array([5.3]), np.array([412.3])]
\end{Python*} 


\section*{Initialisation}
Pour pouvoir réaliser une simulation, il convient de disposer d'une situation initiale, c'est-à-dire d'un ensemble
de particules réparties dans le récipient et dotées d'une vitesse initiale connue. Cette partie s'intéresse au
positionnement aléatoire d'un ensemble de particules. L'attribution de vitesses initiales à ces particules ne sera
pas abordé ici.

\subsection*{A -- Placement en dimension 1}
Nous cherchons d'abord comment placer $N$ particules (sphères de rayon $R$) le long d'un segment de longueur $L$
sans qu'elles se chevauchent ni qu'elles sortent du segment. La figure 2 montre quelques exemples de placements
possibles avec $N = 5$, $R = 0,5$ et $L = 10$.
Exemples de placement de $5$ particules de rayon $0,5$ sur un segment de longueur $10$
La fonction \code{placement1D} construit aléatoirement, à partir des paramètres géométriques du problème (nombre
et rayon des particules, taille du récipient), une liste de coordonnées correspondant à la position initiale du
centre de chaque particule.
\begin{Python}
def placement1D(N:int, R:float, L:float) -> [np.ndarray]:

    def possible(c:np.ndarray) -> bool:
        if c[0] < R or c[0] > L - R: return False
        for p in res:
            if abs(c[0] - p[0]) < 2*R: return False
        return True

    res = []
    while len(res) < N:
        p = L * np.random.rand(1)
        if possible(p): res.append(p)
    return res
\end{Python}
\renewcommand{\labelenumi}{\textbf{Q \theenumi.}}
\begin{enumerate}
\item
Détailler l'action de la ligne 11.
\item
Quelle est la signification du paramètre \code{c} de la fonction possible (ligne 3) ?
\item
Expliquer le rôle de la ligne 4.
\item
Expliquer le rôle des lignes 5 et 6.
\item
Donner en une phrase le rôle de la fonction \code{possible}.
\item
Proposer une nouvelle version de la ligne 11 permettant d'éviter certains rejets de la part de la fonction
possible.
\item
On considère l'appel \code{placement1D(4, 0.5, 5)} et on suppose que les trois premières particules ont
été placées aux points d'abscisses $1$, $2,5$ et $4$ (figure 3). 
Quelle sera la suite du déroulement de la fonction \code{placement1D} ?
\item
Quelle est la complexité temporelle de la fonction \code{placement1D} dans le cas où $N\ll Nmax$, nombre
maximal de particules de rayon $R$ pouvant être placées sur un segment de longueur $L$ ?
\item
Pour remédier de manière simple (mais non optimale) à la situation de la question 7, on décide
de recommencer à zéro le placement des particules dès qu'une particule est rejetée par la fonction possible.
Réécrire les lignes 9 à 13 de la fonction \code{placement1D} pour mettre en œuvre cette décision.
\end{enumerate}

\subsection*{B -- Optimisation du placement en dimension 1}
Pour placer aléatoirement $N$ particules le long d'un segment, nous envisageons une approche plus efficace que
celle étudiée dans la partie A.
L'idée est de calculer l'espace laissé libre sur le segment cible par $N$ particules puis de répartir aléatoirement cet
espace libre entre les particules. Afin de conserver une répartition uniforme des particules dans tout le segment,
nous utilisons l'algorithme suivant :
\begin{itemize}
  \item[1.] déterminer $\ell$, espace laissé libre par les $N$ particules dans le segment $[0, L[$ ;
  \item[2.] placer aléatoirement dans le segment $[0, \ell[$, $N$ particules virtuelles ponctuelles ($R = 0$) ; à cette étape, deux
particules peuvent parfaitement occuper la même abscisse : il n'y a pas de conflit ;
  \item[3.] remplacer chaque particule virtuelle par une particule réelle de rayon $R$ en décalant toutes les particules
(réelles et virtuelles) situées plus à droite de façon à dégager l'espace nécessaire.
\end{itemize}
\begin{enumerate}\setcounter{enumi}{9}
  \item
Écrire la fonction d'entête
\begin{Python*}
def placement1Drapide(N:int, R:float, L:float) -> [np.ndarray]:
\end{Python*}
qui implante cet algorithme et renvoie la liste des coordonnées des centres de $N$ particules de rayon $R$ réparties
aléatoirement le long d'un segment situé entre les abscisses $0$ et $L$. On précise que l'ordre de la liste résultat
n'est pas important.
\item
Quelle est la complexité de la fonction \code{placement1Drapide} ? Commenter.
\end{enumerate}
% \subsection*{C -- Analyse statistique}
% Afin de vérifier que la fonction \code{placement1Drapide} produit une répartition de particules uniformément répartie
% sur le segment cible, on l'appelle un grand nombre de fois et on comptabilise pour chaque résultat obtenu la
% position initiale de chaque particule. Le résultat final est présenté sous forme d'un histogramme dont l'axe
% horizontal correspond à l'abscisse du centre de la particule dans l'intervalle $[0, L]$ et l'axe vertical au nombre
% total de particules placées à cette abscisse au cours des différentes exécutions de la fonction.
% \begin{enumerate}\setcounter{enumi}{11}
% \item Tracer et justifier l'allure des histogrammes pour $N$ = 1, $N$ = 2 et $N$ = 5 dans le cas où $R$ = 1 et
% $L$ = 10.
% \end{enumerate}
% \subsection*{D -- Dimension quelconque}
% L'algorithme optimisé pour un segment, n'est pas utilisable pour des espaces de dimensions supérieures. Nous
% allons donc généraliser la fonction \code{placement1D} pour la transformer en une fonction utilisable dans un espace
% de dimension 1, 2 ou 3.
% \begin{enumerate}\setcounter{enumi}{12}
%   \item
% En s'inspirant de la fonction \code{placement1D}, écrire la fonction d'entête
% def placement(D:int, N:int, R:float, L:float) -> [np.ndarray]:
% qui renvoie la liste des coordonnées des centres de N particules sphériques de rayon R placées aléatoirement dans
% un récipient de côté L dans un espace à D dimensions. Les modifications prévues aux questions 6 et 9 seront
% prises en compte dans cette fonction.
% \end{enumerate}

\end{document}