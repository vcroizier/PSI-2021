\documentclass[french,11pt,twoside]{VcCours}

\begin{document}

\Titre{PSI}{Promotion 2021--2022}{Informatique}{TP d'info n°7 -- Calcul approché d'intégrales.}

\tableofcontents
\separationTitre


\subsection*{Préliminaires.}

\emph{N'oubliez pas de créer un nouveau répertoire pour ce TP et de
le mettre comme répertoire de travail pour Python.}

\begin{Exercice}{}
Créer une fonction \code{f(x)} qui calcule la valeur de la fonction
$f(x)=\frac{1}{x+2}$.
\end{Exercice}

\begin{Exercice}{}
Créer une fonction \code{subdivision(a,b,n)} qui donne la liste des points de
la subdivision régulière de l'intervalle $[a;b]$ ayant $n+1$
points.

C'est-à-dire les valeurs des $x_k=a+k\frac{b-a}{n}$, où
$k\in[\![0;n]\!]$.
\end{Exercice}

Voici un exercice de compréhension de code:

\begin{Exercice}{}
Lire attentivement et essayez de déterminer la formule mathématique 
calculée par la fonction Python suivante:

Quelle est la limite de cette expression quand $n$ tend vers $+\infty$ ?
\end{Exercice}

\begin{Python}
  def defi(n) :
      S = 0
      for k in range(1,n+1) :
          S = S + 1/(1+k/n)
      return(S/n)
  \end{Python}
  

\subsection{Méthodes numériques classiques.}

On souhaite maintenant calculer l'intégrale $I=\int_a^b
f=\int_{-1}^1\frac{1}{t+2}dt$.

\subsubsection{Méthode des rectangles.}
Ici on approche l'aire correspondant à l'intégrale par l'aire de la
zone hachurée.

\begin{multicols}{3}
  \begin{center}
{\shorthandoff{:}
\begin{tikzpicture}[scale=0.7]
\draw[thick,->] (0,0) -- (5.3,0);%
\draw[thick,->] (0,0) -- (0,5.3);%
\draw[samples=100,semithick,domain=1:5] plot ({\x},{(-\x^3+11*\x^2-32*\x+40)/6});%
\foreach \x/\y/\xx in {1/3/1, 2/2/2, 3/2.666666/3, 4/4/4}%
{
  \draw[pattern=north east lines] (\x,0) rectangle (\x+1,\y);%
  \fill[black] (\xx,\y) circle (1.5pt);%
}%
\foreach \x in {0,1,2,3,4}%
  \draw[shift={(\x+1,0)}] (0pt,2pt) -- (0pt,-2pt) node[below] {$x_\x$};%
\end{tikzpicture}}
\og rectangles à gauche\fg

{\shorthandoff{:}
\begin{tikzpicture}[scale=0.7]
\draw[thick,->] (0,0) -- (5.3,0);%
\draw[thick,->] (0,0) -- (0,5.3);%
\draw[samples=100,semithick,domain=1:5] plot ({\x},{(-\x^3+11*\x^2-32*\x+40)/6});%
\foreach \x/\y/\xx in {1/2.229166667/1.5, 2/2.187500000/2.5, 3/3.312499997/3.5, 4/4.604166667/4.5}%
{
  \draw[pattern=north east lines] (\x,0) rectangle (\x+1,\y);%
  \fill[black] (\xx,\y) circle (1.5pt);%
}%
\foreach \x in {0,1,2,3,4}%
  \draw[shift={(\x+1,0)}] (0pt,2pt) -- (0pt,-2pt) node[below] {$x_\x$};%
\end{tikzpicture}}
Points médians

{\shorthandoff{:}
\begin{tikzpicture}[scale=0.7]
\draw[thick,->] (0,0) -- (5.3,0);%
\draw[thick,->] (0,0) -- (0,5.3);%
\draw[samples=100,semithick,domain=1:5] plot ({\x},{(-\x^3+11*\x^2-32*\x+40)/6});%
\foreach \x/\y/\xx in {1/2/2, 2/2.666666/3, 3/4/4, 4/5/5}%
{
  \draw[pattern=north east lines] (\x,0) rectangle (\x+1,\y);%
  \fill[black] (\xx,\y) circle (1.5pt);%
}%
\foreach \x in {0,1,2,3,4}%
  \draw[shift={(\x+1,0)}] (0pt,2pt) -- (0pt,-2pt) node[below] {$x_\x$};%
\end{tikzpicture}}
\og rectangles à droite\fg
  \end{center}
\end{multicols}




\begin{Exercice}{}
Créer une fonction \code{medians(f,a,b,n)} qui calcul la valeur
approchée de $\int_a^b f$ par la méthode des point médians.

C'est-à-dire la valeur
\[R_n=\frac{b-a}{n}\sum_{k=1}^nf\left(\frac{x_k+x_{k-1}}{2}\right)=\frac{b-a}{n}\sum_{k=1}^nf\left(x_{k-\frac12}\right)\]
\end{Exercice}

\subsubsection{Méthode des trapèzes.}
On approche l'intégrale à l'aide de trapèzes ainsi:

\begin{center}
{\shorthandoff{:}
\begin{tikzpicture}[scale=1.5]
\draw[thick,->] (0,0) -- (5.3,0);%
\draw[thick,->] (0,0) -- (0,5.3);%
\draw[samples=100,semithick,domain=1:5] plot ({\x},{(-\x^3+11*\x^2-32*\x+40)/6});%
\foreach \x/\y/\yy in {1/3/2, 2/2/2.666666, 3/2.666666/4, 4/4/5}%
{
  \draw[pattern=north west lines] (\x,0) -- (\x,\y) -- (\x+1,\yy) -- (\x+1,0) -- cycle;%
  \fill[black] (\x,\y) circle (1.5pt);%
  \fill[black] (\x+1,\yy) circle (1.5pt);%
}%
\foreach \x in {0,1,2,3,4}%
  \draw[shift={(\x+1,0)}] (0pt,2pt) -- (0pt,-2pt) node[below] {$x_\x$};%
\end{tikzpicture}}
  \end{center}

On en déduit la formule des trapèzes:
\[T_n=\frac{b-a}{n}\sum_{k=1}^{n}\frac{f(x_k)+f(x_{k-1})}{2}=\frac{b-a}{n}\left(\frac{f(a)+f(b)}{2}+\sum_{k=1}^{n-1}f(x_k)\right)\]
où les $x_k$ sont les points de la subdivision régulière.

\begin{Exercice}{}
Créer une fonction \code{trapezes(f,a,b,n)} qui calcul la valeur
approchée de $\int_a^b f$ par la méthode des trapèzes.
\end{Exercice}

\subsubsection{Méthode de Simpson.}
La méthode Simpson consiste à approché l'intégrale de $f$ sur chaque
segment $[x_{k-1};x_k]$ par l'intégrale de la fonction polynôme $P$
de degré maximum 2 qui coïncide avec $f$ aux points $x_{k-1}$, $x_k$
et $\frac{x_{k-1}+x_{k}}{2}=x_{k-\frac12}$:

\begin{center}
{\shorthandoff{:}
\begin{tikzpicture}[scale=1.5]
\draw[thick,->] (0,0) -- (5.3,0);%
\draw[thick,->] (0,0) -- (0,5.3);%
\draw[samples=100,semithick,domain=1:5] plot ({\x},{(-\x^3+11*\x^2-32*\x+40)/6});%
\foreach \x/\y/\yy/\yyy/\yyyy in {1/3/2.229166667/2/2.15, 2/2/2.187500000/2.666666/2.15, 3/2.666666/3.312499997/4/3.31, 4/4/4.604166667/5/4.65}%
{
  \draw[pattern=north west lines] (\x,0) -- (\x,\y) .. controls (\x+0.5,\yyyy) and (\x+0.5,\yyyy) .. (\x+1,\yyy) -- (\x+1,0) -- cycle;%
  \fill[black] (\x,\y) circle (1.5pt);%
  \fill[black] (\x+0.5,\yy) circle (1.5pt);%
  \fill[black] (\x+1,\yyy) circle (1.5pt);%
}%
\foreach \x in {0,1,2,3,4}%
  \draw[shift={(\x+1,0)}] (0pt,2pt) -- (0pt,-2pt) node[below] {$x_\x$};%
\end{tikzpicture}}
  \end{center}

\begin{Exercice}{}
Créer une fonction \code{simpson(f,a,b,n)} qui calcul la valeur
approchée de $\int_a^b f$ par la méthode de Simpson. c'est-à-dire la
valeur de
\[S_n=\frac{b-a}{n}\sum_{k=1}^{n}\left(\frac16f(x_{k-1})+\frac23f(x_{k-\frac12})+\frac16f(x_{k})\right)=\frac{T_n+2R_n}{3}\]
\end{Exercice}

\subsubsection{Comparaison des vitesses de convergence.}
Pour chacune des trois méthodes précédentes, l'erreur
d'approximation est de la forme $\frac{M}{n^p}$, où $M\in\R$ et
$p\in\N$ sont tous deux à peu près indépendants de $n$.

On se propose d'évaluer $p$.

\begin{Exercice}{}
Indiquer les erreurs (en notation scientifique avec 2 chiffres
significatifs seulement) obtenues dans l'approximation de
$\int_{-1}^1\frac{dt}{t+2}$ en utilisant $n = 10^k$ points de
subdivision, quand l'entier $k$ varie de $1$ à $4$.

\begin{center}
\begin{tabular}{|l|c|c|c|c||c|}
  \hline
  & $n=10^1$ & $n=10^2$ & $n=10^3$ & $n=10^4$ & $p$ \\
  \hline
  Rectangles & \hspace{2cm} & \hspace{2cm} & \hspace{2cm} & \hspace{2cm} & \hspace{1cm} \\
  \hline
  Trapèzes   &  &  &  &  &  \\
  \hline
  Simpson    &  &  &  &  &  \\
  \hline
\end{tabular}
\end{center}

Avec ces listes évaluer numériquement $p$ pour chacune des $3$
méthodes.
\end{Exercice}

\subsection{Méthode probabiliste -- Algorithme de Monte-Carlo.}
L'idée directrice de cette méthode est que, si on prends $n$ valeurs
$x_k$ au hasard sur $[a;b]$ (en suivant une loi uniforme), alors la
moyenne des $f(x_k)$ donne une valeur approchée de la moyenne de $f$
sur $[a;b]$.

(C'est une méthode dite probabiliste.)

\begin{Exercice}{}
Créer une fonction \code{hasard(a,b)} qui donne une valeur au hasard
sur $[a;b]$. (Utiliser \code{random} après l'avoir importée avec \code{from random import random}.)

Ensuite, écrire une fonction \code{montecarlo(f,a,b,n)} qui calcule
par la méthode de Monte-Carlo une valeur approchée de $\int_a^bf$,
c'est-à-dire $\frac{b-a}{n}\sum_{k=1}^nf(x_k)$ où les $x_k$ sont des
valeur prisent au hasard sur $[a;b]$.
\end{Exercice}

Ce type de méthode est très utilisé pour des intégrales multiples,
mais très peu pour intégrales simples comme c'est le cas ici.

\subsection{Accélération de convergence -- Méthode de Romberg.}
On peut montrer, pour $f$ suffisamment régulière et à l'aide de la
formule d'Euler-MacLaurin que l'on verra plus tard dans l'année, que
$T_n$ (méthode des trapèzes) admet un développement limité de la
forme~: \[T_n=\int_a^bf+c_1\frac{1}{n^2}+c_2\frac{1}{n^4}+
\dots+c_k\frac{1}{n^{2k}}+o\left({\frac{1}{n^{2k}}}\right)\]

La signification exacte de cette écriture sera vue plus tard, pour
l'instant il suffit de voir que cela veut dire que $T_n$ tend vers
$\int_a^bf$ et que l'erreur commise est équivalente à
$c_1\frac{1}{n^2}$.

\medskip
Alors on a
\[\frac{4T_{2n}-T_n}{4-1}=\int_a^bf-\frac{1}{4}c_2\frac{1}{n^4}-
\dots-\frac{2^{2k}-4}{3\times2^{2k}}c_k\frac{1}{n^{2k}}
+o\left(\frac{1}{n^{2k}}\right)\]

En réitérant cette man{\oe}uvre on peut éliminer les termes
suivants du développement limité, tout en conservant la même limite.

\begin{Exercice}{}
Écrire une fonction \code{romberg(f,a,b,n)} qui applique la
méthode de Romberg~:
\begin{itemize}
  \item Pour $i=1\ldots n$, $T_{1,i}=T_{2^{i-1}}$.
  \item Pour $j=1\ldots n-1$ et pour $i=1\ldots n-j$, $T_{j+1,i}=\frac{2^{2j}T_{j,i+1}-T_{j,i}}{2^{2j}-1}$
\end{itemize}
Le résultat est $T_{n,1}$.

(Tout ceci peut être calculé dans un tableau à 2 dimensions.)

Observer la précision de cette méthode par rapport aux autres.
\end{Exercice}



\end{document}
