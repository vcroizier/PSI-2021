\documentclass[french,11pt,twoside]{VcCours}
%cSpell:ignore Fact mystere Fibo Puiss
\begin{document}

\Titre{PSI}{Promotion 2021--2022}{Informatique}{TP d'info n°6 -- Complexité}

\tableofcontents
\separationTitre


\section{Cours}

\subsection{Introduction}

L'analyse de la complexité d'un programme en informatique consiste 
à mesurer deux grandeurs :

\begin{enumerate}
\item Le temps de calcul pour exécuter les opérations.
\item La place mémoire nécessaire pour exécuter ce programme.
\end{enumerate}

\medskip 

Le premier point est l'étude de la \emph{complexité en temps} 
(combien d'opérations sont nécessaires pour exécuter le programme ?) 
et le deuxième est l'étude de la \emph{complexité en mémoire} 
(combien d'espace mémoire est nécessaire pour exécuter le programme?). 

\medskip

Dans ce chapitre, on étudiera presque uniquement la complexité en temps. 
On considérera que les opérations élémentaires (test, affectation, 
opération arithmétique) ont le même coût. Il faut bien comprendre que c'est 
une simplification du problème : les coûts ne sont pas nécessairement 
les mêmes et deux types d'ordinateurs différents ne fonctionnent pas 
de la même manière.

\medskip

\begin{Exercice}{} Soient $n \geq 1$ et 
$P : x \mapsto a_0 + a_1 x + \cdots + a_n x^n$ une fonction polynomiale. 
Déterminons le coût du calcul de $P(x)$ pour un réel $x$ :

\vspace*{4cm}

%Pour calculer $P(x)$ de manière directe, on a besoin :
%
%\begin{itemize}
%\item de $n$ additions.
%\item de $1+2+ \cdots + n = \dfrac{n(n+1)}{2}$ multiplications.
%\end{itemize}
%Ainsi, le coût du calcul de $P(x)$ vaut :
%$$ \mathcal{C}(n) = n + \frac{n(n+1)}{2} = \frac{n(n+3)}{2}$$
\end{Exercice}

\subsection{Une remarque sur la taille des instances}
On exprime la complexité à l'aide des données nécessaires pour écrire l'algorithme. Ces données sont appelées les \emph{instances}. Par exemple :

\begin{itemize}
\item Si on souhaite trier une liste à $n$ éléments, on peut choisir $n$ comme instance.
\item Si on crée un algorithme lié à un polynôme, on peut choisir son degré comme instance. 
\item Si l'on souhaite savoir si un entier $n$ est premier, on peut choisir $n$ comme instance.
\end{itemize}

\medskip

\begin{Remarque}[\alerte]{}

    Considérons un entier naturel $n$ non nul. Celui-ci s'écrit de manière unique sous la forme :
$$ n = a_{p-1}2^{p-1} + \cdots + a_1 2^1 + a_0 2^0$$
où $p \in \mathbb{N}^*$, pour tout $i \in \ii{0}{p-2}$, $a_i \in \lbrace 0,1 \rbrace$ et $a_{p-1}=1$.

\medskip

On a donc besoin de $p$ bits pour stocker le développement binaire de $n$. 
On peut alors choisir $p$ comme instance plutôt que $n$. 
Exprimons $p$ en fonction de $n$ :

\vspace*{4cm}~

\end{Remarque}

\subsection{Comment comparer les ordres de grandeur ?}

Si l'on considère un algorithme dont l'instance est un entier $n$ et que l'on note $\mathcal{C}(n)$ le coût (en calcul ou en mémoire selon ce que l'on veut faire) de cet algorithme, il semble naturel (par exemple si $n$ est la longueur d'une liste) que plus $n$ est grand, plus le coût devrait augmenter. Dans ce cas, quels sont les ordres de grandeurs classiques ?

\medskip

On considère dans la suite que toutes les suites sont à termes strictement positifs.

\begin{Definition}{} Soient $u$ et $v$ deux suites. 
    On dit que $u$ est dominée par $v$ et on note $u=O(v)$ (ou $u\ll v$) 
    lorsque la suite $\dfrac{u}{v}$ est bornée, c'est-à-dire si il existe 
    un réel $M>0$ tel que pour tout $n \geq 0$,
    $$ \frac{u_n}{v_n} \leq M$$
\end{Definition} 

Remarquons que $O$ est transitive : si $u,v,w$ sont trois suites 
et si $u=O(v)$ et $v=O(w)$ alors $u=O(w)$.

\medskip

Il est important de retenir les ordres de grandeurs suivants (en notant abusivement $u_n$ au lieu de $u$) : 
$$ 1\ll \ln(n) \ll n \ll n \ln(n) \ll n^2 \ll n^3 \ll 2^n \ll n! \ll n^n$$

\begin{Exercice}{} Soient $n \geq 1$ et $P : x \mapsto a_0 + a_1 x + \cdots + a_n x^n$ une fonction polynomiale. Pour calculer $P(x)$, on a montré que le coût des opérations est égal à :
$$\mathcal{C}(n)=\frac{n(n+3)}{2} \sim \frac{1}{2}n^2$$
Donc $\left(\frac{\mathcal{C}(n)}{n^2} \right)$ est convergente donc bornée.
Donc
$$ \mathcal{C}(n) = O(n^2)$$
\end{Exercice}

\vspace{-1em}
\begin{Remarque}{} Dire que la complexité est en $O(n^2)$ manque de précision 
    car d'une part cela implique aussi que la complexité est en $O(n^3)$, 
    $O(n!)$ mais d'autre part rien ne précise si l'on ne peut pas faire mieux. 
    On définit alors une nouvelle notion : on dit par exemple que la complexité 
    est $O(n^2)$ \og et pas mieux \fg{} si l'on a $\mathcal{C}(n)=O(n^2)$ mais 
    aussi $n^2 = O(\mathcal{C}(n))$.
\end{Remarque}

\begin{Exercice}{} Dans l'exemple précédent, la complexité est en $O(n^2)$ et pas mieux. Il suffit de remarquer que :
$$ \mathcal{C}(n) \sim \frac{n^2}{2}$$
\end{Exercice}

\vspace{-1em}
\begin{Remarque}{} Si il existe $\lambda >0$ tel que :
$$ \mathcal{C}(n) \sim \lambda u_n$$
alors la complexité est en $O(u_n)$ et pas mieux.
\end{Remarque}

\subsection{Quels sont les ordres de grandeurs les plus utilisés?}

Notons $\mathcal{C}(n)$ le coût d'exécution d'un programme (par exemple obtenu à l'aide de $n$ opérations élémentaires).  Voici les ordres de grandeurs les plus utilisés (par ordre de croissance). On dit que la complexité est :

\begin{itemize}
\item au plus constante si $\mathcal{C}(n)=O(1)$ et constante si $\mathcal{C}(n)=O(1)$ et pas mieux.
\item au plus logarithmique si $\mathcal{C}(n)=O(\ln(n))$ et logarithmique si $\mathcal{C}(n)=O(\ln(n))$ et pas mieux.
\item au plus linéaire si $\mathcal{C}(n)=O(n)$ et linéaire si $\mathcal{C}(n)=O(n)$ et pas mieux.
\item au plus quasi-linéaire si $\mathcal{C}(n)=O(n\ln(n))$ et quasi-linéaire si $\mathcal{C}(n)=O(n\ln(n))$ et pas mieux.
\item au plus quadratique si $\mathcal{C}(n)=O(n^2)$ et quadratique si $\mathcal{C}(n)=O(n^2)$ et pas mieux.
\item au plus exponentielle si $\mathcal{C}(n)=O(q^n)$ (pour un certain $q>1$) et exponentielle si $\mathcal{C}(n)=O(q^n)$ et pas mieux.
\end{itemize}

\subsection{Quelques méthodes pour trouver l'ordre de grandeur}
Soient $j,k$ des entiers tels que $j<k$ et $A$, $B$ et $C$ des instructions ne \emph{dépendant pas} de $j$ ou de $k$.

\begin{itemize}
\item L'instruction \og Si A alors B sinon C \fg{} a un coût au plus égal à :
$$ \hbox{Cout(A)+ Maximum(Cout(B),Cout(C))}$$
\item L'instruction \og Pour i allant de j à k faire A \fg{} a pour coût :
$$ (k-j+1)\hbox{Cout(A)}$$
\item  L'instruction \og Tant que A faire B \fg{} a pour coût :
$$ \hbox{(Nombre d'itérations)} \times \hbox{(Cout(A)+Cout(B))}$$
\end{itemize}

\vspace{-1em}
\begin{Exercice}{} Donnons le coût de l'algorithme suivant : 
\end{Exercice}
\begin{minipage}{\linewidth}
\begin{Python}
S=0
for i in range(n):
    for j in range(n):
        S=S+1
\end{Python} 
\end{minipage}

\vspace{3cm}



\subsection{Un exemple : le tri par Sélection}

\begin{Python}
def TriSelect(L):
    n=len(L)
    for i in range(n-1):
        M,k=L[i],i
        for j in range(i+1,n):
            if L[j]<M:
                M,k=L[j],j
        L[i],L[k]=L[k],L[i]
    return L
\end{Python} 
Comment fonctionne le tri par sélection ?

\vspace{3cm}

Pour l'étude de la complexité d'un tri, on s'intéresse généralement au nombre de comparaisons effectuées. Ce tri par sélection, pour une liste de longueur $n$, effectue un nombre de comparaisons égal à :

\vspace*{3cm}

\newpage
\section{TP}
\subsection{Extrait modifié CCP MP}

\begin{enumerate}
\item
\begin{enumerate}
\item Écrire à l'aide d'une boucle for une fonction \code{factorielle} qui prend en argument un entier naturel $n$ et qui renvoie $n!$.
\item Déterminer le nombre de multiplications qu'effectue \code{factorielle}($n$).
\end{enumerate}
\item On considère la fonction suivante nommée \code{mystere} :
\begin{Python}
def mystere(n):
    s=0
    for k in range(n+1):
        s=s+factorielle(k)
    return s
\end{Python} 
\begin{enumerate}
\item Quelle valeur renvoie \code{mystere}(4) ?
\item Déterminer le nombre de multiplications qu'effectue \code{mystere}($n$).
\item Proposer une amélioration du script de la fonction \code{mystere} afin d'obtenir une complexité linéaire (liée au nombre de multiplications uniquement).
\end{enumerate}
\end{enumerate}
\subsection{Algorithme de Horner}
Soient $x \in \mathbb{R}$ et $P \in \mathbb{R}[X]$ de degré $n \geq 1$ de la forme :
$$ P(X) = a_n X^n + a_{n-1} X^{n-1} + \cdots + a_1 X + a_0 $$
Pour calculer $P(x)$, on peut remarquer que :
\begin{align*}
P(x) & = (a_n x^{n-1} + \cdots + a_1)x + a_0 \\
& = ((a_n x^{n-2} + \cdots + a_2)x + a_1)x + a_0 \\
& = ((\cdots(a_n x + a_{n-1})x + \cdots)x + a_1)x + a_0
\end{align*}
Ainsi, pour calculer $P(x)$, il suffit d'utiliser une variable $R$ valant $a_n$, puis $a_nx+a_{n-1}$ puis ... On pourra identifier le polynôme $P$ avec la liste $[a_n, \ldots, a_0]$.

\begin{enumerate}
\item Écrire une fonction Horner prenant en argument un polynôme et un réel $x$ et qui renvoie $P(x)$ en utilisant l'algorithme de Horner.
\item Déterminer la complexité en temps (addition, multiplication) selon le degré de $P$.
\end{enumerate}

\subsection{Matrices creuses}

Une matrice carrée est dite creuse si elle contient majoritairement des coefficients nuls (disons $x/100$ dans ce TP). On a alors deux manières de représenter cette matrice :

\begin{itemize}
\item En utilisant le type array de numpy : on représente cette matrice par un tableau à deux dimensions.
\item En utilisant une liste qui ne contient que les coefficients non nuls de la matrice. Dans ce cas, chaque élément de cette liste est une liste à 3 éléments $(i,j,a)$ où $i$ est le numéro de la ligne de la matrice, $j$ le numéro de la colonne de la matrice et $a$ le coefficient $a_{i,j}$.
\end{itemize}

\begin{enumerate}
\item
\begin{enumerate}
\item Écrire une fonction \code{somme1} prenant en argument un tableau à deux dimensions (correspondant à une matrice creuse de taille $n \times n$) et qui renvoie la somme des coefficients.
\item Que peut-on dire de la complexité en mémoire dans ce cas ? De la complexité en temps ?
\end{enumerate}
\item \begin{enumerate}
\item Écrire une fonction \code{somme2} prenant en argument une liste (correspondant à une matrice creuse de taille $n \times n$) et qui renvoie la somme des coefficients.
\item Que peut-on dire de la complexité en mémoire dans ce cas ? De la complexité en temps ?
\end{enumerate}
\item Déterminer l'existence critique d'une valeur de $x$ à partir de laquelle une méthode l'emporte sur l'autre.
\end{enumerate}

\subsection{Extrait modifié PT}

\begin{enumerate}
\item Soit l'entier $n=1234$. Quel est le quotient de la division euclidienne de $n$ par $10$? Et le reste? Que se passe t-il si on recommence la division par $10$ à partir de $q$ ?
\item Écrire une suite d'instructions calculant la somme des cubes des chiffres de l'entier $1234$.
\item Écrire une fonction \code{somcube}, d'argument $n$, renvoyant la somme des cubes des chiffres de l'entier $n$.
\item Déterminer la complexité de la fonction \code{somcube} (on comptera les affectations et les opérations).
\end{enumerate}

\end{document}