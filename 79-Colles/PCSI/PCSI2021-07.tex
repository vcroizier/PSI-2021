\documentclass[twoside,a4paper,landscape,french,10pt]{VcCours}

%%%%%%%%%%%%%%%%%%%%%%%%%%%%
\setlength\columnseprule{0.4pt}%
\addtolength\columnsep{1cm}

%%%%%%%%%%%%%%%%%%%%%%%%%%%%
\newenvironment{colle}[4]{{\large\bf\makebox[0cm][l]{#1\hspace{\stretch{-1}}}\hspace{\stretch{1}}#2\hspace{\stretch{1}}\makebox[0cm][r]{\hspace{\stretch{-1}}#3}}
\noindent\rule{\linewidth}{0.4pt}
\begin{multicols*}{#4}}{\end{multicols*}\pagebreak}
\newenvironment{eleve}[1]{\textbf{\large #1}}{\vspace{\stretch{1}}
\columnbreak}
\newenvironment{eleveF}[1]{\textbf{\large #1}}{}
\newcommand{\cours}{\medskip\uwave{Cours :}

\smallskip}
\newcommand{\exo}[1]{\vspace{\stretch{1}}\uwave{Exercice #1:}

\smallskip}

\begin{document}

\pagestyle{empty}

%%%%%%%%%%%%%%%%%%%%%%%%%%%%%%%%%%%%%%%%%%%%%
%+-----------------------------------------+%
%%%%%%%%%%%%%%%%%%%%%%%%%%%%%%%%%%%%%%%%%%%%%


\begin{colle}{PCSI -- Groupe n°}{Colleur: Vincent Croizier}{Semaine
n\degres7 -- le jeudi 18/11/2021}{3}
%%%%%%%%%%%%%%%%%%%%%%%%%%%%%%%%%%%%%%%%%%%%%%%%%%%%%%%%%%%%%%%%%%%%%%%%%%%%%%%
\begin{eleve}{}

    \cours
    Tout sur exp et ln : Def.13.1, énoncé complet du Thm.16, tracé des graphes de exp et ln, tableaux
de variations, preuve du Thm.16.2.


    \exo1
    Résoudre $2\arcsin(x)=\arcsin(2x\sqrt{1-x^2})$

    \exo2
    Étudier et simplifier $f:x\longmapsto2\arccos\left(\frac{1-x^2}{1+x^2}\right)$.
    

    \vspace*{\stretch{2}}

\end{eleve}
%%%%%%%%%%%%%%%%%%%%%%%%%%%%%%%%%%%%%%%%%%%%%%%%%%%%%%%%%%%%%%%%%%%%%%%%%%%%%%%%
\begin{eleve}{}

  \cours
  Tout sur arcsin : énoncé complet du Thm 7, tracé des graphes de sin et arcsin, tableaux de variations,
interprétation (Ex 8 : si $x \in [−1; 1]$, arcsin($x$) est l'unique...) et preuve du Thm 7.2 en admettant
la Prop 9.


  \exo1
  Étudier et simplifier\\ $f:x\longmapsto2\arctan\left(\frac{x^2-2x-1}{x^2+2x-1}\right)$.

  \exo2
  Résoudre $\log_x(10)+2\log_{10x}(10)+3\log_{100x}(10)=0$.


  \vspace*{\stretch{2}}
    
\end{eleve}
%%%%%%%%%%%%%%%%%%%%%%%%%%%%%%%%%%%%%%%%%%%%%%%%%%%%%%%%%%%%%%%%%%%%%%%%%%%%%%%%
\begin{eleveF}{}
    
  \cours
  Tout sur arctan : énoncé complet du Thm 10, tracé des graphes de tan et arctan, tableaux de
  variations, interprétation (Ex 11 : si $x \in \R$, arctan($x$) est l'unique...) et preuve du Thm 10.2.

  \exo1
  Étudier et simplifier $f:x\longmapsto2\arctan(\sqrt{1+x^2}-x)+\arctan(x)$. 

  \exo2
  Résoudre $\sys{\log_y(x)+\log_x(y)=\frac{50}7\\xy=256}$


  \vspace*{\stretch{2}}
  
\end{eleveF}
%%%%%%%%%%%%%%%%%%%%%%%%%%%%%%%%%%%%%%%%%%%%%%%%%%%%%%%%%%%%%%%%%%%%%%%%%%%%%%%%
\end{colle}%
\end{document}
    