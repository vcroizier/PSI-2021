\documentclass[twoside,a4paper,landscape,french,10pt]{VcCours}

%%%%%%%%%%%%%%%%%%%%%%%%%%%%
\setlength\columnseprule{0.4pt}%
\addtolength\columnsep{1cm}

%%%%%%%%%%%%%%%%%%%%%%%%%%%%
\newenvironment{colle}[4]{{\large\bf\makebox[0cm][l]{#1\hspace{\stretch{-1}}}\hspace{\stretch{1}}#2\hspace{\stretch{1}}\makebox[0cm][r]{\hspace{\stretch{-1}}#3}}
\noindent\rule{\linewidth}{0.4pt}
\begin{multicols*}{#4}}{\end{multicols*}\pagebreak}
\newenvironment{eleve}[1]{\textbf{\large #1}}{\vspace{\stretch{1}}
\columnbreak}
\newenvironment{eleveF}[1]{\textbf{\large #1}}{}
\newcommand{\cours}{\medskip\uwave{Cours :}

\smallskip}
\newcommand{\exo}[1]{\vspace{\stretch{1}}\uwave{Exercice #1:}

\smallskip}

\begin{document}

\pagestyle{empty}

%%%%%%%%%%%%%%%%%%%%%%%%%%%%%%%%%%%%%%%%%%%%%
%+-----------------------------------------+%
%%%%%%%%%%%%%%%%%%%%%%%%%%%%%%%%%%%%%%%%%%%%%


\begin{colle}{PCSI -- Groupe n°}{Colleur: Vincent Croizier}{Semaine
n\degres3 -- le jeudi 07/10/2021}{3}
%%%%%%%%%%%%%%%%%%%%%%%%%%%%%%%%%%%%%%%%%%%%%%%%%%%%%%%%%%%%%%%%%%%%%%%%%%%%%%%
\begin{eleve}{}

    \cours
    $1+e^{i\theta}=\ldots$\\
    Simplifier $\sum_{k=2}^n\cos^k(x)$.


    \exo1
    Déterminer les racines $4^{\text{ème}}$ de $28-96i$.

    %Résoudre $2z^2-(4-4i)z-5i=0$.

    \exo2
    Soit $n\in \N^*$ et $\omega =e^{i\frac{2\pi }{n}}$.
\begin{enumerate}
\item  Que vaut $\sum\limits_{k=1}^{n}\omega ^{k}$ ?
\item  Soit $p\in \Z$. Calculer $\sum\limits_{k=1}^{n}(\omega
^{p})^{k}$.
\item  Montrer que $\forall z\in \C,\sum\limits_{k=1}^{n}(z+\omega
^{k})^{n}=n(z^{n}+1)\!\!\!\!\!$.
\end{enumerate}

    \exo3
    Résoudre $\cos(2x)-\sqrt3\sin(2x)=1$.

    \vspace*{\stretch{2}}

\end{eleve}
%%%%%%%%%%%%%%%%%%%%%%%%%%%%%%%%%%%%%%%%%%%%%%%%%%%%%%%%%%%%%%%%%%%%%%%%%%%%%%%%
\begin{eleve}{}

  \cours
  $1-e^{i\theta}=\ldots$\\
  Simplifier $\sum_{k=5}^n\sin^k(x)$.

  \exo1
  Résoudre $X^2-(7-2i)X+13-i=0$.

  \exo2
  Soit $n\in\N$. Résoudre dans $\C$: $(z+i)^n = (i - z)^n$.

  \exo3
  Résoudre $\sin^2(3x)+3\cos(3x)-1<0$.

  \vspace*{\stretch{2}}
    
\end{eleve}
%%%%%%%%%%%%%%%%%%%%%%%%%%%%%%%%%%%%%%%%%%%%%%%%%%%%%%%%%%%%%%%%%%%%%%%%%%%%%%%%
\begin{eleveF}{}
    
  \cours
  $e^{ia}+e^{ib}=\ldots$\\
  Simplifier $\sum_{k=p}^ne^{ikx}$.

  \exo1
  Résoudre $X^2-(5-2i)X+9-7i=0$.
  
  (Calculer $17^2$.)
  
  \exo2
  Résoudre dans $\C$ l'équation: $(z + 1)^n = e^{2ina}$.

  \exo3
  Résoudre $\cos(2x+\tfrac{\pi}{3})=\sin(2x)$

  \vspace*{\stretch{2}}
  
\end{eleveF}
%%%%%%%%%%%%%%%%%%%%%%%%%%%%%%%%%%%%%%%%%%%%%%%%%%%%%%%%%%%%%%%%%%%%%%%%%%%%%%%%
\end{colle}%
\end{document}
    