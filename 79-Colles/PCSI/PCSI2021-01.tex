\documentclass[twoside,a4paper,landscape,french,10pt]{VcCours}

%%%%%%%%%%%%%%%%%%%%%%%%%%%%
\setlength\columnseprule{0.4pt}%
\addtolength\columnsep{1cm}

%%%%%%%%%%%%%%%%%%%%%%%%%%%%
\newenvironment{colle}[4]{{\large\bf\makebox[0cm][l]{#1\hspace{\stretch{-1}}}\hspace{\stretch{1}}#2\hspace{\stretch{1}}\makebox[0cm][r]{\hspace{\stretch{-1}}#3}}
\noindent\rule{\linewidth}{0.4pt}
\begin{multicols*}{#4}}{\end{multicols*}\pagebreak}
\newenvironment{eleve}[1]{\textbf{\large #1}}{\vspace{\stretch{1}}
\columnbreak}
\newenvironment{eleveF}[1]{\textbf{\large #1}}{}
\newcommand{\cours}{\medskip\uwave{Cours :}

\smallskip}
\newcommand{\exo}[1]{\vspace{\stretch{1}}\uwave{Exercice #1:}

\smallskip}

\begin{document}

\pagestyle{empty}

%%%%%%%%%%%%%%%%%%%%%%%%%%%%%%%%%%%%%%%%%%%%%
%+-----------------------------------------+%
%%%%%%%%%%%%%%%%%%%%%%%%%%%%%%%%%%%%%%%%%%%%%


\begin{colle}{PCSI -- Groupe n°}{Colleur: Vincent Croizier}{Semaine
n\degres1 -- le jeudi 23/09/2021}{3}
%%%%%%%%%%%%%%%%%%%%%%%%%%%%%%%%%%%%%%%%%%%%%%%%%%%%%%%%%%%%%%%%%%%%%%%%%%%%%%%
\begin{eleve}{}

    \cours
    Def8 Minorant, majorant, min, max, partie bornée.\\
    Soit $A\subset \R$. Écrire avec des quantificateurs les énoncés 
    "A est majorée", "A est minorée", "A n'est pas majorée" 
    et "A n'est pas minorée".

    \exo1
    Mettre sous forme algébrique $z=\frac{(2+i)^4}{3+2i}$.

    \exo2
    On considère : $u_0=1$ et $u_{n+1}=5u_n-2^n$.
    \begin{enumerate}
      \item Montrer que pour tout $n\in\N$, $u_n\geq2^n$.
      \item On pose $v_n=\frac{u_n}{2^n}$. Exprimer $v_{n+1}$ en en fonction de $v_n$.
      \item En déduire la forme explicite de $u_n$.
    \end{enumerate}

    \exo3
    Simplifier $S_n=\sum_{k=1}^n\ln\left(1-\frac{1}{k^2}\right)$.

    \vspace*{\stretch{2}}

\end{eleve}
%%%%%%%%%%%%%%%%%%%%%%%%%%%%%%%%%%%%%%%%%%%%%%%%%%%%%%%%%%%%%%%%%%%%%%%%%%%%%%%%
\begin{eleve}{}

  \cours
  Prop18 (def de partie entière)\\
  Prop20 (monotonie et régularité de partie entière). + Courbe.

  \exo1
  Mettre sous forme algébrique $z=\left(\frac{1+i}{1+2i}\right)^3$.

  \exo2
  Montrer que pour tout $n\in\N$ que\\ $\sum_{k=0}^{n}(2k-1)^2=\frac{n(2n-1)(2n+1)}{3}$

  \exo3
  Forme explicite de $u_n$ défini par $u_0=5$ et $u_{n+1}=\frac{2-3u_n}{7}$.

  \exo4
\begin{enumerate}
  \item Montrer que, pour $0\leqslant k\leqslant p\leqslant n$, 
  $\binom{n}{k}\binom{n-k}{p-k}=\binom{p}{k}\binom{n}{p}$.
  \item Si $0<p\leqslant n$, simplifier\\
  $\sum_{k=1}^p(-1)^k\binom{n}{k}\binom{n-k}{p-k}$.
\end{enumerate}

  \vspace*{\stretch{2}}
    
\end{eleve}
%%%%%%%%%%%%%%%%%%%%%%%%%%%%%%%%%%%%%%%%%%%%%%%%%%%%%%%%%%%%%%%%%%%%%%%%%%%%%%%%
\begin{eleveF}{}
    
  \cours
  Def14 Valeur absolue.\\
  Prop15c produit/quotient de valeur absolue, double inégalité triangulaire

  \exo1
  Mettre sous forme algébrique $z=\frac{1+3i}{(1-i)^5}$.
  
  \exo2
  Simplifier $S_n=\sum_{k=2}^{n}\binom{n}{k-1}2^kx^k$.

  \exo3
  On considère la suite $u$ définie par:
\[u_0=1\et u_{n+1}=\frac{5u_n-2}{u_n+2}\]
\begin{enumerate}
  \item Montrer que pour tout $n\geqslant3$, $u_n>1$.
  \item on pose $v_n=\frac{u_n-2}{u_n-1}$. Montrer que $(v_n)$ est géométrique.
  \item En déduire la forme explicite de $u_n$.
  \item Déterminer la limite de $u_n$.
\end{enumerate}

  
  \vspace*{\stretch{2}}
  
\end{eleveF}
%%%%%%%%%%%%%%%%%%%%%%%%%%%%%%%%%%%%%%%%%%%%%%%%%%%%%%%%%%%%%%%%%%%%%%%%%%%%%%%%
\end{colle}%
\end{document}
    