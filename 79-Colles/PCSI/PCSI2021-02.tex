\documentclass[twoside,a4paper,landscape,french,10pt]{VcCours}

%%%%%%%%%%%%%%%%%%%%%%%%%%%%
\setlength\columnseprule{0.4pt}%
\addtolength\columnsep{1cm}

%%%%%%%%%%%%%%%%%%%%%%%%%%%%
\newenvironment{colle}[4]{{\large\bf\makebox[0cm][l]{#1\hspace{\stretch{-1}}}\hspace{\stretch{1}}#2\hspace{\stretch{1}}\makebox[0cm][r]{\hspace{\stretch{-1}}#3}}
\noindent\rule{\linewidth}{0.4pt}
\begin{multicols*}{#4}}{\end{multicols*}\pagebreak}
\newenvironment{eleve}[1]{\textbf{\large #1}}{\vspace{\stretch{1}}
\columnbreak}
\newenvironment{eleveF}[1]{\textbf{\large #1}}{}
\newcommand{\cours}{\medskip\uwave{Cours :}

\smallskip}
\newcommand{\exo}[1]{\vspace{\stretch{1}}\uwave{Exercice #1:}

\smallskip}

\begin{document}

\pagestyle{empty}

%%%%%%%%%%%%%%%%%%%%%%%%%%%%%%%%%%%%%%%%%%%%%
%+-----------------------------------------+%
%%%%%%%%%%%%%%%%%%%%%%%%%%%%%%%%%%%%%%%%%%%%%


\begin{colle}{PCSI -- Groupe n°}{Colleur: Vincent Croizier}{Semaine
n\degres2 -- le jeudi 30/09/2021}{3}
%%%%%%%%%%%%%%%%%%%%%%%%%%%%%%%%%%%%%%%%%%%%%%%%%%%%%%%%%%%%%%%%%%%%%%%%%%%%%%%
\begin{eleve}{}

    \cours
    Tout sur le module.


    \exo1
    Résoudre $\cos(2x)-\sqrt3\sin(2x)=1$.

    \exo2
    Soit $x\in \R,\,n\in\N^*$ et $p\in\N^*$.
    \begin{enumerate}
    \item Simplifier $C=\sum_{k=0}^{n}\cos(kx)$.
    \item En déduire $\sum\limits_{k=1}^{n}\cos^3(kx)$.
    \end{enumerate}

    \exo3
    Montrer que pour tout $\lambda \in \R$, le nombre complexe $z =
    \frac{1+\lambda i}{1-\lambda i}$ est de module $1$. Pour quels $z
    \in\C$, existe-t-il $\lambda \in\R$ tel que $z = \frac{1+\lambda
    i}{1-\lambda i}$?

    \vspace*{\stretch{2}}

\end{eleve}
%%%%%%%%%%%%%%%%%%%%%%%%%%%%%%%%%%%%%%%%%%%%%%%%%%%%%%%%%%%%%%%%%%%%%%%%%%%%%%%%
\begin{eleve}{}

  \cours
  Inégalité triangulaire + démo

  \exo1
  Linéariser $\cos^3x\sin(3x)$.

  \exo2
  Résoudre $X^2-(7-2i)X+13-i=0$.

  \exo3
  Soit $n\in\N$. Résoudre dans $\C$: $(z+i)^n = (i - z)^n$.


  \vspace*{\stretch{2}}
    
\end{eleve}
%%%%%%%%%%%%%%%%%%%%%%%%%%%%%%%%%%%%%%%%%%%%%%%%%%%%%%%%%%%%%%%%%%%%%%%%%%%%%%%%
\begin{eleveF}{}
    
  \cours
  Tout sur les exponentielles imaginaires.

  \exo1
  Résoudre $X^2-(5-2i)X+9-7i=0$.
  
  (Calculer $17^2$.)
  
  \exo2
  Résoudre dans $\C$ l'équation: $(z + 1)^n = e^{2ina}$.

  \exo3
  Montrer que: \[\forall(z_1, z_2)\in\C^2,\ |z_1 + z_2|^2 + |z_1 -
  z_2|^2 = 2 \left( |z_1|^2 + |z_2|^2\right)\] Interpréter
  géométriquement.

  
  \vspace*{\stretch{2}}
  
\end{eleveF}
%%%%%%%%%%%%%%%%%%%%%%%%%%%%%%%%%%%%%%%%%%%%%%%%%%%%%%%%%%%%%%%%%%%%%%%%%%%%%%%%
\end{colle}%
\end{document}
    