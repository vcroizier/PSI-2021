\documentclass[twoside,a4paper,landscape,french,10pt]{VcCours}

%%%%%%%%%%%%%%%%%%%%%%%%%%%%
\setlength\columnseprule{0.4pt}%
\addtolength\columnsep{1cm}

%%%%%%%%%%%%%%%%%%%%%%%%%%%%
\newenvironment{colle}[4]{{\large\bf\makebox[0cm][l]{#1\hspace{\stretch{-1}}}\hspace{\stretch{1}}#2\hspace{\stretch{1}}\makebox[0cm][r]{\hspace{\stretch{-1}}#3}}
\noindent\rule{\linewidth}{0.4pt}
\begin{multicols*}{#4}}{\end{multicols*}\pagebreak}
\newenvironment{eleve}[1]{\textbf{\large #1}}{\vspace{\stretch{1}}
\columnbreak}
\newenvironment{eleveF}[1]{\textbf{\large #1}}{}
\newcommand{\cours}{\medskip\uwave{Cours :}

\smallskip}
\newcommand{\exo}[1]{\vspace{\stretch{1}}\uwave{Exercice #1:}

\smallskip}

\begin{document}

\pagestyle{empty}

%%%%%%%%%%%%%%%%%%%%%%%%%%%%%%%%%%%%%%%%%%%%%
%+-----------------------------------------+%
%%%%%%%%%%%%%%%%%%%%%%%%%%%%%%%%%%%%%%%%%%%%%


\begin{colle}{PSI -- Groupe n°}{Colleur: Vincent Croizier}{Semaine
n\degres1 -- le vendredi 17/09/2021}{3}
%%%%%%%%%%%%%%%%%%%%%%%%%%%%%%%%%%%%%%%%%%%%%%%%%%%%%%%%%%%%%%%%%%%%%%%%%%%%%%%
\begin{eleve}{}

    \cours
      DL de $ln(1-x)$ et $ch(x)$\\
      Définition et Th des suites adjacentes\\
      Forme explicite de $u_{n+1} = \frac{7-2u_n}{5}$

    \exo1
    Déterminer $\lim_{n\to+\infty}\left(\frac{n^2-1}{n^2+1}\right)^{\frac{n}{2}}$

    \exo2
    Étudier $u_0\in\R$ et $u_{n+1}=\sqrt{\frac{16+u_n^2}{2}}$

    \vspace*{\stretch{2}}

\end{eleve}
%%%%%%%%%%%%%%%%%%%%%%%%%%%%%%%%%%%%%%%%%%%%%%%%%%%%%%%%%%%%%%%%%%%%%%%%%%%%%%%%
\begin{eleve}{}

  \cours
      DL de $e^x$ et $(1+x)^{\tfrac13}$\\
      Équivalents usuels\\
      Forme explicite de $u_{n+2} = 2u_{n+1}-4u_n$

  \exo1
  $S_n=\sum_{k=0}^n\frac{(-1)^k}{k\ln(k)}$
  \begin{enumerate}
    \item Montrer que $(S_2n)$ et $(S_{2n+1})$ sont adjacentes.
    \item En déduire la convergence de $(S_n)$.
  \end{enumerate}

  \exo2
  $\text{DL}_4(0)$ de $u_n=(\cos(x))^{\tfrac1{x^2}}$.

  \vspace*{\stretch{2}}
    
\end{eleve}
%%%%%%%%%%%%%%%%%%%%%%%%%%%%%%%%%%%%%%%%%%%%%%%%%%%%%%%%%%%%%%%%%%%%%%%%%%%%%%%%
\begin{eleveF}{}
    
  \cours
  DL de $sin(x)$ et $tan(x)$\\
  Inégalité des accroissements finis\\
  Exercice 4

  \exo1
  $\text{DL}_3(0)$ de $u_n=\ln(\ln(e+\tfrac1n))$.
  
  \exo2
  $u_{n+1}=\frac1n e^{-u_n}$
  \begin{enumerate}
    \item Déterminer la limite de $u_n$.
    \item En déduire un équivalent puis un $\text{DL}_3(0)$.
  \end{enumerate}

  \vspace*{\stretch{2}}
  


\end{eleveF}
%%%%%%%%%%%%%%%%%%%%%%%%%%%%%%%%%%%%%%%%%%%%%%%%%%%%%%%%%%%%%%%%%%%%%%%%%%%%%%%%
\end{colle}%
\end{document}
    