\documentclass[twoside,a4paper,landscape,french,10pt]{VcCours}

%%%%%%%%%%%%%%%%%%%%%%%%%%%%
\setlength\columnseprule{0.4pt}%
\addtolength\columnsep{1cm}

%%%%%%%%%%%%%%%%%%%%%%%%%%%%
\newenvironment{colle}[4]{{\large\bf\makebox[0cm][l]{#1\hspace{\stretch{-1}}}\hspace{\stretch{1}}#2\hspace{\stretch{1}}\makebox[0cm][r]{\hspace{\stretch{-1}}#3}}
\noindent\rule{\linewidth}{0.4pt}
\begin{multicols*}{#4}}{\end{multicols*}\pagebreak}
\newenvironment{eleve}[1]{\textbf{\large #1}}{\vspace{\stretch{1}}
\columnbreak}
\newenvironment{eleveF}[1]{\textbf{\large #1}}{}
\newcommand{\cours}{\medskip\uwave{Cours :}

\smallskip}
\newcommand{\exo}[1]{\vspace{\stretch{1}}\uwave{Exercice #1:}

\smallskip}

\begin{document}

\pagestyle{empty}

%%%%%%%%%%%%%%%%%%%%%%%%%%%%%%%%%%%%%%%%%%%%%
%+-----------------------------------------+%
%%%%%%%%%%%%%%%%%%%%%%%%%%%%%%%%%%%%%%%%%%%%%


\begin{colle}{PSI -- Groupe n°}{Colleur: Vincent Croizier}{Semaine
n\degres3 -- le jeudi 30/09/2021}{3}
%%%%%%%%%%%%%%%%%%%%%%%%%%%%%%%%%%%%%%%%%%%%%%%%%%%%%%%%%%%%%%%%%%%%%%%%%%%%%%%
\begin{eleve}{}

  \cours
  la série exponentielle : la fonction exponentielle est bien 
  définie sur $\mathbb{C}$, la série associée est absolument convergente et 
  on a pour tout $(z,z') \in \mathbb{C}^2$, $\exp(z)\exp(z')= \exp(z+z')$.

  \exo1
  Nature de la série $\sum_{n\geq0}\frac{2^nn!}{n^n}$.

  \exo2
  Considérons la série $\sum_{n\geq2}\frac{(-1)^n}{n\ln^2(x)}$.
  \begin{enumerate}
    \item Montrer la convergence de la série. Donner un encadrement de sa somme.
    \item Montrer l'absolue convergence de la série.
  \end{enumerate}

  \vspace*{\stretch{2}}

\end{eleve}
%%%%%%%%%%%%%%%%%%%%%%%%%%%%%%%%%%%%%%%%%%%%%%%%%%%%%%%%%%%%%%%%%%%%%%%%%%%%%%%%
\begin{eleve}{}

  \cours
  Critère spécial des séries alternées.\\
  Montrer que $\sum_{k=1}^n \frac{1}{k} \underset{+ \infty}{\sim} \ln(n)$.
  
  \exo1
  On considère la série $\sum_{n\geq0}\arctan\left(\frac{1}{n^2+n+1}\right)$.
  \begin{enumerate}
    \item Nature de la série.
    \item Simplifier $\tan(\arctan(n+1)-\arctan(n))$.
    En déduire la somme de la série.
  \end{enumerate}

  \exo2
  $\alpha>0$. Nature de la série $\sum_{n\geq1}\frac{(-1)^n}{n^{\alpha}+(-1)^n}$.  

  \vspace*{\stretch{2}}

\end{eleve}
%%%%%%%%%%%%%%%%%%%%%%%%%%%%%%%%%%%%%%%%%%%%%%%%%%%%%%%%%%%%%%%%%%%%%%%%%%%%%%%%
\begin{eleveF}{}
    
  \cours
  Formule de Stirling.\\
  Montrer que $\sum_{n \geq 0} {\dfrac{( - 1)^n 8^n}{(2n)!}}$ 
  est convergente et que sa somme est négative.

  \exo1
  Considérons $u_n=\sum_{k=1}^n\ln(k)$.
  \begin{enumerate}
    \item Nature de $\sum_{n\geq1}\ln(k)$.
    \item Nature de la série de terme général $((-1)^n/u_n)$.
    \item Déterminer un équivalent de $u_n$.
    \item La série du \textbf{2)} est-elle absolument convergente ?
  \end{enumerate}

  \exo2
  $\alpha>0$. Nature de la série $\sum_{n\geq0}\sqrt{1+\frac{(-1)^n}{n^{\alpha}}}-1$.  


  \vspace*{\stretch{2}}
  


\end{eleveF}
%%%%%%%%%%%%%%%%%%%%%%%%%%%%%%%%%%%%%%%%%%%%%%%%%%%%%%%%%%%%%%%%%%%%%%%%%%%%%%%%
\end{colle}%
\end{document}
    