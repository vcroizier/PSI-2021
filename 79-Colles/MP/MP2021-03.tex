\documentclass[twoside,a4paper,landscape,french,10pt]{VcCours}

%%%%%%%%%%%%%%%%%%%%%%%%%%%%
\setlength\columnseprule{0.4pt}%
\addtolength\columnsep{1cm}

%%%%%%%%%%%%%%%%%%%%%%%%%%%%
\newenvironment{colle}[4]{{\large\bf\makebox[0cm][l]{#1\hspace{\stretch{-1}}}\hspace{\stretch{1}}#2\hspace{\stretch{1}}\makebox[0cm][r]{\hspace{\stretch{-1}}#3}}
\noindent\rule{\linewidth}{0.4pt}
\begin{multicols*}{#4}}{\end{multicols*}\pagebreak}
\newenvironment{eleve}[1]{\textbf{\large #1}}{\vspace{\stretch{1}}
\columnbreak}
\newenvironment{eleveF}[1]{\textbf{\large #1}}{}
\newcommand{\cours}{\medskip\uwave{Cours :}

\smallskip}
\newcommand{\exo}[1]{\vspace{\stretch{1}}\uwave{Exercice #1:}

\smallskip}

\begin{document}

\pagestyle{empty}

%%%%%%%%%%%%%%%%%%%%%%%%%%%%%%%%%%%%%%%%%%%%%
%+-----------------------------------------+%
%%%%%%%%%%%%%%%%%%%%%%%%%%%%%%%%%%%%%%%%%%%%%


\begin{colle}{MP -- Groupe n°}{Colleur: Vincent Croizier}{Semaine
n\degres1 -- le vendredi 17/09/2021}{3}
%%%%%%%%%%%%%%%%%%%%%%%%%%%%%%%%%%%%%%%%%%%%%%%%%%%%%%%%%%%%%%%%%%%%%%%%%%%%%%%
\begin{eleve}{}

    \cours
    Définition d’une projection vectorielle et propriétés associées\\
    Exercice 2

    \exo1
    $E\!=\!\R^3$, $F\!=\!\ensemble{(x,y,z)\in\R^3}{x+4y+z=0}$,\\
    $G\!=\!\ensemble{(x,\!y,\!z)\in\R^3}{x+y+z=0\textrm{ et }x+2y+3z=0}$.\hspace*{-1cm}
    \begin{enumerate}
      \item Montrer que $F$ et $G$ sont en somme directe.
      \item Montrer que $F\oplus G=E$.
    \end{enumerate}
    
    \exo2
    $E$ un $\K$-ev. $p,q\in\LL(E)$ deux projecteurs.
    \begin{enumerate}
      \item Montrer que $p+q$ est un projecteur\\
      ssi $p\circ q=q\circ p=0$.
      \item Si $p+q$ est un projecteur, montrer que
      \[\Ima(p+q)=\Ima p\oplus\Ima q\]
      \[\Ker(p+q)=\Ker p\cap \Ker q\]
    \end{enumerate}

    \vspace*{\stretch{2}}

\end{eleve}
%%%%%%%%%%%%%%%%%%%%%%%%%%%%%%%%%%%%%%%%%%%%%%%%%%%%%%%%%%%%%%%%%%%%%%%%%%%%%%%%
\begin{eleve}{}

  \cours
  Définition du noyau et de l’image d’une application linéaire.\\
  Exercice 3
  
  \exo1
  $u$ et $v$ deux endomorphismes de $E$ qui commutent.
  
  Montrer que $\Ker u$ et $\Ima u$ sont stables par $v$.
  
  \exo2
  $(e_1,\ldots,e_n)$ une famille libre de $E$, un $\K$-ev.
  
  $(\lambda_1,\ldots,\lambda_n)\in\K^n$ et $u=\sum_{i=1}^n\lambda_ie_i$.
  
  On pose $v_i=u+e_i$.
  
  Montrer que $(v_1,\ldots,v_n)$ est libre ssi $\sum_{i=1}^n\lambda_i\neq-1$.
  
  \exo3
  $F$ et $G$ deux sev d'un $\K$-ev $E$ tels que $F+G=E$.
  Soit $F'$ un supplémentaire de $F\cap G$ dans $F$.
  
  Montrer que $F'\oplus G=E$.

  \vspace*{\stretch{2}}
    
\end{eleve}
%%%%%%%%%%%%%%%%%%%%%%%%%%%%%%%%%%%%%%%%%%%%%%%%%%%%%%%%%%%%%%%%%%%%%%%%%%%%%%%%
\begin{eleveF}{}
    
  \cours
  Lien entre injectivité et noyau.\\
  Exercice 1

  \exo1
  $a\in\C$ et $f:\C\mapsto\C$ avec $f(z)=z+a\overline{z}$.
  \begin{enumerate}
    \item Montrer que $f$ est $\R$-linéaire.
    \item À quelle condition sur $a$, $f$ est-elle $\C$-linéaire ?
    \item En considérant $\C$ comme un $\R$-ev, déterminer $\Ker f$.
  \end{enumerate}
  
  \exo2
  $f\in\LL(E)$ tel que $f^3-3f+2\id_E=0$.
  \begin{enumerate}
    \item Montrer que $f$ est un isomorphisme et donner $f^{-1}$ en fonction de $f$.
    \item Montrer que $\Ker(f-\id_E)\oplus \Ker(f^2+f-2\id_E)=E$.
    \item Montrer que $\Ima(f-\id_E)=\Ker(f^2+f-2\id_E)$.
  \end{enumerate}
  
  \vspace*{\stretch{2}}
  
\end{eleveF}
%%%%%%%%%%%%%%%%%%%%%%%%%%%%%%%%%%%%%%%%%%%%%%%%%%%%%%%%%%%%%%%%%%%%%%%%%%%%%%%%
\end{colle}%
\end{document}
    