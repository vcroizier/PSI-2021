\documentclass[twoside,a4paper,landscape,french,10pt]{VcCours}

%%%%%%%%%%%%%%%%%%%%%%%%%%%%
\setlength\columnseprule{0.4pt}%
\addtolength\columnsep{1cm}

%%%%%%%%%%%%%%%%%%%%%%%%%%%%
\newenvironment{colle}[4]{{\large\bf\makebox[0cm][l]{#1\hspace{\stretch{-1}}}\hspace{\stretch{1}}#2\hspace{\stretch{1}}\makebox[0cm][r]{\hspace{\stretch{-1}}#3}}
\noindent\rule{\linewidth}{0.4pt}
\begin{multicols*}{#4}}{\end{multicols*}\pagebreak}
\newenvironment{eleve}[1]{\textbf{\large #1}}{\vspace{\stretch{1}}
\columnbreak}
\newenvironment{eleveF}[1]{\textbf{\large #1}}{}
\newcommand{\cours}{\medskip\uwave{Cours :}

\smallskip}
\newcommand{\exo}[1]{\vspace{\stretch{1}}\uwave{Exercice #1:}

\smallskip}

\begin{document}

\pagestyle{empty}

%%%%%%%%%%%%%%%%%%%%%%%%%%%%%%%%%%%%%%%%%%%%%
%+-----------------------------------------+%
%%%%%%%%%%%%%%%%%%%%%%%%%%%%%%%%%%%%%%%%%%%%%


\begin{colle}{MP -- Groupe n°}{Colleur: Vincent Croizier}{Semaine
n\degres3 -- le vendredi 01/10/2021}{3}
%%%%%%%%%%%%%%%%%%%%%%%%%%%%%%%%%%%%%%%%%%%%%%%%%%%%%%%%%%%%%%%%%%%%%%%%%%%%%%%
\begin{eleve}{}

    \cours
    Théorème de Bolzano-Weierstrass.\\
    Exercice 2

    \exo1
    $\alpha>0$. Nature de la série $\sum_{n\geq0}\left(\sqrt{1+\frac{(-1)^n}{n^{\alpha}}}-1\right)$.
    
    \exo2
    Nature de la série $\sum_{n\geq2}\frac{(-1)^2)}{\sqrt{n^{\alpha}+(-1)^n}}$.

    \exo3
    Considérons la suite définie par $u_0=u_1=1$ et $\forall n\in\N$,
    $u_{n+2}=\frac49(3u_{n+1}-u_n)$.

    Déterminer la nature le série $\sum_{n\geq0}u_n$.

    \vspace*{\stretch{2}}

\end{eleve}
%%%%%%%%%%%%%%%%%%%%%%%%%%%%%%%%%%%%%%%%%%%%%%%%%%%%%%%%%%%%%%%%%%%%%%%%%%%%%%%%
\begin{eleve}{}

  \cours
  Théorème des valeurs intermédiaires et théorème de bijection.\\
  Exercice 3
  
  \exo1
  On considère la série $\sum_{n\geq0}\arctan\left(\frac{1}{n^2+n+1}\right)$.
  \begin{enumerate}
    \item Nature de la série.
    \item Simplifier $\tan(\arctan(n+1)-\arctan(n))$.
    En déduire la somme de la série.
  \end{enumerate}
  
  \exo2
  Nature de la série $\sum_{n\geq0}(n+(-1)^n)3^{-n}\sin(n)$.
  
  \exo3
  Étudier $u_0\in\R$ et $u_{n+1}=\sqrt{\frac{16+u_n^2}{2}}$

  \vspace*{\stretch{2}}
    
\end{eleve}
%%%%%%%%%%%%%%%%%%%%%%%%%%%%%%%%%%%%%%%%%%%%%%%%%%%%%%%%%%%%%%%%%%%%%%%%%%%%%%%%
\begin{eleveF}{}
    
  \cours
  Définition : domination, négligeabilité, équivalence.\\
  Exercice 1

  \exo1
  Nature de la série $\sum_{n\geq0}\frac{2^nn!}{n^n}$.

  
  \exo2
  Nature de la série $\sum_{n\geq0}\frac{\ln^3(n)}{n^2+1}$.

  \exo3

  
  \vspace*{\stretch{2}}
  
\end{eleveF}
%%%%%%%%%%%%%%%%%%%%%%%%%%%%%%%%%%%%%%%%%%%%%%%%%%%%%%%%%%%%%%%%%%%%%%%%%%%%%%%%
\end{colle}%
\end{document}
    