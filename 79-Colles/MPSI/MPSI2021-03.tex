\documentclass[twoside,a4paper,landscape,french,10pt]{VcCours}

%%%%%%%%%%%%%%%%%%%%%%%%%%%%
\setlength\columnseprule{0.4pt}%
\addtolength\columnsep{1cm}

%%%%%%%%%%%%%%%%%%%%%%%%%%%%
\newenvironment{colle}[4]{{\large\bf\makebox[0cm][l]{#1\hspace{\stretch{-1}}}\hspace{\stretch{1}}#2\hspace{\stretch{1}}\makebox[0cm][r]{\hspace{\stretch{-1}}#3}}
\noindent\rule{\linewidth}{0.4pt}
\begin{multicols*}{#4}}{\end{multicols*}\pagebreak}
\newenvironment{eleve}[1]{\textbf{\large #1}}{\vspace{\stretch{1}}
\columnbreak}
\newenvironment{eleveF}[1]{\textbf{\large #1}}{}
\newcommand{\cours}{\medskip\uwave{Cours :}

\smallskip}
\newcommand{\exo}[1]{\vspace{\stretch{1}}\uwave{Exercice #1:}

\smallskip}

\begin{document}

\pagestyle{empty}

%%%%%%%%%%%%%%%%%%%%%%%%%%%%%%%%%%%%%%%%%%%%%
%+-----------------------------------------+%
%%%%%%%%%%%%%%%%%%%%%%%%%%%%%%%%%%%%%%%%%%%%%


\begin{colle}{MPSI -- Groupe n°}{Colleur: Vincent Croizier}{Semaine
n\degres3 -- le lundi 04/10/2021}{3}
%%%%%%%%%%%%%%%%%%%%%%%%%%%%%%%%%%%%%%%%%%%%%%%%%%%%%%%%%%%%%%%%%%%%%%%%%%%%%%%
\begin{eleve}{}

    \cours
    Définition de l'injectivité.\\
    Soit $A = \ensemble{(1 + 2t, 2 + t)}{t \in\R}$
    et $B = \ensemble{(x, y)\in\R^2}{x − 2y + 3 = 0}$.
    Démontrer que $A = B$.

    \exo1
    Simplifier $S_n=\sum_{j=1}^n\sum_{i=j}^n\frac{j^3}{(i+1)^2}$.


    \exo2
    Soit $f\in\FF(E,F)$ tel que $f\circ f\circ f=f$.
    Montrez que $f\text{ injective }\Longleftrightarrow f\text{ surjective}$. 

    \exo3
    Soit $f:E\longrightarrow F$ une application, et $G$ un troisième ensemble
    ayant au moins deux éléments.
    
    On construit deux nouvelles applications :

$
   \application{f_*}{E^G}{F^G}{\phi}{f\circ \phi}
   \et 
   \application{f^*}{G^F}{G^E}{\phi}{\phi\circ f}
$

Montrer que :
\begin{enumerate}
\item $f$ est injective  $\Longleftrightarrow f_*$ est injective  $\Longleftrightarrow f^*$ est surjective.
\item $f$ est surjective $\Longleftrightarrow f_*$ est surjective $\Longleftrightarrow f^*$ est injective.
\end{enumerate}
    \vspace*{\stretch{2}}

\end{eleve}
%%%%%%%%%%%%%%%%%%%%%%%%%%%%%%%%%%%%%%%%%%%%%%%%%%%%%%%%%%%%%%%%%%%%%%%%%%%%%%%%
\begin{eleve}{}

  \cours
  Définition de la surjectivité.\\
  Montrer que $g\circ f$ injective $\Longrightarrow$ $f$ injective.\\
  Montrer que $g\circ f$ surjective $\Longrightarrow$ $g$ surjective.

  \exo1
  Simplifier $S_n=\sum_{j=1}^n\sum_{i=j}^n$.

  \exo2
  $A,B,C,D$ des parties de $E$.\\
  Montrez que $\systeme{B\setminus C\subset A\\C\setminus D\subset A}
  \Longrightarrow B\setminus D\subset A$

  \exo3
  Soit $E$ un ensemble, et $A,B$ deux parties fixées de $E$.
  Soit $\application{\phi}{\PP(E)}{\PP(A)\times\PP(B)}{X}{(X\cup A, X\cap B).}$
  \begin{enumerate}
    \item Qu'est-ce que $\phi(\varnothing)$ ? $\phi(E\setminus(A\cup B))$ ?
    \item A quelle condition sur $A$ et $B$, $\phi$ est-elle injective ?
    \item Est-ce que le couple $(\varnothing,B)$ possède un antécédent par $\phi$ ?
    \item A quelle condition sur $A$ et $B$, $\phi$ est-elle surjective ?
  \end{enumerate}

  \vspace*{\stretch{2}}
    
\end{eleve}
%%%%%%%%%%%%%%%%%%%%%%%%%%%%%%%%%%%%%%%%%%%%%%%%%%%%%%%%%%%%%%%%%%%%%%%%%%%%%%%%
\begin{eleveF}{}
    
  \cours
  Définition de la bijectivité.\\
  $g$ et $f$ injectives $\Longrightarrow$ $g\circ f$ injective. Idem avec surjective.

  \exo1
  Montrer que $\systeme{A\cap B \subset A\cap C\\A\cup B \subset A\cup C}
  \Longrightarrow B \subset C$.
  
  \exo2
  Simplifier $S_n=\sum_{j=1}^n\sum_{i=j}^n$.

  \exo3
  Soit $f:E\longrightarrow E$ bijective.
  La conjugaison par $f$ est l'application
  $\application{\phi_f}{E^E}{E^E}{\phi}{f\circ \phi\circ f^{-1}}$
  \begin{enumerate}
    \item Montrer que $\phi_f$ est une bijection de $E^E$.
    \item Simplifier $\phi_f \circ \phi_g$.
    \item Simplifier $\phi_f(\phi) \circ \phi_f(\psi)$.
    \item Soient $\II$ et $\SS$ les sous-ensembles de $E^E$ constitués
        des injections et des surjections.
        Montrer que $\II$ et $\SS$ sont invariants par $\phi_f$.
    \item Lorsque $\phi$ est bijective, qu'est-ce que $(\phi_f(\phi))^{-1}$ ?
  \end{enumerate}
  
  \vspace*{\stretch{2}}
  
\end{eleveF}
%%%%%%%%%%%%%%%%%%%%%%%%%%%%%%%%%%%%%%%%%%%%%%%%%%%%%%%%%%%%%%%%%%%%%%%%%%%%%%%%
\end{colle}%
\end{document}
    