\documentclass[twoside,a4paper,landscape,french,10pt]{VcCours}

%%%%%%%%%%%%%%%%%%%%%%%%%%%%
\setlength\columnseprule{0.4pt}%
\addtolength\columnsep{1cm}

%%%%%%%%%%%%%%%%%%%%%%%%%%%%
\newenvironment{colle}[4]{{\large\bf\makebox[0cm][l]{#1\hspace{\stretch{-1}}}\hspace{\stretch{1}}#2\hspace{\stretch{1}}\makebox[0cm][r]{\hspace{\stretch{-1}}#3}}
\noindent\rule{\linewidth}{0.4pt}
\begin{multicols*}{#4}}{\end{multicols*}\pagebreak}
\newenvironment{eleve}[1]{\textbf{\large #1}}{\vspace{\stretch{1}}
\columnbreak}
\newenvironment{eleveF}[1]{\textbf{\large #1}}{}
\newcommand{\cours}{\medskip\uwave{Cours :}

\smallskip}
\newcommand{\exo}[1]{\vspace{\stretch{1}}\uwave{Exercice #1:}

\smallskip}

\begin{document}

\pagestyle{empty}

%%%%%%%%%%%%%%%%%%%%%%%%%%%%%%%%%%%%%%%%%%%%%
%+-----------------------------------------+%
%%%%%%%%%%%%%%%%%%%%%%%%%%%%%%%%%%%%%%%%%%%%%


\begin{colle}{MPSI -- Groupe n°}{Colleur: Vincent Croizier}{Semaine
n\degres4 -- le lundi 11/10/2021}{3}
%%%%%%%%%%%%%%%%%%%%%%%%%%%%%%%%%%%%%%%%%%%%%%%%%%%%%%%%%%%%%%%%%%%%%%%%%%%%%%%
\begin{eleve}{}

    \cours
    Définition d'images directes et réciproques.\\
    Liens entre images directes et $\cap/\cup$.\\
    Exercice 4.

    \exo1
    Résoudre $2z^2-(4-4i)z-5i=0$.


    \exo2
    Soit $f:E\longrightarrow E$.
Soit $A\subset E$ et $B = \bigcup_{n\in\N}f^n(A)$.
\begin{enumerate}
\item Montrer que $f(B) \subset B$.
\item Montrer que $B$ est la plus petite partie de $E$ stable par $f$ et
    contenant $A$.
  \end{enumerate}

    \exo3
    Soit $n\in \N^*$ et $\omega =e^{i\frac{2\pi }{n}}$.
\begin{enumerate}
\item  Que vaut $\sum\limits_{k=1}^{n}\omega ^{k}$ ?
\item  Soit $p\in \Z$. Calculer $\sum\limits_{k=1}^{n}(\omega
^{p})^{k}$.
\item  Montrer que $\forall z\in \C,\sum\limits_{k=1}^{n}(z+\omega
^{k})^{n}=n(z^{n}+1)\!\!\!\!\!$.
\end{enumerate}

    \vspace*{\stretch{2}}

\end{eleve}
%%%%%%%%%%%%%%%%%%%%%%%%%%%%%%%%%%%%%%%%%%%%%%%%%%%%%%%%%%%%%%%%%%%%%%%%%%%%%%%%
\begin{eleve}{}

  \cours
  Définition de recouvrement et partition.\\
  Complémentaire de l'union/intersection d'une famille d'ensembles.\\
  Exercice 1

  \exo1
  Résoudre $X^2-(7-2i)X+13-i=0$.

  \exo2
  Soit $f:E\longrightarrow F$ et 
  
  $S = \ensemble{X\subset E}{f^{-1}(f(X)) = X}$.
\begin{enumerate}
\item Pour $A \subset E$, montrer que $f^{-1}(f(A))\in S$.
\item Montrer que $S$ est stable par intersection et réunion.
\item Soient $X\in S$ et $A\subset E$ tels que $X\cap A = \varnothing$.
Montrer que $X\cap f^{-1}(f(A)) = \varnothing$.
\item Soient $X,Y\in S$. Montrer que $E\setminus X$ et $Y\setminus X$ appartiennent à $S$.
\item Montrer que l'application

$\application{\varphi}{S}{\PP(f(E))}{A}{f(A)}$
    est une bijection.
\end{enumerate}

  \exo3
  Soit $n\in\N$. Résoudre dans $\C$: $(z+i)^n = (i - z)^n$.

  \vspace*{\stretch{2}}
    
\end{eleve}
%%%%%%%%%%%%%%%%%%%%%%%%%%%%%%%%%%%%%%%%%%%%%%%%%%%%%%%%%%%%%%%%%%%%%%%%%%%%%%%%
\begin{eleveF}{}
    
  \cours
  Définition de racine d'un polynôme, Théorème de D'Alembert-Gauss.\\
  Lien entre racine et factorisation.\\
  Exercice 3.

  \exo1
  Résoudre $X^2-(5-2i)X+9-7i=0$.
  
  \exo2
  Soit $f:E→F$. On définit
$\application{\varphi}{\PP(E)}{\PP(F)}A{f(A)}$ et
$\application{\psi}{\PP(F)}{\PP(E)}B{f^{-1}(B)}$.

Montrer que :
\begin{enumerate}
\item $f$ est injective  $\Longleftrightarrow \varphi$ est injective  $\Longleftrightarrow \psi$ est surjective.
\item $f$ est surjective $\Longleftrightarrow \varphi$ est surjective $\Longleftrightarrow \psi$ est injective.
\end{enumerate}

  \exo3
  Résoudre dans $\C$ l'équation: $(z + 1)^n = e^{2ina}$.
  
  \vspace*{\stretch{2}}
  
\end{eleveF}
%%%%%%%%%%%%%%%%%%%%%%%%%%%%%%%%%%%%%%%%%%%%%%%%%%%%%%%%%%%%%%%%%%%%%%%%%%%%%%%%
\end{colle}%
\end{document}
    