\documentclass[twoside,a4paper,landscape,french,10pt]{VcCours}

%%%%%%%%%%%%%%%%%%%%%%%%%%%%
\setlength\columnseprule{0.4pt}%
\addtolength\columnsep{1cm}

%%%%%%%%%%%%%%%%%%%%%%%%%%%%
\newenvironment{colle}[4]{{\large\bf\makebox[0cm][l]{#1\hspace{\stretch{-1}}}\hspace{\stretch{1}}#2\hspace{\stretch{1}}\makebox[0cm][r]{\hspace{\stretch{-1}}#3}}
\noindent\rule{\linewidth}{0.4pt}
\begin{multicols*}{#4}}{\end{multicols*}\pagebreak}
\newenvironment{eleve}[1]{\textbf{\large #1}}{\vspace{\stretch{1}}
\columnbreak}
\newenvironment{eleveF}[1]{\textbf{\large #1}}{}
\newcommand{\cours}{\medskip\uwave{Cours :}

\smallskip}
\newcommand{\exo}[1]{\vspace{\stretch{1}}\uwave{Exercice #1:}

\smallskip}

\begin{document}

\pagestyle{empty}

%%%%%%%%%%%%%%%%%%%%%%%%%%%%%%%%%%%%%%%%%%%%%
%+-----------------------------------------+%
%%%%%%%%%%%%%%%%%%%%%%%%%%%%%%%%%%%%%%%%%%%%%


\begin{colle}{MPSI -- Groupe n°}{Colleur: Vincent Croizier}{Semaine
n\degres8 -- le lundi 22/11/2021}{3}
%%%%%%%%%%%%%%%%%%%%%%%%%%%%%%%%%%%%%%%%%%%%%%%%%%%%%%%%%%%%%%%%%%%%%%%%%%%%%%%
\begin{eleve}{}

  \cours
  Définition de dérivabilité en un point.
  
  Théorème de dérivabilité d'une composée.
  
  Théorème de dérivabilité de la réciproque.
  
  \exo1
  Résoudre $\sys{\log_y(x)+\log_x(y)=\frac{50}7\\xy=256}$
  
  \exo2
  Étudier et simplifier $f:x\longmapsto2\arctan(\sqrt{1+x^2}-x)+\arctan(x)$. 
  
  \exo3
  Résoudre $(x^{\lambda})^x=x^{(x^{\lambda})}$ où $\lambda\in\R\setminus\{1\}$.
  
  \end{eleve}
  %%%%%%%%%%%%%%%%%%%%%%%%%%%%%%%%%%%%%%%%%%%%%%%%%%%%%%%%%%%%%%%%%%%%%%%%%%%%%%%%%%%%%%%%%%%%%%%%%%%%%%%
  \begin{eleve}{}
  
    \cours
  Définition de $\arccos$ et $\arcsin$.
  
  Théorème de la bijection monotone.
  
  Décomposition unique d'une fonction en somme de fonctions pair/impair.
  
  \exo1
  Résoudre $2\arcsin(x)=\arcsin(2x\sqrt{1-x^2})$
  
  \exo2
  Étudier et simplifier $f:x\longmapsto2\arccos\left(\frac{1-x^2}{1+x^2}\right)$. 
  
  \exo3
  Résoudre $\sys{\ch(x)+\ch(y)=a\ch(\alpha)\\\sh(x)+\sh(y)=a\sh(\alpha)}$ où $a,\alpha\in\R$.
  
  \end{eleve}
  %%%%%%%%%%%%%%%%%%%%%%%%%%%%%%%%%%%%%%%%%%%%%%%%%%%%%%%%%%%%%%%%%%%%%%%%%%%%%%%%%%%%%%%%%%%%%%%%%%%%%%%
  \begin{eleve}{}
  
  \cours
  Définition de $\ch$ et $\sh$.
  
  Formules trigo hyperbolique.
  
  Étude de $\arctan$.
  
  \exo1
  Résoudre $\log_x(10)+2\log_{10x}(10)+3\log_{100x}(10)=0$.
  
  \exo2
  Étudier et simplifier\\ $f:x\longmapsto2\arctan\left(\frac{x^2-2x-1}{x^2+2x-1}\right)$.
  
  \exo3
  Calculer $\sum_{k=0}^nk\sh(kx)$.
  
  \end{eleve}
  %%%%%%%%%%%%%%%%%%%%%%%%%%%%%%%%%%%%%%%%%%%%%%%%%%%%%%%%%%%%%%%%%%%%%%%%%%%%%%%%%%%%%%%%%%%%%%%%%%%%%%%
  \end{colle}
  
  \end{document}
  