\documentclass[twoside,a4paper,landscape,french,10pt]{VcCours}

%%%%%%%%%%%%%%%%%%%%%%%%%%%%
\setlength\columnseprule{0.4pt}%
\addtolength\columnsep{1cm}

%%%%%%%%%%%%%%%%%%%%%%%%%%%%
\newenvironment{colle}[4]{{\large\bf\makebox[0cm][l]{#1\hspace{\stretch{-1}}}\hspace{\stretch{1}}#2\hspace{\stretch{1}}\makebox[0cm][r]{\hspace{\stretch{-1}}#3}}
\noindent\rule{\linewidth}{0.4pt}
\begin{multicols*}{#4}}{\end{multicols*}\pagebreak}
\newenvironment{eleve}[1]{\textbf{\large #1}}{\vspace{\stretch{1}}
\columnbreak}
\newenvironment{eleveF}[1]{\textbf{\large #1}}{}
\newcommand{\cours}{\medskip\uwave{Cours :}

\smallskip}
\newcommand{\exo}[1]{\vspace{\stretch{1}}\uwave{Exercice #1:}

\smallskip}

\begin{document}

\pagestyle{empty}

%%%%%%%%%%%%%%%%%%%%%%%%%%%%%%%%%%%%%%%%%%%%%
%+-----------------------------------------+%
%%%%%%%%%%%%%%%%%%%%%%%%%%%%%%%%%%%%%%%%%%%%%


\begin{colle}{MPSI -- Groupe n°}{Colleur: Vincent Croizier}{Semaine
n\degres7 -- le lundi 15/11/2021}{3}
%%%%%%%%%%%%%%%%%%%%%%%%%%%%%%%%%%%%%%%%%%%%%%%%%%%%%%%%%%%%%%%%%%%%%%%%%%%%%%%
\begin{eleve}{}

    \cours
    Définition de noyaux et images d'un morphisme de groupes\\
    Lien avec l'injectivité + démo.\\
    Exercice 4

    \exo1
    Étudier la fonction $f(x)=\frac{\ln(1+x^2)}{x}$.

    \exo2
    Soit $G$ un groupe multiplicatif, $E$ un ensemble, et 
    $\varphi:G\longrightarrow E$ une bijection. On définit 
    une opération $*$ sur $E$ par :
    \[\forall x,y\in E, x * y = \varphi(\varphi^{-1}(x)\varphi^{-1}(y))\]
    Montrer que $(E,*)$ est un groupe et $G$ et $E$ sont isomorphes.

    \vspace*{\stretch{2}}

\end{eleve}
%%%%%%%%%%%%%%%%%%%%%%%%%%%%%%%%%%%%%%%%%%%%%%%%%%%%%%%%%%%%%%%%%%%%%%%%%%%%%%%%
\begin{eleve}{}

  \cours
  Tout sur $\ln$ + démo $\ln(ab)=\ln(a)+\ln(b)$.\\
  Exercice 7

  \exo1
  Étudier la fonction $f(x)=xe^{\tfrac1x}$

  \exo2
  Soit $G$ un groupe multiplicatif et $H$ une partie finie de $G$ non vide,
  stable par multiplication. Montrer que $H$ est un
  sous-groupe de $G$.

  \vspace*{\stretch{2}}
    
\end{eleve}
%%%%%%%%%%%%%%%%%%%%%%%%%%%%%%%%%%%%%%%%%%%%%%%%%%%%%%%%%%%%%%%%%%%%%%%%%%%%%%%%
\begin{eleveF}{}
    
  \cours
  Définitions de morphisme de groupes.\\ 
  Composée et réciproque d'un isomorphisme de groupes.\\
  Exercice 3.

  \exo1
  Étudier $f(x)=e^{x\ln(x)}$.

  \exo2
  $(G,+)$ un groupe et $\varphi:G\longrightarrow G'$ un morphisme de groupes.
  \begin{enumerate}
    \item Montrer que pour tout sous-groupe $H$ de $G$ on a :
    $f^{-1}(f(H)) = H + \Ker f$.
    \item  Montrer que pour tout sous-groupe $H'$ de $G'$ on a : 
    $f(f^{−1}(H')) = H'\cap \Ima f$.    
  \end{enumerate}
  
  \vspace*{\stretch{2}}
  
\end{eleveF}
%%%%%%%%%%%%%%%%%%%%%%%%%%%%%%%%%%%%%%%%%%%%%%%%%%%%%%%%%%%%%%%%%%%%%%%%%%%%%%%%
\end{colle}%
\end{document}
    