\documentclass[twoside,a4paper,landscape,french,10pt]{VcCours}

%%%%%%%%%%%%%%%%%%%%%%%%%%%%
\setlength\columnseprule{0.4pt}%
\addtolength\columnsep{1cm}

%%%%%%%%%%%%%%%%%%%%%%%%%%%%
\newenvironment{colle}[4]{{\large\bf\makebox[0cm][l]{#1\hspace{\stretch{-1}}}\hspace{\stretch{1}}#2\hspace{\stretch{1}}\makebox[0cm][r]{\hspace{\stretch{-1}}#3}}
\noindent\rule{\linewidth}{0.4pt}
\begin{multicols*}{#4}}{\end{multicols*}\pagebreak}
\newenvironment{eleve}[1]{\textbf{\large #1}}{\vspace{\stretch{1}}
\columnbreak}
\newenvironment{eleveF}[1]{\textbf{\large #1}}{}
\newcommand{\cours}{\medskip\uwave{Cours :}

\smallskip}
\newcommand{\exo}[1]{\vspace{\stretch{1}}\uwave{Exercice #1:}

\smallskip}

\begin{document}

\pagestyle{empty}

%%%%%%%%%%%%%%%%%%%%%%%%%%%%%%%%%%%%%%%%%%%%%
%+-----------------------------------------+%
%%%%%%%%%%%%%%%%%%%%%%%%%%%%%%%%%%%%%%%%%%%%%


\begin{colle}{MPSI -- Groupe n°}{Colleur: Vincent Croizier}{Semaine
n\degres1 -- le lundi 27/09/2021}{3}
%%%%%%%%%%%%%%%%%%%%%%%%%%%%%%%%%%%%%%%%%%%%%%%%%%%%%%%%%%%%%%%%%%%%%%%%%%%%%%%
\begin{eleve}{}

    \cours
    Définition de $A\cap B$, $A\cup B$, $A\setminus B$ et $\overline{A}$.\\
    Démo de $A\subset B \Longleftrightarrow \overline{B}\subset \overline{A}$.

    \exo1
    Montrez que $\forall n\geqslant 1$,
    $\sum_{k=1}^{n}\dfrac{k}{(k+1)!}=1-\dfrac{1}{(n+1)!}$.


    \exo2
    Résoudre dans $\C$ l'équation: $(z + 1)^n = e^{2ina}$.

    \exo3
    Montrez que $(\forall \varepsilon >0,|a|\leqslant \varepsilon)\Longrightarrow a=0$.  

    \vspace*{\stretch{2}}

\end{eleve}
%%%%%%%%%%%%%%%%%%%%%%%%%%%%%%%%%%%%%%%%%%%%%%%%%%%%%%%%%%%%%%%%%%%%%%%%%%%%%%%%
\begin{eleve}{}

  \cours
  Démo $\overline{A}=\overline{A\cap B}\cup\overline{B}$.\\
  $u_0=2$, $u_1=10$, $u_{n+2}=4u_{n+1}-4u_n$.\\
  Montrer que $\forall n\in\N$, $u_n=(3n+2)2^n$.

  \exo1
  Résoudre $\cos(2x)-\sqrt3\sin(2x)=1$.

  \exo2
  \begin{enumerate}
    \item Étudier pour $n\in\N$, la proposition $2n+1\leq 2^n$.
    \item Montrer que $\forall n\in\N$, $n^2\leq 2^n+1$.
  \end{enumerate}

  \exo2
  $A,B,C,D$ des parties de $E$.\\
  Montrez que $\systeme{B\setminus C\subset A\\C\setminus D\subset A}
  \Longrightarrow B\setminus D\subset A$

  \vspace*{\stretch{2}}
    
\end{eleve}
%%%%%%%%%%%%%%%%%%%%%%%%%%%%%%%%%%%%%%%%%%%%%%%%%%%%%%%%%%%%%%%%%%%%%%%%%%%%%%%%
\begin{eleveF}{}
    
  \cours
  Démo $A\cap B=\varnothing \Longleftrightarrow A\subset\overline{B}
  \Longleftrightarrow B\subset\overline{A}$.\\
  Montrez que $\forall z\in\C$, $\exists!(a,b)\in\R^2/$ $z = a + bj$ 
  où $j=e^{i2\pi/3}$.


  \exo1
  Montrez que $\forall n\geqslant 1$,
  $\sum_{k=1}^{n}k\times k!=(n+1)!-1$.
  
  \exo2
  Linéariser $\cos^{2}x\sin(4x)$.

  \exo3
  Montrer que pour tout $\lambda \in \R$, le nombre complexe $z =
  \frac{1+\lambda i}{1-\lambda i}$ est de module $1$. Pour quels $z
  \in\C$, existe-t-il $\lambda \in\R$ tel que $z = \frac{1+\lambda
  i}{1-\lambda i}$?
  
  \vspace*{\stretch{2}}
  
\end{eleveF}
%%%%%%%%%%%%%%%%%%%%%%%%%%%%%%%%%%%%%%%%%%%%%%%%%%%%%%%%%%%%%%%%%%%%%%%%%%%%%%%%
\end{colle}%
\end{document}
    