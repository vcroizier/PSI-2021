\documentclass[twoside,a4paper,landscape,french,10pt]{VcCours}

%%%%%%%%%%%%%%%%%%%%%%%%%%%%
\setlength\columnseprule{0.4pt}%
\addtolength\columnsep{1cm}

%%%%%%%%%%%%%%%%%%%%%%%%%%%%
\newenvironment{colle}[4]{{\large\bf\makebox[0cm][l]{#1\hspace{\stretch{-1}}}\hspace{\stretch{1}}#2\hspace{\stretch{1}}\makebox[0cm][r]{\hspace{\stretch{-1}}#3}}
\noindent\rule{\linewidth}{0.4pt}
\begin{multicols*}{#4}}{\end{multicols*}\pagebreak}
\newenvironment{eleve}[1]{\textbf{\large #1}}{\vspace{\stretch{1}}
\columnbreak}
\newenvironment{eleveF}[1]{\textbf{\large #1}}{}
\newcommand{\cours}{\medskip\uwave{Cours :}

\smallskip}
\newcommand{\exo}[1]{\vspace{\stretch{1}}\uwave{Exercice #1:}

\smallskip}

\begin{document}

\pagestyle{empty}

%%%%%%%%%%%%%%%%%%%%%%%%%%%%%%%%%%%%%%%%%%%%%
%+-----------------------------------------+%
%%%%%%%%%%%%%%%%%%%%%%%%%%%%%%%%%%%%%%%%%%%%%


\begin{colle}{MPSI -- Groupe n°}{Colleur: Vincent Croizier}{Semaine
n\degres6 -- le lundi 08/11/2021}{3}
%%%%%%%%%%%%%%%%%%%%%%%%%%%%%%%%%%%%%%%%%%%%%%%%%%%%%%%%%%%%%%%%%%%%%%%%%%%%%%%
\begin{eleve}{}

    \cours
    Définition de groupe + démo intersection de sous-groupes.\\
    Exercice 6.

    \exo1
    Étudier la fonction $f(x)=\sqrt{|x^2-3x+2|}$.

    \exo2
    Exercice 1 de la feuille annexe.


    \vspace*{\stretch{2}}

\end{eleve}
%%%%%%%%%%%%%%%%%%%%%%%%%%%%%%%%%%%%%%%%%%%%%%%%%%%%%%%%%%%%%%%%%%%%%%%%%%%%%%%%
\begin{eleve}{}

  \cours
  Définition de sous-groupe et ses 2 caractérisations.\\
  Exercice 4

  \exo1
  Étudier la fonction $f(x)=xe^{\tfrac1x}$

  \exo2
  exo 2.2.2

  \vspace*{\stretch{2}}
    
\end{eleve}
%%%%%%%%%%%%%%%%%%%%%%%%%%%%%%%%%%%%%%%%%%%%%%%%%%%%%%%%%%%%%%%%%%%%%%%%%%%%%%%%
\begin{eleveF}{}
    
  \cours
  Définitions de associativité, commutativité, élément neutre, élément symétrisable.\\
  Exercice 1.

  \exo1
  Étudier $f(x)=\ln(1+e^x+e^{2x})$.

  \exo2
  Exercice 2 de la feuille annexe.
  
  \vspace*{\stretch{2}}
  
\end{eleveF}
%%%%%%%%%%%%%%%%%%%%%%%%%%%%%%%%%%%%%%%%%%%%%%%%%%%%%%%%%%%%%%%%%%%%%%%%%%%%%%%%
\end{colle}%
\end{document}
    