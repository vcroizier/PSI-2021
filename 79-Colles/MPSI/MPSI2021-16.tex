\documentclass[twoside,a4paper,landscape,french,10pt]{VcCours}

%%%%%%%%%%%%%%%%%%%%%%%%%%%%
\setlength\columnseprule{0.4pt}%
\addtolength\columnsep{1cm}

%%%%%%%%%%%%%%%%%%%%%%%%%%%%
\newenvironment{colle}[4]{{\large\bf\makebox[0cm][l]{#1\hspace{\stretch{-1}}}\hspace{\stretch{1}}#2\hspace{\stretch{1}}\makebox[0cm][r]{\hspace{\stretch{-1}}#3}}
\noindent\rule{\linewidth}{0.4pt}
\begin{multicols*}{#4}}{\end{multicols*}\pagebreak}
\newenvironment{eleve}[1]{\textbf{\large #1}}{\vspace{\stretch{1}}
\columnbreak}
\newenvironment{eleveF}[1]{\textbf{\large #1}}{}
\newcommand{\cours}{\medskip\uwave{Cours :}

\smallskip}
\newcommand{\exo}[1]{\vspace{\stretch{1}}\uwave{Exercice #1:}

\smallskip}

\begin{document}

\pagestyle{empty}

%%%%%%%%%%%%%%%%%%%%%%%%%%%%%%%%%%%%%%%%%%%%%
%+-----------------------------------------+%
%%%%%%%%%%%%%%%%%%%%%%%%%%%%%%%%%%%%%%%%%%%%%


\begin{colle}{MPSI -- Groupe n°}{Colleur: Vincent Croizier}{Semaine
n\degres16 -- le lundi 31/01/2022}{3}
%%%%%%%%%%%%%%%%%%%%%%%%%%%%%%%%%%%%%%%%%%%%%%%%%%%%%%%%%%%%%%%%%%%%%%%%%%%%%%%
\begin{eleve}{}

\cours
Définition concave, inégalité de Jensen\\
Exo2: $f$ et $g$ cont. sur $[a,b]$ dér. sur $]a,b[$, $g'$ ne s'annule pas. $\exists c\in]a,b[/\frac{f(b)-f(a)}{g(b)-g(a)}=\frac{f'(c)}{g'(c)}$.

\exo1
\begin{enumerate}
  \item $f:\R\longrightarrow\R$ et $g:\R\longrightarrow\R$ convexes implique $g\circ f$ convexe.
  \item $f:\R\longrightarrow\R_+^*$. Montrer que $\ln\circ f$ convexe implique $f$ convexe.  
\end{enumerate}

\exo2
Critère spécial des séries alternées.

\vspace*{\stretch{1}}



\end{eleve}
%%%%%%%%%%%%%%%%%%%%%%%%%%%%%%%%%%%%%%%%%%%%%%%%%%%%%%%%%%%%%%%%%%%%%%%%%%%%%%%%%%%%%%%%%%%%%%%%%%%%%%%
\begin{eleve}{}

\cours
Théorème des trois pentes.\\
exo:$f(x)=x\exp(1/\ln(x))$ se prolonge par continuité sur $[0;1]$.

\exo1
$f:[a,b]\longrightarrow[c,d]$ convexe bijective et croissante. $a\leq u_0\leq v_0\leq b$
\[u_{n+1}=\frac{u_n+v_n}{2}\et v_{n+1}=f^{-1}\left(\frac{f(u_n)+f(v_n)}{2}\right)\]
Mq $u$ et $v$ convergent vers une même limite.

\exo2
$f$ une fonction dérivable sur $\R$ et admettant une même limite en $-\infty$ et $+\infty$.

Montrer qu'il existe $c\in\R$ tel que $f'(c)=0$.
\vspace*{\stretch{1}}

\end{eleve}
%%%%%%%%%%%%%%%%%%%%%%%%%%%%%%%%%%%%%%%%%%%%%%%%%%%%%%%%%%%%%%%%%%%%%%%%%%%%%%%%%%%%%%%%%%%%%%%%%%%%%%%
\begin{eleveF}{}

\cours
$f$ convexe ssi $f'$ croissante + démo\\
Mq $\frac2{\pi}x\leq\sin(x)\leq x$ sur $]0;\pi/2[$.

\exo1
Monter que $f(x)=\ln(1+e^x)$ est convexe sur $\R$, puis
\[\forall x_1,\ldots,x_n\in\R_+^*,
1+\left(\prod_{k=1}^nx^k\right)^{\frac1n}\leq\left(\prod_{k=1}^n(1+x^k)\right)^{\frac1n}\]

\exo2
$f$ une fonction dérivable sur $\R$ telle que $f(0)=f(a)=f'(0)=0$ avec $a>0$.

Montrer que la courbe de $f$ admet une tangente en $b>0$ passant par l'origine.


\vspace*{\stretch{1}}

\end{eleveF}
%%%%%%%%%%%%%%%%%%%%%%%%%%%%%%%%%%%%%%%%%%%%%%%%%%%%%%%%%%%%%%%%%%%%%%%%%%%%%%%%%%%%%%%%%%%%%%%%%%%%%%%
\end{colle}


\end{document}
