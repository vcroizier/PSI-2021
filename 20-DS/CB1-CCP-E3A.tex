\documentclass[twoside,french,11pt]{VcCours}
\renewcommand{\l}{\LL(E)}
\newcommand{\dx}{\text{d}x}
\newcommand{\dt}{\text{d}t}

\begin{document}

\Titre{PSI}{Promotion 2021--2022}{Concours Blanc 2022}{Épreuve de Mathématiques\\{\large Type CCINP-E3A}}
\fancyhead[CE,CO]{\bf{Concours Blanc 2022 -- Épreuve de Mathématiques}}

\begin{center}
\large 
Le mardi $1^{\text{er}}$ février 2022

\bigskip
\textbf{Durée: 4h}

\bigskip
\large\underline{\textbf{Calculatrice interdite}}
\end{center}

\bigskip
\begin{itemize}
  \item Le candidat attachera la plus grande importance à la clarté, à la précision et à la concision de la rédaction. 
  Si un candidat est amené à repérer ce qui peut lui sembler être une erreur d'énoncé, il le signalera sur sa copie et 
  devra poursuivre sa composition en expliquant les raisons des initiatives qu'il a été amené à prendre.
  \item Il est conseillé au candidat de lire l'intégralité du sujet et de repérer les parties qui lui semblent plus abordables.
  \item Le candidat \fbox{encadrera} ou \underline{soulignera} les résultats.
  \end{itemize}
\separationTitre


\newpage
\section*{Exercice 1}
Soit $\C[X]$ l'ensemble des polynômes à coefficients complexes. Dans tout cet  exercice, on identifie les éléments de $\C[X]$ et leurs fonctions polynomiales associées. Soit $P\in \C[X]$ un polynôme non nul vérifiant la relation
\[(*)\qquad P(X^2-1)=P(X-1)P(X+1)\]
\begin{enumerate}
\item Montrer que si $a$ est racine de $P$ alors $(a+1)^2-1$ et $(a-1)^2-1$ sont aussi des racines de $P$.
\item Soit $a_0\in \C$. On définit la suite de nombres complexes $(a_n)_{n\geq 0}$ en posant, pour tout $n\geq 0$, $a_{n+1}=a_n^2+2a_n$.
\begin{enumerate}
\item Vérifier que lorsque $a_0$ est une racine de $P$, pour tout entier naturel $n$ le nombre complexe $a_n$ est une racine de $P$.
\item Montrer que lorsque $a_0$ est un réel strictement positif, la suite $(a_n)_{n\geq 0}$ est une suite strictement croissante de réels positifs.
\item En déduire que $P$ n'admet pas de racine réelle strictement positive.
\item Montrer que $-1$ n'est pas racine de $P$.
\item Montrer que pour tout $n\in \N$, $a_{n}+1=(a_0+1)^{2^n}$.
\end{enumerate}
\item Déduire des questions précédentes que si $a$ est une racine complexe de $P$ alors $|a+1|=1$. On admettra que l'on a aussi $|a-1|=1$.
\item Montrer que si le degré de $P$ est strictement positif alors $P$ a pour unique racine $0$.
\item Déterminer tous les polynômes $P\in \C[X]$ qui vérifient la relation $(*)$.
\end{enumerate}

\section*{Exercice 2}
Pour tout entier naturel $n$ dans $\N^*$, on note :
\[h_n=\sum_{k=1}^n\frac{1}{k},\ f_n=h_n-\ln(n)\]
On considère les suites $(u_n)_{n\in\N^*}$ et $(v_n)_{n\in \N^*}$ définies par~:
$$ u_1=1\ \textrm{ et pour }\ n\geq 2,\ u_n=\frac{1}{n}+\ln \left(1-\frac{1}{n}\right)$$
et pour tout entier $n \geq 1$, 
$$ v_n=\frac{1}{n}-\ln \left(1+\frac{1}{n}\right)$$
\begin{enumerate}
\item Rappeler l'ensemble de définition de la fonction $x\mapsto x+\ln(1-x)$. Préciser son développement limité à l'ordre 2 en 0.
\item Étudier la fonction $f :  x\mapsto x-\ln(1+x)$ sur $[0,1]$.
\item Soit $n$ un entier naturel non nul. Quel est le signe de $u_n$ ? 
\item Justifier que la série $\sum_{n\geq 1} u_n$ est convergente.
\item Justifier que la série $\sum_{n\geq 1} v_n$ est convergente.
\item Soit $n$ un entier naturel non nul. Exprimer en fonction de $n$, $v_n-u_n$.\\
En déduire une expression de $\sum\limits_{n=1}^N(v_n-u_n)$ en fonction de $N$ pour tout entier naturel $N$ supérieur ou égal à $2$.
\item Que peut-on dire des suites $\left(\sum\limits_{n=1}^Nv_n \right)_{N\in \N^*}$ et $\left(\sum\limits_{n=1}^Nu_n\right)_{N\in \N^*}$~? 

Justifier que $\sum\limits_{n\geq 1}v_n$ et $\sum\limits_{n\geq 1}u_n$ ont la même somme.
\end{enumerate}

\medskip

\noindent Dans la suite de l'exercice, on note $\gamma$ la somme des séries $\sum\limits_{n\geq 1}v_n$ et $\sum\limits_{n\geq 1}u_n$.

\begin{enumerate}[resume]
\item Démontrer que $\gamma$ est dans l'intervalle $]0,1[$.
\item Soit $n$ un entier naturel non nul. Justifier que~:
\[\ln(n+1)\leq h_n\leq 1+\ln(n)\]
\item Justifier que la suite $(f_n)_{n\in \N^*}$ est croissante.
\item Démontrer que la suite $(f_n)_{n\in \N^*}$ est convergente et de limite $\gamma$.\\
\textit{Indication} : exprimer les sommes partielles de la série $\sum\limits_{n\geq 1}^Nu_n$ en fonction des termes de la suite $(f_n)$.
\item Soit $r$ un entier naturel $>1$.
\begin{enumerate}
\item Dessiner l'allure du graphe de la fonction $x\mapsto \dfrac{1}{x^r}$ sur $\R^{+*}$.
\item Soit $a$ un nombre réel strictement positif. Exprimer en fonction de $a$ et $r$~:
\[I(a)=\int_a^{+\infty}\frac{dt}{t^r}\]
\item Soit $(w_n)$ une suite de nombres réels qui converge vers $0$.\\
On suppose que la suite $(n^r(w_{n+1}-w_n))_{n\in \N}$ est convergente vers une limite $\ell$ telle que $\ell >0$.
\begin{enumerate}
\item Soient $a,b$ dans $\R^{+*}$ tels que $0<a<\ell<b$. Justifier l'existence d'un entier naturel $N$ supérieur ou égal à $2$ tel que pour tout entier naturel $n\geq N$, on ait les inégalités~:
\[a\leq n^r(w_{n+1}-w_n)\leq b\]
\item Démontrer que pour tout entier naturel $n$ supérieur ou égal à $N$~:
\[a\int_{N}^{n+1}\frac{dt}{t^r}\leq w_{n+1}-w_N\leq b\int_{N-1}^n\frac{dt}{t^r}\]
\item En déduire l'encadrement~:
\[-bI(N-1)\leq w_N\leq -aI(N)\]
o\`u $I$ a été défini dans la question 12(b).
\item Démontrer que la suite $(n^{r-1}w_n)_{n\in \N}$ est convergente et expliciter en fonction de $\ell$ et $r$ sa limite.
\item Ce résultat reste-t-il vrai si la limite $\ell$ de la suite $(n^{r}(w_{n+1}-w_n))_{n\in \N}$ est $0$~?
\end{enumerate} 
\end{enumerate}
\item Démontrer qu'il existe un nombre réel $\alpha$ que l'on explicitera tel que~:
\[ \sum_{k=1}^n\frac{1}{k} \underset{+ \infty}{=}\ln(n)+\gamma+\frac{\alpha}{n}+o \left(\frac{1}{n} \right)\]

\noindent \textit{Indication} : on appliquera les résultats de la question 12 à une suite bien choisie.
\end{enumerate}
  
\section*{Problème 3}
\centerline{\bf Notations et objectifs.}
\bigskip
\noindent Dans tout ce problème, $n$ est un entier naturel supérieur ou égal à $2$ et $E$ est un espace vectoriel de dimension finie $n$ sur $\R$.\\
$\l$ désigne l'ensemble des endomorphismes de $E$ et $GL(E)$ l'ensemble des endomorphismes de $E$ qui sont bijectifs.\\
On note $0$ l'endomorphisme nul et $\id$ l'application identité.\\
Pour tout endomorphisme $f$, $\Ker (f)$ et $\Ima(f)$ désigneront respectivement le noyau et l'image de $f$.\\
L'ensemble des valeurs propres de $f$ sera noté Sp($f$) et on notera~:
$$
\RR(f)= \{h \in \l | h^2=f \}
$$
où $h^2=h \circ h$.\\
$\R[X]$ désigne l'espace des polynômes à coefficients réels.\\
Étant donné $f \in \l$ et $P \in \R[X]$ donné par $P(X)= \sum_{k=0}^{\ell} a_k X^k$, on définit $P(f) \in \l$ par~:
$$
P(f)=\sum_{k=0}^\ell a_k f^k
$$
o\`u $f^0=\id$ et pour $k \in \N^*$, $f^k=\underset{\rm k \ fois} {\underbrace{f \circ \cdots \circ f}}$.\\
Si $f_1,\ldots,f_q$ désignent $q$ endomorphismes de $E$ ($q \in \N^*$) alors $\prod_{1 \le i \le q} f_i$ désignera l'endomorphisme de $E$ suivant : $f_1 \circ \cdots \circ f_q$.\\
Pour tout entier $p$ non nul, $\MM_p(\R)$ désigne l'espace des matrices carrées à $p$ lignes et $p$ colonnes à coefficients dans $\R$.\\
$I_p$ est la matrice identité de $\MM_p(\R)$.\\
L'objectif du problème est d'étudier des conditions nécessaires ou suffisantes à l'existence de racines carrées d'un endomorphisme $f$ et de décrire dans certains cas l'ensemble $\RR(f)$. 



%%%%%%%%%%%%%%%%%%%%%%%%%%%%%%%%%%%%%%%%%%%%%%%%%%%%%%%%
\vskip 1.8cm
\centerline{\bf Partie I}
\bigskip
\noindent {\bf A)} On désigne par $f$ l'endomorphisme de $\R^3$ dont la matrice dans la base canonique est donnée par~:
$$
A=
\begin{pmatrix}
8 & 4 & -7 \\
-8 & -4 & 8 \\
0 & 0 & 1
\end{pmatrix}
$$

\begin{enumerate}
\item Montrer que $f$ est diagonalisable.
\item Déterminer une base $(v_1,v_2,v_3)$ de $\R^3$ formée de vecteurs propres de $f$ et donner la matrice $D$ de $f$ dans cette nouvelle base. \textit{On ordonnera les vecteurs propres dans l'ordre croissant des valeurs propres}.
\item Soit $P$ la matrice de passage de la base canonique à la base $(v_1,v_2,v_3)$. Soit un entier $m \ge 1$. Sans calculer l'inverse de $P$, exprimer $A^m$ en fonction de $D$, $P$ et $P^{-1}$.
\item Calculer $P^{-1}$, puis déterminer la matrice de $f^m$ dans la base canonique.
\item Déterminer toutes les matrices de $\MM_3(\R)$ qui commutent avec la matrice $D$ trouvée à la question 2.
\item Montrer que si $H \in \MM_3(\R)$ vérifie $H^2=D$, alors $H$ et $D$ commutent.
\item Déduire de ce qui précède toutes les matrices $H$ de $\MM_3(\R)$ vérifiant $H^2=D$, puis déterminer tous les endomorphismes $h$ de $\R^3$ vérifiant $h^2=f$ en donnant leur matrice dans la base canonique.
\end{enumerate}



\bigskip


\noindent {\bf B)} Soient $f$ et $j$ les endomorphismes de $\R^3$ dont les matrices respectives $A$ et $J$ dans la base canonique sont données par~:
$$
A=
\begin{pmatrix}
2 & 1 & 1 \\
1 & 2 & 1 \\
1 & 1 & 2
\end{pmatrix}
\ \hbox{et} \ 
J=
\begin{pmatrix}
1 & 1 & 1 \\
1 & 1 & 1 \\
1 & 1 & 1 \\
\end{pmatrix}
$$
\begin{enumerate}
\item Calculer $J^m$ pour tout entier $m \ge 1$.
\item En déduire que pour tout $m \in \N^*$, $f^m=\id + \frac13 (4^m-1) j$. Cette relation est-elle encore valable pour $m=0$ ?
\item Montrer que $f$ admet deux valeurs propres distinctes $\lambda$ et $\mu$ telles que $\lambda<\mu$.
\item Montrer qu'il existe un unique couple $(p,q)$ d'endomorphismes de $\R^3$ tel que pour tout entier $m \ge 0$, $f^m=\lambda^m p + \mu^m q$ et montrer que ces endomorphismes $p$ et $q$ sont linéairement indépendants.
\item Après avoir calculé $p^2$, $q^2$, $p \circ q$ et $q \circ p$, trouver tous les endomorphismes $h$, combinaisons linéaires de $p$ et $q$ qui vérifient $h^2=f$.
\item Montrer que $f$ est diagonalisable et trouver une base de vecteurs propres de $f$. Écrire la matrice $D$ de $f$, puis la matrice de $p$ et de $q$ dans cette nouvelle base.
\item Déterminer une matrice $K$ de $\MM_2(\R)$ non diagonale telle que $K^2=I_2$, puis une matrice $Y$ de $\MM_3(\R)$ non diagonale telle que $Y^2=D$. \textit{On pourra construire $Y$ par blocs à l'aide de $K$.}
\item En déduire qu'il existe un endomorphisme $h$ de $\R^3$ vérifiant $h^2=f$ qui n'est pas combinaison linéaire de $p$ et $q$.
\item Montrer que tous les endomorphismes $h$ de $\R^3$ vérifiant $h^2=f$ sont diagonalisables.
\end{enumerate}


%%%%%%%%%%%%%%%%%%%%%%%%%%%%%%%%%%%%%%%%%%%%%%%%%%%%%%%%
\vskip 2cm
\centerline{\bf Partie II}
\bigskip
\noindent Soit $f$ un endomorphisme de $E$. On suppose qu'il existe $(\lambda,\mu) \in \R^2$ et deux endomorphismes non nuls $p$ et $q$ de $E$ tels que :
$$
\lambda \not= \mu \ \hbox{et} \ 
\left\{
\begin{array}{l}
\id = p+ q \\
f = \lambda p + \mu q \\
f^2 = \lambda^2 p + \mu^2 q
\end{array}
\right.
$$
\begin{enumerate}
\item \textit{Question de cours.} Montrer que si $Q$ est un polynôme annulateur de $f$ alors toute valeur propre de $f$ est racine de $Q$.
\item Calculer $(f-\lambda \id) \circ (f-\mu \id)$. En déduire que $f$ est diagonalisable.
\item Montrer que $\lambda$ et $\mu$ sont valeurs propres de $f$ et qu'il n'y en a pas d'autres.
\item Déduire de la relation trouvée dans la question 2 que $p \circ q = q \circ p =0$ puis montrer que $p^2=p$ et $q^2=q$.
\item On suppose jusqu'à la fin de cette partie que $\lambda \mu \not= 0$.\\
Montrer que $f$ est un isomorphisme et écrire $f^{-1}$ comme combinaison linéaire de $p$ et $q$.
\item Montrer que pour tout $m \in \Z$~:
$$
f^m = \lambda^m p + \mu^m q
$$
\item  Soit $F$ le sous-espace de $\l$ engendré par $p$ et $q$. Déterminer la dimension de $F$.
\item On suppose dans la suite de cette partie que $\lambda$ et $\mu$ sont strictement positifs. Déterminer $\RR(f) \cap F$.
\item Soit $k$ un entier supérieur ou égal à $2$. Déterminer une matrice $K$ de $\MM_k(\R)$ non diagonale et vérifiant $K^2=I_k$. \textit{On utilisera la question 7 de I.B en construisant une matrice par blocs définie à l'aide de $I_{k-2}$.}
\item Montrer que si l'ordre de multiplicité de la valeur propre $\lambda$ est supérieur ou égal à $2$, alors il existe un endomorphisme $p' \in \l \setminus F$ tel que ${p'}^2=p$ et $p' \circ q = q \circ p' = 0$.
\item En déduire que si $\dim(E) \ge 3$, alors $\RR(f) \not\subset F$.
\end{enumerate}

\end{document}