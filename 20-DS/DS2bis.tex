\documentclass[twoside,french,11pt]{VcCours}

\begin{document}

\Titre{PSI}{Promotion 2021--2022}{Mathématiques}{Devoir surveillé n°2${}^{\text{bis}}$}

\begin{center}
\large 
Le jeudi 14 octobre 2021

\bigskip
\textbf{Durée: 3h30}

\bigskip
\large\underline{\textbf{Calculatrice interdite}}
\end{center}

\bigskip
\begin{itemize}
  \item Le candidat attachera la plus grande importance à la clarté, à la précision et à la concision de la rédaction. Si un candidat est amené à repérer ce qui peut lui sembler être une erreur d'énoncé, il le signalera sur sa copie et devra poursuivre sa composition en expliquant les raisons des initiatives qu'il a été amené à prendre.
  \item Il est conseillé au candidat de lire l'intégralité du sujet et de repérer les parties qui lui semblent plus abordables.
  \item Le candidat \fbox{encadrera} ou \underline{soulignera} les résultats. 
  %\item Les étudiants visant un concours plus dur que CCP traiteront \textbf{obligatoirement} l'exercice $2$.
  \end{itemize}
\separationTitre


\section*{Exercice 1 -- Séries}
  Déterminer la nature des séries suivantes. 
  
  \begin{multicols}{3}
  \begin{enumerate}
  \item $\sum_{n \geq 1} \dfrac{n\sqrt{n}}{n^7+\sqrt{n}}$
  \item $\sum_{n \geq 1} \dfrac{\ln^{5}(n)}{2\sqrt{n}+1}$
  \columnbreak
  \item $\sum_{n \geq 1} \dfrac{n!\pi^n}{n^n}$
  \item $\sum_{n \geq 1} \sqrt{\ch\left(\frac{1}{n^2}\right)-1}$
  \columnbreak
  \item $\sum_{n \geq 1} \arctan \left( \dfrac{(-1)^n}{n}\right)$
  \end{enumerate}
  \end{multicols}

\section*{Exercice 2 -- Espaces vectoriels}
\begin{enumerate}
  \item Montrer que $F=\ensemble{(x;y;z)\in\R^3}{2x-y-5z=0}$ est un sous-espace
  vectoriel de $\R^3$, en donner une base et la dimension.
  \item Montrer que $\SS_n(\R)$ et $\AA_n(\R)$ sont supplémentaires dans $\MM_n(\R)$.
\end{enumerate}

  
\section*{Exercice 3 -- Espaces vectoriels}
On note $\mathcal{M}_3(\mathbb{R})$ l'espace vectoriel des matrices carrées 
d'ordre $3$ à coefficients réels. On note $I_3$ la matrice identité de 
$\mathcal{M}_3(\mathbb{R})$.

\subsection*{Partie 1 : un ensemble de matrices}
Soit $F$ l'ensemble défini par :
$$ F = \ensemble{\begin{pmatrix}
a & 0 & 0 \\
0 & b & -c \\
0 & c & b \\
\end{pmatrix}}{(a,b,c) \in \R^3}$$

\begin{enumerate}
\item Montrer que $F$ est un espace vectoriel et donner-en une base.
\item Montrer que $F$ est stable par produit :
$$ \forall (A,B) \in F^2, \; AB \in F$$
Dans la suite de l'exercice, on pose :
$$ M = \begin{pmatrix}
-2 & 0 & 0 \\
0 & 1 & -1 \\
0 & 1 & 1 \\
\end{pmatrix}$$
\item Montrer que $(I_3,M,M^2)$ est une base de $F$.
\item Exprimer $M^3$ en fonction de $I_3$ et $M$.
\end{enumerate}


\subsection*{Partie 2 : étude d'une suite}


Soit $u$ la suite définie par $u_0=3$, $u_1=0$, $u_2=4$ et pour tout entier $n \geq 0$,
$$ u_{n+3}= 2u_{n+1}-4u_n$$

\medskip

\begin{enumerate}
\item Écrire, en langage Python, une fonction {\tt Suite} ayant pour argument un entier naturel $n$ et renvoyant le réel $u_n$.
\item Montrer que pour tout entier $k \geq 0$, il existe des réels $a_k$, $b_k$ et $c_k$ tels que :
$$ M^k = \begin{pmatrix}
a_k & 0 & 0 \\
0 & b_k & -c_k \\
0 & c_k & b_k \\
\end{pmatrix}$$
Préciser une relation de récurrence définissant $(a_k)_{k \geq 0}$ et deux relations de récurrences liant $(b_k)_{k \geq 0}$ et $(c_k)_{k \geq 0}$.
\item En déduire pour tout entier $k \geq 0$, $a_k$ en fonction de $k$.
\item Pour tout $k \in \mathbb{N}$, on pose $z_k$ le nombre complexe défini par :
$$ z_k = b_k + i c_k$$
Exprimer $z_{k+1}$ en fonction de $z_k$ et en déduire que $b_k = \Re e((1+i)^k)$ puis une expression simple de $b_k$.
% \item Montrer que pour tout entier $n \geq 0$,
% $$ u_n = \textrm{Tr}(M^n)$$
% et en déduire la valeur de $u_n$.
\end{enumerate}

  
\section*{Problème -- séries dont le terme général est un reste}
  Soit $n_0$ un entier naturel fixé.

  Soient $(a_n)_{n\geq n_0}$ une suite numérique et $\sum_{n \geq n_0} a_n$ la
  série de terme général $a_n.$ 
  
  \medskip
  
  On suppose que cette série est convergente et on définit pour $n\geq n_0,$ son reste
  d'ordre $n$ par :
  \begin{center}$r_n=\sum_{k=n+1}^{+\infty}a_k$
  \end{center}
  Dans ce problème, on s'intéresse à la
  série $\sum_{n \geq n_0} r_n$ dans divers exemples.
  
  \medskip
  
  
  \begin{enumerate}
  
  \item {\it Exemple 1}. Dans cette question, on pose pour tout entier $n\geq 0,$ 
  $$a_n=\lambda^n$$
  où $\lambda\in\R.$
  
  \begin{enumerate}
    \item Donner, sans preuve, une condition nécessaire et suffisante sur $\lambda$ pour la série $\sum_{n \geq 0} a_n$ soit convergente. Rappeler la valeur de la somme de cette série. 
    
    \item On suppose cette condition réalisée. \\
    Déterminer pour tout entier $n \geq 0$, la valeur de $r_n$ (en fonction de $\lambda$). \\
    En déduire que $\sum_{n \geq 0} r_n$ converge et calculer sa somme.
  \end{enumerate}
  
  
  \item {\it Exemple 2}. Pour tout entier $n\geq 1,$ on pose maintenant :
  $$a_n=\frac{1}{n^2}$$
  
  \begin{enumerate}
  
  %\item Démontrer que pour tout entier $k\in\N^*$, on a :
  %$$\frac{1}{(k+1)^2}\leq\int_{k}^{k+1}\frac{dt}{t^2}\leq
  %\frac{1}{k^2}$$
  
  
  
  \item Montrer que pour tout $n\in\N^*$ et pour tout entier $N\geq n+1$, on a l'encadrement suivant :
  $$\sum_{k=n+1}^{N}\frac{1}{(k+1)^2}\leq
  \int_{n+1}^{N+1}\frac{1}{t^2}dt\leq
  \sum_{k=n+1}^{N}\frac{1}{k^2}$$
  
  
  
  \item En déduire que pour tout entier $n\in\N^*,$ on a :
  $$ \frac{1}{n+1}\leq r_n \leq
  \frac{1}{n+1}+\frac{1}{(n+1)^2}$$
  
  
  
  \item Déterminer alors un équivalent de $r_n$ quand $n$ tend vers $+\infty.$\\
  Que peut-on en conclure sur la nature de la série $\sum_{n \geq 1}
  r_n$?
  
  
  \end{enumerate}

  
  \item {\it Exemple 3}. Dans cet question, on suppose simplement que $\sum_{n \geq 0} a_n$ est une série à termes positifs.
  
  
  
  \begin{enumerate}
    \item Démontrer que pour tout $n\in\N^*,$ on a :
    $$\sum_{k=1}^n ka_k=\sum_{k=0}^{n-1} r_k-nr_n$$
    
    \item En déduire que si $\sum_{n \geq 0} r_n$ converge alors $\sum_{n \geq 0} na_n$ converge aussi.
    
    \item Retrouver alors le résultat de la question $2(c).$ 
    
  \end{enumerate}
  
  \item {\it Exemple 4.} Pour tout entier $n\geq 1,$ on pose dorénavant :
  $$a_n=\frac{(-1)^n}{n}$$
  
  \begin{enumerate}
  
  \item Justifier la convergence de la série $\sum_{n \geq 1} a_n.$ \textit{Il ne suffira pas de donner son nom}.
  
  
  \item On définit la suite $(I_n)_{n \geq 1}$ par :
  $$\forall n\in\N^*,\;
  I_n=(-1)^n\int_0^1\frac{x^n}{1+x}dx$$
  
  \begin{enumerate}
  
  \item Démontrer que 
  $\lim_{n\to +\infty}I_n=0.$
  
  
  \item Soit $n\in\N^*$. Calculer $I_{n+1}-I_n.$
  En déduire que :
  $$ \forall n\in\N^*, \; I_n=\ln(2)+\sum_{k=1}^n\frac{(-1)^k}{k}$$
  \item En déduire la valeur de :
  $$\sum_{k=1}^{+\infty}\frac{(-1)^k}{k}$$
  Exprimer enfin $r_n$ en fonction de $I_n.$
  
  
  
  \end{enumerate}
  
  \item \textit{Conclusion.}
  
  \begin{enumerate}
  
  \item Montrer que :
  $$ I_n \underset{+ \infty}{=} \frac{(-1)^n}{a(n+1)}+O\left(\frac{1}{n^{\alpha}}\right)$$
  o\`u $a\in\R^*$ et $\alpha>1$ sont des constantes à déterminer.
  \item En déduire la nature de la série $\sum_{n \geq 1} r_n$.
  \end{enumerate}
  \end{enumerate}
  \item {\it Exemple 5.}  Pour tout entier $n\geq 0,$ on pose dorénavant :
  $$a_n=\frac{1}{n!}$$
  \begin{enumerate}
  \item Déterminer un équivalent de $r_n$ quand $n$ tend vers $+ \infty$.
  \item En déduire la nature de la série de terme général $r_n$.
  \item Déterminer, à l'aide de (a), la nature de la série de terme général $\sin(2 \pi e n!)$.
  \end{enumerate}
  \end{enumerate}
  
  
  \end{document}