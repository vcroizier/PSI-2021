\documentclass[a4paper,twoside,french,10pt]{VcCours}

\begin{document}
\Titre{PSI}{Promotion 2021--2022}{Mathématiques}{Devoir surveillé n\degres1}

\begin{center}
\large\bf
Correction
\end{center}
\separationTitre


\section*{Exercice 1}
Soit $n \in \mathbb{N}^*$. Calculer $\sum_{k=0}^{n} 2^k \binom{n}{k}$.

\section*{Exercice 2}
Soit $f$ la fonction définie par $f(x) =  \arcsin \left( \frac{x}{\sqrt {1 + x^{2}}} \right) \cdot$

\begin{enumerate}
\item Rappeler l'ensemble de définition de la fonction arcsinus.
\item Justifier que $f$ est définie sur $\mathbb{R}$.
\item Donner une expression simple de $f$.
\end{enumerate}

\medskip
\section*{Exercice 3}
Calculer $\int_{0}^{1} \arctan(x) \text{d}x$.

\medskip
\section*{Exercice 4}
Déterminer $\lim_{n \rightarrow + \infty} \left( 1+ \frac{1}{n} \right)^n$.

\medskip
\section*{Exercice 5}
Déterminer le terme général de la suite définie par $u_0=2$ et pour tout $n \in \mathbb{N}$ par :
$$u_{n+1} = 5 u_n -6$$

\medskip
\section*{Exercice 6}
Déterminer la nature de $\sum_{n \geq 1} \dfrac{1}{n^3+n^2+n} \cdot$

\medskip
\section*{Exercice 7}
 

\begin{enumerate}
\item Factoriser dans $\mathbb{C}[X]$ le polynôme $X^2+X+1$.
\item Factoriser dans $\mathbb{C}[X]$ le polynôme $X^4+X^2+1$.
\end{enumerate}

\medskip
 \section*{Exercice 8}
Soit $f$ l'application définie sur $\mathbb{R}_2[X]$ par $f(P)=P-(X+1)P'+X^2 P''$.
\begin{enumerate}
\item Montrer que $f$ définit un endomorphisme de $\mathbb{R}_2[X]$.
\item Déterminer le noyau de $f$. On donnera une base de celui-ci.
\item Que vaut le rang de $f$? Déterminer une base de l'image de $f$.
\end{enumerate}

\medskip
\section*{Exercice 9}
Soient $F$ et $G$ les ensembles définis par :
$$ F = \lbrace (x,y,z) \in \mathbb{R}^3, \, x-y+z=0 \rbrace \quad \hbox{ et }  \quad G = \textrm{Vect}((1,1,1)) $$

\begin{enumerate}
\item Donner une base de $F$ et préciser sa dimension.
\item Montrer que $F$ et $G$ sont supplémentaires dans $\mathbb{R}^3$.
\item Soit $p$ la projection sur $F$ parallèlement à $G$. Donner l'expression de $p$.
\item Soit $q$ la projection sur $G$ parallèlement à $F$. Donner l'expression de $q$.
\end{enumerate}


\end{document}