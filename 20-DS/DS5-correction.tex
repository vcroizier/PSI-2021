\documentclass[twoside,french,11pt]{VcCours}
\newcommand{\enc}[1]{\fbox{#1}}
\newcommand{\dt}{\text{d}t}
\newcommand{\du}{\text{d}u}
\renewcommand{\d}{\text{d}}
\renewcommand{\l}{\langle }
\renewcommand{\r}{\rangle }
\begin{document}

\Titre{PSI}{Promotion 2021--2022}{Mathématiques}{Devoir surveillé n°5}

\begin{center}
\large\bf
Correction
\end{center}
\separationTitre


\section*{Problème 1}
\subsection*{Partie 1 : polynômes d'Hermite}
  \begin{enumerate}
    
    \item Pour tout $x\in\R,$ on a :
    \begin{itemize}
    \item $w(x)=e^{-x^2}$
    \item $w'(x)=-2xe^{-x^2}$
    \item $w''(x)=(4x^2-2)e^{-x^2}$ 
    \item $w'''(x)=(-8x^3+12x)e^{-x^2}$
    \end{itemize}
  Ainsi, pour tout $x \in \mathbb{R}$,
  
  \enc{$H_1(x)=2x, \, H_2(x)=4x^2-2$ et $H_3(x)=8x^3-12x$}
    
  
    \item Soit $n \in \mathbb{N}$. On sait que pour tout $x \in \mathbb{R}$,
    $$H_n(x)=(-1)^nw^{(n)}(x)e^{x^2}$$
    donc par dérivation d'un produit :
    $$ H_n'(x)= (-1)^n w^{(n+1)}(x) e^{x^2} + (-1)^n 2x w^{(n)}(x) e^{x^2} $$
    Or on a :
    $$H_{n+1}(x)=(-1)^{n+1}w^{({n+1})}(x)e^{x^2} = - (-1)^{n}w^{({n+1})}(x)e^{x^2}$$
  ce qui implique que :
    
    
    $$ \boxed{H_{n+1}(x)=2xH_n(x)-H'_n(x)}$$
    %En déduire l'expression de $H_4(x).$
    
    \item La première question permet de conjecturer la valeur du coefficient dominant. Montrons par récurrence que pour tout entier $n \geq 0$, $H_n$ est un polynôme de degré $n$ et a un coefficient dominant égal à $2^n$. 
    
    \begin{itemize}
    \item La propriété est vraie au rang $0$ car $H_0(X)=1$ et $2^0=1$.
    \item Soit $n \in \mathbb{N}$ tel que la proposition soit vraie au rang $n$. On sait que :
    $$ \forall x \in \mathbb{R}, \;  H_{n+1}(x)=2xH_n(x)-H'_n(x)$$
    Or $H_n$ est un polynôme par hypothèse de récurrence donc $H_n'$ aussi et ainsi, $H_{n+1}$ aussi. Par hypothèse, $H_n$ est de degré $n$ donc $2X H_n$ est de degré $n+1$ et $H_n'$ a un degré strictement plus petit que $H_n$ ($H_n$ n'est pas nul car $n \geq 0$). Par somme (les degrés sont différents), on en déduit que $H_{n+1}$ est de degré $n+1$. Pour les mêmes raisons, le coefficient dominant de $H_{n+1}$ est le coefficient dominant de $2XH_n$ donc $2\times 2^n=2^{n+1}$ par hypothèse. La propriété est donc vraie au rang $n+1$.
    \end{itemize}
  La propriété est vraie au rang $0$ et est héréditaire donc par principe de récurrence, elle est vraie pour tout entier $n \geq 0$. Ainsi, pour tout entier $n \geq 0$,
  
  \enc{$H_n$ est un polynôme de degré $n$ et a un coefficient dominant égal à $2^n$}
    
    
    \item Pour tout $x\in\R,$ $w(-x)=w(x)$ donc pour tout entier $n \geq 0$, en dérivant $n$ fois, on obtient que pour tout $x \in \mathbb{R}$,
  $$(-1)^nw^{(n)}(-x)=w^{(n)}(x)$$
  En multipliant par $e^{x^2},$ on obtient que :
  $$ \boxed{\forall x\in\R,\quad H_n(-x)=(-1)^nH_n(x)}$$
  Si $n$ est pair, $H_n$ est paire et si $n$ est impair, $H_n$ est impaire.
  
  \end{enumerate}
  
  \subsection*{Partie 2 : un produit scalaire sur $\R[X]$}  
  
  \begin{enumerate}    
    \item Soient $P,Q\in\R[X]$. Si l'un des polynômes $P$ ou $Q$ est nul, il est évident que $\l P|Q\r$ existe. Supposons que $P$ et $Q$ sont non nuls. La fonction $t \mapsto P(t)Q(t) e^{-t^2}$ est continue sur $\mathbb{R}$. Le polynôme $PQ$ étant non nul ($P$ et $Q$ le sont), il existe une constante $C$ et un entier $k \geq 0$ tel que :
  $$ P(t)Q(t) \underset{+ \infty}{\sim} c t^d$$
  donc
  $$ P(t)Q(t) e^{-t^2}\underset{+ \infty}{\sim} c t^d e^{-t^2}$$
  ce qui implique par théorème des croissances comparées que :
  $$ P(t)Q(t) e^{-t^2}  \underset{+ \infty}{=} o \left( \dfrac{1}{t^2} \right)$$
  La fonction $t \mapsto \tfrac{1}{t^2}$ est intégrable sur $[1, + \infty[$ donc par critère de comparaison, $t \mapsto P(t)Q(t) e^{-t^2}$ aussi. La même méthode permet de montrer l'intégrabilité de celle-ci sur $]- \infty,-1]$. La fonction étant continue sur $[-1,1]$, on en déduit que pour tous $P,Q\in\R[X]$,
  \enc{$\l P|Q\r$ est bien défini}
  
    \item Vérifions les différents points.
    
    \begin{itemize}
    \item Pour tous $P,Q \in \mathbb{R}[X]$, $\l P|Q\r$ existe et est un réel.
    \item La symétrie est évidente par commutativité du produit de réels.
    \item La linéarité à gauche est évidente par linéarité de l'intégrale. La bilinéarité est donc prouvée par symétrie.
    \item Soit $P \in \mathbb{R}[X]$. Alors :
    $$ \l P|P\r = \int_{- \infty}^{+ \infty} P(t)^2 e^{-t^2} \dt \geq 0$$
    par positivité de l'intégrale (intégrale convergente d'une fonction continue positive et les bornes sont dans le bon sens). Si $\l P|P\r =0$ alors par stricte positivité de l'intégrale (la fonction est continue) on en déduit que :
    $$ \forall t \in \mathbb{R}, \, P(t)^2 e^{-t^2} = 0$$
    donc :
    $$ \forall t \in \mathbb{R}, \, P(t) = 0$$
    Le polynôme $P$ est donc le polynôme nul.
    \end{itemize}
  Ainsi,	
  \enc{$(P,Q)\longmapsto \l P|Q\r$ définit un produit scalaire sur $\R[X]$}
    
    
    \item
    \begin{enumerate}
      \item On a  :
      $$\|1\|^2=\int_{-\infty}^{+\infty}e^{-t^2} \dt=\sqrt{\pi}$$
  donc 
  $$ \boxed{\|1\|=\pi^{1/4}}$$
  Par définition, on a :
  $$\l X| 1\r=\int_{-\infty}^{+\infty}te^{-t^2}\dt$$
  Soit $A \in \mathbb{R}_+$. On a :
  $$ \int_0^A te^{-t^2}dt = \left[ - \dfrac{e^{-t^2}}{2} \right]_0^A = - \dfrac{e^{-A^2}}{2} + \dfrac{1}{2}$$
  Par passage à la limite, on en déduit que :
  $$ \int_{0}^{+\infty}te^{-t^2} \dt = \dfrac{1}{2}$$
  De même, on obtient :
  $$  \int_{- \infty}^{0}te^{-t^2} \dt = -\dfrac{1}{2}$$
  On obtient donc que :
  $$ \boxed{\l X| 1\r = 0}$$
  
      
      
      \item Soit $n \in \mathbb{N}$. Posons $f_n : \mathbb{R} \rightarrow \mathbb{R}$ définie par :
      $$ \forall t \in \mathbb{R}, \, f_n(t) = t^{2n+1}e^{-t^2}$$
      La fonction $f_n$ est impaire sur $\mathbb{R}$. Le changement de variable $ u : t \mapsto -t$ est une bijection de $\mathcal{C}^1$ strictement décroissante de $\mathbb{R}_+$ sur $\mathbb{R}_{-}$ donc d'après le théorème de changement de variable, on en déduit que les intégrales suivantes sont de même nature et égale en cas de convergence :
  $$ \int_{0}^{+ \infty} f_n(t) \dt \; \hbox{ et } \; \int_{0}^{- \infty} f_n(-u) - \textrm{d}u = - \int_{- \infty}^0 f_n(u) \textrm{d}u$$
  Or les deux intégrales convergent (raisonnement de la première question). On en déduit que :
  \begin{align*}
  \int_{-\infty}^{+\infty} f_n(t) \dt & = \int_{-\infty}^{0} f_n(t) \dt + \int_{0}^{+\infty} f_n(t) \dt \\
  & = \int_{-\infty}^{0} f_n(t) \dt - \int_{-\infty}^{0} f_n(t) \dt \\
  & = 0
  \end{align*}
  Ainsi, pour tout $n\in\N,$ 
  $$ \boxed{\int_{-\infty}^{+\infty}t^{2n+1}e^{-t^2}\dt=0}$$
      
      \item Soit $n \in \mathbb{N}$. On a :
      $$ I_{n+1} = \int_{-\infty}^{+\infty}t^{2n+2}e^{-t^2}\dt = \int_{-\infty}^{+\infty}t^{2n+1} te^{-t^2}\dt$$
  Soit $(A,B) \in \mathbb{R}^2$ tel que $A<B$. Les fonctions $t \mapsto - \tfrac{1}{2} e^{-t^2}$ et $t \mapsto t^{2n+1}$ sont de classe $\mathcal{C}^1$ sur $[A,B]$ donc par intégration par parties, on a :
  \begin{align*}
  \int_{A}^{B}t^{2n+1} te^{-t^2}\dt & = \left[ - \dfrac{1}{2} e^{-t^2} t^{2n+1} \right]_A^B + \dfrac{2n+1}{2} \int_{A}^{B} t^{2n} e^{-t^2} \dt \\
  & =  - \dfrac{1}{2} e^{-B^2} B^{2n+1} +   \dfrac{1}{2} e^{-A^2} A^{2n+1} + \dfrac{2n+1}{2} \int_{A}^{B} t^{2n} e^{-t^2} \dt
  \end{align*}
  Les intégrales $I_n$ et $I_{n+1}$ étant convergentes et par croissance comparées, on en déduit par passage à la limite quand $A$ tend vers $- \infty$ et $B$ tend vers $+ \infty$ que :
  $$ \boxed{I_{n+1}=\frac{2n+1}{2}I_n}$$
  On en déduit que pour tout entier $n \geq 0$,
  $$I_n=\frac{2n-1}{2}I_{n-1}=\dots=\frac{(2n-1)\times(2n-3)\times\cdots \times1}{2^n}I_{0}$$
  En multipliant par $(2n)\times(2n-2)\times\cdots\times 2=2^nn!$ le numérateur et le dénominateur, on en déduit que :
  $$ \boxed{ I_{n}=\int_{-\infty}^{+\infty}t^{2n}e^{-t^2}dt=\frac{(2n)!}{2^{2n}n!}\sqrt{\pi}}$$ 
  \end{enumerate}
    
    \item Soient $P \in F$ et $Q \in G$. Alors le polynôme $PQ$ est impair (produit d'une fonction paire et d'une fonction impaire). Le même raisonnement que dans la question 3.(b) (avec $f_n : t \mapsto P(t)Q(t) e^{-t^2}$) montre que $\l P|Q\r$ est nul. Ainsi,
  \enc{$F\perp G$}
    
    \item Procédons par étapes.
    \begin{itemize}
  \item Posons $Q_0=1$. 
  \item On a montré que $\l 1|X\r =0$ donc on pose $Q_1=X$.
  \item On pose $Q_2 = X^2+ \alpha Q_1+ \beta Q_0 = X^2 + \alpha X+ \beta$ où $\alpha, \beta \in \mathbb{R}$. On a :
  $$ \l Q_2|Q_0\r = \int_{- \infty}^{+ \infty} (t^2+ \alpha t+ \beta) e^{-t^2} \dt = I_1 + \alpha \times 0 + \beta I_0 $$
  d'après 3.(b). D'après la question précédente, on obtient que :
  $$ \l Q_2|Q_0\r = \dfrac{1}{2} \sqrt{\pi} + \beta \sqrt{\pi}$$
  On choisit donc $\beta = - \tfrac{1}{2}$ pour que le produit scalaire soit nul. De même, on a :
  $$  \l Q_2|Q_1\r = \int_{- \infty}^{+ \infty} (t^3+ \alpha t^2+ \beta t) e^{-t^2} \dt = 0 + \alpha I_1 + \beta \times 0 $$
  d'après 3.(b). On pose donc $\alpha=0$ pour que le produit scalaire soit nul. Ainsi,
  $$ Q_2(X) = X^2 - \dfrac{1}{2}$$
    \end{itemize}
  On a déjà montré que la norme de $Q_0$ (le polynôme constant égal à $1$) vaut $\pi^{1/4}$. On pose donc :
  $$ \boxed{P_0 = \pi^{-1/4}}$$
  On a :
  $$ \|Q_1\|^2=\int_{-\infty}^{+\infty}t^2e^{-t^2}dt=I_1=\frac{\sqrt{\pi}}{2}$$
  On pose donc :
  $$\boxed{P_1=\frac{\sqrt{2}}{\pi^{1/4}}X}$$
  On a pour finir :
  \begin{align*}
  \|Q_2\|^2 & =\int_{-\infty}^{+\infty}\left(t^2-\frac{1}{2}\right)^2e^{-t^2}dt \\
  &=I_2-I_1+\frac{1}{4}I_0 \\
  &=\sqrt{\pi}\left(\frac{3}{4}-\frac{1}{2}+\frac{1}{4}\right)
  \\& =\frac{\sqrt{\pi}}{2}
  \end{align*}
  On pose donc : 
  $$ \boxed{P_2=\frac{\sqrt{2}}{\pi^{1/4}}\left(X^2-\frac{1}{2}\right)}$$ 
  
    
    \item On a :
    $$\inf_{(a,b)\in\R^2}\int_{-\infty}^{+\infty}(t^2-a-bt)^2e^{-t^2}dt=\inf_{(a,b)\in\R^2}\|X^2-(a+bX)\|^2=d(X^2,\R_1[X])^2$$
  Soit $p : \mathbb{R}[X] \rightarrow \mathbb{R}[X]$ la projection orthogonale sur $\mathbb{R}_1[X]$ (qui existe car cet espace est de dimension finie). On sait que $(P_0,P_1)$ est une base orthonormée de $\R_1[X]$ d'après la question précédente. On a alors :
  $$p(X^2)  =\l X^2| P_0\r P_0+\l X^2| P_1\r P_1=  I_1 \times \dfrac{1}{\sqrt{\pi}} = \dfrac{1}{2}$$
  Les calculs précédents permettent de montrer que :
  $$\boxed{d(X^2,\R_1[X])^2=\|X^2-P(X)\|^2=\|Q_2(X)\|^2=\frac{\sqrt{\pi}}{2}}$$
    
  \end{enumerate}
   
  \subsection*{Partie 3 : construction d'une base orthonormée de $\R_n[X].$}

  \begin{enumerate}
    \item Soient $P\in\R[X]$ et $n\in\N^*$. On a : 
    $$\l P'|H_{n-1}\r \ = \ \int_{-\infty}^{+\infty}P'(t)H_{n-1}(t)e^{-t^2} \dt$$
  Soit $(A,B) \in \mathbb{R}^2$ tel que $A<B$. Les fonctions $t \mapsto P(t)$ et $t \mapsto H_{n-1}(t) e^{-t^2}$ sont de classe $\mathcal{C}^1$ sur $[A,B]$ donc par intégration par parties :
  $$ \int_{A}^{B} P'(t)H_{n-1}(t)e^{-t^2} \dt = \left[P(t)H_{n-1}(t)e^{-t^2}\right]_{A}^{B}- \int_{A}^{B} P(t)(H'_{n-1}-2tH_{n-1})(t)e^{-t^2} \dt$$
  On sait que $H_n(X)=-H'_{n-1}(X)+2XH_{n-1}(X)$ donc :
  $$  \int_{A}^{B} P'(t)H_{n-1}(t)e^{-t^2} \dt = \left[P(t)H_{n-1}(t)e^{-t^2}\right]_{A}^{B} + \int_{A}^{B} P(t) H_n(t)e^{-t^2} \dt$$
    Par passage à la limite (les intégrales convergent) quand $A$ tend vers $- \infty$ et $B$ tend vers $+ \infty$ et par croissances comparées, on en déduit que :
   
    $$ \boxed{\l P'|H_{k-1}\r \ = \ \l P|H_k\r}$$
    
    \item Soit $P\in\R_{n-1}[X].$ On applique le résultat précédent avec $P,P',\cdots $ :
    $$\l P|H_n\r \ =\ \l P'|H_{n-1}\r =\ \l P''|H_{n-2}\r  =  \cdots  =   \l P^{(n)}|H_{0}\r$$
  Or $P\in\R_{n-1}[X]$ donc $P^{(n)}$ est le polynôme nul donc on a bien :
    
  $$ \boxed{\forall P\in\R_{n-1}[X], \; \l P|H_n\r=0}$$  
    
  \item Soient $i,j\in\{0,\dots,n\}$. Sans perte de généralité, on peut supposer que $i<j$. Le degré $H_i$ vaut donc $i$ ce qui implique que $H_i\in\R_{j-1}[X].$ D'après la question précédente avec $n=j$ et $P=H_i$ on a alors $ \l H_i|H_j\r=0.$ Ainsi,
    
  \enc{La famille $(H_0,H_1,\dots,H_n)$ est orthogonale}
    
    \item La famille $(H_0,H_1,\dots,H_n)$ est une famille orthogonale de vecteurs non nuls de $\mathbb{R}_n[X]$ donc c'est une famille libre. Elle contient $n+1$ éléments qui est la dimension de $\mathbb{R}_n[X]$. Ainsi,
  
  \enc{$(H_0,H_1,\dots,H_n)$ est une base de $\R_n[X]$}
  
    
    \item 
    
    
    \begin{enumerate}
      \item On reprend le raisonnement de la question 2 avec $P=H_k$ :
      
      $$\| H_k\|^2=\l H_k|H_k\r \ =\ \l H_k'|H_{k-1}\r =\ \l H_k''|H_{k-2}\r \ = \ \cdots \ = \  \l H_k^{(k)}|H_{0}\r$$
  Ainsi,
  \enc{  $\| H_k\|^2=\l H_k^{(k)}| H_0\r$ }

      
      \item On sait que $H_k$ est de degré $k$ et de coefficient dominant $2^k.$ Ainsi,
       $$H_k^{(k)}=k!2^k$$
  Or $H_0=1$ donc 
       $$\| H_k\|^2=\l H_k^{(k)}| H_0\r=k!2^k\l 1| 1\r=\sqrt{\pi}k!2^k$$
      On obtient donc que :
       \enc{ $\|H_k\|=\pi^{1/4}\sqrt{k!}\sqrt{2}^k$ }
  La famille $(H_0,H_1,\dots,H_n)$ est orthogonale donc la famille suivante une base orthonormée de $\R_n[X]$ :
    $$\boxed{\left(\frac{H_0}{\|H_0\|},\frac{H_1}{\|H_1\|},\dots,\frac{H_n}{\|H_n\|}\right)}$$		
    
    \end{enumerate}
    
    
    
  \end{enumerate}
  
  \subsection*{Partie 4 : étude d'un endomorphisme}

  \begin{enumerate}
    \item Si $P \in \mathbb{R}_n[X]$, $XP'$ aussi ($P'$ est au maximum de degré $n-1$) et $P''$ aussi. Ainsi, $f(P) \in \mathbb{R}_n[X]$. La linéarité de $f$ est évidente par linéarité de la dérivation. 
    
    Ainsi, \enc{$f$ est un endomorphisme de $\R_n[X]$}
    \item 
    \begin{enumerate}
    \item Soient $P,Q\in\R_n[X]$. On a :
    
  $$\l P'|Q'\r  = \int_{-\infty}^{+\infty}P'(t)Q'(t)e^{-t^2} \dt$$
  Soit $(A, B) \in \mathbb{R}^2$ tel que $A<B$. Les fonctions $Q$ et $t \mapsto P'(t) e^{-t^2}$ sont de classe $\mathcal{C}^1$ sur $[A,B]$ donc par intégration par parties, on a :
  \begin{align*}
   \int_{A}^{B}P'(t)Q'(t)e^{-t^2} \dt & = \left[ Q(t) P'(t) e^{-t^2} \right]_A^B - \int_A^B Q(t) (P''(t)e^{-t^2} - 2tP'(t) e^{-t^2}) \dt \\
   & = \left[ Q(t) P'(t) e^{-t^2} \right]_A^B + \int_A^B Q(t) (-P''(t)+2tP'(t)) e^{-t^2} \dt \\
    & = \left[ Q(t) P'(t) e^{-t^2} \right]_A^B + \int_A^B Q(t) (f(P)(t)-P(t)) e^{-t^2} \dt \\
    & = \left[ Q(t) P'(t) e^{-t^2} \right]_A^B + \int_A^B Q(t) f(P)(t) e^{-t^2} \dt  -\int_A^B Q(t) P(t) e^{-t^2} \dt 
    \end{align*}
  D'après le théorème des croissances comparées, le crochet du terme précédent vers $0$ quand $A$ tend vers $- \infty$ et $B$ tend vers $+ \infty$. On en déduit alors (les intégrales sont convergentes) : 
  $$ \boxed{\l P'|Q'\r=\l f(P)|Q\r-\l P|Q\r}$$
      
    \item Soient $P,Q\in\R_n[X]$. Alors :
    $$\l f(P)|Q\r\ =\ \l P'|Q'\r+\l P|Q\r\ =\ \l Q'|P'\r+\l Q|P\r\ =\ \l f(Q)|P\r \ =\ \l P|f(Q)\r.$$
    Ainsi,
    $$\boxed{\l f(P)|Q\r = \l P|f(Q)\r}$$
    
    
    \end{enumerate}
    
    
    \item 
    \begin{enumerate}
    \item On a $H_0=1$ et $H_1=2X$. Par simple calcul, on trouve :
    
  \enc{ $f(H_0)=1=H_0$ et $f(H_1)=6X=3H_1$}
  
  \item Soit $k\in\{0,\dots, n\}$. On sait que $H_{k+1}=2XH_k-H'_k$ donc par dérivation :
  $$ H'_{k+1}=2H_k+2XH'_k-H''_k$$
  donc
  $$ \boxed{H'_{k+1}=H_k+f(H_k)}$$
    
    \item Montrons par récurrence que pour tout $k\in\{0,\dots,n\},$ on a $f(H_k)=(2k+1)H_k.$
    
    \begin{itemize}
    \item D'après la question précédente, $f(H_0)= H_0 = (2 \times 0+1)H_0$. 
    \item Soit $k \in \{0,\dots,n-1\}$ tel que :
    $$ f(H_k)=(2k+1)H_k$$
  Par définition de $f$, on a :
  $$ f(H_{k+1})= -H''_{k+1}+2XH'_{k+1}+H_{k+1}=(2k+2)(-H'_k+2XH_k)+H_{k+1}=(2k+3)H_{k+1}$$
  La propriété est donc vraie au rang $k+1$.
  \end{itemize}
  La propriété est vraie au rang $0$ et est héréditaire donc par principe de récurrence, elle est vraie pour tout entier $k \in \{0,\dots,n\}$.
    
    \item La famille $(H_0,\dots,H_n)$ est une base orthonormée de $\R_n[X]$ formée de vecteurs propres de $f$ donc
  
    \enc{$\left(\tfrac{H_0}{\|H_0\|},\dots,\tfrac{H_n}{\|H_n\|}\right)$ est une base orthonormée de $\R_n[X]$ formée de vecteurs propres de $f$}
      
    \end{enumerate}
    
  \end{enumerate}
  

\section*{Problème 2}

\subsection*{Partie 1}

  \begin{enumerate}
  \item Soit $\varepsilon \in ]0,1]$. Les fonctions $t \mapsto \dfrac{t^{s+1}}{s+1}$ et $t \mapsto \ln(t)$ sont de classe $\mathcal{C}^1$ sur $[\varepsilon,1]$ donc par intégration par parties :
  \begin{align*}
   \int_{\varepsilon}^1 t^s \ln(t) \dt & = \left[ \dfrac{t^{s+1}}{s+1} \ln(t) \right]_{\varepsilon}^1 - \dfrac{1}{s+1}\int_{\varepsilon}^1 t^{s} \dt \\
   & = -  \dfrac{\varepsilon^{s+1}}{s+1} \ln(\varepsilon)  - \dfrac{1}{(s+1)^2} \left[ t^{s+1} \right]_{\varepsilon}^1 \\
   & = -  \dfrac{\varepsilon^{s+1}}{s+1} \ln(\varepsilon)  - \dfrac{1}{(s+1)^2} + \dfrac{\varepsilon^{s+1}}{(s+1)^2}
   \end{align*}
  On sait que $s>-1$ donc $s+1>0$ donc par croissance comparées, on a :
  $$ \lim_{\varepsilon \rightarrow 0} -  \dfrac{\varepsilon^{s+1}}{s+1} \ln(\varepsilon) = 0$$
  et on a aussi :
  $$ \lim_{\varepsilon \rightarrow 0} \dfrac{\varepsilon^{s+1}}{(s+1)^2} = 0$$
  Par définition de convergence, on en déduit que $J_s$ converge et on a :
  $$ \boxed{J_s = - \dfrac{1}{(s+1)^2} }$$
  \item \underline{Étude de la fonction H}
  \begin{enumerate}
  \item Soit $x \in \mathbb{R}$. La fonction :
  $$ t \mapsto \frac{t^x\ln(t)}{t-1}$$
  est continue sur $]0,1[$ donc l'intégrale définissant $H(x)$ est impropre en $0$ et $1$.
  \begin{itemize}
  \item On a :
  $$ \frac{t^x\ln(t)}{t-1} \underset{t \rightarrow 0}{\sim} - t^x \ln(t)$$
  Si $x>-1$, $t \mapsto t^{x} \ln(t)$ est intégrable (conséquence de la convergence de $J_s$). Si $x \leq -1$, on a :
  $$ -t \times t^{x} \ln(t) = -t^{x+1} \ln(t) \underset{t \rightarrow 0}{\longrightarrow} + \infty$$
  car $x+1 \geq 0$. Ainsi, sur un voisinage de $0$, on a :
  $$ -t \times t^{x} \ln(t) \geq 1$$
  ou encore 
  $$  -t^{x} \ln(t) \geq \dfrac{1}{t} \geq 0$$
  La fonction $t \mapsto \tfrac{1}{t}$ a une intégrale divergente sur $]0,1/2]$ donc par critère de comparaison, $t \mapsto -t^x \ln(t)$ aussi. Ainsi, l'intégrale suivante diverge :
  $$ \int_{0}^{1/2} \frac{t^x\ln(t)}{t-1} \dt$$
  \item On a :
  $$ \frac{t^x\ln(t)}{t-1} \underset{t \rightarrow 1}{\sim} t^x$$
  donc
  $$ \lim_{t \rightarrow 1} \frac{t^x\ln(t)}{t-1} = 1$$
  L'intégrale définissant $H(x)$ est faussement impropre en $1$.
  \end{itemize}
  Finalement, l'intégrale définissant $H(x)$ converge si et seulement si $x<-1$. On en déduit que 
  \enc{l'ensemble de définition de la fonction $H$ est $D_H=]-1,+\infty[$}
  \item Soit $(x,y) \in D_H$ tel que $x \leq y$. Pour tout $t \in ]0,1[$, $\ln(t) \leq 0$ donc 
  $$ x \ln(t) \geq y \ln(t)$$
  donc par croissance de la fonction exponentielle sur $\mathbb{R}$ :
  $$ t^x \geq t^y$$
  Par positivité de $\dfrac{\ln(t)}{t-1}$, on a :
  $$ \dfrac{ t^x \ln(t)}{t-1} \geq \dfrac{t^y \ln(t)}{t-1}$$
  Par croissance de l'intégrale (les deux intégrales convergent et les bornes sont dans le bon sens), on en déduit que $H(x) \geq H(y)$. Ainsi,
  \enc{$H$ est décroissante sur $D_H$}
  \item Utilisons le théorème de Leibniz.
  \begin{itemize}
  \item Pour tout $x \in D_H$, $t \mapsto \dfrac{ t^x \ln(t)}{t-1}$ est continue et intégrable sur $]0,1[$ (vu précédemment).
  \item Pour tout $t \in ]0,1[$, $x \mapsto \dfrac{ t^x \ln(t)}{t-1}$ est de classe $\mathcal{C}^1$ sur $D_H$ et de dérivée :
  $$ x \mapsto \dfrac{ t^x \ln(t)^2}{t-1}$$
  \item Pour tout $x \in D_H$, $t \mapsto \dfrac{ t^x \ln(t)^2}{t-1}$ est continue sur $]0,1[$.
  \item Soit $[a,b] \subset D_H$. Pour tout $x \in [a,b]$, sachant que $t \in ]0,1[$, on a :
  $$ \vert t^x \vert = e^{x \ln(t)} \leq e^{a \ln(t)} = t^a$$
  donc :
  $$ \left\vert \dfrac{ t^x \ln(t)^2}{t-1} \right\vert \leq \dfrac{t^a \ln(t)^2}{1-t}$$
  On a :
  $$ \frac{t^a\ln(t)^2}{1-t} \underset{t \rightarrow 1}{\sim} \dfrac{t^a (t-1)^2}{1-t} = t^a (1-t)$$
  donc
  $$ \lim_{t \rightarrow 1}  \frac{t^a\ln(t)^2}{1-t} = 0$$
  Ainsi, $t \mapsto \dfrac{t^a \ln(t)^2}{1-t}$ est prolongeable par continuité en $1$ (donc intégrable sur $[1/2,1[$). On a de plus :
  $$ \frac{t^a\ln(t)^2}{1-t} \underset{t \rightarrow 0}{=} o \left( \dfrac{1}{t^{(1-a)/2}} \right)$$
  car
  $$  t^{(1-a)/2} \frac{t^a\ln(t)^2}{1-t} = \frac{t^{(a+1)/2}\ln(t)^2}{1-t} \underset{t \rightarrow 0}{\longrightarrow} 0$$
  par théorème des croissances comparées car $a+1>0$. De plus, $-a<1$ donc $1-a<2$ donc $\dfrac{1-a}{2}<1$ ce qui implique que $t \mapsto  \dfrac{1}{t^{(1-a)/2}}$ est intégrable sur $]0,1/2[$. Par critère de comparaison, on en déduit que $t \mapsto \dfrac{t^a \ln(t)^2}{1-t}$ aussi. Finalement, notre fonction dominante est intégrable sur $]0,1[$.
  \end{itemize}
  D'après le théorème de Leibniz, on en déduit que $H$ est de classe $\mathcal{C}^1$ sur $D_H$ et on a pour tout $x \in D_H$,
  $$ H'(x) = \int_0^1\dfrac{ t^x \ln(t)^2}{t-1} \dt$$
  L'intégrande est négative et les bornes sont dans le bon sens donc par positivité de l'intégrale, on en déduit que $H'(x) \leq 0$. Ainsi,
  \enc{$H$ est décroissante sur $D_H$}
  
  \item Soit $x>0$. La fonction $g : t\mapsto \dfrac{t\ln(t)}{t-1}$ est continue sur $]0,1[$ et prolongeable par continuité en $0$ (valeur $0$) et $1$ (valeur $1$). C'est donc une fonction bornée sur $]0,1[$. Une majoration grossière (les bornes sont dans le bon sens) implique que :
  \[ |H(x)|\leq \|g\|_\infty \int_0^1t^{x-1}\dt=\frac{\|g\|_\infty}{x}\]
  Le théorème d'encadrement permet de conclure :
  \[\lim_{x\to +\infty}H(x)=0\]
  
  \item Soit $x>-1$. On a :
  $$H(x)-H(x+1)=\int_0^1\frac{t^x(1-t)\ln(t)}{t-1}\dt=-J_x=\frac{1}{(x+1)^2}$$
  Ainsi, pour tout $x>-1$,
  \[\boxed{ H(x)-H(x+1)=\frac{1}{(x+1)^2}}\]
  \item La fonction $H$ est continue (car dérivable) en $0$ donc :
  $$ \lim_{x \rightarrow -1} H(x+1) = H(0)$$
  Pour tout $x>-1$, on a alors :
  $$ H(x) = \dfrac{1}{(x+1)^2} + H(x+1) = \dfrac{1}{(x+1)^2} ( 1 + (x+1)^2 H(x))$$
  Par la remarque précédente, on a :
  $$ \lim_{x \rightarrow -1} 1 + (x+1)^2 H(x) = 1$$
  donc 
  $$ \boxed{ H(x) \underset{-1}{\sim}\dfrac{1}{(x+1)^2}}$$
  \item Soit $x>-1$.
  \begin{enumerate}
  \item C'est une série à termes positifs ($x>-1$) dont le terme général est équivalent à $\dfrac{1}{k^2}$ qui est le terme général d'une série de Riemann convergente. D'après le critère de comparaison des séries à termes positifs, on en déduit que :
  
  \enc{$\sum_{k\geq 1}\frac{1}{(x+k)^2}$ converge}
  \item On sait que pour tout $x>-1$,
  $$ H(x)-H(x+1)=\frac{1}{(x+1)^2}$$
  Soit $n \in \mathbb{N}^*$. Pour tout $k \geq 1$, $x+k-1 >-1+1-1=-1$ donc :
  $$ H(x+k-1)-H(x+k) = \frac{1}{(x+k)^2}$$
  On somme pour $k$ variant de $1$ à $n$ :
  $$ \sum_{k=1}^n H(x+k-1)-H(x+k) =  \sum_{k=1}^n \frac{1}{(x+k)^2}$$
  Par télescopage, on a :
  $$ H(x) - H(x+n) =  \sum_{k=1}^n \frac{1}{(x+k)^2}$$
  donc
  \[ \boxed{H(x)=\sum_{k=1}^n\frac{1}{(x+k)^2}+H(x+n)}\]
  \item On sait que la limite de $H$ en $+ \infty$ est $0$ donc par passage à la limite dans l'égalité précédemment prouvée, on a 
  \[ \boxed{H(x)=\sum_{k=1}^{+ \infty} \frac{1}{(x+k)^2}}\]
  \item D'après la question précédente, on a :
  $$ H(0) = \sum_{k=1}^{+ \infty} \dfrac{1}{k^2}$$
  donc
  $$ \boxed{ H(0) = \dfrac{\pi^2}{6}}$$
  De même, on a :
  $$ H(0) = \sum_{k=1}^{+ \infty} \dfrac{1}{(k+1)^2} = \sum_{k=2}^{+ \infty} \dfrac{1}{k^2}$$
  donc
  $$ \boxed{ H(1) = \dfrac{\pi^2}{6} -1}$$
  \end{enumerate}
  \end{enumerate}
  \end{enumerate}
  
  \subsection*{Partie 2}
  
  \begin{enumerate}
  \item Simple comparaison en utilisant la décroissance de l'intégrande.
  \item On somme les inégalités précédente pour $k$ variant de $1$ à $n$ : 
  \[\sum_{k=1}^n\frac{1}{(x+k+1)^2}\leq \int_1^{n+1}\frac{dt}{(x+t)^2}=\left [-\frac{1}{t+x}\right ]_{1}^{n+1}\leq \sum_{k=1}^n\frac{1}{(x+k)^2}\]
  Tous les termes admettent une limite quand $n\to +\infty$ ce qui permet d'obtenir :
  \[H(x)-\frac{1}{(x+1)^2}\leq \frac{1}{1+x}\leq H(x)\]
  ou encore :
  \[\frac{1}{1+x}\leq H(x)\leq \frac{1}{1+x}+\frac{1}{(1+x)^2}\]
  puis :
  $$ 1 \leq (x+1) H(x) \leq 1 + \dfrac{1}{x+1}$$
  Le théorème d'encadrement permet d'obtenir que :
  $$ \lim_{x \rightarrow + \infty} (x+1) H(x) = 1$$
  donc 
  $$ H(x) \underset{+ \infty}{\sim} \dfrac{1}{x+1}$$
  ou encore 
  $$ \boxed{  H(x) \underset{+ \infty}{\sim} \dfrac{1}{x}}$$
  
  \item 
  \begin{enumerate}
  \item  Pour tout $n \geq 0$, $u_n \geq 0$ ($H$ et la somme d'une série de fonctions positives) et 
  $$ u_n = H(n) \underset{+ \infty}{\sim} \dfrac{1}{n}$$
  Par comparaison à la série harmonique, on en déduit que :
  \enc{$\sum_{n\geq 0}u_n$ diverge}
  On sait que $(u_n)$ est une positive, convergente vers $0$ ($H$ admet pour limite $0$ en $+ \infty$) et est décroissante (car $H$ l'est). D'après le critère spécial des séries alternées, on en déduit que :
  \enc{$\sum_{n\geq 0}(-1)^nu_n$ converge}
  \item Soit $t \in ]0,1[$. La série de terme général $f_n(t)$ est, à constante multiplicative près, le terme général d'une série géométrique convergente (car $-t \in ]-1,1[$). 
  
  Ainsi, \enc{$\sum f_n$ converge simplement sur $]0,1[$}
  et on a pour tout $t \in ]0,1[$,
  $$ \sum_{n=0}^{+ \infty} f_n(t) =  \dfrac{\ln(t)}{t-1} \sum_{n=0}^{+ \infty} (-t)^n = \dfrac{\ln(t)}{(t-1)(t+1)}$$
  donc
  $$ \boxed{S(t) = \dfrac{\ln(t)}{t^2-1}}$$
  \item Le théorème d'intégration terme à terme ne peut pas s'appliquer (la série des intégrales des valeurs absolues diverge). Pour tout $t \in ]0,1[$, on a :
  $$ S(t) = \lim_{n \rightarrow + \infty} g_n(t)$$
  où pour tout $n \geq 0$,
  $$g_n(t) = \sum_{k=0}^n f_k(t)$$
  \begin{itemize}
  \item Pour tout entier $n \geq 0$, $g_n$ est continue sur $]0,1[$.
  \item La suite de fonctions $(g_n)$ converge simplement vers $S$ sur $]0,1[$.
  \item Pour tout $n\in \N$ et tout $t\in ]0,1[$,
  \begin{align*}
  \vert g_n(t) \vert & =  \left |\sum_{k=0}^nf_k(t)\right | \\
  & =\left |\frac{(1-(-t)^{n+1})\ln(t)}{t^2-1}\right | \qquad (-t \neq 1) \\
  & \leq \frac{2|\ln(t)|}{1-t^2}
  \end{align*}
  La fonction $t \mapsto  \frac{2|\ln(t)|}{1-t^2}$ est continue sur $]0,1[$, prolongeable par continuité en $1$ (car $\ln(t) \underset{1}{\sim} t-1$) et on a :
  $$  \frac{2|\ln(t)|}{1-t^2} \underset{0}{\sim} 2 \vert \ln(t) \vert$$
  La fonction $t \mapsto \ln(t)$ est intégrable sur $]0,1]$ (intégrale de référence). Ainsi, notre fonction dominante est intégrable sur $]0,1[$.
  \end{itemize}
  D'après le théorème de convergence dominée, toute les fonctions $g_n$ sont intégrables, $S$ est intégrable, et on a :
  $$ \lim_{n \rightarrow + \infty} \int_0^1 g_n(t) \dt = \int_0^1 S(t) \dt $$
  donc
  $$ \boxed{\sum_{n=0}^\infty (-1)^nu_n=\sum_{n=0}^\infty\int_0^1f_n(t) \dt =\int_0^1\frac{\ln(t)}{t^2-1}\dt}$$
  \item La fonction $v\mapsto v^2$ est une bijection de classe $\mathcal{C}^1$  strictement croissante de $]0,1[$ dans $]0,1[$ donc d'après le théorème de changement de variable, les intégrales suivantes sont de même nature et égales en cas de convergence :
  $$ \int_0^1\frac{\ln(v)}{v^2-1} \; dv \; \hbox{ et } \; \int_0^1\frac{\ln(\sqrt{u})}{u-1}\frac{du}{2\sqrt{u}}$$
  La première converge (question précédente) donc 
  $$\int_0^1\frac{\ln(v)}{v^2-1}\;dv=\int_0^1\frac{\ln(\sqrt{u})}{u-1}\frac{du}{2\sqrt{u}} $$
  Ceci implique que :
  $$ \boxed{\int_0^1\frac{\ln(v)}{v^2-1}\;dv = \frac{1}{4}H(-1/2)}$$
  
  \end{enumerate}
  \end{enumerate}
  
  \subsection*{Partie 3}
  
  \begin{enumerate}
  \item
  \begin{enumerate}
  \item Simple intégration par parties bien justifiée.
  \item On conjecture et on prouve par récurrence (sur $q$) que :
  $$ \boxed{I_{p,q}=(-1)^q\frac{q!}{(p+1)^q}I_{p,0}=(-1)^q\frac{q!}{(p+1)^{q+1}}}$$
  \end{enumerate}
  \item
  \begin{enumerate}
  \item La fonction $t\mapsto \frac{(\ln(t))^{n+1}}{t-1}$ est continue sur $]0,1[$, négligeable devant $1/\sqrt{t}$ au voisinage de $0$ (par croissance croissances comparées) et prolongeable par continuité en $1$ (valeur $1$ si $n=0$ et $0$ si $n\geq 1$). C'est donc une fonction intégrable sur $]0,1[$. Ainsi, pour tout entier $n \geq 0$,
  \enc{$B_n$ existe}
  
  \item On sait que pour tout $t \in ]0,1[$,
  \[ \frac{1}{1-t}=\sum_{k=0}^{+\infty}t^k\]
  Posons donc  pour tout entier $k \geq 0$, $g_k\ :\ t\mapsto \ln(t)^{n+1}t^k$ ($n$ est fixé).
  \begin{itemize}
  \item Pour tout $k \geq 0$, $g_k$ est continue sur $]0,1[$.
  \item  $\sum_{k \geq 0} g_k$ converge simplement sur $]0,1[$ et sa somme est $t\mapsto \frac{(\ln(t))^{n+1}}{t-1}$ qui est continue sur $]0,1[$.
  \item Pour tout entier $k \geq 0$,
  $$\int_0^1|g_k|=(-1)^{n+1}\int_0^1g_k=(-1)^{n+1}I_{k,n+1}=\frac{(n+1)!}{(k+1)^{n+2}}$$
  qui est le terme général d'une série de Riemann convergente car $n+2\geq 2>1$.
  \end{itemize}
  D'après le théorème d'intégration terme à terme, on obtient :
  \[ \boxed{B_n=\sum_{k=0}^{+\infty}\int_0^1g_k=\sum_{k=0}^{+\infty}I_{k,n+1}}\]
  \item Avec la question 1, on en déduit (avec changement d'indice) que :
  \[ \boxed{B_n=\sum_{k=0}^{+\infty}(-1)^{n+1}\frac{(n+1)!}{(k+1)^{n+2}}=(-1)^{n+1}(n+1)!Z_{n+2}}\]
  \end{enumerate}
  \item On utilise le développement en série entière de l'exponentielle : 
  \[\forall x>-1,\ H(x)=\int_0^1\sum_{k=0}^{+\infty}\frac{(\ln(t))^{k+1}}{t-1}\frac{x^k}{k!}\;dt\]
  Fixons $x\in ]-1,1[$ et notons pour tout entier $k \geq 0$, $h_k\ :\ t\mapsto \dfrac{(\ln(t))^{k+1}}{t-1}\dfrac{x^k}{k!} \cdot$
  \begin{itemize}
  \item Pour tout entier $k \geq 0$, $h_k$ est continue sur $]0,1[$.
  \item $\sum h_k$ converge simplement sur $]0,1[$ et sa somme est $t\mapsto \frac{t^x\ln(t)}{t-1}$ qui est continue sur $]0,1[$.
  \item On a pour tout $n \geq 0$ et tout $t \in ]0,1[$,
  \[ \left |\sum_{k=0}^nh_k(t)\right |\leq \sum_{k=0}^{+\infty}|h_k(t)|=\frac{t^{-|x|}|\ln(t)|}{|t-1|}\]
  Comme $|x|<1$, $-|x|>-1$ et le majorant est intégrable sur $]0,1[$ (à détailler).
  \end{itemize}
  D'après le théorème de convergence dominée, on en déduit que :
  \[\boxed{\forall x\in ]-1,1[,\ H(x)=\sum_{k=0}^{+\infty}(-1)^k(k+1)Z_{k+2}x^k}\]
  \item Le rayon de convergence est au moins égal à $1$ (l'égalité étant valable sur $]-1,1[$). Si, par l'absurde, il était strictement plus grand que $1$, la somme de la série entière serait continue en $-1$ et $H$ admettrait une limite finie en $-1$ ce qui est faux. Ainsi,
  \enc{le rayon de convergence vaut donc $1$}
  \end{enumerate}
  


\end{document}