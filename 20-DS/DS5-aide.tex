\documentclass[twoside,french,10pt]{VcCours}
%cspell:ignore orthonormaliser
\newcommand{\dx}{\text{d}x}
\newcommand{\dt}{\text{d}t}

\begin{document}

\Titre{PSI}{Promotion 2021--2022}{Mathématiques}{Devoir surveillé n°5}

\begin{center}
	\textbf{\Large Indications}
\end{center}
	
\separationTitre

\section*{Problème 1}

\subsection*{Partie 1 : polynômes d'Hermite}

\begin{enumerate}
	\item Calculer les dérivées successives de $w$.
	\item Calculer $H_n'(x)+H_{n+1}(x)$.
	\item Récurrence, degré et terme dominant d'une somme de deux polynômes.
	\item Utiliser la parité de $w$ pour en déduire celle de $w^{(n)}$.
\end{enumerate}

\subsection*{Partie 2 : un produit scalaire sur $\R[X]$}

\begin{enumerate}
	\item $P(t)Q(t)\equi_{+\infty}ct^d$ le produit des monômes dominants, croissances comparées.
	\item Définition de produit scalaire...
	\item 
	\begin{enumerate}
		\item Utiliser l'intégrale de Gauss rappelée au début du sujet.
	
		Pour $\langle X|1\rangle$, calcul direct avec une primitive.
		\item Utiliser un argument de parité, ou alors poser $t=-u$.
		\item Intégration par parties. Puis récurrence.
	\end{enumerate}
	\item Utiliser un argument de parité, ou alors poser $t=-u$.
	\item Suivre la méthode, certain produits scalaires sont nuls grâce à la question précédente.
	\item Y voir $\inf_{P\in F}||X^2-P||^2$ où $F$ est un sev à reconnaître.
\end{enumerate}
 
\subsection*{Partie 3 : construction d'une base orthonormée de $\R_n[X].$}

\begin{enumerate}
	\item Intégration par parties et utiliser Partie 1 : 2).
	\item Itérer la relation précédente.
	\item Utiliser la question précédente. 
	\item Famille orthogonale...
	\item
	\begin{enumerate}
		\item Itérer la formule du 1).
		\item Utiliser Partie 1 : 3).
	\end{enumerate}
\end{enumerate}

\subsection*{Partie 4 : étude d'un endomorphisme}

\begin{enumerate}
	\item Penser à justifier que $f(P)\in\R_{n}[X]$.
	\item 
	\begin{enumerate}
		\item Intégration par parties.
		\item \ldots
	\end{enumerate}
	\item  
	\begin{enumerate}
		\item Calcul. 
		\item Utiliser Partie 1 : 2)
		\item \ldots
		\item Que sais-t-on des $H_k$ ?
	\end{enumerate}
\end{enumerate}

\section*{Problème 2}

\subsection*{Partie 1}

\begin{enumerate}
\item Intégration par parties.
\item
\begin{enumerate}
	\item Utiliser des équivalents\ldots
	\item Pour $x\leq y$, comparer $H(x)$ et $H(y)$. 
	\item Dérivation sous le signe intégrale.
	\item Montrer que $g:x\longmapsto\frac{t\ln(t)}{t-1}$ est bornée sur $]0;1[$.
	\item Simplifier par $t-1$.
	\item $\lim_{x\to-1}H(x+1)$ ?
	\item 
\begin{enumerate}
		\item Équivalent.
		\item Grâce à (e) : récurrence, ou alors utiliser un télescopage.
		\item Passage à la limite.
		\item $\sum_{k=1}^{+\infty}\frac{1}{k^2}$ donné au début de l'énoncé.
\end{enumerate}
\end{enumerate}
\end{enumerate}

\subsection*{Partie 2}

\begin{enumerate}
\item \ldots 
\item Comparaison série intégrale.
\item 
\begin{enumerate}
	\item Équivalent pour l'un, critère spécial pour l'autre.
	\item \ldots
	\item Théorème de convergence dominée.
	\item Poser $v=\sqrt{u}$.
\end{enumerate}
\end{enumerate}

\subsection*{Partie 3}

\begin{enumerate}
\item
\begin{enumerate}
	\item Intégration par parties.
	\item Itérer la relation (récurrence sur $q$).
\end{enumerate}
\item 
\begin{enumerate}
	\item \ldots
	\item Indication du sujet avec intégration terme à terme.
	\item Utiliser 1)(b).
\end{enumerate}
\item Revenir à la définition de $H(x)$ et remplacer $t^x$ à l'aide la série 
exponentielle.
\item Si $x=-1$ ?
\end{enumerate}

\end{document}