\documentclass[twoside,french,11pt]{VcCours}
%cspell:ignore orthonormaliser
\newcommand{\dx}{\text{d}x}
\newcommand{\dt}{\text{d}t}

\begin{document}

\Titre{PSI}{Promotion 2021--2022}{Mathématiques}{Devoir surveillé n°5}

\begin{center}
\large 
Le samedi 5 mars 2022

\bigskip
\textbf{Durée: 3h30}

\bigskip
\large\underline{\textbf{Calculatrice interdite}}
\end{center}

\bigskip
\begin{itemize}
  \item Le candidat attachera la plus grande importance à la clarté, à la précision et à la concision de la rédaction. Si un candidat est amené à repérer ce qui peut lui sembler être une erreur d'énoncé, il le signalera sur sa copie et devra poursuivre sa composition en expliquant les raisons des initiatives qu'il a été amené à prendre.
  \item Il est conseillé au candidat de lire l'intégralité du sujet et de repérer les parties qui lui semblent plus abordables.
  \item Le candidat \fbox{encadrera} ou \underline{soulignera} les résultats.
  \item Si un étudiant a traité tout ce qu'il savait faire, il peut rendre sa copie et recevoir en contrepartie une feuille
  contenant des indications. Il composera alors sur une nouvelle copie les questions qu'il souhaite.
  \end{itemize}
\separationTitre

\section*{Problème 1}

Dans ce problème, $\R[X]$ désigne l'espace vectoriel des polynômes à coefficients réels et $\R_n[X]$ celui des polynômes à coefficients réels de degré inférieur ou égal à $n$ (entier naturel non nul).

\medskip
On rappelle aussi la valeur de l'intégrale de Gauss :
$$\int_{-\infty}^{+\infty}e^{-x^2} \dx=\sqrt{\pi}=\pi^{1/2}$$
Les parties 1 et 2 sont indépendantes. Les parties 3 et 4 utilisent les résultats précédents.

\subsection*{Partie 1 : polynômes d'Hermite}

On note $w$ l'application de $\R$ dans $\R$ définie par :
$\ \forall x\in\R,\quad w(x)=e^{-x^2}.$

Pour tout $n\in\N,$ on note alors $H_n$ l'application  de $\R$ dans $\R$ définie par :
$$\forall x\in\R,\qquad H_n(x)=(-1)^nw^{(n)}(x)e^{x^2}$$
où $w^{(n)}$ désigne la dérivée $n$-ième de $w$. En particulier, $H_0(x)=1.$

\begin{enumerate}
	\item Calculer pour tout $x\in\R,$ $H_1(x), H_2(x)$ et $H_3(x).$
	\item Montrer que pour tout $n\in\N$ et pour tout $x\in\R$, on a :
	$$H_{n+1}(x)=2xH_n(x)-H'_n(x).$$
	%En déduire l'expression de $H_4(x).$
	\item En déduire que pour tout $n\in\N$, $H_n$ est un polynôme et déterminer son degré et son terme dominant.
	\item Démontrer que pour tout $n\in\N$ et tout $x\in\R$, on a $H_n(-x)=(-1)^nH_n(x).$\\
	En déduire, en fonction de $n$, la parité de $H_n.$
	
\end{enumerate}

\subsection*{Partie 2 : un produit scalaire sur $\R[X]$}

Pour tous $P,Q\in\R[X],$ on pose :
$$\langle P|Q\rangle=\int_{-\infty}^{+\infty}P(t)Q(t)e^{-t^2}\dt$$
Pour $P\in\R[X],$ on note $\|P\|=\sqrt{\langle P|P\rangle }.$
\begin{enumerate}
	\item Justifier l'existence de $\langle P|Q\rangle $ pour tous $P,Q\in\R[X].$
	\item Démontrer que $(P,Q)\longmapsto \langle P|Q\rangle $ définit un produit scalaire sur $\R[X].$
	\item
	\begin{enumerate}
		\item Que valent $\|1\|$ et $\langle X| 1\rangle $ ?
		\item Démontrer que pour tout $n\in\N,$ $\int_{-\infty}^{+\infty}t^{2n+1}e^{-t^2}\dt=0.$
		\item Pour $n\in\N,$ on pose $I_n=\int_{-\infty}^{+\infty}t^{2n}e^{-t^2}\dt.$

		Démontrer que pour tout $n\in\N,$ $I_{n+1}=\frac{2n+1}{2}I_n.$
		
		En déduire l'expression de $I_n$ à l'aide de factorielles.
\end{enumerate}
	\item On note $F$ le sous-espace vectoriel des polynômes pairs, et $G$ celui des polynômes impairs.

	Démontrer que $F\perp G.$
	\item Orthonormaliser la famille $(1,X,X^2)$ par le procédé de Gram-Schmidt.
	
	On notera $(P_0, P_1, P_2)$ la famille obtenue.
	\item Déterminer $\inf_{(a,b)\in\R^2}\int_{-\infty}^{+\infty}(t^2-a-bt)^2e^{-t^2} \dt.$
\end{enumerate}
 
\subsection*{Partie 3 : construction d'une base orthonormée de $\R_n[X].$}

Soit $n \in \mathbb{N}^*$. Dans cette partie, $\R_n[X]$ est muni du produit scalaire introduit dans la partie 2. On travaille avec les polynômes de Hermite $H_k$ défini dans la partie $1.$ 

\begin{enumerate}
  \item Démontrer que pour tout $P\in\R[X]$ et pour tout $k \in\N^*,$ on a :
	$$\langle P'|H_{k-1}\rangle  \ = \ \langle P|H_k\rangle  $$
	\item En déduire que pour tout $P\in\R_{n-1}[X],$ on a 
	$\langle P|H_n\rangle =0.$
	\item Montrer alors que la famille $(H_0,H_1,\dots,H_n)$ est orthogonale.
	\item En déduire que $(H_0,H_1,\dots,H_n)$ est une base de $\R_n[X].$
	\item Soit $k\in\N$.
	\begin{enumerate}
		\item Montrer que $\| H_k\|^2=\langle H_k^{(k)}| H_0\rangle .$
		\item En déduire la valeur de $\|H_k\|$ et donner une base orthonormée de $\R_n[X].$
	\end{enumerate}
\end{enumerate}

\subsection*{Partie 4 : étude d'un endomorphisme}

Soit $n\in\N^*.$ On considère l'application $f$ définie par :
$$\forall P\in\R_n[X],\quad f(P)=-P''+2XP'+P.$$

\begin{enumerate}
	\item Montrer que $f$ est un endomorphisme de $\R_n[X].$
	\item 
	\begin{enumerate}
	\item Démontrer que pour tous $P,Q\in\R_n[X]$, on a $\langle P'|Q'\rangle =\langle f(P)|Q\rangle -\langle P|Q\rangle .$
	\item En déduire que pour tous $P,Q\in\R_n[X]$, $\langle f(P)|Q\rangle  = \langle P|f(Q)\rangle $.
	\end{enumerate}
	\item 
	\begin{enumerate}
	\item Calculer $f(H_0)$ et $f(H_1).$
	\item Soit $k\in\{0,\dots, n\}$. Démontrer que $H'_{k+1}=H_k+f(H_k).$
	\item Démontrer par récurrence que pour tout $k\in\{0,\dots,n\},$ on a $f(H_k)=(2k+1)H_k.$
	\item Déterminer enfin une base orthonormée de $\R_n[X]$ formée de vecteurs propres de $f.$
	\end{enumerate}
\end{enumerate}

\section*{Problème 2}

On admet dans ce problème que : $\sum_{k=1}^{+ \infty} \dfrac{1}{k^2} = \dfrac{\pi^2}{6} \cdot$

\subsection*{Partie 1}
Soit $x\in \R$. On note, lorsque cela a un sens, $H(x)=\int_0^1\frac{t^x\ln(t)}{t-1}\dt$.
\begin{enumerate}
\item Démontrer que pour $s>-1$, l'intégrale $J_s=\int_0^1t^s\ln(t)\dt$ existe et déterminer sa valeur.
\item \underline{Étude de la fonction H}
\begin{enumerate}
\item Montrer que l'ensemble de définition de la fonction $H$ est $D_H=]-1,+\infty[$.
\item Montrer que $H$ est monotone sur $D_H$.
%\item Montrer que pour tout réel $\alpha >0$, la fonction $t\longmapsto\frac{t^\alpha(\ln(t))^2}{1-t}$ est prolongeable en une fonction bornée sur le segment $[0,1]$.
\item Démontrer que $H$ est de classe ${\cal C}^1$ sur $D_H$. Retrouver alors la monotonie de $H$.
\item À l'aide d'une majoration judicieuse, démontrer que $\lim_{x\to +\infty}H(x)=0$.
\item Démontrer que :
\[\forall x>-1,\quad H(x)-H(x+1)=\frac{1}{(x+1)^2}\]
\item Déterminer alors un équivalent simple de $H(x)$ lorsque $x$ tend vers $-1$ par valeurs supérieures.
\item Soit $x>-1$.
\begin{enumerate}
\item Justifier la convergence de la série $\sum_{k\geq 1}\frac{1}{(x+k)^2} \cdot$
\item Prouver que pour tout $n\in \N^*$, 
\[H(x)=\sum_{k=1}^n\frac{1}{(x+k)^2}+H(x+n)\]
\item En déduire que :
\[H(x)=\sum_{k=1}^{+ \infty} \frac{1}{(x+k)^2}\]
\item Calculer $H(0)$ et $H(1)$.
\end{enumerate}
\end{enumerate}
\end{enumerate}

\subsection*{Partie 2}

\begin{enumerate}
\item Prouver que pour tout $x>-1$ et tout entier naturel $k$ non nul,
\[\frac{1}{(x+k+1)^2}\leq \int_k^{k+1}\frac{dt}{(x+t)^2}\leq \frac{1}{(x+k)^2}\]
\item Déterminer un équivalent de $H(x)$ lorsque $x$ tend vers $+\infty$.
\item Pour tout entier naturel $n$, on pose $u_n=H(n)$.
\begin{enumerate}
\item Étudier la convergence des séries $\sum_{n\geq 0}u_n$ et $\sum_{n\geq 0}(-1)^nu_n$.
\item Pour $n\in\N$ et pour $t\in]0,1[,$ on pose $f_n(t)=(-1)^n\frac{t^{n}\ln(t)}{t-1} \cdot$\\
Montrer que $\sum f_n$ converge simplement sur $]0,1[$ et calculer sa somme $S$.
\item Démontrer alors que
$\sum_{n=0}^{+\infty}(-1)^nu_n=\int_0^1\frac{\ln(v)}{v^2-1}\;dv.$
\item Donner la valeur de cette intégrale en fonction de $H\left (-\frac{1}{2}\right )$.
\end{enumerate}
\end{enumerate}

\subsection*{Partie 3}

Pour tout entier naturel $k\geq 2$, on note :
\[Z_k=\sum_{p=1}^{+\infty}\frac{1}{p^k}\]
\begin{enumerate}
\item Pour tout couple d'entiers naturels $(p,q)$, on pose $I_{p,q}=\int_0^1t^p(\ln(t))^q\dt$ et on admettra que cette intégrale existe.
\begin{enumerate}
\item Démontrer que si $q\geq 1$, $I_{p,q}=-\frac{q}{p+1}I_{p,q-1}$.
\item En déduire la valeur de $I_{p,q}$.
\end{enumerate}
\item
\begin{enumerate}
\item Justifier l'existence pour tout $n\in \N$ de $B_n=\int_0^1 \frac{(\ln(t))^{n+1}}{t-1}\dt$.
\item Exprimer $B_n$ à l'aide des intégrales $I_{p,q}$.

\indication{On pourra utiliser la série de terme général $t^p$.}
\item Prouver enfin que pour tout $n\in \N,\ B_n=(-1)^n(n+1)!Z_{n+2}$.
\end{enumerate}
\item En déduire alors que :
\[\forall x\in ]-1,1[,\ H(x)=\sum_{k=0}^{+\infty}(-1)^k(k+1)Z_{k+2}x^k\]
\item Quel est le rayon de convergence de la série entière dont $H$ est la somme sur $]-1,1[$?
\end{enumerate}

\end{document}