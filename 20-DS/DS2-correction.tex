\documentclass[a4paper,twoside,french,10pt]{VcCours}

\begin{document}
\Titre{PSI}{Promotion 2021--2022}{Mathématiques}{Devoir surveillé n\degres2}

\begin{center}
\large\bf
Correction
\end{center}
\separationTitre


\section*{Exercice 1 -- Séries}
\begin{enumerate}
    \item La série étudiée est à termes positifs. On a :
    $$ \dfrac{\sqrt{n}}{n^7+\sqrt{n}} \underset{+ \infty}{\sim} \dfrac{\sqrt{n}}{n^7} = \dfrac{1}{n^{6.5}}$$
    La série de terme général positif $1/n^{6.5}$ converge ($6.5>1$). Par critère de comparaison des séries à termes positifs, on en déduit que la série étudiée converge.
    \item Pour tout entier $n \geq 1$,
    $$ n \times \dfrac{\ln(n)^{12}}{\sqrt{n}+1}\underset{+ \infty}{\sim} \sqrt{n} \ln(n)^{12} \underset{n\rightarrow + \infty}{\longrightarrow} + \infty$$
    donc :
    $$ \lim_{n \rightarrow + \infty} n \times \dfrac{\ln(n)^{12}}{\sqrt{n}+1} = + \infty$$
    Ainsi, à partir d'un certain rang,
    $$ n \times \dfrac{\ln(n)^{12}}{\sqrt{n}+1} \geq 1$$
    et ainsi :
    $$ \dfrac{\ln(n)^{12}}{\sqrt{n}+1} \geq  \dfrac{1}{n} \geq 0$$
    La série harmonique diverge donc par critère de comparaison des séries à termes positifs, la série étudiée diverge.
    \item Pour tout entier $n \geq 1$,
    $$ u_n = \dfrac{3^n n!}{n^n} >0$$
    et on a :
    \begin{align*}
    \dfrac{u_{n+1}}{u_n} & = \dfrac{3^{n+1} (n+1)!}{(n+1)^{n+1}} \times \dfrac{n^n}{3^n n!} \\
    & = \dfrac{3(n+1)n^n}{(n+1)^{n+1}} \\
    & = 3  \left( \dfrac{n}{n+1} \right)^n \\
    & = 3 \left( \dfrac{n+1}{n} \right)^{-n}\\
    & = 3 \left( 1+ \dfrac{1}{n} \right)^{-n}\\
    & = \dfrac{3}{\left( 1+ \dfrac{1}{n} \right)^{n}}
    \end{align*}
    On a déjà montré de nombreuses fois que :
    $$ \lim_{n \rightarrow + \infty} \left( 1+ \dfrac{1}{n} \right)^{n} = e$$
    Ainsi,
    $$ \lim_{n \rightarrow + \infty} \dfrac{u_{n+1}}{u_n} = \dfrac{3}{e}>1$$
    D'après le critère de D'Alembert, on en déduit que la série étudiée diverge grossièrement.
    \item La série étudiée est à termes positifs. On sait que :
    $$ \ch(x) -1 \underset{0}{\sim} \dfrac{x^2}{2}$$
    Si $n$ tend vers $+ \infty$, $1/n$ tend vers $0$ donc :
    $$ \ch(1/n)-1 \underset{+ \infty}{\sim} \dfrac{1}{2n^2}$$
    puis :
    $$ \sqrt{\ch(1/n)-1} \underset{+ \infty}{\sim} \dfrac{1}{\sqrt{2}} \times \dfrac{1}{n}$$
    La série harmonique (qui est à termes positifs) diverge donc par critère de comparaison des séries à termes positifs, la série étudiée diverge.
    %\item On sait que :
    %$$ \exp(x) \underset{0}{=} 1 + x + \dfrac{x^2}{2} + o(x^2)$$
    %Si $n$ tend vers $+ \infty$, $1/n$ tend vers $0$ donc :
    %\begin{align*}
    % \exp(1/n)-a- \dfrac{b}{n} & \underset{+ \infty}{=} 1 + \dfrac{1}{n} + \dfrac{1}{2n^2} - a - \dfrac{b}{n} + o \left( \dfrac{1}{n^2} \right) \\
    % & = (1-a) + \dfrac{1-b}{n}  + \dfrac{1}{2n^2} + o \left( \dfrac{1}{n^2} \right) 
    %\end{align*}
    %Distinguons plusieurs cas.
    %\begin{itemize}
    %\item Si $a \neq 1$ alors le terme général converge vers une constante non nulle donc la série diverge grossièrement.
    %\item Si $a=1$ et $b \neq 1$ alors :
    %$$ \exp(1/n)-1- \dfrac{b}{n} = \dfrac{1-b}{n}  + \dfrac{1}{2n^2} + o \left( \dfrac{1}{n^2} \right) $$
    %donc :
    %$$ \exp(1/n)-1- \dfrac{b}{n} \underset{+ \infty}{\sim} \dfrac{1-b}{n}$$
    %La série de terme général $(1-b)/n$ est de signe constant et diverge (série harmonique). A partir d'un certain rang, le terme général de la série étudiée a le même signe. Par critère de comparaison, on en déduit que la série étudiée diverge.
    %\item Si $a=1$ et $b=1$, on a :
    %$$  \exp(1/n)-1- \dfrac{1}{n}  \underset{+ \infty}{=}  \dfrac{1}{2n^2} + o \left( \dfrac{1}{n^2} \right) $$
    %donc :
    %$$  \exp(1/n)-1- \dfrac{1}{n} \underset{+ \infty}{\sim}  \dfrac{1}{2n^2}$$
    %La série de terme général $1/(2n^2)$ est à termes positifs et converge (série harmonique). A partir d'un certain rang, le terme général de la série étudiée est aussi positif (c'est en fait clair car pour tout réel $x$, $\exp(x) \geq 1+x$). Par critère de comparaison des séries à termes positifs, on en déduit que la série étudiée converge. Finalement, la série étudiée converge si et seulement si $a$ et $b$ sont égaux à $1$.
    %\end{itemize}
    \item On sait que :
    $$ \tan(x) \underset{0}{=} x + \dfrac{x^3}{3} + o(x^3)$$
    Par produit d'une suite bornée et d'une suite convergeant vers $0$, on a :
    $$ \lim_{n \rightarrow + \infty} \dfrac{(-1)^n}{n}=0$$
    donc :
    \begin{align*}
    \tan \left( \dfrac{(-1)^n}{n} \right) & \underset{+ \infty}{=} \dfrac{(-1)^n}{n} + \dfrac{1}{3}\left(\dfrac{(-1)^n}{n}\right)^3 + o \left( \dfrac{1}{n^3} \right) \\
    &\underset{+ \infty}{=}  \dfrac{(-1)^n}{n} + \dfrac{(-1)^n}{3n^3} + o \left( \dfrac{1}{n^3} \right) 
    \end{align*}
    Les suites $(1/n)_{n \geq 1}$ et $(1/(3n^3))_{n \geq 1}$ sont décroissantes, positives et convergent vers $0$. D'après le critère spécial des séries alternées, on en déduit que les séries :
    $$ \Sum{n \geq 1}{} \dfrac{(-1)^n}{n} \; \hbox{ et } \;  \Sum{n \geq 1}{} \dfrac{(-1)^n}{3n^3}$$
    convergent. De plus, la série de terme général $1/n^3$ converge ($3>1$) donc par critère de comparaison, la série de terme général $o \left( \dfrac{1}{n^3} \right)$ converge absolument donc converge. Ainsi, par somme, la série étudiée converge.
    \end{enumerate}

\section*{Exercice 2 -- Espaces vectoriels}
\begin{enumerate}
    \item 
    \item 
\end{enumerate}

  
\section*{Exercice 3 -- Espaces vectoriels}


\subsection*{Partie 1 : un ensemble de matrices}
\begin{enumerate}
    \item De manière évidente, on a :
    $$ F = \Vect(M,N,P)$$
    où
    $$ M = \begin{pmatrix}
    1 & 0 & 0 \\
    0 & 0 & 0 \\
    0 & 0 & 0 \\
    \end{pmatrix}, \, N= \begin{pmatrix}
    0 & 0 & 0 \\
    0 & 1 & 0 \\
    0 & 0 & 1 \\
    \end{pmatrix} \, \hbox{ et } \, P = \begin{pmatrix}
    0 & 0 & 0 \\
    0 & 0 & 1 \\
    0 & -1 & 0 \\
    \end{pmatrix}$$
    Ainsi,
    \enc{$F$ est un espace vectoriel}
    La famille $(M,N,P)$ est libre car pour tout $(a,b,c) \in \mathbb{R}^3$, si 
    $$ aM+bN+cP = 0_3$$
    alors 
    $$ \begin{pmatrix}
    a & 0 & 0 \\
    0 & b & c \\
    0 & -c & b \\
    \end{pmatrix} = 0_3$$
    donc $a=b=c=0$. On en déduit que :
    \enc{$(M,N,P)$ est une base de $F$}
    \item Soit $A$ et $B$ deux matrices de $F$. Alors il existe des réels $a,b,c,a',b',c'$ tels que :
    $$ A= \begin{pmatrix}
    a & 0 & 0 \\
    0 & b & c \\
    0 & -c & b \\
    \end{pmatrix} \, \hbox{ et } \, B=\begin{pmatrix}
    a' & 0 & 0 \\
    0 & b' & c'\\
    0 & -c' & b' \\
    \end{pmatrix}$$
    Par simple calcul, on a :
    $$ AB =\begin{pmatrix}
    aa' & 0 & 0 \\
    0 & bb'-cc' & bc'+cb' \\
    0 & -b'c-bc' & -cc'+bb' \\
    \end{pmatrix} = \begin{pmatrix}
    a_1 & 0 & 0 \\
    0 & b_1& c_1 \\
    0 & -c_1 & b_1 \\
    \end{pmatrix}$$
    où $a_1=aa' \in \mathbb{R}$, $b_1=bb'-cc' \in \mathbb{R}$ et $c_1 = bc'+cb' \in \mathbb{R}$. On en déduit que $AB$ appartient à $F$ donc :
    \enc{$F$ est stable par produit}
    \item Remarquons que $I_3$ appartient à $F$ (on considère $a=1$, $b=1$ et $c=0$), $M$ appartient à $F$ (on considère $a=-2$, $b=1$ et $c=1$). La matrice $M^2$ appartient à $F$ comme produit de deux matrices de $F$. Par simple calcul, on a :
    $$ M^2 = \begin{pmatrix}
    4 & 0 & 0 \\
    0 & 0 & 2 \\
    0 & -2 & 0 
    \end{pmatrix}$$
    Soit $(a,b,c) \in \mathbb{R}^3$ tel que :
    $$ a I_3 + bM + cM^3 = 0_3$$
    Alors :
    $$ \begin{pmatrix}
    a & 0 & 0 \\
    0 & b &b+2c \\
    0 &-b-2c &b \\
    \end{pmatrix} = 0_3$$
    donc $a=b=0$ puis on obtient $c=0$. Ainsi, $(I_3,M,M^2)$ est une famille libre de $F$ de cardinal $3$ qui est la dimension de cet espace (d'après la question $1$) donc 
    \enc{$(I_3,M,M^2)$ est une base de $F$}
    \item Par simple calcul, on a :
    $$ M^3 = \begin{pmatrix}
    -8 & 0 & 0 \\
    0 & -2 & 2 \\
    0 & -2 & -2 \\
    \end{pmatrix}$$
    donc 
    $$ \boxed{M^3 = 2M -4I_3}$$
    \end{enumerate}

\subsection*{Partie 2 : étude d'une suite}
\begin{enumerate}
    \item Voici une proposition :
    \begin{verbatim}
    def Suite(n):
    u=3
    v=0
    w=4
    for i in range(2,n+1):
        u,v,w=v,w,2*v-4*u
    if n==0:
        return 3
    if n==1:
        return 0
    return w
    \end{verbatim}
    \item Posons pour tout entier $k \geq 0$ la propriété $\mathcal{P}_k$ définie par :
    
    \begin{center}
    \og $\exists (a_k,b_k,c_k) \in \mathbb{R}^3 \, \vert \,  M^k = \begin{pmatrix}
    a_k & 0 & 0 \\
    0 & b_k & c_k \\
    0 & -c_k & b_k \\
    \end{pmatrix}$ \fg
    \end{center}
    \begin{itemize}
    \item \textit{Initialisation.} On a :
    $$ M^0 = I_3$$
    Posons $a_0=1$, $b_0=1$ et $c_0=0$. Alors on a :
    $$ M^0 = \begin{pmatrix}
    a_0 & 0 & 0 \\
    0 & b_0 & c_0 \\
    0 & -c_0 & b_0 \\
    \end{pmatrix}$$
    La propriété est vérifiée au rang $0$.
    \item Soit $k \in \mathbb{N}$ tel que la propriété soit vérifiée. Ainsi :
    $$ \exists (a_k,b_k,c_k) \in \mathbb{R}^3 \, \vert \,  M^k = \begin{pmatrix}
    a_k & 0 & 0 \\
    0 & b_k & c_k \\
    0 & -c_k & b_k \\
    \end{pmatrix}$$
    On a $M^{k+1}= M \times M^k$ donc par simple calcul, on a :
    $$ M^{k+1} = \begin{pmatrix}
    -2a_k & 0 & 0 \\
    0 & b_k-c_k & c_k+b_k \\
    & -c_k -b_k & b_k-c_k
    \end{pmatrix}$$
    La propriété est donc vérifiée au rang $k+1$ en posant :
    $$ \boxed{a_{k+1}= -2 a_k, \; b_{k+1} = b_k - c_k \, \hbox{ et } \, c_{k+1}= b_k+c_k}$$
    \item Par principe de récurrence, on en déduit que la propriété est vraie pour tout entier $k \geq 0$.
    \end{itemize}
    Ainsi, pour tout entier $k \geq 0$, il existe des réels $a_k$, $b_k$ et $c_k$ tels que :
    $$ \boxed{M^k = \begin{pmatrix}
    a_k & 0 & 0 \\
    0 & b_k & c_k \\
    0 & -c_k & b_k \\
    \end{pmatrix}}$$
    \item D'après la question précédente, on sait que $(a_k)_{k \geq 0}$ est une suite géométrique de raison $-2$ et $a_0=1$ donc :
    $$ \boxed{\forall k \geq 0, \; a_k = (-2)^k}$$
    \item Soit $k \in \mathbb{N}$. Alors :
    \begin{align*}
    z_{k+1} & = b_{k+1} + i c_{k+1} \\
    & = b_k - c_k + i (b_k+c_k) \\
    & = b_k (1+i) + c_k (-1+i) \\
    & = b_k (1+i) + c_k (i^2+i) \\
    & = (1+i)(b_k+i c_k) \\
    & = (1+i) z_k
    \end{align*}
    Ainsi, $(z_k)$ est une suite géométrique de raison $1+i$ et on a :
    $$ z_0 = b_0+i c_0 = 1$$
    Ainsi, pour tout entier $k \geq 0$,
    $$ \boxed{z_k = (1+i)^k}$$
    On sait que pour tout entier $k \geq 0$, $b_k$ et $c_k$ sont des réels donc :
    $$ \boxed{b_k = \Re e((1+i)^k)}$$
    Remarquons maintenant que pour tout entier $k \geq 0$,
    $$ (1+i)^k = \left( \sqrt{2} e^{i \pi/4} \right)^k = \left(\sqrt{2}\right)^k (\cos(k \pi/4)+i \sin(k \pi/4))$$
    Ainsi,
    $$ \boxed{b_k = \left(\sqrt{2}\right)^k \cos(k \pi/4)}$$
    % \item On sait que $u_0=3$, $u_1=0$ et $u_2=4$. On sait aussi que :
    % $$ M^0=I_3, \, M=\begin{pmatrix}
    % -2 & 0 & 0 \\
    % 0 & 1 & 1 \\
    % 0 & -1 & 1 \\
    % \end{pmatrix} \, \hbox{ et } M^2 = \begin{pmatrix}
    % 4 & 0 & 0 \\
    % 0 & 0 & 2 \\
    % 0 & -2 & 0 
    % \end{pmatrix}$$
    % Ainsi, $\textrm{Tr}(M^0)=3$, $\textrm{Tr}(M^1)=0$ et $\textrm{Tr}(M^2)= 4$. Pour montrer que $(u_n)_{n \geq 0}$ et $(\textrm{Tr}(M^n))$ sont égales, il suffit donc de montrer qu'elles vérifient la même relation de récurrence d'ordre $3$ (car leurs $3$ premiers termes sont égaux). Pour tout entier $n \geq 0$, on a :
    % \begin{align*}
    % \textrm{Tr}(M^{n+3}) & = \textrm{Tr}(M^n \times M^3) \\
    % & =  \textrm{Tr}(M^n \times(2M -4I_3)) \\
    % & =  \textrm{Tr}(2M^{n+1} - 4M^n) \\
    % & = 2  \textrm{Tr}(M^{n+1}) - 4 \textrm{Tr}(M^n)
    % \end{align*}
    % par linéarité de la trace. Or on sait aussi que pour tout entier $n \geq 0$,
    % $$ u_{n+3} = 2u_{n+1}-4u_n$$
    % Finalement, on en déduit bien que pour tout entier $n \geq 0$,
    % $$ \boxed{u_n = \textrm{Tr}(M^n)}$$
    % Or pour tout entier $n \geq 0$,
    % $$ \textrm{Tr}(M^n) = a_n + 2b_n$$
    % On en déduit donc que :
    % $$ \boxed{u_n = (-2)^n + 2  (\sqrt{2})^n \cos(n\pi/4)}$$
    
    
    \end{enumerate}
    


  
\section*{Problème -- séries dont le terme général est un reste}
\begin{enumerate}

    \item 
    
    \begin{enumerate}
        \item D'après le cours, 
        \enc{$\Sum{n \geq 0}{} a_n$ converge si et seulement $\vert \lambda \vert <1$}
        et on a en cas de convergence :
        \enc{$\Sum{n = 0}{+ \infty} \lambda^n= \dfrac{1}{1- \lambda}$}
        \item On sait (voir le cours pour la preuve) que pour tout $n \geq 0$,
        $$ r_n = \dfrac{\lambda^{n+1}}{1-\lambda}$$
    La série de terme général $\lambda^{n+1}$ converge car $\vert \lambda \vert <1$ donc 
    \enc{$\Sum{n \geq 0}{} r_n$ converge}
    et on a :
    \begin{align*}
    \sum_{k=0}^{+ \infty} \lambda^{k+1} & = \lambda \sum_{k=0}^{+ \infty} \lambda^k \\
    & = \dfrac{\lambda}{1-\lambda}
    \end{align*}
    On en déduit que :
    \enc{$\dis \sum_{k=0}^{+ \infty} r_k = \dfrac{\lambda}{(1- \lambda)^2}$}
    \end{enumerate}
    
    
    \item 
    \begin{enumerate}
    
    \item Soit $k \in \mathbb{N}^*$. La fonction $x \mapsto \dfrac{1}{x^2}$ est décroissante sur $[k,k+1]$ donc pour tout réel $t \in [k,k+1]$,
    $$ \dfrac{1}{(k+1)^2} \leq \dfrac{1}{t^2} \leq \dfrac{1}{k^2}$$
    Par croissance de l'intégrale (les bornes sont dans le bon sens), on en déduit que :
    $$  \int_{k}^{k+1} \dfrac{1}{(k+1)^2} \dt \leq \int_{k}^{k+1} \dfrac{1}{t^2} \dt  \leq \int_{k}^{k+1} \dfrac{1}{k^2} \dt$$
    et ainsi :
    $$ \frac{1}{(k+1)^2}\leq\dis\int_{k}^{k+1}\dis\frac{dt}{t^2}\leq
    \dis\frac{1}{k^2}$$
    ce qui donne en sommant pour $k$ variant de $n+1$ à $N$ et d'après la relation de Chasles :
    $$ \boxed{\Sum{k=n+1}{N}  \dfrac{1}{(k+1)^2}\leq\dis\int_{n+1}^{N+1}\dis\dfrac{dt}{t^2}\leq
    \Sum{k=n+1}{N} \dfrac{1}{k^2}}$$
    
    \item Soit $n\in\N^*$. D'après la question précédente, on a pour  tout entier $N\geq n+1$, 
    $$\dis\sum_{k=n+1}^{N}\dis\frac{1}{(k+1)^2}\leq
    \dis\int_{n+1}^{N+1}\dis\frac{1}{t^2}dt\leq
    \dis\sum_{k=n+1}^{N}\dis\frac{1}{k^2}$$
    donc :
    $$\dis\sum_{k=n+1}^{N}\dis\frac{1}{(k+1)^2}\leq
    - \dfrac{1}{N+1} + \dfrac{1}{n+1}    \leq
    \dis\sum_{k=n+1}^{N}\dis\frac{1}{k^2}$$
    Par passage à la limite dans l'égalité précédente quand $N$ tend vers $+ \infty$ (la série de terme général $1/k^2$ est convergente), on en déduit que :
    $$ \dis\sum_{k=n+1}^{+ \infty}\dis\frac{1}{(k+1)^2}\leq
    \dfrac{1}{n+1}    \leq
    \dis\sum_{k=n+1}^{+ \infty}\dis\frac{1}{k^2}$$
    Ainsi, la deuxième inégalité implique que :
    $$ \dfrac{1}{n+1} \leq r_n$$
    La première inégalité implique que :
    $$   \dis\sum_{k=n+2}^{+ \infty}\dis\frac{1}{k^2}      \leq
    \dfrac{1}{n+1}$$
    ou encore :
    $$ r_{n}- \dfrac{1}{(n+1)^2} \leq \dfrac{1}{n+1}$$
    On en déduit que que pour tout entier $n\in\N^*,$ on a :
    $$ \boxed{\frac{1}{n+1}\leq r_n \leq
    \dis\frac{1}{n+1}+\dis\frac{1}{(n+1)^2}}$$
    \item D'après la question précédente, on a pour tout entier $n \geq 1$,
    $$ 1 \leq (n+1) r_n \leq 1+ \dfrac{1}{n+1} = 1$$
    On a :
    $$ \lim_{n \rightarrow + \infty} 1 =  \lim_{n \rightarrow + \infty}1+ \dfrac{1}{n+1}=1$$
    D'après le théorème d'encadrement, on en déduit que $((n+1)r_n)_{n \geq 1}$ converge et on a :
    $$ \lim_{n \rightarrow + \infty} (n+1)r_n = 1$$
    Ainsi,
    \enc{$r_n \underset{+ \infty}{\sim} \dfrac{1}{n+1}$}
    La série de terme général $1/(n+1)$ est divergente (série harmonique). Les séries étudiées sont à termes positifs. Par critère de comparaison des séries à termes positifs, on en déduit que :
    \enc{$\Sum{n \geq 1}{} r_n$ diverge}
    
    
    \end{enumerate}
    
    \item 
    
    
    
    \begin{enumerate}
        \item Montrons par récurrence que pour tout entier $n \geq 1$,
        $$ \sum_{k=1}^n ka_k=\dis\sum_{k=0}^{n-1} r_k-nr_n$$
    \begin{itemize}
    \item On a :
    \begin{align*}
    \sum_{k=0}^{1-1} r_k-1 \times r_1 & =  \sum_{k=0}^{0} r_k-r_1 \\
    & = r_0 - r_1 \\
    & = \sum_{k=1}^{+ \infty} a_k -\sum_{k=2}^{+ \infty} a_k \\
    & = a_1 \\
    & = 1 \times a_1
    \end{align*}
    La propriété est donc vraie au rang $1$.
    \item Soit $n \in \mathbb{N}^*$ tel que :
    $$ \sum_{k=1}^n ka_k=\dis\sum_{k=0}^{n-1} r_k-nr_n$$
    Alors :
    \begin{align*}
     \sum_{k=1}^{n+1} ka_k & =  \sum_{k=1}^n ka_k + (n+1)a_{n+1} \\
     & = \sum_{k=0}^{n-1} r_k-nr_n + (n+1)a_{n+1} \quad \hbox{(par hypothèse de récurrence)} \\
     & = \sum_{k=0}^{n-1} r_k-nr_n + (n+1)(r_n-r_{n+1}) \\
     & = \sum_{k=0}^{n-1} r_k + r_n -(n+1)r_{n+1} \\
     & = \sum_{k=0}^{n} r_k  -(n+1)r_{n+1} 
    \end{align*}
    La propriété est donc vraie au rang $n+1$.
    \item La propriété est vraie au rang $1$ et est héréditaire donc par principe de récurrence, elle est vraie pour tout entier $n \geq 1$.
    \end{itemize}
    Ainsi, pour tout $n\in\N^*,$ on a :
        $$\boxed{\sum_{k=1}^n ka_k=\dis\sum_{k=0}^{n-1} r_k-nr_n}$$
    
    \medskip
    
    \noindent \textit{Remarque.} On peut s'en sortir sans récurrence en remarquant un télescopage  :
    $$ k a_k = k (r_{k-1}-r_k) = (k r_{k-1} - (k+1) r_k) + r_k $$
        \item Pour tout entier $n \geq 1$, $a_n \geq 1$ donc pour tout entier $n \geq 1$, $n r_n \geq 0$ et ainsi d'après la question précédente :
        $$ \sum_{k=1}^n ka_k=\dis\sum_{k=0}^{n-1} r_k-nr_n \leq \sum_{k=1}^n ka_k=\dis\sum_{k=0}^{n-1} r_k$$
    La série à termes positifs de terme général $r_k$ est convergente donc sa suite des sommes partielles est majorée. D'après la question précédente, la suite des sommes partielles de la série à termes positifs $ka_k$ est donc aussi majorée et donc converge. Ainsi,
        
    \enc{si $\Sum{n \geq 0}{} r_n$ converge alors $\Sum{n \geq 0}{} na_n$ converge aussi}
        
        \item Reprenons les notations de la question $2$. Supposons par l'absurde que $\Sum{n \geq 1}{} r_n$ converge. Alors $\Sum{n \geq 1}{} na_n$ converge aussi. Or pour tout entier $n \geq 1$,
        $$ n a_n = \dfrac{1}{n}$$
        et la série harmonique diverge : c'est absurde ! Ainsi,
        \enc{$\Sum{n \geq 1}{} r_n$ diverge}
    
    \end{enumerate}
    
    \item 
    
    \begin{enumerate}
    
    \item La série harmonique alternée converge d'après le critère spécial des séries alternées. Ainsi,
    \enc{$\Sum{n \geq 1}{} a_n$ converge}
    
    
    \item 
    
    \begin{enumerate}
    
    \item Pour tout réel $x \in [0,1]$,
    $$ 1+x \geq 1 >0$$
    donc par décroissance de la fonction inverse sur $\mathbb{R}_+^*$,
    $$ 0 \leq \dfrac{1}{1+x} \leq 1$$
    Pour tout entier $n \geq 0$, $x^n \geq 0$ donc :
    $$ 0 \leq \dfrac{x^n}{1+x} \leq x^n$$
    Par croissance de l'intégrale (les bornes sont dans le bon sens), on en déduit que :
    $$ 0 \leq \int_{0}^1 \dfrac{x^n}{1+x} \dx \leq \int_{0}^1 x^n \dx = \dfrac{1}{n+1}$$
    On sait que :
    $$ \lim_{n \rightarrow + \infty} 0 =  \lim_{n \rightarrow + \infty} \dfrac{1}{n+1}=0$$
    donc par théorème d'encadrement, on en déduit que :
    $$ \lim_{n \rightarrow + \infty} \int_{0}^1 \dfrac{x^n}{1+x} \dx = 0$$
    et par produit avec une suite bornée, 
    \enc{$\dis \lim_{n\to +\infty}I_n=0$}
    
    
    \item Soit $n\in\N^*$. Alors :
    \begin{align*}
    I_{n+1}- I_n & = (-1)^{n+1}  \int_{0}^1 \dfrac{x^{n+1}}{1+x} \dx - (-1)^{n}\int_{0}^1 \dfrac{x^{n+1}}{1+x} \dx \\
    & = (-1)^{n+1} \left(\int_{0}^1 \dfrac{x^{n+1}}{1+x} \dx + \int_{0}^1 \dfrac{x^{n}}{1+x} \dx \right) \\
    & = (-1)^{n+1} \int_{0}^1 \dfrac{x^{n+1}+x^n}{1+x} \dx \\
    &  = (-1)^{n+1} \int_{0}^1 \dfrac{x^n(x+1)}{1+x} \dx \\
    & = (-1)^{n+1} \int_{0}^1x^n \dx \\
    & = \dfrac{(-1)^{n+1}}{n+1}
    \end{align*}
    On en déduit que pour tout entier $n \geq 1$,
    $$ \sum_{k=1}^{n-1} I_{k+1}- I_k = \sum_{k=1}^{n-1} \dfrac{(-1)^{k+1}}{k+1}$$
    donc par télescopage et changement d'indice :
    $$ I_n- I_1 = \sum_{k=2}^{n} \dfrac{(-1)^{k}}{k}$$
    Or on a :
    \begin{align*}
     I_1 & = \sum_{k=2}^{n} \dfrac{(-1)^{k}}{k} \\
    & = - \int_{0}^1 \dfrac{x}{1+x} \dx  \\ 
    & = - \int_{0}^1 \dfrac{x+1-1}{1+x} \dx  \\ 
    & = - \int_{0}^1 1 - \dfrac{1}{1+x} \\
    & = - \left[x - \ln(1+x) \right]_0^1 \\
    & = - 1+ \ln(2)
    \end{align*}
    Ainsi,
    $$ I_n = \ln(2)-1 + \sum_{k=2}^{n} \dfrac{(-1)^{k}}{k}$$
    et donc 
    $$ \boxed{ I_n=\ln(2)+\dis\sum_{k=1}^n\dis\frac{(-1)^k}{k}}$$
    \item D'après la question précédente, on a pour tout entier $n \geq 1$,
    $$ \sum_{k=1}^n\dis\frac{(-1)^k}{k}  =  I_n - \ln(2)$$
    Sachant que $(I_n)_{n \geq 1}$ converge vers $0$, on en déduit que :
    
    $$\boxed{\dis\sum_{k=1}^{+\infty}\dis\frac{(-1)^k}{k} = - \ln(2)}$$
    Pour tout entier $n \geq 1$,
    \begin{align*}
     r_n & = \sum_{k=n+1}^{+ \infty} \frac{(-1)^k}{k}  \\
     & = \sum_{k=1}^{+ \infty} \frac{(-1)^k}{k}  - \sum_{k=1}^{n} \frac{(-1)^k}{k}  \\
     & = - \ln(2) - (I_n - \ln(2)) \\
     & = - I_n
    \end{align*}
    Ainsi,
    \enc{$r_n= - I_n$}
    
    
    
    \end{enumerate}
    
    \item 
    
    \begin{enumerate}
    
    \item Soit $n \geq 1$. On sait que :
    $$ I_n = (-1)^n \int_0^1 \dfrac{x^n}{1+x} \dx$$
    Les fonctions $x \mapsto \dfrac{x^{n+1}}{n+1}$ et $x \mapsto \dfrac{1}{1+x}$ sont de classe $\mathcal{C}^1$ sur $[0,1]$, de dérivées respectives $x \mapsto x^n$ et $x \mapsto - \dfrac{1}{(1+x)^2}$ donc par intégration par parties :
    \begin{align*}
    \int_0^1 \dfrac{x^n}{1+x} \dx & = \left[ \dfrac{x^{n+1}}{(n+1)(x+1)} \right]_0^1 + \dfrac{1}{n+1}\int_0^1  \dfrac{x^{n+1}}{(1+x)^2} \dx \\
    & =\dfrac{1}{2(n+1)} + \dfrac{1}{n+1}\int_0^1  \dfrac{x^{n+1}}{(1+x)^2} \dx
    \end{align*}
    et ainsi :
    $$ I_{n} = \dfrac{(-1)^n}{2(n+1)} + \dfrac{(-1)^n}{n+1}\int_0^1  \dfrac{x^{n+1}}{(1+x)^2} \dx$$
    D'après l'inégalité triangulaire (les bornes sont dans le bon sens), on sait que :
     $$ \left\vert   \dfrac{(-1)^n}{n+1} \int_0^1  \dfrac{x^{n+1}}{(1+x)^2} \dx \right\vert \leq \dfrac{1}{n+1} \int_0^1  \left\vert \dfrac{x^{n+1}}{(1+x)^2} \right\vert \dx = \dfrac{1}{n+1} \int_0^1  \dfrac{x^{n+1}}{(1+x)^2} \dx$$
    Or pour tout réel $x \in [0,1]$, 
    $$ (1+x)^2 \geq 1 > 0$$
    donc par décroissance de la fonction inverse sur $\mathbb{R}_+^*$ :
    $$ \dfrac{1}{(1+x)^2} \leq 1$$
    et par croissance de l'intégrale (les bornes sont dans le bon sens), on en déduit que : 
    $$  \dfrac{1}{n+1} \int_0^1  \dfrac{x^{n+1}}{(1+x)^2} \dx \leq  \dfrac{1}{n+1} \int_0^1  x^{n+1} \dx = \dfrac{1}{(n+1)(n+2)}$$
    Sachant que $((-1)^n)_{n \geq 1}$ est bornée et que :
    $$ \dfrac{1}{(n+1)(n+2)} \leq \dfrac{1}{n^2}$$
    On a :
    $$ \dfrac{(-1)^n}{n+1}\int_0^1  \dfrac{x^{n+1}}{(1+x)^2} \dx \underset{+ \infty}{=} O \left( \dfrac{1}{n^2} \right)$$
    et ainsi,
    $$\boxed{ I_n \underset{+ \infty}{=} \dis\frac{(-1)^n}{2(n+1)}+O\left(\dis\frac{1}{n^{2}}\right)}$$
    \item On sait que pour tout entier $n \geq 1$, $r_n = -I_n$ donc :
    $$ r_n \underset{+ \infty}{=} \dis\frac{(-1)^{n+1}}{2(n+1)}+O\left(\dis\frac{1}{n^{2}}\right)$$
    La série de terme général $\dfrac{(-1)^{n+1}}{2(n+1)}$ est une série alternée convergente d'après le critère spécial ses séries alternées (hypothèses clairement vérifiées) et la série de terme général $O\left(\dis\frac{1}{n^{2}}\right)$ converge absolument donc converge par critère de comparaison ($2>1$). Ainsi, par somme, on en déduit que :
    \enc{$\Sum{n \geq 1}{} r_n$ converge}
    \end{enumerate}
    \end{enumerate}
    \item 
    
    \begin{enumerate}
    \item La série de terme général $\dfrac{1}{k!}$ converge donc pour tout $n \geq 0$,
    $$ r_n = \sum_{k=n+1}^{+ \infty} \dfrac{1}{k!}$$
    existe. 
    
    \medskip
    
    \noindent Le terme prépondérant de $r_n$ semble être $\dfrac{1}{(n+1)!}\cdot$
    
    \medskip
    
    \noindent Pour tout $n \geq 0$, on a :
    $$ \dfrac{1}{(n+1)!} \leq r_n$$
    car les termes de la somme sont positifs. On a aussi :
    \begin{align*}
    r_n & = \sum_{k=n+1}^{+ \infty} \dfrac{1}{k!} \\
    & = \dfrac{1}{(n+1)!} \sum_{k=n+1}^{+ \infty} \dfrac{(n+1)!}{k!} \\
    & = \dfrac{1}{(n+1)!} \sum_{k=0}^{+ \infty} \dfrac{(n+1)!}{(k+n+1)!} \\
    & = \dfrac{1}{(n+1)!} \left(1 + \sum_{k=1}^{+ \infty} \dfrac{1}{(n+2)(n+3) \times \cdots \times (n+k+1)} \right)
    \end{align*}
    Pour tout entier $n \geq 0$ et $k \geq 1$,
    $$ (n+2)(n+3) \times \cdots (n+k+1) \geq (n+2)^k$$
    La série de terme général $\dfrac{1}{(n+2)^k}$ converge car $\left\vert \dfrac{1}{n+2} \right\vert<1$, on obtient alors par décroissance de la fonction inverse sur $\mathbb{R}_+^*$ :
    $$ \dfrac{1}{(n+1)!} \leq r_n \leq \dfrac{1}{(n+1)!} \left(1+ \sum_{k=1}^{+ \infty} \dfrac{1}{(n+2)^k} \right)$$
    ou encore :
    $$ 1 \leq (n+1)! r_n \leq 1 + \dfrac{1/(n+2)}{1-1/(n+2)} = 1 + \dfrac{1}{n+1}$$
    Par théorème d'encadrement, on en déduit que $((n+1)! r_n)_{n \geq 0}$ converge vers $1$ donc :
    $$ \boxed{r_n \underset{ + \infty}{\sim} \dfrac{1}{(n+1)!}}$$
    \item Pour tout entier $n \geq 0$, $r_n$ et $1/(n+1)!$ sont positifs. La série de terme général $1/(n+1)!$ converge (série exponentielle) donc par critère de comparaison des séries à termes positifs, on en déduit que :
    \enc{la série de terme général $r_n$ converge}
    \item On sait que pour tout entier $n \geq 0$,
    $$e=\dis\sum_{k=0}^{+\infty}\dis\frac{1}{k!}=\dis\sum_{k=0}^{n}\dis\frac{1}{k!}+r_n$$
    Notons $N$ l'entier défini par :
    $$ N= n!\dis\sum_{k=0}^{n}\dis\frac{1}{k!}$$ 
    Puisque $\sin$ est $2\pi-$périodique, on a :
    \begin{align*}
    \sin(2\pi e n!) & =\sin(2\pi N + 2\pi n!r_n) \\
    & =\sin(2\pi n!r_n) \\
    & =\sin\left(2\pi\dis\frac{(n+1)!r_n}{n+1}\right)
    \end{align*}
    Or d'apr\`es la question (a) :
     $$\dis\lim_{n\to+\infty} 2\pi\dis\frac{(n+1)!r_n}{n+1}=0$$
    donc 
    $$\sin(2\pi e n!) \underset{+ \infty}{\sim} 2\pi\dis\frac{(n+1)!r_n}{n+1}\underset{+ \infty}{\sim}\dis\frac{2\pi}{n}$$
    La série harmonique diverge et est à termes positifs (donc il en est de même pour $\sin(2 \pi e n!)$ à partir d'un certain rang) donc par critère de comparaison des séries à termes positifs, on en déduit que :
    \enc{la série de terme général $\sin(2\pi e n!)$ diverge}
    \end{enumerate}
    \end{enumerate}
    
  
\end{document}