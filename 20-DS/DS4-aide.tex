\documentclass[twoside,french,11pt]{VcCours}

\newcommand{\dx}{\text{d}x}
\newcommand{\dt}{\text{d}t}

\begin{document}

\Titre{PSI}{Promotion 2021--2022}{Mathématiques}{Devoir surveillé n°4}

\begin{center}
\textbf{\Large Indications}
\end{center}

\separationTitre

\section*{Exercice 1 : Intégrales impropres}
\begin{enumerate}
  \item Critère de comparaison pour la convergence.
  \item Lors d'un changement de variables, les deux intégrales ont même nature.
\end{enumerate}

\section*{Exercice 2 : Probabilités discrètes}
\begin{enumerate}
  \item \begin{enumerate}
          \item $A_n$ est une intersection de $F_k$ et $\overline{F_k}$, on utilise l'indépendance mutuelle.
          \item Calcul de somme en utilisant $P(A_n)=q^{n-1}p$.
          \item Utiliser les probabilités totales. On y voit la somme d'une série géométrique.
          \item Définition de probabilité conditionnelle suivit des probabilités composées.
        \end{enumerate}
  \item \begin{enumerate}
          \item $P(R_k)=\frac{1}{(k+1)^2}$.
          
                Pour $P(C_n)$, écrire $C_n$ comme une intersection des $R_k$.
                Puis utiliser l'indépendance. Pour simplifier le produit, on peut faire un télescopage.
          \item Utiliser la continuité décroissante.
        \end{enumerate}
      \end{enumerate}

\section*{Problème 3 : Matrices nilpotentes d'ordre $3$}
  \begin{enumerate}
  \item Trouver l'unique valeur propre, puis la dimension de son sous espace propre associé.
  \item Produit matriciel.
  \item \begin{enumerate}
  \item Montrer que $AM=MA$.
  \item Calculer $AM$ et $MA$.
  \item \ldots
  \end{enumerate}
  \item \begin{enumerate}
  \item Pour montrer que $M^2\neq0$, on pourra raisonner par l'absurde.
  \item \begin{enumerate}
  \item Calculer $M^2$ et $M^3$.
  \item Donner un exemple.
  \item On peut calculer le déterminant de la matrice de la famille.
  \item Calculer les images de la base et les décomposer sur cette même base.
  \item Lien entre les matrices d'un endomorphisme.
  \end{enumerate}
  \end{enumerate}
  \end{enumerate}
  

  
\section*{Problème 4 : Étude d'un endomorphisme sur un espace de polynômes}
  \begin{enumerate}
  \item 
  \begin{enumerate}
  \item Montrer que $t^ne^{-t}=o\left(\frac{1}{n^2}\right)$.
  \item Intégration par partie. Récurrence avec $I_n=n!$.
  \end{enumerate}
  \item \begin{enumerate}
  \item Calculer $T(1)$, $T(X)$, $T(X^2)$.
  \item Déterminer la dimension de l'unique sous-espace propre.
  \end{enumerate}
  \item \begin{enumerate}
  \item Utiliser la formule de Taylor pour le polynôme $t\mapsto P(x+t)$.
  \item Utiliser la linéarité de l'intégrale et la question 1)b).
  \item Écrire la matrice de $T$ dans la base canonique de $\R_n[X]$. 
  En déduire l'unique valeur propre. Déterminer la dimension du sous-espace propre. Il y a un vecteur propre évident.
  \end{enumerate}
  \item L'énoncé souffle que $h(x)=e^x \int_x^{+ \infty} e^{-t} g(t) \dt$ est une solution particulière de l'équation différentielle. Il faut le vérifier.

  Pour cela on fait $\int_{x}^{+\infty}=\int_{0}^{+\infty}-\int_{0}^{x}$ et on utilise le théorème fondamental de l'analyse. 
  \begin{enumerate}
  \item ...
  \item Trouver une relation entre $g'$ et $g$. Résoudre l'équation différentielle ainsi trouvée. Prendre en compte le fait que $g$ est borné.
  \item Majoration de $|T_g|$.
  \item Montrer que si $g$ tend vers $0$ en $+ \infty$, $T_g$ aussi.
  \end{enumerate}
  \item ...
  \begin{enumerate}
  \item Utiliser la definition de "tendre vers $0$".
  \item Calculer $T_g(x)$ pour $g(t)=e^{ix}$.
  
  En déduire la matrice de $T_g$ dans la base donnée.
  \end{enumerate}
  \end{enumerate}
  
  
  
  
  

  
\section*{Problème 5 : Racine carré}
  \begin{enumerate}
  \item \begin{enumerate}
  \item \ldots
  \item \ldots
  \end{enumerate}
  \item 
  \begin{enumerate}
  \item 
  \begin{itemize}
  \item $u$ et $v$ commutent.
  \item Déterminer un polynôme annulateur de $v$ en regardant $v^2$.
  \item \ldots
  \item Image d'une base bien choisie.
  \end{itemize}
  \item La question précédente est "l'Analyse", ici c'est la "synthèse".
  \item Le penser sur des matrices diagonales.
  \end{enumerate}
  \end{enumerate}
  

\end{document}