\documentclass[twoside,french,11pt]{VcCours}

\newcommand{\dx}{\text{d}x}
\newcommand{\dt}{\text{d}t}

\begin{document}

\Titre{PSI}{Promotion 2021--2022}{Mathématiques}{Devoir surveillé n°4}

\begin{center}
\large 
Le samedi 15 janvier 2022

\bigskip
\textbf{Durée: 3h30}

\bigskip
\large\underline{\textbf{Calculatrice interdite}}
\end{center}

\bigskip
\begin{itemize}
  \item Le candidat attachera la plus grande importance à la clarté, à la précision et à la concision de la rédaction. Si un candidat est amené à repérer ce qui peut lui sembler être une erreur d'énoncé, il le signalera sur sa copie et devra poursuivre sa composition en expliquant les raisons des initiatives qu'il a été amené à prendre.
  \item Il est conseillé au candidat de lire l'intégralité du sujet et de repérer les parties qui lui semblent plus abordables.
  \item Le candidat \fbox{encadrera} ou \underline{soulignera} les résultats.
  \item Les étudiants visant e3a-CCP traiteront : \textbf{les exercices 1 et 2 et les problèmes 3 et 4}.
  \item Les étudiants visant un concours plus dur traiteront :
  \textbf{les exercices 1 et 2 et les problèmes 4 et 5}.
  \item Si un étudiant a traité tout ce qu'il savait faire, il peut rendre sa copie et recevoir en contrepartie une feuille
  contenant des indications. Il composera alors sur une nouvelle copie les questions qu'il souhaite.
  \end{itemize}
\separationTitre

\vspace{1cm}
\section*{Questions de cours}
\begin{enumerate}
  \item Citer la définition de probabilité $P$ sur $(\Omega,\AAA)$ (dont fait partie la $\sigma$-additivité).
  \item Citer la formule des probabilités composées.
  \item Citer le théorème de la formule des probabilités totales dans le cas dénombrable.
  \item Citer le théorème de la continuité croissante et décroissante.
\end{enumerate}

\newpage
\section*{Exercice 1 : Intégrales impropres}
\begin{enumerate}
  \item On considère l'intégrale $I=\int_{1}^{+\infty}\frac{\ln(t)}{t^3}\dt$.
  
  Justifier son existence puis la calculer en intégrant par parties.
  \item On considère l'intégrale $J=\int_{0}^{1}\frac{x}{\sqrt{1-x}}\dx$.
  
  En posant $t=\sqrt{1-x}$, montrer que l'intégrale $J$ converge et calculer sa valeur.
  % \item On considère la série $\sum_{n\geq0}f_n$ où
  %       \[\application{f_n}{[1;+\infty}{\R}{x}{e^{-nx}}\]

  %       Déterminer la fonction somme de cette série est $S(x)=\frac{e^x}{e^x-1}$ et montrer que
  %       \[\sum_{n=0}^{+\infty}\frac{e^{-n}}{n}=\ln(e-1)\]
\end{enumerate}

\section*{Exercice 2 : Probabilités discrètes}
Cet exercice propose l'étude de deux jeux indépendants.
\begin{enumerate}
  \item Le premier jeu consiste à lancer une pièce jusqu'à ce qu'elle donne pile.
        La probabilité que la pièce donne pile est noté $p\in]0;1[$ et on pose $q=1-p$.

        Puis, si on note $N$ le nombre de faces obtenu, avant d'avoir pile,
        on place dans une urne $1$ pièce d'or et $2^N-1$ fausses pièces 
        (identiques au touché et de même masse que celle en or). On tire une pièce dans l'urne.

        On pose les événements suivants :
        \begin{itemize}
          \item $F_k$ :\og{}Obtenir face au $k$-ème lancer.\fg{}
          \item $A_n$ :\og{}Obtenir pile pour la première fois au $n$-ème lancer.\fg{}
          \item $G$ :\og{}Tirer la pièce en or.\fg{}
        \end{itemize}
        \begin{enumerate}
          \item En exprimant l'événement $A_n$ à l'aide des $F_k$, déterminer $P(A_n)$.

                Donner la valeur de $P_{A_n}(G)$.
          \item La suite $(A_n)_{n\geq1}$ est-elle un système complet d'événement ?

                Montrer que $\sum_{n=1}^{+\infty}P(A_n)=1$.
          \item Montrer que la probabilité de gagner la pièce d'or est $\frac{2p}{1+p}$.
          \item On a gagné la pièce en or, quelle est la probabilité d'avoir 
          obtenu pile au premier lancer ?
        \end{enumerate}
  \item On considère une urne qui contient $1$ boule rouge et $3$ boules noires.
  
        On tire une boule et :
        \begin{itemize}
          \item Si elle est rouge, le jeu s'arrête.
          \item Si elle est noire, on rajoute des boules noire puis on procède au tirage suivant.
        \end{itemize}
        L'ajout de boules noires est fait de sorte que pour le $n$-ème tirage on ait exactement $(n+1)^2$ 
        boules dont $1$ rouge (ce qui est cohérent avec le contenu de l'urne au départ).

        On pose les événements suivants :
        \begin{itemize}
          \item $R_k$ :\og{}Obtenir la boule rouge au $k$-ème tirage.\fg{}
          \item $C_n$ :\og{}N'obtenir que des boules noires aux $n$ premiers tirages.\fg{}
          \item $T$ :\og{}Le jeu s'arrête.\fg{}
        \end{itemize}
        \begin{enumerate}
          \item Donner $P(R_k)$ puis montrer que $P(C_n)=\frac{n+2}{2(n+1)}$.

          \item Exprimer $\overline{T}$ à l'aide des $C_n$.

                Quelle est la probabilité que le ne s'arrête jamais ?
        \end{enumerate}
      \end{enumerate}

\section*{Problème 3 : Matrices nilpotentes d'ordre $3$}
  On considère la matrice $A = \begin{pmatrix}
  0 & 0 & 0 \\
  1 & 0 & 0 \\
  0 & 1 & 0 \\
  \end{pmatrix} \cdot$
  
  \begin{enumerate}
  \item Déterminer les valeurs propres et les sous-espaces propres de la matrice $A$.
  
  La matrice $A$ est-elle diagonalisable ?
  \item Calculer $A^2$ et $A^3$.
  \item On considère $\mathcal{S}$ l'ensemble des matrices $M$ de $\mathcal{M}_3(\mathbb{R})$ telles que $AM=MA$.
  \begin{enumerate}
  \item Soit $\alpha$, $\beta$ et $\gamma$ trois réels et $M=\alpha I_3 + \beta A + \gamma A^2$. Vérifier que $M \in \mathcal{S}$.
  \item Réciproquement, si $a$, $b$, $c$, $d$, $e$, $f$, $g$, $h$ et $i$ des réels tels que $M = \begin{pmatrix}
  a & b & c \\
  d & e & f \\
  g & h & i \\
  \end{pmatrix} \in \mathcal{S}$. 
  
  Déterminer, en fonction des coefficients de $M$, trois réels $\alpha$, $\beta$ et $\gamma$ tels que $M= \alpha I_3 + \beta A + \gamma A^2$.
  \item En déduire une condition nécessaire et suffisante pour appartenir à $\mathcal{S}$.
  \end{enumerate}
  
  \item On considère $\mathcal{S}'$ l'ensemble des matrices $M$ de $\mathcal{M}_3(\mathbb{R})$ telles que $M^3=0_3$ et $M^2 \neq 0_3$.
  \begin{enumerate}
  \item Soit $P \in \mathcal{M}_3(\mathbb{R})$ inversible et $M=PAP^{-1}$. Vérifier que $M \in \mathcal{S}'$.
  
  Dans la suite, tout vecteur de $\mathbb{R}^3$ sera assimilé à une matrice colonne de $\mathcal{M}_{3,1}(\mathbb{R})$ de sorte que, pour tout vecteur $X$ de $\mathbb{R}^3$, le produit matriciel $MX$ soit correctement défini.
  \item Soit $M = \begin{pmatrix}
  -1 & 1 & 1 \\
  1 & 1 & -1 \\
  0 & 2 & 0 \\
  \end{pmatrix} \cdot$
  \begin{enumerate}
  \item Vérifier que $M \in \mathcal{S}'$.
  
  On considère $f$ l'endomorphisme de $\mathbb{R}^3$ canoniquement associé à $M$.
  \item Prouver qu'il existe un vecteur $X \in \mathbb{R}^3$ tel que $M^2X$ soit non nul.
  \item Montrer que la famille $\mathcal{B} = (X,MX,M^2X)$ est une base de $\mathbb{R}^3$.
  \item Déterminer la matrice de $f$ dans la base $\mathcal{B}$.
  \item En déduire qu'il existe $P \in \mathcal{M}_3(\mathbb{R})$ inversible telle que $M=PAP^{-1}$.
  \end{enumerate}
  \end{enumerate}
  \end{enumerate}
  %\item On cherche à généraliser ce qui a été montré dans la question précédente. Pour cela on considère $M \in \mathcal{S}'$ et $f$ l'endomorphisme de $\mathbb{R}^3$ dont $M$ est la matrice dans la base canonique.
  %\begin{enumerate}
  %\item Prouver qu'il existe un vecteur $X \in \mathbb{R}^3$ tel que $M^2X$ soit non nul.
  %\item En déduire qu'il existe $P \in \mathcal{M}_3(\mathbb{R})$ inversible telle que $M=PAP^{-1}$.
  %\end{enumerate}
  %\item En déduire une condition nécessaire et suffisante pour appartenir à $\mathcal{S}'$.
  %\end{enumerate}
    

  
\section*{Problème 4 : Étude d'un endomorphisme sur un espace de polynômes}
  Pour tout entier naturel $n$, on note $\mathbb{R}_n[X]$ l'espace vectoriel des polynômes à coefficients réels de degré inférieur ou égal à $n$. Si $I$ est un intervalle de $\mathbb{R}$ et $f : I \rightarrow \mathbb{R}$ une fonction bornée sur $I$, on rappelle que :
  $$ \Vert f \Vert_{\infty} = \sup_{x \in I} \vert f(x) \vert$$
    \begin{enumerate}
  \item 
  \begin{enumerate}
  \item Pour tout entier naturel $n$, montrer l'existence de $I_n =  \int_0^{+ \infty} t^n e^{-t} \dt$.
  \item Justifier que pour tout entier $n$, $I_{n+1}=(n+1)I_n$. En déduire la valeur de $I_n$.
  \end{enumerate}
  \item Pour tout $P \in \mathbb{R}_2[X]$, on pose :
  $$ \forall x \in \mathbb{R}, \; T(P)(x) = \int_0^{+ \infty} e^{-t} P(x+t) \dt$$
  \begin{enumerate}
  \item Montrer que $T$ est un endomorphisme de $\mathbb{R}_2[X]$ et déterminer sa matrice $M$ dans la base canonique de $\mathbb{R}_2[X]$.
  \item La matrice $M$ est-elle diagonalisable ?
  \end{enumerate}
  \item Soit $D$ l'endomorphisme de $\mathbb{R}_n[X]$ associant à tout polynôme $P$ son polynôme dérivé $P'$.
  \begin{enumerate}
  \item Soient $P$ et $\mathbb{R}_n[X]$ et $(x,t) \in \mathbb{R}^2$. Déterminer des réels $b_0(x)$, $b_1(x)$, $\ldots$, $b_n(x)$ tels que :
  $$ P(x+t) = \sum_{k=0}^n b_k(x) t^k$$
  \item Pour tout $P \in \mathbb{R}_n[X]$, on pose :
  $$ \forall x \in \mathbb{R}, \; T(P)(x) = \int_0^{+ \infty} e^{-t} P(x+t) \dt$$
  Montrer que $T$ est un endomorphisme de $\mathbb{R}_n[X]$ et déterminer des réels $a_0$, $a_1$, $\ldots$, $a_n$ tels que pour tout $P \in \mathbb{R}_n[X]$,
  $$ T(P) = \sum_{k=0}^n a_k D^k(P)$$
  \item Déterminer les éléments propres de $T$.
  \end{enumerate}
  \item Soit $g : \mathbb{R} \rightarrow \mathbb{R}$ une fonction continue et bornée. Justifier que les solutions sur $\mathbb{R}$ de l'équation différentielle :
  $$ y'-y+g=0$$
  sont les fonctions de la forme :
  $$ x \mapsto k e^x + e^x \int_x^{+ \infty} e^{-t} g(t) \dt$$
  où $k \in \mathbb{R}$.
  \item Soit $g : \mathbb{R} \rightarrow \mathbb{R}$ une fonction continue et bornée.
  \begin{enumerate}
  \item On définit $T_g : \mathbb{R} \rightarrow \mathbb{R}$ par :
  $$ \forall x \in \mathbb{R}, \; T_g(x) = \int_0^{+ \infty} e^{-t} g(t+x) \dt$$
  Justifier que $T_g$ est bien définie puis que :
  $$ \forall x \in \mathbb{R}, \; T_g(x) = e^x \int_x^{+ \infty} e^{-u} g(u) \textrm{d}u$$
  En déduire que $T_g$ est de classe $\mathcal{C}^1$ sur $\mathbb{R}$ et préciser la relation entre $(T_g)'$, $T_g$ et $g$.
  \item En supposant $g$ non nulle, déterminer si il existe $\lambda \in \mathbb{R}$ tel que $T_g= \lambda g$.
  \item Montrer que $T_g$ est bornée sur $\mathbb{R}$ et majorer $\Vert T_g \Vert_{\infty}$ à l'aide de $\Vert g \Vert_{\infty}$.
  \item Montrer que si $g$ tend vers $0$ en $+ \infty$, $T_g$ aussi.
  \end{enumerate}
  \item 
  \begin{enumerate}
  \item Pour tout réel $A$, justifier l'existence et calculer $ \int_A^{+ \infty} e^{(i-1)t} \dt$.
  \item Soit $F$ le sous-espace vectoriel  de $\mathcal{C}(\mathbb{R}, \mathbb{R})$ engendré par les fonctions cosinus et sinus. Montrer que $g \mapsto T_g$ définit un endomorphisme de $F$ et écrire sa matrice dans la base (cosinus, sinus). $N$ est-elle diagonalisable dans $\mathcal{M}_2(\mathbb{R})$?
  \end{enumerate}
  \end{enumerate}
  
  
  
  
  

  
\section*{Problème 5 : Racine carré}
  Soient $E$ un $\C$-espace vectoriel de dimension $n>0$ et $u$ un endomorphisme de $E$.
  \begin{enumerate}
  \item On suppose que $u$ est un automorphisme de $E$ tel que $u^2$ soit diagonalisable.
  \begin{enumerate}
  \item Rappeler une condition nécessaire et suffisante concernant les propriétés d'un polynôme annulateur d'un endomorphisme $f$ de $E$ pour que $f$ soit diagonalisable.
  \item Montrer que $u$ est diagonalisable.
  \end{enumerate}
  \item Soit $v$ un automorphisme de $E$ diagonalisable à valeurs propres réelles strictement positives.
  \begin{enumerate}
  \item On suppose qu'il existe un automorphisme $u$ de $E$, à valeurs propres réelles strictement positives et tel que $u^2=v$.
  \begin{itemize}
  \item Montrer que les sous-espaces propres de $v$ sont stables par $u$.
  \item Montrer que l'endomorphisme induit par $u$ sur tout sous-espace propre de $v$ est diagonalisable.
  \item Soit $\lambda$ une valeur propre de $v$, montrer que le sous-espace propre de $v$ associé à la valeur propre $\lambda$ est égal au sous-espace propre de $u$ associé à la valeur propre $\sqrt{\lambda}$.
  \item En déduire que $u$ est unique.
  \end{itemize}
  \item Montrer l'existence d'un automorphisme $u$ de $E$, à valeurs propres réelles strictement positives, tel que $u^2=v$.
  \item Soit $u$ l'unique automorphisme de $E$  à valeurs propres réelles strictement positives vérifiant $u^2=v$, montrer qu'il existe un polynôme $P$ tel que $u=P(v)$.
  \end{enumerate}
  \end{enumerate}
  

\end{document}