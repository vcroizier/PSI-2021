\documentclass[twoside,french,11pt]{VcCours}

\begin{document}

\Titre{PSI}{Promotion 2021--2022}{Mathématiques}{Devoir surveillé n°1}

\begin{center}
\large 
Le samedi 11 septembre 2021

\bigskip
\textbf{Durée: 1h45}

\bigskip
\large\underline{\textbf{Calculatrice interdite}}
\end{center}

\bigskip
\begin{itemize}
  \item Le candidat attachera la plus grande importance à la clarté, à la précision et à la concision de la rédaction. Si un candidat est amené à repérer ce qui peut lui sembler être une erreur d'énoncé, il le signalera sur sa copie et devra poursuivre sa composition en expliquant les raisons des initiatives qu'il a été amené à prendre.
  \item Il est conseillé au candidat de lire l'intégralité du sujet et de repérer les parties qui lui semblent plus abordables.
  \item Les questions de cours sont à traiter \underline{sur la feuille d'énoncé} et les exercices sont à traiter sur une autre feuille.
  \item Le candidat \fbox{encadrera} ou \underline{soulignera} les résultats. 
  %\item Les étudiants visant un concours plus dur que CCP traiteront \textbf{obligatoirement} l'exercice $2$.
  \end{itemize}
\separationTitre


\section*{Questions de cours}


\begin{enumerate}
\item Compléter les formules suivantes : pour tout $(a,b) \in \mathbb{R}^2$,

\medskip
\begin{itemize}
\item$\sin(a)-\sin(b) = $

\bigskip
\item $\cos(a-b) = $
\end{itemize}

\medskip
\item Soit $n \geq 1$. Donner l'expression des racines $n$-ièmes de l'unité :

\vspace{3cm}

\item Donner les développements limités en $0$ à l'ordre $n \in \mathbb{N}$ :

\medskip
\begin{itemize}
\item $e^x =$

\bigskip
\item $\ln(1+x) = $
\end{itemize}

\medskip
\item Donner les développements limités en $0$ à l'ordre $3$ :

\medskip
\begin{itemize}
\item $\tan(x)=$

\bigskip
\item $\sqrt{1+x}=$
\end{itemize}

\medskip
\item Énoncer le théorème de bijection monotone.

\vspace{4.5cm}


\item Énoncer le théorème de Rolle.

\vspace{4.5cm}



\item Soit $z \in \mathbb{C}$. Compléter :

\medskip
\begin{itemize}
\item $\sum_{n \geq 0} z^n$ converge si et seulement si

\item En cas de convergence, on a : $\sum_{k=0}^{+\infty} z^k =$
\end{itemize}

\medskip
\item Soient $E$ un espace vectoriel et $F$, $G$ deux sous-espaces vectoriels de $E$. Compléter :

\medskip
\begin{itemize}
\item $F$ et $G$ sont en somme directe si 

\bigskip
\item $F$ et $G$ sont supplémentaires dans $E$ si 
\end{itemize}

\medskip
\item Soient $E$ et $F$ deux espaces vectoriels et $f : E \rightarrow F$ une application linéaire. Compléter :

\medskip
\begin{itemize}
\item $\textrm{Ker}(f) =$

\bigskip
\item $\textrm{Im}(f) = $
\end{itemize}
\end{enumerate}

\newpage
\section*{Exercice 1}
Soit $n \in \mathbb{N}^*$. Calculer $\sum_{k=0}^{n} 2^k \binom{n}{k}$.

\section*{Exercice 2}
Soit $f$ la fonction définie par $f(x) =  \arcsin \left( \frac{x}{\sqrt {1 + x^{2}}} \right) \cdot$

\begin{enumerate}
\item Rappeler l'ensemble de définition de la fonction arcsinus.
\item Justifier que $f$ est définie sur $\mathbb{R}$.
\item Donner une expression simple de $f$.
\end{enumerate}

\medskip
\section*{Exercice 3}
Calculer $\int_{0}^{1} \arctan(x) \text{d}x$.

\medskip
\section*{Exercice 4}
Déterminer $\lim_{n \rightarrow + \infty} \left( 1+ \frac{1}{n} \right)^n$.

\medskip
\section*{Exercice 5}
Déterminer le terme général de la suite définie par $u_0=2$ et pour tout $n \in \mathbb{N}$ par :
$$u_{n+1} = 5 u_n -6$$

\medskip
\section*{Exercice 6}
Déterminer la nature de $\sum_{n \geq 1} \dfrac{1}{n^3+n^2+n} \cdot$

\medskip
\section*{Exercice 7}
 

\begin{enumerate}
\item Factoriser dans $\mathbb{C}[X]$ le polynôme $X^2+X+1$.
\item Factoriser dans $\mathbb{C}[X]$ le polynôme $X^4+X^2+1$.
\end{enumerate}

\medskip
 \section*{Exercice 8}
Soit $f$ l'application définie sur $\mathbb{R}_2[X]$ par $f(P)=P-(X+1)P'+X^2 P''$.
\begin{enumerate}
\item Montrer que $f$ définit un endomorphisme de $\mathbb{R}_2[X]$.
\item Déterminer le noyau de $f$. On donnera une base de celui-ci.
\item Que vaut le rang de $f$? Déterminer une base de l'image de $f$.
\end{enumerate}

\medskip
\section*{Exercice 9}
Soient $F$ et $G$ les ensembles définis par :
$$ F = \lbrace (x,y,z) \in \mathbb{R}^3, \, x-y+z=0 \rbrace \quad \hbox{ et }  \quad G = \textrm{Vect}((1,1,1)) $$

\begin{enumerate}
\item Donner une base de $F$ et préciser sa dimension.
\item Montrer que $F$ et $G$ sont supplémentaires dans $\mathbb{R}^3$.
\item Soit $p$ la projection sur $F$ parallèlement à $G$. Donner l'expression de $p$.
\item Soit $q$ la projection sur $G$ parallèlement à $F$. Donner l'expression de $q$.
\end{enumerate}


\end{document}