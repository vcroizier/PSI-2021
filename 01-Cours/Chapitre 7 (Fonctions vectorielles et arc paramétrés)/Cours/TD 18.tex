\documentclass[a4paper,10pt]{report}
\usepackage{cours}

\begin{document}
\everymath{\displaystyle}


\begin{center}
 \shadowbox{{\huge TD 18 : Fonctions vectorielles et arc param�tr�s}}
\end{center}

\medskip

\noindent $I$ d�signe un intervalle de $\mathbb{R}$ contenant au moins deux points et $n$ est un entier naturel non nul.

\medskip



\begin{center}
{\large \textit{\underline{Fonctions vectorielles}}}
\end{center}

\medskip

\exo Soit $f : I \rightarrow \R^n$ une fonction de classe ${\cal C}^1$ sur $I$.  On suppose que l'application $\|f\|$ est constante. Montrer que pour tout $t\in I,$ $f(t)\perp f'(t).$

\exo Soient $a$, $b$, $c$ et $d$, des fonctions d�rivables sur $I$. Donner de deux mani�res la d�riv�e de la fonction $f$ d�finie par :
$$ f(x)  = \left\vert \begin{array}{cc}
a(x) 	& b(x) \\
c(x) & d(x) \\
\end{array}\right\vert$$

\exo Soit $f : ]-1,1[ \rightarrow \mathbb{R}^2$ d�fini par :
$$ f(t) = \left\lbrace \begin{array}{cl}
(0,0) & \hbox{ si } t \in ]-1,0] \\
\left( t^2  \cos \left( \dfrac{1}{t} \right), t^2  \sin \left( \dfrac{1}{t} \right) \right) & \hbox{ si } t \in ]0,1[ \\
\end{array}\right.$$
Montrer que $f$ est d�rivable sur $]-1,1[$ et donner $f'$.



\begin{center}
{\large \textit{\underline{Arcs param�tr�s}}}
\end{center}

\medskip

\exo �tudier l'arc param�tr� du plan d�fini par :
$$ x(t) = t^2+ \dfrac{2}{t}, \; y(t) = t^2+\dfrac{1}{t^2} $$

\exo �tudier l'arc param�tr� du plan d�fini par :
$$ x(t) = \dfrac{t^2}{t-1}, \; y(t) = \dfrac{t}{t^2-1} $$

\exo �tudier l'arc param�tr� du plan, appel� cardio�de,  d�fini par :
$$ x(t) = 2 \cos(t) - \cos(2t), \; y(t) = 2 \sin(t) - \sin(2t) $$

\exo Consid�rons l'arc param�tr� du plan d�fini par :
$$ x(t) =\cos(t)^3, \; y(t) = \sin(t)^3 $$

\begin{enumerate}
\item Calculer pour tout $t \in \mathbb{R}$, $x(\pi-t)$, $y(\pi-t)$, $x(\pi/2-t)$ et $y(\pi/2-t)$.
\item Justifier que l'�tude sur $[0,\pi/4]$ suffit � tracer enti�rement la courbe.
\item �tudier l'arc param�tr�.
\end{enumerate}

\exo On consid�re l'arc param�tr� plan d�fini pour tout $t \in [0,2\pi]$ par :
$$ x(t) = t - \sin(t), \; y(t) = 1- \cos(t) $$
D�terminer la longueur de cet arc.
\end{document}
