\documentclass[a4paper,10pt]{report}
\usepackage{cours}

\begin{document}
\everymath{\displaystyle}


\begin{center}
\textit{{ {\huge TD 7 : Fonctions vectorielles et arc paramétrés}}}
\end{center}


\medskip

%\noindent $I$ désigne un intervalle de $\mathbb{R}$ contenant au moins deux points et $n$ est un entier naturel non nul.
%
%\medskip

%
%
%\begin{center}
%{\large \textit{\underline{Fonctions vectorielles}}}
%\end{center}
%
%\medskip
%
%\exo Soit $f : I \rightarrow \R^n$ une fonction de classe ${\cal C}^1$ sur $I$.  On suppose que l'application $\|f\|$ est constante. Montrer que pour tout $t\in I,$ $f(t)\perp f'(t).$
%
%\exo Soient $a$, $b$, $c$ et $d$, des fonctions dérivables sur $I$. Donner de deux manières la dérivée de la fonction $f$ définie par :
%$$ f(x)  = \left\vert \begin{array}{cc}
%a(x) 	& b(x) \\
%c(x) & d(x) \\
%\end{array}\right\vert$$
%
%\exo Soit $f : ]-1,1[ \rightarrow \mathbb{R}^2$ défini par :
%$$ f(t) = \left\lbrace \begin{array}{cl}
%(0,0) & \hbox{ si } t \in ]-1,0] \\
%\left( t^2  \cos \left( \dfrac{1}{t} \right), t^2  \sin \left( \dfrac{1}{t} \right) \right) & \hbox{ si } t \in ]0,1[ \\
%\end{array}\right.$$
%Montrer que $f$ est dérivable sur $]-1,1[$ et donner $f'$.
%
%
%
%\begin{center}
%{\large \textit{\underline{Arcs paramétrés}}}
%\end{center}
%
%\medskip

\begin{Exercice}{} Étudier l'arc paramétré du plan défini par :
$$ x(t) = t^2+ \dfrac{2}{t}, \; y(t) = t^2+\dfrac{1}{t^2} $$
\end{Exercice}

\begin{Exercice}{} Étudier l'arc paramétré du plan défini par :
$$ x(t) = \dfrac{t^2}{t-1}, \; y(t) = \dfrac{t}{t^2-1} $$
\end{Exercice}

\begin{Exercice}{} Étudier l'arc paramétré du plan, appelé cardioïde,  défini par :
$$ x(t) = 2 \cos(t) - \cos(2t), \; y(t) = 2 \sin(t) - \sin(2t) $$
\end{Exercice} 

%\begin{Exercice}{} Considérons l'arc paramétré du plan défini par :
%$$ x(t) =\cos(t)^3, \; y(t) = \sin(t)^3 $$
%
%\begin{enumerate}
%\item Calculer pour tout $t \in \mathbb{R}$, $x(\pi-t)$, $y(\pi-t)$, $x(\pi/2-t)$ et $y(\pi/2-t)$.
%\item Justifier que l'étude sur $[0,\pi/4]$ suffit à tracer entièrement la courbe.
%\item Étudier l'arc paramétré.
%\end{enumerate}
%\end{Exercice}

\begin{Exercice}{} Étudier l'arc paramétré défini par :
$$ x(t) = \dfrac{1}{t} + \ln(2+t), \; y(t) = t + \dfrac{1}{t}$$
On précisera en particulier les tangentes parallèles aux axes, les branches infinies et les points particuliers de l'arc.
\end{Exercice} 

\begin{Exercice}{} On considère l'arc paramétré plan défini pour tout $t \in [0,2\pi]$ par :
$$ x(t) = t - \sin(t), \; y(t) = 1- \cos(t) $$
Déterminer la longueur de cet arc.
\end{Exercice}

\end{document}
