\documentclass[a4paper,10pt]{report}
\usepackage{cours}
\usepackage{pifont}

\begin{document}
\everymath{\displaystyle}

\everymath{\displaystyle}
\begin{center}
\textit{{ {\huge TD 3 : Espaces vectoriels}}}
\end{center}
\bigskip


\noindent Dans tout le TD, $n$ sera un entier naturel non nul et $\mathbb{K}$ d�signera $\mathbb{R}$ ou $\mathbb{C}$.

\medskip

\begin{center}
\textit{{ {\large Sous-espaces vectoriels}}}
\end{center}

\begin{Exa} Les ensembles suivants sous-ils des sous-espaces vectoriels d'un espace vectoriel de r�f�rence ?

\begin{multicols}{2}
\begin{small}
\begin{enumerate}
\item $A = \lbrace (u_n)_{n \geq 0} \in \mathbb{R}^{\mathbb{N}} \, \vert \, (u_n)_{n \geq 0} \hbox{ born�e} \rbrace $
\item $B = \lbrace (u_n)_{n \geq 0} \in \mathbb{R}^{\mathbb{N}} \, \vert \, (u_n)_{n \geq 0} \hbox{ converge} \rbrace $
\item $C = \lbrace (u_n)_{n \geq 0} \in \mathbb{R}^{\mathbb{N}} \, \vert \, (u_n)_{n \geq 0} \hbox{ g�om�trique} \rbrace $
\item $D = \lbrace f \in \mathcal{F}(\mathbb{R}, \mathbb{R}) \, \vert \, f \hbox{ est croissante} \rbrace$
\item $E = \lbrace f \in \mathcal{F}(\mathbb{R}, \mathbb{R}) \, \vert \, f \hbox{ est continue en 0} \rbrace$
\columnbreak
\item $F = \lbrace f \in \mathcal{F}(\mathbb{R}, \mathbb{R}) \, \vert \, f(x) \underset{ x \rightarrow + \infty}{\rightarrow} 1 \rbrace$
\item $G= \lbrace f \in \mathcal{F}(\mathbb{R}, \mathbb{R}) \, \vert \, f \hbox{ est paire} \rbrace$
\item $H= \lbrace P \in \mathbb{C}[X] \, \vert \, P(0)=1  \rbrace$
\item $I= \lbrace P \in \mathbb{C}[X] \, \vert \, P(X^2)=(X^2+1)P(X)  \rbrace$
%\item $K= \lbrace M \in \mathcal{M}_n(\mathbb{R}) \, \vert \, M^2=I_n  \rbrace$
\item $J= \lbrace M \in \mathcal{M}_n(\mathbb{R}) \, \vert \, AM=MA \rbrace$ o� $A \in \mathcal{M}_n(\mathbb{R})$
\end{enumerate}
\end{small}
\end{multicols}
\vspace{0.01cm}
\end{Exa}



\begin{Exa} Montrer que $F$ d�fini par :
$$ F = \lbrace (u_n)_{n \geq 0} \in \mathbb{R}^{\mathbb{N}} \, \vert \, \forall n \geq 0, \, u_{n+3}=u_{n+2}+u_{n+1}+u_n \rbrace$$
est un sous-espace vectoriel de $\mathbb{R}^{\mathbb{N}}$.
\end{Exa}

\begin{Exa} Soient $E = \mathcal{F}(\mathbb{R}, \mathbb{R})$, $C$ l'ensemble des fonctions croissantes de $E$ et $\Delta = \lbrace f-g \, \vert \, (f,g) \in C^2 \rbrace\cdot$

\noindent Montrer que $\Delta$ est un sous-espace vectoriel de $E$.
\end{Exa}



\begin{Exa} Soit $F$ le sous-ensemble de $\mathbb{R}^3$ d�fini par $F=$Vect((1,0,1), (-1,2,3)). 
\begin{enumerate}
\item Le vecteur $(-1,1,6)$ appartient-il � $F$?
\item Comparer $F$ et $G=$Vect((0,2,4), (3,-2,-1)). 
\end{enumerate}
\end{Exa}

\begin{Exa} Soit $E$ un $\mathbb{K}$-espace vectoriel de $F$, $G$ deux sous-espaces vectoriels de $E$. Montrer que $F \cup G$ est un sous-espace vectoriel de $E$ si et seulement si $F \subset G$ ou $G \subset F$.
\end{Exa}

\medskip

\begin{center}
\textit{{ {\large Familles libres, g�n�ratrices, bases}}}
\end{center}

\medskip

\begin{Exa} Les familles suivantes de vecteurs de $\R^3$ sont-elles libres?
   \begin{enumerate}
  \item
        $(x_1 ,x_2)$ avec $x_1 = (1,0,1)$ et $x_2 = (1,1,1)$
      \item
        $(x_1 ,x_2 ,x_3)$ avec $x_1 = (1, - 1,1)$, $x_2 = (2, - 1,3)$ et $x_3 = ( - 1,1, - 2)$.
      \item
        $(x_1 ,x_2 ,x_3)$ avec $x_1 = (1,2,1)$, $x_2 = (2,1, - 1)$ et $x_3 = (1, - 1, - 2)$
    \end{enumerate}
\end{Exa}


\begin{Exa} Soit $f : \mathbb{R} \rightarrow \mathbb{R}$ d�finie par $f(x)=e^x$. Montrer que $(f,f^2,f^3)$ est une famille libre de l'espace vectoriel des fonctions de $\mathbb{R}$ dans $\mathbb{R}$.
\end{Exa}

\begin{Exa}[\ding{80}] Pour $a \in \mathbb{R}$, on note $f_a$ l'application de $\mathbb{R}$ vers $\mathbb{R}$ d�finie par 
$$f_a(x) = \vert x - a \vert$$
Soit $a_1, \ldots, a_n \in \mathbb{R}$ $(n \geq 2)$ deux � deux distincts. Montrer que $(f_{a_i})_{1 \leq i \leq n}$ est une famille libre de $\mathcal{F}(\mathbb{R}, \mathbb{R})$.
\end{Exa} 

\begin{Exa} �crire les sous-ensembles suivants comme des sous-espaces vectoriels engendr�s puis donner une base des ces espaces. 

\begin{multicols}{2}
\begin{enumerate}
\item $A=\lbrace (x,y,z) \in \mathbb{R}^3 \, \vert  \, 2x-y+z=0 \rbrace$
\item $B= \lbrace (x,y,z) \in \mathbb{R}^3 \, \vert \, z+y=0 \rbrace$
\item $C= \lbrace (x,y,z) \in \mathbb{R}^3 \, \vert \, z-y=0 \hbox{ et } x-y+z=0 \rbrace$

\item $D = \left\lbrace \begin{pmatrix}
a+b & 0 & c \\
0 & b+c & 0 \\
c+a & 0& a+b 
\end{pmatrix} \bigg{\vert} (a,b,c) \in \mathbb{R}^3 \right\rbrace$

\columnbreak
\item $E = \left\lbrace M \in \mathcal{M}_3(\mathbb{R}) \,\vert  MA=AM \right\rbrace$ o� \newline \begin{center}
$A = \begin{pmatrix}
0 & 2 & 0 \\
1 & 1 & 0 \\
0 & 0 & 1 
\end{pmatrix}$
\end{center}
\item $F= \lbrace a + bX^2 + (c-b)(X-1) \, \vert \, (a,b,c) \in \mathbb{R}^3 \rbrace$
\item $G = \lbrace P \in \mathbb{R}_3[X] \, \vert \, P(1)=P'(1)=0 \rbrace$
\end{enumerate}
\end{multicols}

\vspace{0.05cm}
\end{Exa}



\begin{Exa} Montrer que $F$ d�fini par :
$$ F=\lbrace f \in \mathcal{C}^2(\mathbb{R}, \mathbb{R}) \, \vert \,  f''-3f'-10f= 0 \rbrace $$
est un sous-espace vectoriel de $\mathcal{C}^2(\mathbb{R}, \mathbb{R})$ et donner-en une base.
\end{Exa} 

\begin{Exa} Soient $E= \mathbb{R}_n[X]$ et $\mathcal{F} = (X^k(1-X)^{n-k})_{0 \leq k \leq n}$. Montrer que $\mathcal{F}$ est une base de $E$.
\end{Exa}


\begin{Exa} Soient $n \in \mathbb{N}^*$ et $(a,b) \in \mathbb{R}^2$ avec $a \neq b$.
\begin{enumerate}
\item Montrer que $((X-a)^k)_{0 \leq k \leq 2n}$ est une base de $\mathbb{R}_{2n}[X]$.
\item D�terminer les coordonn�es de $(X-a)^n(X-b)^n$ dans la base pr�c�dente.
\end{enumerate}
\end{Exa} 

\begin{Exa} Soit $E = \mathcal{F}(]-1,1[, \mathbb{R})$. On consid�re les fonctions de $E$ d�finies pour tout $x \in ]-1,1[$ par : 
    \[
    f_1(x) = \sqrt {\frac{1 + x}{1 - x}} , \; f_2(x) = \sqrt {\frac{1 - x}{1 + x}} , \; f_3(x) = \frac{1}{\sqrt {1 - x^2}}, \; f_4(x) = \frac{x}{\sqrt {1 - x^2}}
    \]
Quel est le rang de la famille $(f_1 ,f_2 ,f_3 ,f_4)$?
\end{Exa}

\begin{Exa} Soit $E$ un $\mathbb{K}$-espace vectoriel muni d'une base $\mathcal{B} = (e_1 , \ldots ,e_n)$. Pour tout $i \in \lbrace 1, \ldots ,n \rbrace$, on pose :
$$\varepsilon_i = e_1 + \cdots + e_i$$
Montrer que $\mathcal{B}' = (\varepsilon_1 , \ldots ,\varepsilon_n)$ est une base de $E$.
\end{Exa} 

\begin{Exa}[\ding{80}] Soient $(e_1 , \ldots ,e_n)$ et $(e'_1 , \ldots ,e'_n)$ deux bases d'un $\mathbb{K}$-espace vectoriel $E$. Montrer qu'il existe $j \in \lbrace 1, \ldots ,n \rbrace$ tel que la famille $(e_1 , \ldots ,e_{n - 1} ,e'_j)$ soit encore une base de $E$.
\end{Exa}





\medskip

\begin{center}
\textit{{ {\large Espaces suppl�mentaires}}}
\end{center}

\medskip

\begin{Exa} Soient $F= \lbrace (x+y,y-2x, x) \, \vert \,  (x,y) \in \mathbb{R}^2 \rbrace$ et $G = \textrm{Vect}((1,2,3))$. Montrer que $F$ et $G$ sont suppl�mentaires dans $\mathbb{R}^3$.
\end{Exa} 

\begin{Exa} Soient $E= \mathbb{R}^3$ et $F$, $G$ les ensembles d�finis par :
$$ F = \lbrace (x,y,z) \in \mathbb{R}^3, \, x-y+z=0 \rbrace \quad \hbox{ et }  \quad G = \textrm{Vect}((1,1,1)) $$

\begin{enumerate}
\item Donner une base de $F$ et pr�ciser sa dimension.
\item Montrer que $F$ et $G$ sont suppl�mentaires dans $\mathbb{R}^3$.
\end{enumerate}
\end{Exa}

\begin{Exa} Soit $F = \lbrace P \in \mathbb{R}_4[X] \, \vert \, P(0)=P'(0)=P'(1)=0 \rbrace \cdot$

\begin{enumerate}
\item Montrer que $F$ est un espace vectoriel, d�terminer en une base et pr�ciser sa dimension.
\item Montrer que le sous-espace vectoriel $G= \textrm{Vect}(1,X,1+X+X^2)$ et $F$  sont en somme directe.
\item $F$ et $G$ sont-ils suppl�mentaires dans $\mathbb{R}_4[X]$?
\end{enumerate}
\end{Exa}

\begin{Exa} Montrer que l'ensemble des fonctions paires et l'ensemble des fonctions impaires sont des espaces suppl�mentaires de l'espace vectoriel des fonctions de $\mathbb{R}$ dans $\mathbb{R}$. 
\end{Exa}


\begin{Exa} Montrer que l'ensemble des matrices sym�triques $S_n(\mathbb{R})$ et l'ensemble des matrices antisym�triques $A_n(\mathbb{R})$ sont suppl�mentaires dans $\mathcal{M}_n(\mathbb{R})$.
\end{Exa}


\begin{Exa} Soit $E = \mathcal{F}(\mathbb{R}, \mathbb{R})$. Montrer que l'ensemble des fonctions affines est un suppl�mentaire du sous-espace vectoriel $F$ de $E$ d�fini par $F = \lbrace f \in \mathcal{F}(\mathbb{R}, \mathbb{R}) \, \vert \, f(0)=f(1)=0 \rbrace\cdot$
\end{Exa}



\begin{Exa}[\ding{80}] Soit $E$ l'espace des fonctions continues de $[-1,1]$ � valeurs dans $\mathbb{R}$. On consid�re les sous-espaces vectoriels suivants de $E$ :
\begin{align*}
F_1 &= \lbrace f \in E \, \vert \, f{\text{~est constante}} \rbrace \\
F_2 &= \lbrace f \in E \, \vert \ \forall t \in [ - 1,0], \,  f(t)  = 0 \rbrace \\
F_3 & = \lbrace f \in E \, \vert \ \forall t \in [0,1], \,  f(t) = 0 \rbrace 
\end{align*}
Montrer que $E = F_1 \oplus F_2 \oplus F_3$.
\end{Exa} 



\begin{Exa}[\ding{80}] Soient $n\in\N$, $E=\K_n[X]$, $S$ un polyn�me de degr� $d \in \Interv{1}{n}$ et $F$ l'ensemble des polyn�mes de $E$ divisibles par $S$.
\begin{enumerate}
	\item Montrer que $F$ est un sous-espace vectoriel de $\E$ de dimension $n+1-d$.
	\item Montrer que $\mathbb{K}_{d-1}[X]$  et $F$ sont des sous-espaces vectoriels suppl�mentaires de $E$.
\end{enumerate} 
\end{Exa}


\begin{Exa}[\ding{80}] Consid�rons $E$ un $\mathbb{K}$-espace vectoriel de dimension finie. Montrer que si $F$ et $G$ sont deux sous-espaces vectoriels de m�me dimension alors ils ont un sous-espace suppl�mentaire en commun.
\end{Exa}




\end{document}