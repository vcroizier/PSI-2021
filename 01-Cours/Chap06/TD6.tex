\documentclass[a4paper,twoside,french,11pt]{VcCours}

\newcommand{\dx}{\text{d}x}
\newcommand{\dt}{\text{d}t}
\DeclareMathOperator{\e}{e}
\newcommand{\Sum}[2]{\sum_{#1}^{#2}}
\newcommand{\Int}[2]{\int_{#1}^{#2}}

\begin{document}
\Titre{PSI}{Promotion 2021--2022}{Mathématiques}{TD 6 : Suites et séries de fonctions}

\tableofcontents
\separationTitre

Dans la suite, $I$ sera un intervalle de $\mathbb{R}$ contenant au moins deux points.

\subsection{\large Convergence de suites de fonctions}


\begin{Exercice}{} Pour tout entier $n \geq 1$, on note $f_n$ la fonction définie sur $\mathbb{R}_+$ par :
$$ f_n(x) = \frac{x}{n+x}$$
Étudier la convergence uniforme de $(f_n)_{n \geq 1}$ sur $\mathbb{R}_+$ puis sur un segment de $\mathbb{R}_+$.
\end{Exercice}


\begin{Exercice}{} Soient $\alpha \in \R$ et $(f_n)_{n \geq 1}$ la suite de fonctions définies pour tout $x \in [0,1]$ par :
$$f_n(x) = n^\alpha x(1-x)^n$$

\begin{enumerate}
 \item Étudier la convergence simple de $(f_n)_{n \geq 1}$.
 \item Y a-t-il convergence uniforme ?
  \end{enumerate}
\end{Exercice}


\begin{Exercice}{} Pour tout entier $n \geq 0$, on note $f_n$ la fonction définie sur $\mathbb{R}_+$ par :
$$ f_n(x) = \frac{x^n e^{-x}}{n!}$$

\begin{enumerate}
\item Déterminer la limite simple de $(f_n)_{n \geq 0}$ sur $\mathbb{R}_+$.
\item Montrer qu'il y a convergence uniforme (on pourra utiliser la formule de Stirling).
\end{enumerate}
\end{Exercice} 


\begin{Exercice}{} Soit $(f_n)_{n \geq 1}$ la suite de fonctions définies pour tout $x \in \mathbb{R}_+$ par :
$$f_n(x) = \frac {nx}{1+n^2x^2}$$
\begin{enumerate}
\item Étudier la convergence simple de $(f_n)_{n \geq 1}$ sur $\mathbb{R}_+$.
\item Étudier la convergence uniforme de $(f_n)_{n \geq 1}$ sur $\mathbb{R}_+$ puis sur $[a, + \infty[$ pour $a>0$.
\end{enumerate}
\end{Exercice}


\begin{Exercice}{} Montrer que la limite simple d'une suite de fonctions croissantes de $I$ vers $\mathbb{R}$ est croissante.
\end{Exercice}

\begin{Exercice}{} On pose pour tout entier $n \geq 0$ et tout $x \in \mathbb{R}$,
$$f_{n}(x) =\dfrac{n+2}{n+1}\mathrm{e}^{-n x^{2}}$$
\begin{enumerate}
\item Étudier la convergence simple de la suite de fonctions $\left(f_{n}\right) _{n \geq 0}$.
\item La suite de fonctions  $\left(f_{n}\right) _{n \geq 0}$ converge-t-elle uniformément sur $\left[ 0,+\infty\right[$ ?
\item Soit $a>0$. La suite de fonctions $\left(f_{n}\right) _{n \geq 0}$ converge-t-elle uniformément sur  $[a,+\infty[$ ?	
\item La suite de fonctions  $\left(f_{n}\right) _{n \geq 0}$ converge-t-elle uniformément sur $]0,+\infty[$? 
\end{enumerate}
\end{Exercice}

\begin{Exercice}{} \begin{enumerate}
\item Soient $X$ une partie de $\mathbb{R}$ et $\left( f_{n}\right) _{n \geq 0}$ une suite de fonctions de $X$ dans $\mathbb{R}$ convergeant simplement vers une fonction $f$. \\
On suppose qu'il existe une suite $\left( x_{n}\right)_{n \geq 0}$ d'éléments de $X$ telle que la suite $\left( f_{n}(x_{n})-f\left( x_{n}\right) \right) _{n \geq 0}$ ne tende pas vers $0$. \medskip

Démontrer que la suite de fonctions $\left( f_{n}\right) _{n \geq 0}$ ne converge pas uniformément vers $f$ sur $X$.

\item Pour tout $x\in\mathbb{R}$ et tout entier $n \geq 0$, on pose :
$$f_{n}(x) =\dfrac{\sin \left( nx\right) }{1+n^{2}x^{2}}$$
	\begin{enumerate}
	\item Étudier la convergence simple de la suite $\left( f_{n}\right)_{n \geq 0}$.
	\item Étudier la convergence uniforme de la suite $\left( f_{n}\right)_{n \geq 0}$ sur $[a,+\infty[$ (avec $a>0$),  puis sur $]0,+\infty[$.
	\end{enumerate}
\end{enumerate}
\end{Exercice}


\begin{Exercice}{} 
\begin{enumerate}

\item Soient $(g_n)_{n \geq 0}$ une suite de fonctions de $X$ dans $\mathbb{C}$ et $X$ un ensemble non vide quelconque.\\
On suppose que, pour tout $n\in \mathbb{N}$, $g_n$ est bornée et que la suite $(g_n)_{n \geq 0}$ converge uniformément sur $X$ vers $g$. Démontrer que la fonction  $g$ est bornée.
\item
On considère la suite  $(f_n)_{n \geq 1}$ de fonctions  définies sur $\mathbb{R}$ par : 

$$f_n(x)=\left\lbrace \begin{array}{lll}
n^2x&\:\text{si}\:&|x|\leq \frac{1}{n}\\[0.3cm]
\dfrac{1}{x}&\:\text{si}\:&|x|>\frac{1}{n}
\end{array}\right.$$
Prouver que $(f_n)_{n \geq 1}$  converge simplement sur $\mathbb{R}$.\\
La convergence est-elle uniforme sur $\mathbb{R}$?

\end{enumerate}
\end{Exercice}


\newpage
\begin{Exercice}{} Pour tout $n \geq 1$, on pose $f_n : \mathbb{R} \rightarrow \mathbb{R}$ définie par :
$$ f_n(x)= \left\lbrace \begin{array}{cl}
\dfrac{nx^2}{1+nx} & \hbox{ si } x \geq 0 \\[0.3cm]
\dfrac{nx^3}{1+nx^2} & \hbox{ si } x < 0 \\
\end{array}\right.$$
\begin{enumerate}
\item Montrer que $(f_n)_{n \geq 1}$ converge uniformément sur $\mathbb{R}$ vers une fonction $f$ que l'on précisera.
\item Montrer que $(f_n')_{n \geq 1}$ converge simplement mais pas uniformément sur $[-1,1]$.
\end{enumerate}
\end{Exercice}




\begin{Exercice}[$\bigstar$] Soit $(P_{n})_{n \geq 0}$ une suite de fonctions polynomiales de $\R$ dans $\R$. On suppose que cette suite converge uniformément vers une fonction $f$ sur $\R$. Montrer que la fonction $f$ est polynomiale.
\end{Exercice}


\medskip

\subsection{\large Convergence de séries de fonctions}

\medskip

\begin{Exercice}{} Étudier la convergence simple, normale et uniforme de $\Sum{n\geq 1}{} f_n$ sur $\mathbb{R}_+^*$ où pour tout entier $n \geq 1$ et tout $x \in \mathbb{R}_+^*$, 
$$ f_n(x) = \frac{1}{n+n^3x^2}$$
Cette série converge t-elle uniformément sur tout segment de $\mathbb{R}_+^*$ ?
\end{Exercice} 



\begin{Exercice}[$\bigstar$] Étudier la convergence simple, normale puis uniforme de $\Sum{n\geq 0}{} f_n$ sur $\mathbb{R}_+$ où pour tout entier $n \geq 0$ et tout $x \in \mathbb{R}_+$, $f_n(x) = nx^2e^{-x\sqrt{n}}.$
\end{Exercice}

\begin{Exercice}{} Pour tout entier $n \geq 1$ et tout $x \in \mathbb{R}$, on pose :
$$ f_n(x) = \dfrac{\left(-1\right)^{n}e^{-nx}}{n} $$

\begin{enumerate}
\item Étudier la convergence simple sur $\mathbb{R}$  de la série de fonctions $\Sum{n\geq 1}{} f_n.$
\item Étudier la convergence uniforme sur $\left[ 0,+\infty\right[ $  de la série de fonctions $\Sum{n\geq 1}{} f_n.$
\end{enumerate}
\end{Exercice}

 


\begin{Exercice}[$\bigstar$] Soient $A\subset \mathbb{R}$ et $\left( f_{n}\right)_{n \geq 0}$ une suite de fonctions de $A$ dans $\mathbb{R}$.

\begin{enumerate}
\item Démontrer l'implication:
	\begin{eqnarray*}
	&  \text{La série de fonctions }\Sum{n\geq 0}{} f_n\ \text{converge uniformément sur $A$}& \\
	&\Downarrow &\\
	&\text{La suite de fonctions\ }\left( f_{n}\right) _{n \geq 0}\ \text{converge uniformément vers la fonction nulle sur $A$}&
	\end{eqnarray*}
\item
On pose pour tout $n\in\mathbb{N}$ et tout $x\in\left[ 0,+\infty\right[ $, 
$$f_n(x)=nx^2\mathrm{e}^{-x\sqrt{n}}$$
Prouver que $\Sum{n\geq 0}{} f_n$ converge simplement sur $\left[ 0,+\infty\right[$. La convergence est-elle uniforme ?
\end{enumerate}
\end{Exercice}


\medskip

\subsection{\large Interversion limite-intégrale}

\medskip


\begin{Exercice}{}On pose pour tout entier $n \geq 1$ et tout $x \in [0,1]$,
$$f_{n}\left( x\right) =\left( x^{2}+1\right) \dfrac{ne^{x}+xe^{-x}}{n+x} $$

\begin{enumerate}
\item Démontrer que la suite de fonctions $\left( f_{n}\right) _{n \geq 1}$ converge uniformément sur $[0,1]$.
\item Calculer $\underset{n\rightarrow +\infty }{\lim }\int\limits_{0}^{1}\left( x^{2}+1\right) \dfrac{ne^{x}+xe^{-x}}{n+x}\text{d}x$.
\end{enumerate}
\end{Exercice} 


\medskip

\subsection{\large Fonctions définies par une somme}

\medskip


\begin{Exercice}{} Donner l'ensemble de définition et de continuité de la fonction $f$ définie par :
  \[
  f(x) = \sum_{n = 0}^{ + \infty} \e^{ - x\sqrt n}
  \]
 Déterminer la limite en $ + \infty$ de $f$.
 \end{Exercice}
 

\begin{Exercice}{} On considère la série de fonctions de terme général $u_{n}$ définie par: 
\begin{equation*}
\forall n\in \mathbb{N}^{*},\ \forall x\in \lbrack 0,1], \ \ u_{n}\left(x\right) =\ln \left( 1+\dfrac{x}{n}\right) -\dfrac{x}{n}~
\end{equation*}

On pose, lorsque la série converge, $S(x)=\sum\limits_{n=1}^{+\infty }\left[ \ \ln \left( 1+\dfrac{x}{n}\right) -\dfrac{x}{n}\right] \cdot$

\begin{enumerate}
\item Démontrer que $S$ est dérivable sur $[0,1]$.
\item Calculez $S'(1)$.
\end{enumerate}
\end{Exercice}



\newpage
\begin{Exercice}{} On pose pour tout $x \in \mathbb{R}$,
$$ f(x) = \sum_{n=1}^{+ \infty} \dfrac{\arctan(nx)}{n^2}$$
\begin{enumerate}
\item Déterminer $\lim_{n \rightarrow + \infty} f(x)$ sachant que $\sum_{n=1}^{+ \infty} \dfrac{1}{n^2} = \dfrac{\pi^2}{6} \cdot$
\item Montrer que $f$ est de classe $\mathcal{C}^1$ sur $\mathbb{R}^*$.
\item Déterminer $\lim_{n \rightarrow + \infty} f'(x)$.
\item Que peut-on en déduire sur le graphe de $f$?
\end{enumerate}
\end{Exercice}



\begin{Exercice}{} \begin{enumerate}
\item Déterminer l'ensemble $I$ des réels $x$ pour lesquels la série $\Sum{n\geq 1}{} \dfrac{1}{1+(nx)^2}$ converge.\\
On définit alors la fonction $f$ de $I$ dans $\R$ en posant $f(x)=\sum_{n=1}^{+\infty}\frac{1}{1+(nx)^2}\cdot$
\item Déterminer la parité de $f$ puis son sens de variation de $f$. 
\item Prouver que $f$ est de classe $C^1$ sur $I$.
\item Calculer $\lim_{x\to +\infty}f(x)$.
\item
\begin{enumerate}
\item Vérifier que :
\[\forall p\in \N^*,\ \forall x >0,\ \frac{1}{1+(p+1)^2x^2}\leq \int_p^{p+1}\frac{dt}{1+t^2x^2}\leq \frac{1}{1+p^2x^2}\] 
\item En déduire un équivalent de $f$ au voisinage de $0$.
\end{enumerate}
\item Donner l'allure de la courbe représentative de $f$.
\end{enumerate}
\end{Exercice}


\begin{Exercice}{}  Soit $\Sum{n \geq 1}{} u_n$ la série de fonctions d'une variable réelle de terme général $u_n$ défini pour tout $n \in \N^*$ par : 
$$ \forall x \in \R ,\ u_n(x)=\dfrac{2x}{x^2+n^2\pi^2}$$
\begin{enumerate}
\item Montrer que $\Sum{n \geq 1}{} u_n$ converge simplement sur $\R$. On note $U$ la somme de cette  série de fonctions.
\item Montrer que, pour tout $a > 0$ , $\Sum{n \geq 1}{} u_n$ converge normalement sur $[-a,a]$.
\item La série $\Sum{n \geq 1}{} u_n$ converge-t-elle normalement sur $\mathbb{R}$ ?
\item Montrer que $U$ est continue sur $\R$.
\end{enumerate}
\end{Exercice}


\newpage
\begin{Exercice}{} Soit $S : x \mapsto \sum_{n=0}^{+ \infty}  (-1)^n \dfrac{\cos^3(3^nx)}{3^n} \cdot$

\begin{enumerate}
\item Déterminer l'ensemble de définition, notée $I$, de $S$.
\item Donner une expression simple de $S(x)$ pour $x \in I$.
\end{enumerate}
\end{Exercice}





\begin{Exercice}{} Pour $n\in\N^*,$ on considère la fonction $f_n : \, ] 1,+\infty[ \rightarrow \mathbb{R}$ définie par :
$$ f_n(x)=\frac{(-1)^n}{\ln(nx)}$$

\begin{enumerate}
	\item Justifier que $F : x\longmapsto \sum_{n=1}^{+\infty} f_n(x)$ est bien définie sur $]1,+\infty[.$
	
	\item Démontrer que $\Sum{n\geq 1}{} f_n$ converge uniformément sur $]1,+\infty[.$ 
	
	\item Montrer que $F$ est continue sur $]1,+\infty[$ et déterminer $\lim_{x \rightarrow 1}F(x)$ et  $\lim_{x \rightarrow  +\infty}F(x)$.
	
	\item Démontrer que $F$ est de classe ${\cal C}^1$ sur $]1,+\infty[$ et préciser les variations de $F.$
\end{enumerate}
\end{Exercice}



 

\begin{Exercice}[$\bigstar$] Posons :
$$f: x \mapsto \sum_{n=1}^{+ \infty} \dfrac{1}{n^2+x^2}$$
Montrer que $f$ est définie et de classe $\mathcal{C}^{\infty}$ sur $\mathbb{R}$.
\end{Exercice} 


\end{document}