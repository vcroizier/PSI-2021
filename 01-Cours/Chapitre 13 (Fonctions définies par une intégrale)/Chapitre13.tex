\documentclass[a4paper,10pt]{report}
\usepackage{Cours}
\usepackage{delarray}
\usepackage{fancybox}
\newcommand{\Sum}[2]{\ensuremath{\textstyle{\sum\limits_{#1}^{#2}}}}
\newcommand{\Int}[2]{\ensuremath{\mathchoice%
	{{\displaystyle\int_{#1}^{#2}}}
	{{\displaystyle\int_{#1}^{#2}}}
	{\int_{#1}^{#2}}
	{\int_{#1}^{#2}}
	}}
\usepackage{pstricks-add}



\begin{document}
% \everymath{\displaystyle}

\maketitle{Chapitre 13}{Fonctions définies par une intégrale}

Dans ce chapitre, $I$ est un intervalle de $\mathbb{R}$ non vide et non réduit à un point.
\section{Théorème fondamental de l'analyse}

\begin{Theoreme}{} Soient $f$ une fonction continue sur $I$ à valeurs dans $\mathbb{R}$ et $a \in I$. Alors la fonction 
$$ \begin{array}{cllll}
F & : & I & \rightarrow & \mathbb{R} \\
 & & x & \mapsto & \int_{a}^x f(t) \dt \\
\end{array}$$
est l'unique primitive de $f$ s'annulant en $a$. En particulier $F$ est de classe $\mathcal{C}^1$ sur $I$ et $F'=f$.
\end{Theoreme} 

\begin{Exemple} La fonction $t \mapsto e^{-t^2}$ est continue sur $\mathbb{R}$ donc la fonction $F$ définie pour tout $x \in \mathbb{R}$ par 
$$ F(x) = \int_{0}^x e^{-t^2} \dt$$
est de classe $\mathcal{C}^1$ sur $\mathbb{R}$ et pour tout réel $x$, on a $F'(x)=e^{-x^2}$.
\end{Exemple}

\medskip

\begin{Exemple} Soit $f$ la fonction définie sur $\mathbb{R}$ par $\displaystyle{f(x)=\int_x^{2x} e^{-t^2}\;\text{d}t}$.
\begin{enumerate}
\item Montrons que $f$ est bien définie et que c'est une fonction impaire.

\vspace{5cm}
\item Montrons que $f$ est de classe $\mathcal{C}^1$ sur $\mathbb{R}$ et calculer $f'$.

\vspace{5cm}

\item Étudions les variations de $f$.

\vspace{5cm}
\newpage

\item Étudions les limites de $f$ en $+ \infty$ et $- \infty$.

\vspace{7cm}
\end{enumerate}
\end{Exemple}

\medskip

\begin{Exemple} Soient $f$ une fonction continue et intégrable sur $\mathbb{R}$ et $h$ la fonction définie pour tout réel $x$ par :
$$ h(x) = \int_x^{+ \infty} f(t) \dt$$
Montrons que $h$ est de classe $\mathcal{C}^1$ sur $\mathbb{R}$ et déterminons sa dérivée.


%\begin{Exemple} Soit $f$ définie par $f(x) = \int_{x}^{3x-1} \frac{e^t}{t} dt$. Donnons l'ensemble de définition de $f$, de dérivabilité de $f$ et l'expression de $f'$.

\vspace{11.5cm}
\end{Exemple}

\begin{ApplicationDirecte} Soit $f$ la fonction définie sur $\mathbb{R}^*$ par $f(x) = \int_{x}^{2x} \dfrac{\ch(t)}{t} \dt$.

\begin{enumerate}
\item Justifier que $f$ est bien définie sur $\mathbb{R}^*$. 
\item Étudier la parité de $f$. 
\item Justifier que $f$ est prolongeable par continuité en $0$.
\item Montrer que $f$ est $\mathcal{C}^1$ sur $\mathbb{R}_+^{*}$ et donner pour tout $x>0$, $f'(x)$.
\end{enumerate}
\end{ApplicationDirecte}

\section{Intégrales à paramètres}

Dans cette section, $A$ et $I$ désignent deux intervalles de $\mathbb{R}$ et $\mathbb{K}$ désignera $\mathbb{R}$ ou $\mathbb{C}$. Dans la suite, on considère des fonctions $f : (x,t) \in A \times I \rightarrow \mathbb{K}$.

\subsection{Théorème de continuité}

\begin{Theoreme}{de continuité sous le signe intégrale}

Soit $f : A \times I \rightarrow \mathbb{K}$. On suppose que :
\begin{itemize}
\item Pour tout $x \in A$, $t \mapsto f(x,t)$ est continue par morceaux sur $I$.
\item Pour tout $t \in I$, la fonction $x \mapsto f(x,t)$ est continue sur $A$.
\item Il existe une fonction $\varphi : I \rightarrow \mathbb{R}_+$ continue par morceaux et intégrable sur $I$ telle que pour tout $(x,t) \in A \times I$,
$$ \vert f(x,t) \vert \leq \varphi(t)$$
\end{itemize}
Alors la fonction :
$$ F : x \mapsto \int_{I} f(x,t) \dt$$
est définie et continue sur $A$.
\end{Theoreme}


\begin{Remarques}{}
\begin{itemize}
\item $x$ est la variable de la fonction $F$ et $t$ est la variable d'intégration.
\item La dernière hypothèse s'appelle l'\emph{hypothèse de domination}.
\item Si il est difficile d'obtenir l'hypothèse de domination sur $A$, on peut juste l'obtenir sur tout segment de $A$ (on obtiendra alors la continuité de $F$ sur tout segment de $A$ et donc sur $A$).
\end{itemize}
\end{Remarques}{}

\medskip

\begin{Exemple} Soit $F$ définie par 
$$ F(x) = \int_{1}^{+ \infty} \frac{1}{x+t^3} \dt$$
Justifions que $F$ est bien définie et continue sur $\mathbb{R}_+$.

\vspace{8.5cm}
\end{Exemple}

\begin{Exemple} Soit $F$ définie par 
$$ F(x) = \int_{0}^{+ \infty} \frac{\ln(1+xt)}{1+t^2} \dt$$
Justifions que $F$ est bien définie et continue sur $\mathbb{R}_+$.

\vspace{8.5cm}
\end{Exemple}

\newpage

$\phantom{test}$

\vspace{6cm}

\begin{ApplicationDirecte} Soit $F$ définie par 
$$ F(x) = \int_{0}^{+ \infty} \frac{e^{-xt}}{1+t^2} \dt$$
Justifier que $F$ est bien définie et continue sur $\mathbb{R}_+$.
\end{ApplicationDirecte} 

\subsection{Théorème de dérivation}

\begin{Definition}{} Soit $f : A \times I \rightarrow \mathbb{K}$.

\begin{itemize}
\item Soit $t \in I$. Si la fonction $x \mapsto f(x,t)$ est de classe $\mathcal{C}^1$ sur $A$ alors pour tout $x_0 \in A$, le nombre dérivé de $x \mapsto f(x,t)$ en $x_0$ est noté $\dfrac{\partial f}{\partial x} (x_0,t)$.
\item Si la propriété précédente est vérifiée pour tout $t \in I$, on obtient ainsi une fonction :
$$ \dfrac{\partial f}{\partial x} : (x,t) \in A \times I \mapsto \dfrac{\partial f}{\partial x}(x,t)$$
que l'on appelle la \emph{dérivée partielle} de $f$ par rapport à $x$.
\item En itérant le procédé, on définit de la même manière la dérivée partielle d'ordre $k \geq 2$ par rapport à $x$, que l'on note $\dfrac{\partial^k f}{\partial x^k} \cdot$
\end{itemize}
\end{Definition}

\medskip

\begin{Exemple} Soit $f : \mathbb{R}^2 \rightarrow \mathbb{R}$ définie par $f(x,t) = \frac{e^{-xt}}{1+t^2}\cdot$

\vspace{4cm}
\end{Exemple}

\begin{ApplicationDirecte} Soit $f : \mathbb{R} \times \mathbb{R}_+^{*} \rightarrow \mathbb{R}$ définie par :
$$ f(x,t) = t^x$$
Justifier que $f$ admet une dérivée partielle par rapport à $x$ et donner son expression. 
\end{ApplicationDirecte}

\begin{Theoreme}{dérivation sous le signe intégrale}
Soit $f : A \times I \rightarrow \mathbb{K}$. On suppose que :
\begin{itemize}
\item Pour tout $x \in A$, $t \mapsto f(x,t)$ est continue par morceaux et intégrable sur $I$.
\item Pour tout $t \in I$, $x \mapsto f(x,t)$ est de classe $\mathcal{C}^1$ sur $A$.
\item Pour tout $x \in A$, $t \mapsto \dfrac{\partial f}{\partial x}(x,t)$ est continue par morceaux sur $I$.
\item Il existe une fonction $\varphi : I \rightarrow \mathbb{R}_+$ continue par morceaux et intégrable sur $I$ telle que pour tout $(t,x) \in A \times I$,
$$ \left\vert \dfrac{\partial f}{\partial x}(x,t) \right\vert \leq \varphi(t)$$
\end{itemize}
Alors la fonction 
$$ F : x \mapsto \int_{I} f(x,t) \dt$$
est définie et de classe $\mathcal{C}^1$ sur $A$ et pour tout $x \in A$,
$$ F'(x) = \int_{I}  \dfrac{\partial f}{\partial x}(x,t) \dt$$
\end{Theoreme}

\begin{Remarques}{}
\begin{itemize}
\item L'hypothèse de domination est faite sur la dérivée partielle de $f$ par rapport à $x$, il faut donc remarquer que pour ce théorème, il est important de ne pas oublier de montrer que pour tout $x \in A$, $t \mapsto f(x,t)$ est \emph{intégrable} sur $I$. 
\item Ici aussi, on peut appliquer le théorème sur tout segment de $A$.
\end{itemize}
\end{Remarques}{}

\begin{Exemple} Soit $F$ définie par :
$$ F(x) = \int_{0}^{+ \infty} \frac{\sin(xt)}{t} e^{-t} \dt$$
Justifions que $F$ est de classe $\mathcal{C}^1$ sur $\mathbb{R}$, déterminons $F'$ et déduisons-en une expression simple de $F$.

\vspace{10cm}
\end{Exemple}

\newpage

\begin{ApplicationDirecte} Soit $F$ définie par :
$$ F(x) = \int_{0}^{+ \infty} e^{-t^2} \ch(2xt) \dt$$
Justifier que $F$ est de classe $\mathcal{C}^1$ sur $\mathbb{R}$ (on vérifiera l'hypothèse de domination sur $[-a,a]$ pour tout $a>0$).
\end{ApplicationDirecte}

\subsection{Généralisation : dérivée d'ordre supérieur}
\begin{Theoreme}{}
Soient $f : A \times I \rightarrow \mathbb{K}$ et $k \geq 2$. On suppose que :
\begin{itemize}
\item Pour tout $x \in A$, $t \mapsto f(x,t)$ est continue par morceaux et intégrable sur $I$.
\item Pour tout $t \in I$, $x \mapsto f(x,t)$ est de classe $\mathcal{C}^k$ sur $A$.
\item Pour tout $j \in \Interv{1}{k-1}$, pour tout $x \in A$, $t \mapsto \dfrac{\partial^j f}{\partial x^j}(x,t)$ est continue par morceaux et intégrable sur $I$.
\item Il existe une fonction $\varphi : I \rightarrow \mathbb{R}_+$ continue par morceaux et intégrable sur $I$ telle que pour tout $(t,x) \in A \times I$,
$$ \left\vert \dfrac{\partial^k f}{\partial x^k}(x,t) \right\vert \leq \varphi(t)$$
\end{itemize}
Alors la fonction 
$$ F : x \mapsto \int_{I} f(x,t) \dt$$
est définie et de classe $\mathcal{C}^k$ sur $A$ et pour tout $j \in \Interv{1}{k}$ et tout $x \in A$,
$$ F^{(j)}(x) = \int_{I}  \dfrac{\partial^{j} f}{\partial x^j}(x,t) \dt$$
\end{Theoreme}

\begin{Remarques}{}
\begin{itemize}
\item Encore une fois : on peut travailler sur tout sous-segment de $A$.
\item Pour montrer que $F$ est de classe $\mathcal{C}^{\infty}$ sur $A$, il suffit de montrer que $F$ est de classe $\mathcal{C}^k$ pour tout $k \geq 1$.
\end{itemize}
\end{Remarques}{}
%Centrale PSI 2001
\section{Fonction Gamma}
On définit la fonction $\Gamma$ d'Euler, pour tout réel $x>0$, par :
$$ \Gamma(x) = \int_{0}^{+ \infty} t^{x-1} e^{-t} \dt$$
Montrons que $t \mapsto  t^{x-1} e^{-t}$ est intégrable sur $]0, + \infty[$ si et seulement si $x>0$.

\vspace{6cm}

\newpage

Justifions que $\Gamma$ est de classe $\mathcal{C}^1$ sur $]0, + \infty[$.

\vspace{13cm}



Exprimons, pour tout réel $x>0$, $\Gamma(x+1)$ en fonction de $\Gamma(x)$ et de $x$.

\vspace{7cm}



Déterminer $\Gamma(n)$ pour tout $n \geq 1$.

\vspace{5cm}

\end{document}
