\documentclass[a4paper,10pt]{report}
\usepackage{cours}

\begin{document}
\everymath{\displaystyle}

\begin{center}
\textit{{ {\huge TD 13 : Fonctions définies par une intégrale}}}
\end{center}


\bigskip

\begin{center}
\textit{{ {\large Théorème fondamental de l'analyse}}}
\end{center}

\medskip

\begin{Exa} Soit $f$ définie par 
$$f(x) = \int_{1/x}^{3x}\frac{1}{\sqrt{t^6+1}}\dt$$ 
\begin{enumerate}
\item Déterminer l'ensemble de définition $\mathcal{D}$ de $f$.
\item Étudier la dérivabilité de $f$ sur $\mathcal{D}$ et donner l'expression de sa dérivée.
\end{enumerate}
\end{Exa} 

\corr 

\begin{enumerate}
\item La fonction $h : t \mapsto \dfrac{1}{\sqrt{t^6+1}}$ est continue sur $\mathbb{R}$ (car pour tout réel $t$, $t^6+1>0$). La fonction $f$ n'est pas définie pour $x=0$. Pour tout réel $x>0$, $[1/x,3x]$ (ou $[3x, 1/x]$) est inclus dans $\mathbb{R}_+^*$ donc $h$ est continue sur ce segment et ainsi $f(x)$ existe.  De même, pour $x<0$, on montre que $f(x)$ existe. Ainsi, $f$ est définie sur $\mathcal{D} = \mathbb{R}^*$.
\item Soit $H$ une primitive de $h$ sur $\mathbb{R}$. Pour tout réel $x$ non nul, on a :
$$ f(x) = H(3x)-H(1/x)$$
La fonction $H$ est de classe $\mathcal{C}^1$ sur $\mathbb{R}$ et les fonctions $x \mapsto \dfrac{1}{x}$ et $x \mapsto 3x$ sont de classe $\mathcal{C}^1$ sur $\mathbb{R}^*$ donc par composition, on en déduit que $f$ est de classe $\mathcal{C}^1$ sur $\mathbb{R}^*$ et pour tout réel $x$ non nul,
\begin{align*}
f'(x) & = 3 H'(3x)  + \dfrac{1}{x^2} H'(1/x) \\
& = \dfrac{3}{\sqrt{(3x)^6+1}} + \dfrac{1}{x^2} \times \dfrac{1}{\sqrt{(1/x)^6+1}} 
\end{align*}
\end{enumerate}

\begin{Exa} Justifier que la fonction $H$ définie par 
$$H(x) = \int_{-x}^{x^2} \ln(1+t^4) \dt$$
est de classe $\mathcal{C}^1$ sur $\mathbb{R}$ et donner l'expression de sa dérivée.
\end{Exa}

\corr La fonction $t \mapsto \ln(1+t^4)$ est continue sur $\mathbb{R}$ (car $t \mapsto 1+t^4$ est continue sur $\mathbb{R}$ et à valeurs strictement positives sur $\mathbb{R}$ et que $\ln$ est continue sur $\mathbb{R}_+^{*}$). Les fonctions $x \mapsto x$ et $x \mapsto x^2$ étant définies sur $\mathbb{R}$, pour tout réel $x$,
$$ H(x) = \int_{-x}^{x^2} \ln(1+t^4) \dt$$
existe. Soit $F$ une primitive de $t \mapsto \ln(1+t^4)$ sur $\mathbb{R}$ (qui existe car cette fonction est continue sur $\mathbb{R}$). La fonction $F$ est donc de classe $\mathcal{C}^1$ sur $\mathbb{R}$ et on a pour tout réel $x$,
$$ H(x) = F(x^2)-F(-x)$$
Par composition de fonctions de classe $\mathcal{C}^1$ sur $\mathbb{R}$, $H$ est donc de classe $\mathcal{C}^1$ sur $\mathbb{R}$ et on a pour tout réel $x$,
$$ H'(x) = 2x F'(x^2)-(-F'(-x)) = 2x \ln(1+x^8)+ \ln(1+x^4)$$


\begin{Exa} Trouver toutes les fonctions $f$ continues sur $\mathbb{R}$ telles que pour tout $x \in \mathbb{R}$,
$$ f(x) + \int_{0}^x (x-t)f(t) \dt =1$$
\end{Exa} 

\corr Raisonnons par analyse-synthèse.

\medskip

\noindent \textit{Analyse.} Soit $f$ une fonction continue sur $\mathbb{R}$ telle que pour tout $x \in \mathbb{R}$,
$$ f(x) + \int_{0}^x (x-t)f(t) \dt =1$$
On a alors :
$$ f(x) + x \int_0^x f(t) \dt - \int_0^x t f(t) \dt = 1$$
Les fonctions $f$ et $t \mapsto t f(t)$ sont continues sur $\mathbb{R}$ donc d'après le théorème fondamental de l'analyse, on en déduit que les fonctions :
$$x \mapsto \int_0^x f(t) \dt \, \hbox{ et } \, \int_0^x tf(t) \dt$$ 
sont de classes $\mathcal{C}^1$ sur $\mathbb{R}$ et sont, respectivement, des primitives de ces deux fonctions. Sachant que pour tout $x \in \mathbb{R}$, on a :
$$ f(x) =  -x \int_0^x f(t) \dt + \int_0^x t f(t) \dt - 1$$
On en déduit que $f$ est de classe $\mathcal{C}^1$ sur $\mathbb{R}$ et on a pour tout réel $x$,
$$ f'(x) = - \int_0^x f(t) \dt - xf(x) + xf(x) = - \int_0^x f(t) \dt$$
On en déduit que $f'$ est de classe $\mathcal{C}^1$ sur $\mathbb{R}$ et pour tout réel $x$,
$$ f''(x) = -f(x)$$
Ainsi, $f$ est solution de l'équation différentielle linéaire d'ordre deux homogène suivante :
$$ y''+y=0$$
L'équation caractéristique de cette équation, $x^2+1=0$, admet $i$ et $-i$ pour racines donc il existe deux réels $\lambda$ et $\mu$ tels que pour tout réel $x$,
$$ f(x) = \lambda \cos(x) + \mu \sin(x)$$
Or en reprenant la première égalité, on sait que $f(0)=1$ et la deuxième inégalité implique que $f'(0)=0$. On en déduit que $\lambda=1$ et $\mu=0$. Ainsi, $f$ est la fonction cosinus.

\medskip

\noindent \textit{Synthèse.} Une intégration par parties (bien justifiée) montre que pour tout réel $x$, on a :
$$ \int_0^x t \cos(t) \dt = x \sin(x) - \int_0^x \sin(t) \dt = x \sin(x) + \cos(x) - 1 $$
Ainsi,
\begin{align*}
\cos(x) + x \int_0^x \cos(t) \dt - \int_0^x t \cos(t) \dt & =  \cos(x) + x \int_0^x \cos(t) \dt  - (x \sin(x) + \cos(x) - 1) \\
& = \cos(x) + x \sin(x)  - (x \sin(x) + \cos(x) - 1) \\
& = 1 
\end{align*}
Ainsi, la fonction cosinus est bien solution du problème posé et c'est donc l'unique solution d'après l'analyse.

\begin{Exa} Soit $g : \mathbb{R} \rightarrow \mathbb{R}$ une fonction continue. On pose, pour tout $x \in \mathbb{R}$,
    \[
    f(x) = \int_0^x {\sin(x - t)g(t) \dt}
    \]
    \begin{enumerate}
      \item Montrer que $f$ est dérivable et déterminer sa dérivée.
      \item Montrer que $f$ est solution de l'équation différentielle $y'' + y = g(x)$.
      \item Résoudre cette équation différentielle.
    \end{enumerate}
\end{Exa}

\corr \begin{enumerate}
\item Soit $x \in \mathbb{R}$. On a :
\begin{align*}
f(x) &= \int_0^x \sin(x - t)g(t) \dt \\
& = \int_0^x (\sin(x)\cos(t)-\sin(t)\cos(x))g(t) \dt \\
& = \sin(x) \int_0^x \cos(t)g(t) \dt  -\cos(x) \int_0^x \sin(t)g(t) \dt \\
\end{align*}
La fonction $g$ est continue sur $\mathbb{R}$ donc les fonctions $t \mapsto \cos(t)g(t)$ et $t \mapsto \sin(t)g(t)$ le sont aussi. D'après le théorème fondamental de l'analyse, on en déduit que $f$ est dérivable et on a pour tout réel $x$,
\begin{align*}
f'(x) & = \cos(x) \int_0^x \cos(t)g(t) \dt + \sin(x)\cos(x) g(x) + \sin(x) \int_0^x \sin(t)g(t) \dt - \cos(x) \sin(x)g(x) \\
& = \cos(x) \int_0^x \cos(t)g(t) \dt  + \sin(x) \int_0^x \sin(t)g(t) \dt 
\end{align*}
\item De nouveau en utilisant le théorème fondamental de l'analyse, on sait que $f'$ est de classe $\mathcal{C}^1$ sur $\mathbb{R}$ et on a pour tout réel $x$,
\begin{align*}
f''(x) & = - \sin(x)\int_0^x \cos(t)g(t) \dt + \cos(x)^2 g(x) + \cos(x)  \int_0^x \sin(t)g(t) \dt  + \sin(x)^2 g(x) \\
& =  - \sin(x)\int_0^x \cos(t)g(t) \dt  + \cos(x)  \int_0^x \sin(t)g(t) \dt  + (\cos(x)^2+\sin(x)^2) g(x) \\
& =  - \sin(x)\int_0^x \cos(t)g(t) \dt  + \cos(x)  \int_0^x \sin(t)g(t) \dt + g(x) 
\end{align*}
ce qui donne d'après l'expression de $f(x)$ :
$$ f''(x)+f(x) = g(x)$$
Ainsi, $f$ est solution de $y''+y=g$.
\item L'équation $y''+y=g$ est une équation différentielle linéaire du second ordre à coefficients constants et second membre continue. L'équation homogène associée est :
$$ y''+y=0$$
Elle a pour équation caractéristique $r^2+1=0$ dont les racines sont $-i$ et $i$. Les solutions sont les fonctions de la forme :
$$ x \mapsto A \cos(x) + B \sin(x)$$
où $(A,B) \in \mathbb{R}^2$. La fonction $f$ est une solution particulière de $y''+y=g$ d'après la question précédente. Ainsi, les solutions de cette équation sont les fonctions de la forme :
$$ x \mapsto A \cos(x) + B \sin(x) + f(x)$$
où $(A,B) \in \mathbb{R}^2$.
\end{enumerate}



\bigskip

\begin{center}
\textit{{ {\large Intégrales à paramètres}}}
\end{center}

\medskip

\begin{Exa} Étudier la continuité sur $\mathbb{R}$ de $\dis x \mapsto \int_{0}^1 \sin(xt^2) \dt$.
\end{Exa}

\corr Vérifions les hypothèses du théorème de dérivation sous le signe intégrale.

\begin{itemize}
\item Pour tout réel $x$, $t \mapsto \sin(xt^2)$ est continue sur $[0,1]$.
\item Pour tout réel $t \in [0,1]$,  $x \mapsto \sin(xt^2)$ est continue sur $\mathbb{R}$.
\item Pour tout réel $x$ et tout $t \in [0,1]$,
$$ \vert \sin(x t^2) \vert \leq 1$$
et $t \mapsto 1$ est intégrable sur $[0,1]$ car continue.
\end{itemize}
D'après le théorème de continuité sous le signe intégrale, on en déduit que $f$ est continue sur $\mathbb{R}$.


\begin{Exa} Déterminer l'ensemble de définition de la fonction $f$ définie par :
$$ f(x) = \int_{0}^{+ \infty} \dfrac{\dt}{t^x(1+t)}$$
Étudier ensuite la continuité de $f$ sur cet ensemble de définition.
\end{Exa}

\corr Déterminons l'ensemble de définition de $f$. Soit $x \in \mathbb{R}$. La fonction :
$$ t \mapsto  \dfrac{1}{t^x(1+t)}$$
est continue  et positive sur $]0, + \infty[$ donc l'intégrale définissant $f(x)$ est impropre en $0$ et $+ \infty$. On a :
$$  \dfrac{1}{t^x(1+t)} \underset{t \rightarrow 0}{\sim} \dfrac{1}{t^x}$$
L'intégrale $\dis \int_{0}^1 \dfrac{1}{t^x} \dt$ converge si et seulement si $x<1$ (intégrale de référence). Par critère de comparaison (les intégrandes sont positives), on en déduit que $\dis \int_{0}^1 \dfrac{1}{t^x(1+t)} \dt$ converge si et seulement si $x<1$.

\medskip

\noindent On a aussi :
$$   \dfrac{1}{t^x(1+t)} \underset{t \rightarrow + \infty}{\sim} \dfrac{1}{t^{x+1}}$$
L'intégrale $\dis \int_{1}^{+ \infty} \dfrac{1}{t^{x+1}} \dt$ converge si et seulement si $x+1>1$ (intégrale de référence). Par critère de comparaison (les intégrandes sont positives), on en déduit que $\dis \int_{1}^{+ \infty} \dfrac{1}{t^x(1+t)} \dt$ converge si et seulement si $x>0$.

\medskip

\noindent Finalement l'intégrale définissant $f(x)$ converge si et seulement si $x<1$ et $x>0$ donc $f$ est définie sur $]0,1[$.

\medskip

\noindent Étudions maintenant la continuité de $f$ sur $]0,1[$.
\begin{itemize}
\item Pour tout $x \in ]0,1[$, la fonction $t \mapsto \dfrac{1}{t^x(1+t)}$ est continue sur $\mathbb{R}_+^{*}$.
\item Pour tout $t \in \mathbb{R}_+^{*}$, la fonction $x \mapsto  \dfrac{1}{t^x(1+t)}$ est continue sur $]0,1[$.
\item Soit $[a,b] \subset ]0,1[$. Soit $x \in [a,b]$. Si $t \in ]0,1]$ alors $\ln(t) \leq 0$ et ainsi :
$$ t^x = e^{x \ln(t)} \geq e^{b \ln(t)} = t^b$$
puis $t^x(1+t) \geq t^b (1+t)>0$ et par décroissance de la fonction inverse sur $\mathbb{R}_+^{*}$ :
$$ \dfrac{1}{t^x(1+t)} \leq \dfrac{1}{t^b(1+t)}$$
Si $t \geq 1$, on obtient de la même manière que :
$$  \dfrac{1}{t^x(1+t)} \leq \dfrac{1}{t^a(1+t)}$$
Posons alors $\varphi : \mathbb{R}_+^{*} \rightarrow \mathbb{R}$ définie pour tout $t>0$ par :
$$ \varphi(t) = \left\lbrace  \begin{array}{cl}
\dfrac{1}{t^b(1+t)} & \hbox{ si } t \in ]0,1] \\
\dfrac{1}{t^a(1+t)} & \hbox{ si } t \in [1, + \infty[ 
\end{array}\right.$$
Alors $\varphi$ est continue par morceaux sur $]0, + \infty[$ et intégrable sur $\mathbb{R}_+^{*}$ (il suffit de refaire le raisonnement prouvant la convergence de l'intégrale définissant $f(x)$).
\end{itemize}
Par théorème de continuité sous le signe intégrale, on en déduit que $f$ est continue sur son ensemble de définition.



\begin{Exa} 
\begin{enumerate}
\item Montrer que l'ensemble de définition de $g$, définie par 
$$ g(x) = \int_{0}^1 \dfrac{\ln(1+xt)}{t} \dt$$
contient $]-1,1[$.
\item Montrer que $g$ est de classe $\mathcal{C}^1$ sur $]0,1[$ et déterminer $g'$.
\end{enumerate}
\end{Exa}

\corr 
\begin{enumerate}
\item Soit $x \in ]-1,1[$. Pour tout $t \in ]0,1]$, $xt \in ]-1,1[$ donc $1+xt >0$ et ainsi $t \mapsto \dfrac{\ln(1+xt)}{t} $ est continue sur $]0,1]$. On a :
$$ \dfrac{\ln(1+xt)}{t}  \underset{0}{\sim} \dfrac{xt}{t}= x$$
donc la fonction $t \mapsto \dfrac{\ln(1+xt)}{t} $ est prolongeable par continuité en $0$ et ainsi l'intégrale définissant $g(x)$ est faussement impropre en $0$ donc convergente. Ainsi, $]-1,1[ \subset \mathcal{D}_g$.

\item Utilisons le théorème de dérivation pour des intégrales à paramètres :
\begin{itemize}
\item Pour tout $x \in ]0,1[$, $t \mapsto \dfrac{\ln(1+xt)}{t}$ est continue sur $]0,1]$ et intégrable sur $]0,1]$ (on a déjà montré que l'intégrale convergeait et l'intégrande est positive).
\item Pour tout $t \in ]0,1]$, $x \mapsto \dfrac{\ln(1+xt)}{t}$ est $\mathcal{C}^1$ sur $]0,1[$ de dérivée $x \mapsto \dfrac{1}{1+xt}$ qui est une fonction continue sur $]0,1[$.
\item Pour tout $t \in ]0,1]$ et tout $x \in ]0,1]$,
$$ 0 \leq \dfrac{1}{1+xt} \leq 1$$
et la fonction $t \mapsto 1$ est intégrable sur $]0,1]$.
\end{itemize}
Par théorème de dérivation pour les intégrales à paramètres, on en déduit que pour tout $x \in ]0,1[$,
$$ g'(x) = \int_{0}^1 \dfrac{1}{1+xt} \dt$$
et ainsi :
$$ g'(x) = \dfrac{1}{x} [ \ln(1+xt) ]_0^1 = \dfrac{\ln(1+x)}{x}$$
\end{enumerate}

\begin{Exa} Soit $f : \dis x \mapsto \int_{0}^{+ \infty} \dfrac{\arctan(xt)}{t(1+t^2)} \dt$.
\begin{enumerate}
\item Déterminer l'ensemble de définition de $f$, notée $\mathcal{D}$.
\item Montrer que $f$ est de classe $\mathcal{C}^1$ sur $\mathcal{D}$.
\item Donner l'expression de $f$ à l'aide de fonctions usuelles.
\end{enumerate}
\end{Exa}

\corr  
\begin{enumerate}
\item Posons pour tout $(x,t) \in \mathbb{R} \times \mathbb{R}_+^{*}$,
$$ f(x,t) = \dfrac{\arctan(xt)}{t(1+t^2)}$$
Pour tout $x \in \mathbb{R}$, $t \mapsto f(x,t)$ est continue sur $\mathbb{R}_+^{*}$ et :
$$ f(x,t) \underset{ t \rightarrow 0 }{\sim} \dfrac{xt}{t(1+t^2)} = \dfrac{x}{1+t^2}$$
Ainsi,
$$ \lim_{t \rightarrow 0 } f(x,t) = x$$
On en déduit que $\dis \int_{0}^1 f(x,t) \dt$ est faussement impropre en $0$ donc convergente. On a pour tout $t \geq 1$,
$$ \vert f(x,t) \vert \leq \dfrac{\pi}{2} \times \dfrac{1}{1+t^2} \leq \dfrac{\pi}{2t^2}$$
Par critère de comparaison, on en déduit que $t \mapsto f(x,t)$ est intégrable sur $[1,+ \infty[$ et donc $\dis \int_{1}^{+ \infty} f(x,t) \dt$ est absolument convergente donc convergente.

\medskip

\noindent Finalement, $f$ est définie sur $\mathbb{R}$.
\item Vérifions les hypothèses du théorème de dérivation sous le signe intégrale.
\begin{itemize}
\item Pour tout $x \in \mathbb{R}$, $t \mapsto f(x,t)$ est continue et intégrable sur $]0, + \infty[$ (d'après la question précédente).
\item Pour tout $t \in \mathbb{R}_+^{*}$, $x \mapsto f(x,t)$ est de classe $\mathcal{C}^1$ sur $\mathbb{R}$, et on a pour tout $x \in \mathbb{R}$ :
$$ \dfrac{\partial f(x,t)}{\partial x} = \dfrac{1}{(1+(xt)^2)(1+t^2)}$$
Pour tout $x \in \mathbb{R}$,
$$ t \mapsto \dfrac{1}{(1+(xt)^2)(1+t^2)}$$
est continue sur $]0,+ \infty[$. On a :
$$ \left\vert \dfrac{\partial f(x,t)}{\partial x} \right\vert \leq \dfrac{1}{1+t^2}$$
La fonction $t \mapsto \dfrac{1}{1+t^2}$ est continue et intégrable sur $]0, + \infty[$ car :
$$ \lim_{A + \rightarrow + \infty} \int_{0}^A \dfrac{1}{1+t^2} \dt = \lim_{A \rightarrow + \infty} \arctan(A) = \dfrac{\pi}{2}$$
\end{itemize}
Ainsi, par théorème de dérivabilité sous le signe intégrale, $f$ est de classe $\mathcal{C}^1$ sur $\mathbb{R}$ et on a pour tout $x \in \mathbb{R}$,
$$ f'(x) = \int_{0}^{+ \infty} \dfrac{1}{(1+(xt)^2)(1+t^2)} \dt$$
\item La fonction $f$ est impaire (car $\arctan$ l'est). Déterminons $f(x)$ pour $x \geq 0$. Soit $x \in \mathbb{R}_+$. Si $x$ est différent de $1$, une décomposition en éléments simples montre que pour tout $t \in \mathbb{R}$,
$$ \dfrac{1}{(1+(xt)^2)(1+t^2)} = \dfrac{1}{x^2-1} \left( x\dfrac{x}{1+(xt)^2} - \dfrac{1}{1+t^2} \right)$$
Donc pour tout $A>0$,
$$ \int_{0}^A \dfrac{1}{(1+(xt)^2)(1+t^2)} \dt = \dfrac{1}{x^2-1} [ x \arctan(xt)- \arctan(t) ]_0^1 = \dfrac{x\arctan(xA) - \arctan(A)}{x^2-1}$$
Par passage à la limite, on en déduit que :
$$ f'(x) = \dfrac{\pi}{2} \times \dfrac{x-1}{x^2-1} = \dfrac{\pi}{2} \times \dfrac{1}{x+1}$$
Remarquons que cette égalité est encore vérifiée pour $x=1$ par continuité de $f'$ en $1$. Ainsi, il existe $C \in \mathbb{R}$ tel que pour tout $x \geq 0$,
$$ f(x) = \dfrac{\pi}{2} \ln(x+1) + C$$
On sait que $f(0)=0$ donc $C=0$. Ainsi, pour tout $x \in \mathbb{R}_+$,
$$ f(x) = \dfrac{\pi}{2} \ln(x+1) $$
La fonction $f$ est impaire donc si $x \in \mathbb{R}_{-}$,
$$ f(x) = - f(-x) = -\dfrac{\pi}{2} \ln(1-x)$$
\end{enumerate}


\begin{Exa}
\begin{enumerate}
\item Déterminer l'ensemble de définition de la fonction $f$ définie par :
$$ f(x) = \int_{0}^{+ \infty} e^{-t^2} \cos(xt) \dt$$
\item Déterminer $f$ en vérifiant que celle-ci est solution d'une équation différentielle d'ordre $1$.
\end{enumerate}
\end{Exa}

\corr \begin{enumerate}
\item Pour tout réel $x$, la fonction :
$$t \mapsto e^{-t^2} \cos(xt)$$
est continue sur $\mathbb{R}_+$ et on a par théorème des croissances comparées :
$$ e^{-t^2} \cos(xt)  \underset{t \rightarrow + \infty}{=} o \left( \dfrac{1}{t^2} \right)$$
Par critère de comparaison on en déduit que $t \mapsto e^{-t^2} \cos(xt)$ est intégrable sur $[1, + \infty[$ et donc sur $\mathbb{R}_+$ par continuité de celle-ci sur $\mathbb{R}_+$. L'intégrale définissant $f(x)$ converge donc pour tout réel $x$ et donc $f$ est définie sur $\mathbb{R}$.

\item Montrons que $f$ est de classe $\mathcal{C}^1$ sur $\mathbb{R}$.
\begin{itemize}
\item Pour tout $x \in \mathbb{R}$, $t \mapsto e^{-t^2} \cos(xt)$ est continue et intégrable sur $\mathbb{R}_+$ d'après la question précédente.
\item Pour tout $t \in \mathbb{R}_+$, $x \mapsto e^{-t^2} \cos(xt)$ est $\mathcal{C}^1$ sur $\mathbb{R}$ donc la fonction :
$$g: (x,t) \mapsto e^{-t^2} \cos(xt)$$
admet une dérivée partielle par rapport à la première variable et on a pour tout $(x,t) \in \mathbb{R} \times \mathbb{R}_+$,
$$ \dfrac{\partial f (x,t)}{\partial x} = -t e^{-t^2} \sin(xt)$$
\item Pour tout $x \in \mathbb{R}$,
$$ t \mapsto -t e^{-t^2} \sin(xt)$$
est continue par morceaux sur $\mathbb{R}_{+}$. On a pour tout $(x,t) \in \mathbb{R} \times \mathbb{R}_+$,
$$ \vert  -t e^{-t^2} \sin(xt) \vert \leq t e^{-t^2} $$
et la fonction $t \mapsto t e^{-t^2}$ est continue sur $\mathbb{R}_+$ et intégrable sur $\mathbb{R}_+$ (même preuve que pour $t \mapsto e^{-t^2}$).
\end{itemize}
Par théorème de dérivation sous le signe intégrale, on en déduit que $f$ est $\mathcal{C}^1$ sur $\mathbb{R}$ et que pour tout $x \in \mathbb{R}$,
$$f'(x) =  \dis \int_{0}^{+ \infty} -t e^{-t^2} \sin(xt) \dt$$
Soit $x \in \mathbb{R}$. Les fonctions $t \mapsto \dfrac{1}{2} e^{-t^2}$ et $t \mapsto \sin(xt)$ sont de classe $\mathcal{C}^1$ sur $\mathbb{R}_+$ donc par intégration par parties, on a pour tout $A>0$ :
$$ \int_{0}^{A}  -t e^{-t^2} \sin(xt) \dt = \left[\dfrac{1}{2} e^{-t^2}  \sin(xt) \right]_0^A - \dfrac{x}{2} \int_{0}^A e^{-t^2} \cos(xt) \dt$$ 
Par passage à la limite quand $A$ tend vers $+ \infty$ (la fonction sinus étant bornée sur $\mathbb{R}$), on en déduit que :
$$ f'(x) = - \dfrac{x}{2} f(x)$$
Ainsi $f$ sur solution sur $\mathbb{R}$ de l'équation différentielle linéaire homogène sur premier ordre $y'+ \dfrac{x}{2}y=0$ dont les solutions sont de la forme :
$$ x \mapsto K e^{- \dfrac{x^2}{4}}$$
ou $K \in \mathbb{R}$. Ainsi, il existe $K \in \mathbb{R}$ tel que pour tout $x \in \mathbb{R}$,
$$ f(x) = Ke^{- \dfrac{x^2}{4}}$$
Or $f(0)= \dis \int_{0}^{+ \infty} e^{-t^2} \dt = \dfrac{\sqrt{\pi}}{2}$ (intégrale \og classique \fg) donc on en déduit que pour tout $x \in \mathbb{R}$,
$$ f(x) = \dfrac{\sqrt{\pi}}{2 } e^{- \dfrac{x^2}{4}}$$
\end{enumerate}

\begin{Exa} Soit $f$ la fonction d\'efinie par 
$$f(x)=\int_0^1 \frac{t^x}{1+t} \dt$$
	\begin{enumerate}
	\item Montrer que $f$ est bien d\'efinie et continue sur $I=]-1, + \infty[$.

	\item Montrer que :
$$\forall x\in I,\quad f(x+1)+f(x)=\frac1{x+1}$$

	\item Montrer que :
$$\forall x\in I,\quad 0\leq f(x) \leq \frac1{x+1}$$

	\item Montrer que la fonction $f$ est d\'ecroissante sur $I$.

	\item En d\'eduire que :
$$\forall x\in]0,+\infty[,\quad \frac1{2(x+1)} \leq f(x) \leq \frac1{2x}$$

	\item Donner, en le justifiant, un \'equivalent simple de $f(x)$ au voisinage de $+\infty$.
	\end{enumerate}
\end{Exa}

\corr 

	\begin{enumerate}
	\item Soit $x \in I$. La fonction $t\mapsto\dfrac{t^x}{1+t}$ est continue et positive sur $]0,1]$. On a :
	$$ \frac{t^{x}}{1+t} \underset{t \rightarrow 0}{\sim} \frac1{t^{-x}}$$
	et d'après le résultat sur les intégrales de Riemann, $t \mapsto \dfrac{1}{t^{-x}}$ est intégrable si et seulement si $-x<-1$ ou encore $x>-1$ (et dans ce cas l'intégrale converge donc $f(x)$ existe). Ainsi, $f$ est bien d\'efinie sur $I$.
	
	\medskip

\noindent Utilisons le théorème de continuité sous le signe intégrale.

\begin{itemize}
\item Pour tout $x \in I$, $t \mapsto \dfrac{t^x}{1+t}$ est continue sur $]0,1]$.
\item Pour tout $t \in ]0,1]$, $x \mapsto  \dfrac{t^x}{1+t}$ est continue sur $I$.
\item Soit $a>-1$. Pour tout $t \in ]0,1]$ et tout $x \in [a, + \infty[$, sachant que $\ln(t) \leq 0$,
\begin{align*}
 \left\vert \dfrac{t^x}{1+t} \right\vert & = \dfrac{e^{x \ln(t)}}{1+t} \\
 & \leq \dfrac{e^{a \ln(t)}}{1+t} \\
 & = \dfrac{t^a}{1+t}
 \end{align*}
La fonction $t \mapsto \dfrac{t^a}{1+t}$ est intégrable sur $]0,1]$ (fonction positive, continue par morceaux et son intégrable converge d'après le raisonnement effectué pour montrer que $f$ est bien définie sur $I$).
\end{itemize}
D'après le théorème de continuité pour les intégrales à paramètres, on en déduit que $f$ est continue sur $[a, + \infty[$ pour tout réel $a>-1$ donc $f$ est continue sur $I$.


	\item Soit $x \in I$.  Alors :
$$f(x+1)+f(x)=\int_0^1 \frac{t^x(t+1)}{1+t} \dt =\int_0^1t^x \dt=\frac1{x+1}$$

	\item Pour tout $t\in[0,1]$, sachant que $1+t \geq 1>0$, on a :
	$$ \ 0\le \frac{t^x}{1+t} \le t^x$$
Donc par croissance de l'intégrale (les bornes sont dans le bon sens), on en déduit que : 
$$ \forall x\in I,\quad 0\le f(x) \le \frac1{x+1}$$

	\item Soit ($x,y$)$\in I^2$ tel que $x\le y$. Par croissance de la fonction exponentielle sur $\mathbb{R}$, on a :
$$\forall t\in[0,1], \ \frac{t^x}{1+t} \ge \frac{t^y}{1+t}$$
Donc par croissance de l'intégrale (les bornes sont dans le bon sens), on en déduit que $f(x)\ge f(y)$. Ainsi, $f$ est d\'ecroissante sur $I$.

	\item La fonction $f$ \'etant d\'ecroissante sur $I$ donc pour tout $x \in I$, on en d\'eduit que 
	$$2f(x+1) \le f(x+1)+f(x) \le 2f(x)$$
Ainsi, d'après la question 2, 
$$f(x)\ge \frac1{2(x+1)}$$
 et $$f(x+1) \le \frac1{2(x+1)}$$
ou encore $\dis f(x) \le \frac1{2x}$ (changement de variable).

	\item On a pour tout $x>0$,
$$\frac{2x}{2x+2} \le 2xf(x) \le 1$$ 
donc par théorème d'encadrement, on en déduit que :
$$\lim\limits_{x\to+\infty}2xf(x)=1$$
et ainsi :
$$ f(x) \underset{+\infty}\sim \frac1{2x}$$
	\end{enumerate}



\begin{Exa} On considère la fonction $f : x \mapsto \int_0^{+ \infty} \dfrac{e^{-xt}}{\sqrt{1+t}} \dt$.
\begin{enumerate}
\item Déterminer l'ensemble de définition de $f$. On le notera $\mathcal{D}$.
\item Montrer que $f$ est de classe $\mathcal{C}^1$ sur $\mathcal{D}$ et est solution d'équation différentielle linéaire que l'on précisera.
\item Déterminer les limites de $f$ aux bornes de $\mathcal{D}$.
\end{enumerate}
\end{Exa}

\corr 

\begin{enumerate}
\item Pour tout réel $x$,
$$ t \mapsto \dfrac{e^{-xt}}{\sqrt{1+t}}$$
est continue sur $\mathbb{R}_+$.

\begin{itemize}
\item Soit $x \in \mathbb{R}_{-}$. Alors pour tout réel $t \geq 1$,
$$ \dfrac{e^{-xt}}{\sqrt{1+t}} \geq \dfrac{1}{\sqrt{1+t}}$$
et 
$$ \dfrac{1}{\sqrt{1+t}} \underset{+ \infty}{\sim} \dfrac{1}{\sqrt{t}}$$
On sait que $\dis \int_1^{+ \infty} \dfrac{1}{\sqrt{t}} \dt$ est une intégrale de Riemann divergente donc par critère de comparaison (les fonctions sont positives sur $[1, + \infty[$), $\dis \int_1^{+ \infty}  \dfrac{1}{\sqrt{1+t}}  \dt$ est divergente et de nouveau par critère de comparaison, $\dis \int_1^{+ \infty} \dfrac{e^{-xt}}{\sqrt{1+t}}$ diverge et ainsi, $f(x)$ n'est pas défini.
\item Soit $x \in \mathbb{R}_+^*$. Pour tout réel positif $t$,
$$ 0 \leq  \dfrac{e^{-xt}}{\sqrt{1+t}} \leq e^{-xt}$$
L'intégrale de référence $\dis \int_0^{+ \infty} e^{-xt} \dt$ est convergente (car $x>0$) donc par critère de comparaison (les fonctions sont positives), $ \int_0^{+ \infty} \dfrac{e^{-xt}}{\sqrt{1+t}}$ converge et ainsi, $f(x)$ est bien défini.
\end{itemize}
Finalement, $f$ est définie sur $\mathcal{D}= \mathbb{R}_+^*$.
\item Soit $g : \mathbb{R}_+^* \times \mathbb{R}_+ \rightarrow \mathbb{R}$ définie par :
$$ \forall (x,t) \in \mathbb{R}_+^* \times \mathbb{R}_+, \; g(x,t) = \dfrac{e^{-xt}}{\sqrt{1+t}} $$
Vérifions les hypothèses du théorème de dérivation sous le signe intégrale.
\begin{itemize}
\item Pour tout réel $x>0$, $t \mapsto g(x,t)$ est continue sur $\mathbb{R}_+$ et intégrable (d'après la première question).
\item Pour tout réel $t \geq 0$, $x \mapsto g(x,t)$ est de classe $\mathcal{C}^1$ sur $\mathbb{R}_+^*$. On a :
$$ \forall (x,t) \in \mathbb{R}_+^* \times \mathbb{R}_+, \; \dfrac{\partial g}{\partial x}(x,t) =  -  \dfrac{te^{-xt}}{\sqrt{1+t}}$$
\item Pour tout $x \in \mathbb{R}_+^*$,
$$ t \mapsto   -  \dfrac{te^{-xt}}{\sqrt{1+t}}$$
est continue sur $\mathbb{R}_+$ et on a pour tout $t \in \mathbb{R}_+$,
$$ \left\vert  -  \dfrac{te^{-xt}}{\sqrt{1+t}} \right\vert \leq  \dfrac{te^{-xt}}{\sqrt{1+t}} \leq t e^{-xt}$$
Soit $a>0$. Pour tout réel $x \geq a$, on a alors :
$$ \left\vert  -  \dfrac{te^{-xt}}{\sqrt{1+t}} \right\vert \leq  t e^{-at}$$
La fonction $t \mapsto t e^{-at}$ est continue sur $\mathbb{R}_+$ et on a par théorème des croissances comparées :
$$ t e^{-at} \underset{+ \infty}{=} o \left( \dfrac{1}{t^2} \right)$$
La fonction $t \mapsto 1/t^2$ est intégrable sur $[1, + \infty[$ (fonction de référence) donc par critère de comparaison, $t \mapsto t e^{-at}$ aussi. Par continuité sur $[0,1]$, elle est intégrable sur $\mathbb{R}_+$.
\end{itemize}
Par théorème de dérivation sous le signe intégrale, on en déduit que pour tout réel $a>0$, $f$ est de classe $\mathcal{C}^1$ sur $]a, + \infty[$. Ainsi, $f$ est de classe $\mathcal{C}^1$ sur $\mathbb{R}_+^*$ et on a pour tout réel $x>0$,
$$ f'(x) = \int_{0}^{+ \infty}   -  \dfrac{te^{-xt}}{\sqrt{1+t}} \dt$$
On a alors :
\begin{align*}
f'(x) & = \int_{0}^{+ \infty}   -  \dfrac{((t+1)-1)e^{-xt}}{\sqrt{1+t}} \dt \\
& = -\int_{0}^{+ \infty} \sqrt{1+t} e^{-xt}  -  \dfrac{e^{-xt}}{\sqrt{1+t}} \dt \\
& = -\int_{0}^{+ \infty} \sqrt{1+t} e^{-xt} \dt + f(x)
\end{align*}
Le fait de découper l'intégrale est licite car $f'(x)$ et $f(x)$ sont des intégrales convergentes. Pour tout réel $A>0$, par intégration par parties (bien justifiée), on a :
\begin{align*}
\int_{0}^{A} \sqrt{1+t} e^{-xt} \dt & = \left[ -\sqrt{1+t} \times \dfrac{e^{-xt}}{x} \right]_0^A + \int_0^A \dfrac{1}{2\sqrt{1+t}} \times  \dfrac{e^{-xt}}{x} \dt \\
& =  -\sqrt{1+A} \times \dfrac{e^{-xA}}{x} + \dfrac{1}{x} + \dfrac{1}{2x}\int_0^A \dfrac{e^{-xt}}{\sqrt{1+t}} \dt \\
\end{align*}
Par passage à la limite quand $A$ tend vers $+ \infty$ (les intégrales convergent) et en utilisant le théorème des croissances comparées ($x>0$) on a :
$$ \int_{0}^{+ \infty} \sqrt{1+t} e^{-xt} \dt = \dfrac{1}{x} + \dfrac{f(x)}{2x}$$
Finalement, on a pour tout réel $x>0$,
$$ f'(x) = - \dfrac{1}{x} + f(x) \left( - \dfrac{1}{2x}+1 \right)$$
Ainsi, $f$ est solution sur $\mathbb{R}_+^*$ de l'équation différentielle suivante :
$$ y' = - \dfrac{1}{x} + \dfrac{2x-1}{2x}y$$
\item Déterminons la limite en $+ \infty$. Pour tout réel $t \geq 0$,
$$ \sqrt{1+t} \geq 1$$
donc par décroissance de la fonction inverse sur $\mathbb{R}_+^*$ :
$$ 0 \leq \dfrac{1}{\sqrt{1+t}} \leq 1$$
Pour tout réel $x>0$, $e^{-xt} > 0$ donc :
$$  0 \leq \dfrac{e^{-xt}}{\sqrt{1+t}} \leq e^{-xt}$$
Par croissance de l'intégrale (les intégrales convergent sachant que $x>0$ pour la deuxième) :
$$ 0 \leq f(x) \leq \int_0^{+ \infty} e^{-xt} \dt$$
On a :
\begin{align*}
\int_0^{+ \infty} e^{-xt} \dt & =\lim_{A \rightarrow + \infty} \int_0^{A} e^{-xt} \dt \\
& = \lim_{A \rightarrow + \infty} \left[ \dfrac{e^{-tx}-1}{-x} \right]_0^A \\
&  = \lim_{A \rightarrow + \infty} \dfrac{e^{-Ax}-1}{-x} \\
& = \dfrac{1}{x}
\end{align*}
Ainsi, pour tout réel $x>0$,
$$ 0 \leq f(x) \leq \dfrac{1}{x}$$
On en déduit par théorème d'encadrement que $f$ admet une limite en $+ \infty$ et :
$$ \lim_{x \rightarrow + \infty} f(x) = 0$$
Déterminons la limite en $0$. Soit $A >0$. Pour tout réel $t \in [0,A]$:
$$ 0 \leq t \leq A$$
donc par décroissance de $u \mapsto e^{-ux}$ sur $\mathbb{R}$ (pour $x>0)$, on a :
$$  e^{-tx} \geq  e^{-xA}$$
Par positivité de l'intégrande, on a :
\begin{align*}
 f(x) & \geq \int_0^{A} \dfrac{e^{-xt}}{\sqrt{1+t}}  \dt  \\
 & \geq e^{-xA} \int_0^A \dfrac{1}{\sqrt{1+t}}  \dt  \\ 
 & = e^{-xA} (2 \sqrt{1+A}-2) 
\end{align*}
Remarquons pour tout $(x,y) \in (\mathbb{R}_+^*)^2$, si $x<y$ alors pour tout réel $t \geq 0$,
$$ e^{-xt} \geq e^{-yt}$$
donc 
$$ \dfrac{e^{-xt}}{\sqrt{1+t}} \geq \dfrac{e^{-yt}}{\sqrt{1+t}}$$
puis par croissance de l'intégrale :
$$ f(x) \geq f(y)$$
Ainsi, $f$ est décroissante sur $\mathbb{R}_+^*$ donc admet une limite en $0$. Supposons que $f$ admette une limite finie en $0$, alors sachant que pour tout réel $A>0$ et tout réel $x>0$,
$$ f(x) \geq  e^{-xA} (2 \sqrt{1+A}-2) $$
On en déduit que :
$$ \lim_{x \rightarrow 0} f(x) \geq \lim_{x \rightarrow 0} e^{-xA} (2 \sqrt{1+A}-2) = 2 \sqrt{1+A}-2$$
Or :
$$ \lim_{A \rightarrow + \infty} 2 \sqrt{1+A}-2 = + \infty$$
C'est absurde. Ainsi, $f$ admet pour limite $+ \infty$ en $0$.
\end{enumerate}

\begin{Exa} Pour tout $x \in \mathbb{R}_+$, on pose :
 $$ F(x) = \int_{0}^1 \frac{e^{-x^2(t^2+1)}}{t^2+1} \dt, \; G(x) = \left( \int_{0}^x e^{-t^2} \dt \right)^2, \; H(x)=F(x)+G(x)$$
 
 \begin{enumerate}
 \item Montrer que $F$ est de classe $\mathcal{C}^1$ sur $\mathbb{R}_+$ et donner une expression sous forme d'une intégrale.
 \item Montrer que $G$ est de classe $\mathcal{C}^1$ sur $\mathbb{R}_+$ et donner une expression sous forme d'une intégrale.
 \item Justifier que $H$ est une fonction constante et déterminer celle-ci.
 \item En déduire la valeur de l'intégrale de Gauss $\dis \int_{0}^{+ \infty} e^{-t^2} \dt$.
 \end{enumerate}
\end{Exa}

\corr 

\begin{enumerate}
\item Vérifions les hypothèses du théorème de dérivation sous le signe intégrale.
\begin{itemize}
\item Pour tout réel $x \in \mathbb{R}_+$, 
$$ t \mapsto \frac{e^{-x^2(t^2+1)}}{t^2+1}$$
est continue sur $[0,1]$ (donc intégrable sur $[0,1]$).
\item Pour tout réel $t \in [0,1]$,
$$ x \mapsto \frac{e^{-x^2(t^2+1)}}{t^2+1}$$
est dérivable sur $\mathbb{R}_+$, de dérivée :
$$ t \mapsto - 2xe^{-x^2(t^2+1)}$$
\item Pour tout réel $x \geq 0$,
$$ t \mapsto - 2xe^{-x^2(t^2+1)} $$
est continue sur $[0,1]$ et on a pour tout $t \in [0,1]$,
$$ \vert  - 2xe^{-x^2(t^2+1)} \vert \leq 2$$
La fonction $t \mapsto 2$ est intégrable sur $[0,1]$.
\end{itemize}
Par théorème de dérivation sous le signe intégrale, on en déduit que $F$ est de classe $\mathcal{C}^1$ sur $\mathbb{R}_+$ et on a pour tout réel $x \geq 0$,
$$ F'(x) = - \int_0^1 2x e^{-x^2(t^2+1)} \dt = - 2xe^{-x^2} \int_0^1  e^{-(xt)^2} \dt$$
\item La fonction $t \mapsto e^{-t^2}$ est continue sur $\mathbb{R}_+$. Par théorème fondamental de l'analyse (et par produit), on en déduit que $G$ est de classe $\mathcal{C}^1$ sur $\mathbb{R}_+$ et on a pour tout réel $x \geq 0$,
$$ G'(x) = 2 e^{-x^2} \int_0^x e^{-t^2} \dt$$
\item La fonction $H$ est de classe $\mathcal{C}^1$ sur $\mathbb{R}_+$ par somme de fonctions qui le sont. Remarquons que par le changement de variable $  u : t \mapsto xt$, on a :
$$ x\int_0^1  e^{-(xt)^2} \dt = \int_0^x  e^{-u^2} \dt$$ 
et ainsi :
$$ F'(x) = -2e^{-x^2} \int_0^x  e^{-u^2} \dt$$ 
On en déduit que :
$$ F'(x)+G'(x) = 0$$
Ainsi, $H$ a une dérivée nulle sur l'intervalle $\mathbb{R}_+$. Il existe donc un réel $C$ tel que pour tout $x \in \mathbb{R}_+$,
$$ H(x) = C$$
On a :
$$ H(0) = \int_{0}^1 \frac{1}{t^2+1} \dt = \arctan(1) - \arctan(0) = \dfrac{\pi}{4}$$
Ainsi, pour tout réel $x \geq 0$,
$$ H(x) = \dfrac{\pi}{4}$$
\item La fonction $t \mapsto e^{-t^2}$ est continue sur $\mathbb{R}_+$ et par théorème des croissances comparées :
$$ e^{-t^2} \underset{+ \infty}{=} o \left( \dfrac{1}{t^2} \right)$$
La fonction $t \mapsto 1/t^2$ est intégrable sur sur $[1, + \infty[$ donc par critère de comparaison, $t \mapsto e^{-t^2}$ l'est aussi. Par continuité sur $[0,1]$, $t \mapsto e^{-t^2}$ est intégrable sur $\mathbb{R}_+$. Ainsi, $G$ admet une limite finie en $+ \infty$ et on a :
$$ \lim_{x \rightarrow + \infty} G(x) = \left( \int_{0}^{+ \infty} e^{-t^2} \dt \right)^2$$
Remarquons maintenant que pour tout $x \in \mathbb{R}_+$,
$$ F(x) = e^{-x^2} \int_{0}^1 \frac{e^{-(xt)^2}}{t^2+1} \dt $$
Ainsi, sachant que la fonction exponentielle est majorée par $1$ sur $\mathbb{R}_{-}$ (et par positivité de l'intégrale) :
$$ 0 \leq F(x)   \leq e^{-x^2} \int_{0}^1 \frac{1}{t^2+1} \dt $$
On a :
$$ \lim_{x \rightarrow + \infty} e^{-x^2} \int_{0}^1 \frac{1}{t^2+1} \dt = 0$$
Par théorème d'encadrement, on en déduit que $F$ admet pour limite $0$ en $+ \infty$. Ainsi,
$$ \lim_{x \rightarrow + \infty} H(x) = \lim_{x \rightarrow + \infty} F(x) + G(x) =  \left( \int_{0}^{+ \infty} e^{-t^2} \dt \right)^2$$
Or $H$ est constante, égale à $\pi/4$ donc on a :
$$ \left( \int_{0}^{+ \infty} e^{-t^2} \dt \right)^2 = \dfrac{\pi}{4}$$
et par positivité de l'intégrale :
$$  \int_{0}^{+ \infty} e^{-t^2} \dt  = \dfrac{\sqrt{\pi}}{2}$$
\end{enumerate}




%%\exo
%%Soit $f$ la fonction définie sur $\mathbb{R}$ par $\displaystyle{f(x)=\int_x^{2x} e^{-t^2}\;\text{d}t}$.
%%\begin{enumerate}
%%\item Montrer que $f$ est une fonction impaire.\\
%%\textit{Indication :} On pourra faire le changement de variable $u=-t$.
%%\item Montrer que $f$ est de classe $\mathcal{C}^1$ sur $\mathbb{R}$ et calculer $f'$.
%%\item Étudier les variations de $f$.
%%\item En encadrant $f(x)$, déterminer $\displaystyle{\lim_{x \rightarrow +\infty} f(x)}$.
%%\end{enumerate}
%
%\exo
%Soit $f$ la fonction définie par $\displaystyle{f(x)=\frac 1 {x-1}\int_1^x \frac{t^2}{t^8+1} \;\text{d}t.}$
%\begin{enumerate}
%\item Montrer que $f$ est définie et continue sur $\mathbb{R} \setminus \{1\}$.
%%\item Soit $\varphi : x \mapsto \int_1^x \frac{t^2}{t^8+1} \;\text{d}t$. Montrer que $\varphi$ est de classe $\mathcal C^1$ sur $\R$ donner son développement limité en 1.
%\item Montrer que $f$ est prolongeable par continuité en 1.
%\end{enumerate}
%% Sur DEPUY
%\exo Étudier la fonction $x \mapsto \int_{x}^{2x} \dfrac{\dt}{\sqrt{1+t^2+t^4}} \dt$ sur $\mathbb{R}$ : parité, dérivabilité et variations, tangente au point d'abscisse $0$, limite en $+ \infty$.
%
%\exo Trouver toutes les fonctions $f$ continues sur $\mathbb{R}$ telles que pour tout $x \in \mathbb{R}$,
%$$ f(x) + \int_{0}^x (x-t)f(t) \dt =1$$
%

%\exo \begin{enumerate}
%  \item On pose pour tout $x>0$ :
%    \[
%    f(x) = \int_{0}^{1} \frac{t^{x - 1}}{1 + t} \dt
%    \]
%    Justifier que $f$ est bien définie sur $\mathbb{R}_+^{*}$.
%  \item
%    Justifier la continuité de $f$ sur son ensemble de définition.
%  \item
%    Calculer $f(x) + f(x + 1)$ pour $x > 0$.
%  \item
%    Donner un équivalent de $f(x)$ quand $x \rightarrow 0^{+} $ et la limite de $f$ en $ + \infty$.
%  \end{enumerate}
%  
%\exo Soit $a,b$ deux réels strictement positifs.
%  \begin{enumerate}
%  \item
%    Justifier l'existence pour tout $x \in \R$ de
%    \[
%    F(x) = \int_{0}^{ + \infty} \frac{\e^{ - at} - \e^{ - bt}}{t}\cos(xt) \dt
%    \]
%  \item
%    Justifier que $F$ est de classe $\mathcal{C}^{1}$ sur $\R$ et calculer $F'(x)$.
%  \item Donner une expression simple de $F(x)$ pour $x \in \mathbb{R}$.
%  \end{enumerate}
%  
%  \newpage
%  
%  \exo Pour $x > 0$, on pose
%  \[
%  F(x) = \int_{0}^{\pi / 2} \ln \bigl( \cos^{2}(t) + x^{2} \sin^{2}(t) \bigr) \dt
%  \]
%  \begin{enumerate}
%  \item
%    Justifier que $F$ est définie et de classe $\mathcal{C}^{1}$ sur $]0, + \infty[$.
%  \item Calculer $F'(x)$ pour tout réel $x>0$.
%  \item En déduire un expression simple de $F(x)$ pour $x>0$.
%  \end{enumerate}
%  
%  \exo 
%  \begin{enumerate}
%  \item
%    Justifier la convergence de l'intégrale
%    \[
%    I = \int_{0}^{ + \infty} \frac{\sin t}{t} \dt
%    \]
%  \item
%    Pour tout $x \geq 0$, on pose :
%    \[
%    F(x) = \int_{0}^{ + \infty} \frac{\e^{ - xt} \sin t}{t} \dt
%    \]
%    Déterminer la limite de $F$ en $ + \infty$.
%  \item
%    Justifier que $F$ est dérivable sur $]0, + \infty[$ et calculer $F'$
%  \item En admettant la continuité de $F$ en 0 déterminer la valeur de $I$.
%  \end{enumerate}
%  
% 
% 
% \exo Montrer que $f : x \mapsto \int_{0}^{+ \infty} \dfrac{e^{-tx}}{1+t^2} \dt$ est de classe $\mathcal{C}^{\infty}$ sur $]0, + \infty[$ et montrer qu'elle est solution d'une équation différentielle d'ordre deux.
\end{document}