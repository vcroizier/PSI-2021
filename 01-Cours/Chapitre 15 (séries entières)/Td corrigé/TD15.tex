\documentclass[a4paper,10pt]{report}
\usepackage{cours}
\newcommand{\Sum}[2]{\ensuremath{\textstyle{\sum\limits_{#1}^{#2}}}}
\usepackage{pifont}
\begin{document}
\everymath{\displaystyle}


\begin{center}
\textit{{ {\huge TD 15 : Séries entières}}}
\end{center}

\bigskip

\begin{center}
\textit{{ {\large Rayon de convergence, somme}}}
\end{center}

\medskip

\begin{Exa} Déterminer le rayon de convergence $R$ des séries entières suivantes :

\begin{multicols}{3}
\begin{enumerate}
\item $\dis \Sum{n \geq 1}{} \dfrac{z^n}{n^2}$
\item $\dis \Sum{n \geq 0}{} n^3 z^n$
\item $\dis \Sum{n \geq 1}{} n^{(-1)^n}z^n$
\item $\dis \Sum{n \geq 0}{} (n^3-2n^2+1)z^n$
\columnbreak
\item $\dis \Sum{n \geq 0}{} \dfrac{z^{2n}}{3^n}$
\item $\dis \Sum{n \geq 1}{} \ln(n) z^n$
\item $\dis \Sum{n \geq 1}{} \dfrac{\arctan n}{n^2}z^n$
\item $\dis \Sum{n \geq 1}{} \textrm{sh}(n) z^n$
\columnbreak
\item $\dis \Sum{n \geq 1}{} n! z^n$
\item $\dis \Sum{n \geq 1}{}\ln\left(1+\sin \left(\dfrac{1}{n}\right) \right)$
\item $\dis \Sum{n \geq 0}{} (3+4ni) z^n$
\item $\dis \Sum{n \geq 0}{}   \left(\dis \prod_{k=0}^n\dfrac{1}{(2k+1)}\right) z^n$
\end{enumerate}
\end{multicols}

\medskip

\end{Exa}


\corr

\begin{enumerate}
\item La série étudiée a le même rayon de convergence que $\Sum{n \geq 1}{} n \dfrac{z^n}{n^2}$ qui a elle-même le même rayon de convergence que $\Sum{n \geq 1}{} n^2 \dfrac{z^n}{n^2} = \Sum{n \geq 1}{} z^n$, c'est-à-dire $1$. Ainsi, $R=1$.
\item La série $\Sum{n \geq 0}{} z^n$ a le même rayon de convergence que $\Sum{n \geq 0}{} n z^n$. En recommençant la même propriété deux fois, on en déduit que la série $\Sum{n \geq 0}{} z^n$ a le même rayon de convergence que la série étudiée. Ainsi, $R=1$.
\item Pour tout entier $n \geq 1$,
$$ \dfrac{1}{n} \leq n^{(-1)^n} \leq n$$
La série $\Sum{n \geq 1}{} \dfrac{1}{n} z^n$ a le même rayon de convergence que $\Sum{n \geq 0}{} \dfrac{n}{n} z^n = \Sum{n \geq 1}{}  z^n$ (c'est-à-dire $1$) qui a le même rayon de convergence que $\Sum{n \geq 1}{} n z^n$. Ainsi, par critère de comparaison, on a $R \leq 1$ et $R \geq 1$ donc $R=1$.
\item On a :
$$ n^3-2n^2+1 \underset{+ \infty}{\sim} n^3$$
D'après la question $2$ et par critère de comparaison, on en déduit que $R=1$.
\item Pour tout entier $n \geq 0$ et $z$ complexe non nul, on a :
$$ \left\vert \frac{z^{2n}}{3^n} \right\vert >0$$
et 
$$ \left\vert\dfrac{\frac{z^{2(n+1)}}{3^{n+1}}}{\frac{z^{2n}}{3^n}}\right\vert=\left\vert\dfrac{z^2}{3}\right\vert\underset{n\to +\infty}{\longrightarrow}\dfrac{\vert z\vert^2}{3}$$
D'après le critère de D'Alembert,
\begin{itemize}
\item Si $\vert z\vert<\sqrt{3}$, alors $\dfrac{\vert z\vert^2}{3}<1$ et $\Sum{n \geq 0}{}\dfrac{z^{2n}}{3^n}$ converge absolument. 
\item Si $\vert z\vert>\sqrt{3}$, alors $\dfrac{\vert z\vert^2}{3}>1$ et $\Sum{n \geq 0}{} \dfrac{z^{2n}}{3^n}$ diverge grossièrement. 
\end{itemize}
Ainsi, $R=\sqrt{3}$.
\item Comme $(\ln n)_{n\geq 1}$ ne converge pas vers $0$, $\dsp\sum \ln n \times 1^n$ diverge grossièrement donc $R\leq 1$. De plus, pour tout $r \in [0,1[$, la suite $((\ln n)r^n)_{n\geq 1}$ converge vers $0$ par croissances comparées donc $R\geq 1$. Ainsi, $R=1$.
\item Sachant que :
$$\dsp\lim_{n\to +\infty}\arctan n=\dfrac{\pi}{2}$$
On a :
\[
\dfrac{\arctan n}{n^2} \underset{ +\infty}{\sim}\dfrac{\pi}{2n^2}
\]
Donc la série $\dsp\Sum{n \geq 1}{} \dfrac{\arctan n}{n^2}z^n$ a même rayon de convergence que la série $\dsp\Sum{n \geq 1}{}  \dfrac{z^n}{n^2}$, qui a même rayon de convergence que la série $\dsp\Sum{n \geq 1}{}  z^n$. Ainsi, $R=1$.
\item  On a :
$$\textrm{sh}(n) \underset{+ \infty}{\sim} \dfrac{e^n}{2}$$
La série étudiée a donc le même rayon de convergence que $\Sum{n \geq 0}{}  \dfrac{e^n}{2} \times z^n = \Sum{n \geq 0}{}  \dfrac{(ez)^n}{2}$. Remarquons maintenant que :
$$ \vert ze \vert < 1 \Longleftrightarrow \vert z \vert < e^{-1}$$
D'après le cours sur les séries géométriques, on en déduit que $R=e^{-1}$.
\item Pour tout réel $r>0$, $(n!r^n)$ ne converge pas vers $0$ (elle est strictement positive et il est facile de vérifier qu'elle est strictement croissance à partir d'un certain rang) donc $R=1$.
\item On a :
\[
\ln\left(1+\sin\dfrac{1}{n}\right)\underset{+\infty}{\sim}\sin\dfrac{1}{n}\underset{+\infty}{\sim}\dfrac{1}{n}
\]
Donc la série entière $\Sum{n \geq 1}{}\ln\left(1+\sin\dfrac{1}{n}\right)z^n$ a même rayon de convergence que la série entière $\Sum{n \geq 1}{} \dfrac{z^n}{n}$ qui a le même rayon de convergence que $\Sum{n \geq 1}{} n\dfrac{z^n}{n} = \Sum{n \geq 1}{} z^n$, c'est-à-dire $1$.
\item On a :
\[
\vert 3+4ni\vert=\sqrt{3^2+(4n)^2}\underset{+\infty}{\sim}4n
\]
Donc la série entière $\Sum{n \geq 0}{} (3+4ni)z^n$ a même rayon de convergence que la série entière $\Sum{n \geq 0}{} nz^n$, qui a le même rayon de convergence que la série entière $\Sum{n \geq 0}{} z^n$. Ainsi, $R=1$.
\item Posons pour tout entier $n \geq 0$,
$$a_n = \dis \prod_{k=0}^n\dfrac{1}{(2k+1)} $$
Alors pour tout $z \in \mathbb{C}$ non nul, 
\begin{align*}
\dfrac{\vert a_{n+1} z^{n+1} \vert }{\vert a_n z^n \vert} & =\left\vert\dfrac{\dis \left(\prod_{k=0}^{n+1}\frac{1}{(2k+1)}\right)z^{n+1}}{\dis \left(\prod_{k=0}^n\frac{1}{(2k+1)}\right)z^n}\right\vert \\
& =\dfrac{\vert z\vert }{2n+3}\underset{n \rightarrow +\infty}{\longrightarrow}0<1
\end{align*}
D'après le critère de d'Alembert, la série $\Sum{n \geq 0}{} \left(\dis \prod_{k=0}^n\frac{1}{(2k+1)}\right)z^n$ converge absolument pour tout $z\in\mathbb{C}^*$ donc $R=+\infty$. 
\end{enumerate}

\begin{Exa} Déterminer le rayon de convergence $R$ de la série entière $\Sum{n\geq 1}{} a_n z^n$, où $a_n$ désigne : 
\begin{itemize}
\item Le nombre de diviseurs de $n$ pour tout entier $n\geq 1$.
\item Le terme général d'une suite périodique quelconque. 
\end{itemize}
\end{Exa}

\newpage

\corr 
\begin{itemize}
\item Tout entier $n\geq 1$ a un nombre de diviseurs compris entre $1$ et $n$ :
\[
\forall n\in\mathbb{N}^*,\ 1\leq a_n \leq n
\]
Or les séries $\Sum{n\geq 1}{} z^n$ et $\Sum{n\geq 1}{}n z^n$ ont pour rayon de convergence $1$, ce qui permet d'en déduire respectivement que $R\leq 1$ et $R\geq 1$.  Ainsi, $R=1$.
\item Distinguons deux cas :
\begin{itemize}
\item Si la suite $(a_n)_{n\geq 1}$ est nulle, alors $R=+\infty$.
\item Sinon, la suite $(a_n)_{n\geq 1}$ n'est pas nulle, et étant périodique, ne converge pas vers $0$. Ainsi, la série de terme général $a_n$ diverge grossièrement donc $R\leq 1$. Par ailleurs, étant périodique, $(a_n)_{n\geq 1}$ est bornée par une constante $M>0$ donc $a_n  \underset{+ \infty}{=} O(1)$ donc $R \geq 1$ car $\Sum{n\geq 1}{} z^n$ a pour rayon de convergence $1$. Finalement, $R=1$.
\end{itemize}
\end{itemize}

\begin{Exa}
Déterminer le rayon de convergence de $\Sum{n\geq 0}{} \sin \left( \dfrac{n \pi}{3} \right) x^n$ et déterminer sa somme.
\end{Exa}

\corr Soit $R$ ce rayon de convergence. La suite de terme général $\sin \left( \dfrac{n \pi}{3} \right)$ est bornée donc $R \geq 1$ et cette suite diverge (cyclicité d'une suite non constante) donc la série diverge pour $x=1$ donc $R \leq 1$. Ainsi, $R=1$.

\medskip

\noindent Soit $x \in ]-1,1[$. Remarquons que pour tout $n \geq 0$,
$$ \sin \left( \dfrac{n \pi}{3} \right) x^n = \Im m (e^{in \pi/3} x^n) = \Im m ((xe^{i \pi/3})^n)$$
On sait que $\vert xe^{i \pi/3} \vert <1$ donc la série de terme général $(xe^{i \pi/3})^n$ converge et on a :
\begin{align*}
\sum_{k=0}^{+ \infty} (xe^{i \pi/3})^k & = \dfrac{1}{1-xe^{i \pi/3}} \\
& = \dfrac{1-xe^{-i \pi/3}}{\vert 1-x e^{i \pi/3}\vert^2} \\
& = \dfrac{1-x \cos(-\pi/3)-i x\sin(-\pi/3)}{\vert 1-x e^{i \pi/3}\vert^2} \\
& = \dfrac{1-x \cos(\pi/3)+i x\sin(\pi/3)}{(1-x \cos(\pi/3))^2+ (x\sin(\pi/3))^2} \\
& = \dfrac{1-x \cos(\pi/3)+i x\sin(\pi/3)}{1-2x \cos(\pi/3)+x^2} \\
& = \dfrac{1-x/2 +i x(\sqrt{3}/2)}{1-x+x^2} 
\end{align*}
En prenant la partie imaginaire, on en déduit que pour tout $x \in ]-1,1[$,
$$ \sum_{k=0}^{+ \infty} \sin \left( \dfrac{k \pi}{3} \right) x^k =\dfrac{\sqrt{3}}{2} \times \dfrac{x}{1-x+x^2}$$

\begin{Exa} Déterminer le rayon de convergence de $\Sum{n\geq 1}{}  n^{(-1)^n} x^n$ et déterminer sa somme.
\end{Exa}

\begin{corr} Pour tout $n \geq 1$,
$$ \dfrac{1}{n} \leq n^{(-1)^n} \leq n$$
Ainsi, le rayon de convergence vaut $1$ par critère de comparaison. Soit $N \geq 0$. Alors pour tout $x \in ]-1,1[$,
\begin{align*}
\sum_{n=1}^{2N} n^{(-1)^n} x^n & = \sum_{k=1}^{N} (2k)^{(-1)^{2k}} x^{2k} + \sum_{k=0}^{N-1} (2k+1)^{(-1)^{2k+1}} x^{2k+1} \\
& =  \sum_{k=1}^{N} (2k) x^{2k} + \sum_{k=0}^{N-1} \frac{1}{2k+1} x^{2k+1} \\
& = 2  \sum_{k=1}^{N} k (x^{2})^k + \sum_{k=0}^{N-1} \frac{1}{2k+1} x^{2k+1} \\
\end{align*}
On sait que pour tout $x \in ]-1,1[$,
$$ \dfrac{1}{1-x} = \sum_{n=0}^{+ \infty} x^n $$
et par dérivation terme à terme :
$$ \dfrac{1}{(1-x)^2} = \sum_{n=1}^{+ \infty} n x^{n-1}$$
On a $x^2 \in ]-1,1[$ donc :
$$ \dfrac{1}{(1-x^2)^2} = \sum_{n=1}^{+ \infty} n x^{2n-2}$$
et ainsi :
$$ \dfrac{x^2}{(1-x^2)^2} = \sum_{n=1}^{+ \infty} n x^{2n}$$
De même, on a :
$$ \dfrac{1}{1-x^2} = \sum_{n=0}^{+ \infty} x^{2n} $$
Remarquons que :
$$ \dfrac{1}{1-x^2} = \dfrac{1}{2} \times \dfrac{1+x + (1-x)}{(1+x)(1-x)} = \dfrac{1}{2} \times \dfrac{1}{1-x} + \dfrac{1}{2} \times \dfrac{1}{1+x} $$
donc par intégration terme à terme :
$$ \dfrac{1}{2} ( -\ln(1-x)+\ln(1+x) ) = \sum_{n=0}^{+ \infty} \dfrac{x^{2n+1}}{2n+1}$$
Finalement, en notant $S$ la série de cette série entière, on a pour tout $x \in ]-1,1[$,
$$ S(x) = \dfrac{x^2}{(1-x^2)^2} + \dfrac{1}{2} \ln \left( \dfrac{1+x}{1-x} \right)$$
Remarquons que $S$ n'est pas définie en $1$ ou $-1$ (les suites extraites d'indices pairs et impaires n'ont pas le même comportement asymptotique).
\end{corr}

\begin{Exa} Déterminer le rayon de convergence de $\Sum{n\geq 1}{} \dfrac{3n}{n+2} x^n$ et déterminer sa somme.
\end{Exa}

%\corr On a :
$$  \dfrac{3n}{n+2} \underset{+ \infty}{\sim} 3$$
Par critère de comparaison (à la série géométrique), on en déduit que la série étudiée a pour rayon de convergence $1$. Notons $S$ sa somme. On a pour tout $x \in ]-1,1[$,
$$ S(x) = \sum_{n=1}^{+ \infty}  \dfrac{3n}{n+2} x^n = \sum_{n=1}^{+ \infty}  \dfrac{3n+6-6}{n+2} x^n = \sum_{n=1}^{+ \infty} 3x^n - 6 \sum_{n=1}^{+ \infty} \dfrac{x^n}{n+2}$$
car les deux séries convergents (série géométrique et dérivée). On sait que :
$$ \sum_{n=1}^{+ \infty} 3x^n = \dfrac{3x}{1-x}$$
On remarque aussi que :
$$ x^2 \sum_{n=1}^{+ \infty} \dfrac{x^n}{n+2} = \sum_{n=1}^{+ \infty} \dfrac{x^{n+2}}{n+2} = \sum_{k=3}^{+\infty} \dfrac{x^k}{k} = - \ln(1-x) -x - \dfrac{x^2}{2}   $$
Ainsi, pour $x \neq 0$,
$$ \sum_{n=1}^{+ \infty} \dfrac{x^n}{n+2} = - \dfrac{\ln(1-x)}{x^2} - \dfrac{1}{x} - \dfrac{1}{2}$$
On en déduit alors que pour $x \in ]-1,1[$, $x \neq 0$, on a :
$$ S(x) =  \dfrac{3x}{1-x}  + \dfrac{6\ln(1-x)}{x^2} + \dfrac{6}{x} +3$$
et on a $S(0)= 0$. Remarquons que $S$ n'est pas définie en $-1$ et $1$ (divergence grossière).

\begin{Exa} Soit $\theta\in\mathbb{R}$. Après avoir justifié la convergence, calculer $\dsp\sum_{n=0}^{+\infty}\dfrac{\sin(n\theta)}{3^nn!}$.
\end{Exa}

\corr Pour tout entier naturel $n$, on a : 
\[
\left\vert\dfrac{\sin(n\theta)}{3^n n!}\right\vert\leq \dfrac{1}{3^n n!}
\]
Or $\Sum{n \geq 0}{} \dfrac{(\frac{1}{3})^n}{ n!}$ est une série exponentielle, donc convergente. Par critère de comparaison des séries à termes positifs, on en déduit que $\Sum{n \geq 0}{} \dfrac{\sin(n\theta)}{3^n n!}$ est une série absolument convergente, donc convergente. On sait que pour tout $z \in \mathbb{C}$,
$$\e^{z}=\e^{\Re\mathrm{e}(z)}\big[\cos(\Im\mathrm{m}\,z)+i\sin(\Im\mathrm{m}\,z) \big]= \sum_{n =0}^{+ \infty} \dfrac{z^n}{n!}$$
On a alors : 
$$ \e^{\frac{\e^{i\theta}}{3}}= \sum_{n=0}^{+\infty}\dfrac{(\frac{\e^{i\theta}}{3})^n}{n!} = \sum_{n=0}^{+ \infty} \dfrac{e^{i n \theta}}{3^n n!}$$
Par passage à la partie imaginaire, on en déduit que :
$$ \sum_{n=0}^{+\infty}\dfrac{\sin(n\theta)}{3^nn!} = \e^{\frac{\cos\theta}{3}}\sin\left(\dfrac{\sin\theta}{3}\right)$$

\begin{Exa} Déterminer le rayon de convergence puis la somme de $\Sum{n \geq 0}{} n^2x^n$.
\end{Exa}

\corr La série $\Sum{n \geq 0}{} x^n$ a même rayon de convergence que  la série $\Sum{n \geq 0}{} nx^n$ qui a lui même le même rayon de convergence de $\Sum{n \geq 0}{} n^2x^n$. Ainsi, le rayon de convergence de cette série vaut $1$.

\medskip

\noindent Pour tout $x \in ]-1,1[$, on a :
$$ \sum_{n=0}^{+\infty}x^n  =  \dfrac{1}{1-x} $$
donc par dérivation terme à terme :
$$ \sum_{n=0}^{+\infty} nx^{n-1}  =  \dfrac{1}{(1-x)^2} $$
On multiplie par $x$ :
$$ \sum_{n=0}^{+\infty} nx^{n}  =  \dfrac{x}{(1-x)^2} $$
De nouveau par dérivation terme à terme :
$$ \sum_{n=0}^{+\infty} n^2 x^{n-1} = \dfrac{(1-x)^2\times 1-(-2(1-x))x}{(1-x)^4}$$
puis en multipliant par $x$ :
$$ \sum_{n=0}^{+\infty} n^2 x^{n}= \dfrac{(1-x)^2\times 1-(-2(1-x))x}{(1-x)^4} \times x  = \dfrac{x(x+1)}{(1-x)^3}$$

\begin{Exa}[\ding{80}] Le but de cet exercice est de justifier l'existence et calculer la valeur de la somme :  
\[
A=\sum_{n=1}^{+\infty}\dfrac{1}{n2^n}\cos\left(\dfrac{2n\pi}{3}\right)
\]
\begin{enumerate}
\item Déterminer le rayon de convergence de la série entière $\Sum{n \geq 1}{} \dfrac{1}{n}\cos\left(\dfrac{2n\pi}{3}\right)x^n$.
\item Calculer la somme cette série entière sur son intervalle ouvert de convergence.
\item En déduire l'existence et la valeur de $A$.
\end{enumerate}
\end{Exa}

\corr 

\begin{enumerate}
\item La série $\Sum{n \geq 1}{} \dfrac{1}{n}\cos\left(\dfrac{2n\pi}{3}\right)x^n$ a même rayon de convergence $R$ que $\Sum{n \geq 1}{} \cos\left(\dfrac{2n\pi}{3}\right)x^n$. Remarquons que pour tout entier $n \geq 0$,
\[
 \left\vert \cos\left(\dfrac{2n\pi}{3}\right)\right\vert\leq 1
\]
Or la série $\Sum{n \geq 1}{} x^n$ a pour rayon de convergence $1$ donc on en déduit que $R\geq 1$.

\medskip

D'autre part, la suite $(u_n)_{n \geq 0}=\left(\cos\left(2n\pi/3\right)\right)_{n \geq 0}$ ne converge pas vers $0$, car sa suite extraite $(u_{3n})_{n \geq 0}$ est constante égale à $1$. Ainsi, la série numérique $\Sum{n \geq 1}{} u_n$ est grossièrement divergente donc $R\leq 1$. On en déduit que $R=1$.
\item On note $S$ la somme de la série entière (au moins) définie sur $]-1,1[$.
\begin{itemize}
\item Par dérivation terme à terme, on calcule d'abord $S'(x)$ pour tout $x\in ]-1,1[$ : 
$$ S'(x)  = \sum_{n=1}^{+\infty}\cos\left(\dfrac{2n\pi}{3}\right)x^{n-1} = \Re\mathrm{e}\left[j\sum_{n=1}^{+\infty}(jx)^{n-1}\right]$$
La dernière égalité étant licite car la série est convergente (série exponentielle) et en remarquant que :
$$ \cos\left(\dfrac{2n\pi}{3}\right) = \Re e (e^{2i n \pi/3}) = \Re e (j^n)$$
En remarquant la somme d'une série géométrique, on a :
$$ \sum_{n=1}^{+\infty}(jx)^{n-1} = \sum_{k=0}^{+\infty}(jx)^{k} = \dfrac{1}{1-jx}$$
On a :
$$ \dfrac{j}{1-jx} =  \dfrac{j(1-\overline{j}x)}{(1-jx)(1-\overline{j}x)} =\dfrac{j-x}{1+x+x^2}$$
car $\vert j \vert =1$. Par passage à la partie réelle, on en déduit que :
$$S'(x)  = -\dfrac{1}{2}\dfrac{2x+1}{x^2+x+1}$$

\item Il suffit alors d'intégrer pour en déduire que pour tout $x\in ]-1,1[$ :  
$$ S(x)=S(0)-\dfrac{1}{2}\int_0^x\dfrac{2t+1}{t^2+t+1} \dt=-\dfrac{1}{2}\ln(1+x+x^2)$$

\end{itemize}
\item On remarque que $A=S(1/2)$ avec $1/2\in ]-1,1[$, donc $A$ existe et vaut : 
\[
A=\dsp\sum_{n=1}^{+\infty}\dfrac{1}{n2^n}\cos\left(\dfrac{2n\pi}{3}\right)=S\left(\dfrac{1}{2}\right)=-\dfrac{1}{2}\ln\left(1+\dfrac{1}{2}+\dfrac{1}{4}\right)=-\dfrac{1}{2}\ln\left(\dfrac{7}{4}\right)
\]
\end{enumerate}

\begin{Exa} \begin{enumerate}
\item Déterminer le rayon de convergence $R$ de $\Sum{n \geq 1}{} \sin \left( \frac{1}{\sqrt{n}} \right) x^n$.
\item Déterminer l'ensemble de définition de la somme de cette série entière.
\end{enumerate}
\end{Exa}

\corr 

\begin{enumerate}
\item On a :
$$ \sin \left( \frac{1}{\sqrt{n}} \right) \underset{+ \infty}{\sim} \dfrac{1}{\sqrt{n}}$$
Pour tout nombre complexe non nul $z$ et tout entier $n \geq 1$, posons :
$$  a_n = \left\vert \dfrac{z^n}{ \sqrt{n}} \right\vert$$
Alors $a_n>0$ et 
$$ \dfrac{a_{n+1}}{a_n} = \dfrac{\sqrt{n}}{\sqrt{n+1}} \vert z \vert \underset{ n \rightarrow+ \infty}{\longrightarrow} \vert z \vert$$
D'après le critère de d'Alembert, on en déduit que $R=1$.
\item Soit $\mathcal{D}$ l'ensemble de définition de la somme. On sait que : 
$$ ]-1,1[ \subset \mathcal{D} \subset [-1,1]$$
\begin{itemize}
\item Pour $z=1$, remarquons que :
$$ \sin \left( \frac{1}{\sqrt{n}} \right) \underset{+ \infty}{\sim} \dfrac{1}{\sqrt{n}} $$
Les deux termes généraux sont positifs (la fonction sinus est positive sur $[0,1]$) et la série de terme général $\dfrac{1}{\sqrt{n}}$ diverge (série de Riemann divergente). Par critère de comparaison, on en déduit que la série de terme général $\dis \sin \left( \frac{1}{\sqrt{n}} \right)$ diverge. Ainsi, $1 \notin \mathcal{D}$.
\item La suite de terme général $\dfrac{1}{\sqrt{n}}$ est décroissante et ses termes appartiennent à $[0,1]$ où la fonction sinus est croissante et positive donc la suite de terme général $\dis \sin \left( \frac{1}{\sqrt{n}} \right)$ est décroissante, positive et converge vers $0$. D'après le critère spécial des séries alternées, on en déduit que $-1 \in \mathcal{D}$.
\end{itemize}
Ainsi,
$$ \mathcal{D}= [-1,1[$$
\end{enumerate}

\begin{Exa} Soit $\dis f : x \mapsto \sum_{n = 2}^{ + \infty} \frac{( - 1)^{n}}{n(n - 1)}x^{n}.$

  \begin{enumerate}
  \item
    Déterminer l'ensemble de définition de $f$.
  \item
    Exprimer la fonction $f$ à l'aide des fonctions usuelles sur $] - 1,1[$
  \item Calculer $f(1)$ et $f( - 1)$.
  \end{enumerate}
  \end{Exa} 
  
%  \corr 
  
  
  \begin{enumerate}
  \item Pour tout entier $n \geq 2$,
  $$ \left\vert \frac{( - 1)^{n}}{n(n - 1)} \right\vert = \dfrac{1}{n(n-1)} \underset{+ \infty}{\sim} \dfrac{1}{n^2}$$
Par comparaison, la série entière $\dis \Sum{n \geq 2}{} \frac{( - 1)^{n}}{n(n - 1)}x^{n}$ a le même rayon de convergence que $\dis \Sum{n \geq 2}{} \dfrac{1}{n^2}x^{n}$ qui a le même rayon de convergence que $\dis \Sum{n \geq 2}{} n^2 \times \dfrac{1}{n^2} x^{n} = \dis \Sum{n \geq 2}{} x^n$ qui vaut $1$. Ainsi, $f$ est définie sur $\mathcal{D}$ où
$$ ]-1,1[ \subset \mathcal{D} \subset [-1,1]$$
Remarquons que pour tout entier entier $n \geq 2$ et $x \in [-1,1]$,
$$ \left\vert \dfrac{(-1)^n x^n }{n(n-1)} \right\vert = \dfrac{1}{n(n-1)} $$
Or 
$$ \dfrac{1}{n(n-1)} \underset{+ \infty}{\sim} \dfrac{1}{n^2}$$
La série de terme général $\dfrac{1}{n^2}$ converge (série de Riemann avec $2>1$) donc par critère de comparaison, on en déduit que la série de terme général $\dfrac{1}{n(n-1)}$ converge et finalement, la série de terme général $ \dfrac{(-1)^n x^n }{n(n-1)}$ converge absolument donc converge. Finalement,
$$ \mathcal{D} = [-1,1]$$
\item $f$ est la somme d'une série entière sur $]-1,1[$ donc $f$ est $\mathcal{C}^{\infty}$ sur $]-1,1[$ et on a pour tout $x \in ]-1,1[$,
$$ f'(x) = \sum_{n=2}^{+ \infty} \dfrac{(-1)^n}{n-1} x^{n-1} =  \sum_{n=0}^{+ \infty} \dfrac{(-1)^{n}}{n+1} x^{n+1} = \ln(1+x)$$
Or pour tout réel $t \in ]-1,1[$, par changement de variable affine (donc licite) :
$$ \int_0^t \ln(1+x) \dx = \int_1^{t+1} \ln(u) \textrm{d}u = \left[ u \ln(u)-u \right]_1^{t+1} = (t+1)\ln(t+1)-(t+1) +1$$
Ainsi, il existe une constante $K \in \mathbb{R}$ tel que pour tout réel $x \in ]-1,1[$,
$$ f(x) = (x+1) \ln(x+1) - x + K$$
Or $f(0)=0$ donc $K=0$ et finalement, 
pour tout réel $x \in ]-1,1[$,
$$ f(x) = (x+1) \ln(x+1) - x $$
  \item On a montré que pour tout entier $n \geq 2$ et tout réel $x \in [-1,1]$,
  $$ \left\vert \frac{( - 1)^{n} x^n}{n(n - 1)} \right\vert \leq \dfrac{1}{n(n-1)}$$
La série de terme général $\dfrac{1}{n(n-1)}$ étant convergente, on en déduit la convergence normale de la série entière sur $[-1,1]$ donc la continuité de sa somme sur $[-1,1]$. On en déduit que :
$$ f(1) = \lim_{x \rightarrow 1^{-}} f(x) =  2 \ln(2)-1$$
et par croissances comparées,
$$ f(-1) = \lim_{x \rightarrow 1^{-}} f(x) =  1$$
  \end{enumerate}
  
  
  \begin{Exa} Soit $\dis f : x \mapsto \sum_{n=1}^{+ \infty} \dfrac{(-1)^{n+1}}{n(2n+1)} x^{2n+1}$.

\begin{enumerate}
\item Déterminer l'ensemble de définition de $f$.
\item Montrer que $f$ est continue sur son ensemble de définition.
\item Déterminer une expression simple de $f'(x)$ pour $x \in ]-1,1[$. En déduire une expression simple de $f(x)$ pour $x \in ]-1,1[$.
\item Calculer $ \sum_{n=1}^{+ \infty} \dfrac{(-1)^{n+1}}{n(2n+1)} \cdot$
\end{enumerate}
\end{Exa}


\corr 

\begin{enumerate}
\item Soit $x \in \mathbb{R}^*$. Posons pour tout entier $n \geq 1$,
$$ a_n = \left\vert \dfrac{(-1)^{n+1}}{n(2n+1)} x^{2n+1} \right\vert = \dfrac{1}{n(2n+1)} \vert x \vert ^{2n+1} >0$$
Alors 
$$ \dfrac{a_{n+1}}{a_n} = \dfrac{n(2n+1)}{(n+1)(2n+3)} \vert x \vert^2 \underset{n \rightarrow + \infty}{\longrightarrow} \vert x \vert^2$$
D'après le critère de d'Alembert, on en déduit que la série de terme général $a_n$ converge si $\vert x \vert<1$ et diverge si $\vert x \vert >1$. Ainsi, la série entière $\dis \Sum{n \geq 1}{} \dfrac{(-1)^{n+1}}{n(2n+1)} x^{2n+1}$ a pour rayon de convergence $1$. On en déduit que $f$ est définie sur $\mathcal{D}$ où :
$$ ]-1,1[ \subset \mathcal{D} \subset [-1,1]$$
Remarquons maintenant que pour tout entier $n \geq 1$ et $x \in [-1,1]$,
$$ \left\vert \dfrac{(-1)^{n+1}}{n(2n+1)} x^{2n+1} \right\vert= \dfrac{1}{n(2n+1)} \leq \dfrac{1}{2n^2}$$
La série de terme général $\dfrac{1}{2n^2}$ converge (série de Riemann avec $2>1$) donc par critère de comparaison des séries à termes positifs, on en déduit que la série de terme général $\dfrac{(-1)^{n+1}}{n(2n+1)} x^{2n+1}$ converge absolument donc converge (en particulier pour $x=1$ et $x=-1$). Ainsi,
$$ \mathcal{D}= [-1,1]$$

\item Notons pour tout entier $n \geq 1$,
$$ f_n : x \mapsto \dfrac{(-1)^{n+1}}{n(2n+1)} x^{2n+1}$$
On a montré dans la question précédente que $f_n$ est bornée sur $[-1,1]$ et on a :
$$ 0 \leq \Vert f_n \Vert_{\infty} \leq  \dfrac{1}{2n^2}$$
La série de terme général $\dfrac{1}{2n^2}$ converge (série de Riemann avec $2>1$) donc par critère de comparaison des séries à termes positifs, on en déduit que la série de terme général $f_n$ converge normalement donc converge uniformément sur $[-1,1]$. Or pour tout entier $n \geq 1$, $f_n$ est continue sur $[-1,1]$. Par théorème de continuité, on en déduit que $f$ est continue sur $[-1,1]$.
\item La fonction $f$ est la somme d'une série entière sur $]-1,1[$ donc $f$ est $\mathcal{C}^{\infty}$ sur cet intervalle et par dérivation terme à terme, on a :
$$ f'(x) = \sum_{n=1}^{+ \infty}  \dfrac{(-1)^{n+1}}{n} x^{2n} = \ln(1+x^2)$$
Une intégration par parties (bien justifiée) montre que :
$$ \int_0^{t} \ln(1+x^2) \dx = \left[ x \ln(1+x^2) \right]_0^t - 2\int_0^t \dfrac{x^2}{1+x^2} \dx$$
donc 
\begin{align*}
 \int_0^{t} \ln(1+x^2) \dx & =  t \ln(1+t^2)  - 2\int_0^t \dfrac{x^2+1-1}{1+x^2} \dx \\
 & = t \ln(1+t^2)  - 2\int_0^t 1 - \dfrac{1}{1+x^2} \dx \\
 & = t \ln(1+t^2) - 2 \left[ t - \arctan(t) \right]_0^t \\
 & = t \ln(1+t^2) - 2t + 2 \arctan(t)
\end{align*}
Ainsi, il existe un réel $K$ tel que pour tout réel $x \in ]-1,1[$,
$$ f(x) = x \ln(1+x^2) - 2x + 2 \arctan(x) + K$$
Or $f(0)=0$ donc $K=0$. Finalement, pour tout réel $x \in ]-1,1[$,
$$ f(x) = x \ln(1+x^2) - 2x + 2 \arctan(x) $$
\item Remarquons que :
$$ \sum_{n=1}^{+ \infty} \dfrac{(-1)^{n+1}}{n(2n+1)}  = f(1)$$
Par continuité de $f$ en $1$, on en déduit que :
$$ \sum_{n=1}^{+ \infty} \dfrac{(-1)^{n+1}}{n(2n+1)} = \lim_{x \rightarrow 1^{-}}  x \ln(1+x^2) - 2x + 2 \arctan(x) =\ln(2)-2+ 2 \arctan(1)$$
et ainsi :
$$ \sum_{n=1}^{+ \infty} \dfrac{(-1)^{n+1}}{n(2n+1)} = \ln(2)-2 + \dfrac{\pi}{2}$$
\end{enumerate}

\begin{Exa}[\ding{80}] Soit $\Sum{n \geq 0}{} a_n z^n$ une série entière de rayon de convergence $R>0$. Montrer que  la série $\Sum{n \geq 0}{} \dfrac{a_n}{n!}z^n$ a un rayon de convergence infini. 
\end{Exa}

\corr Soit $r\in ]0,R[$. Par définition du rayon de convergence, $(a_n r^n)_{n \geq 0}$ est donc bornée : 
\[
\exists M\in\mathbb{R}^+\,\mid\, \forall n\in\mathbb{N},\ \vert a_n r^n\vert \leq M
\]
Soit $z\in\mathbb{C}$. On a alors pour tout entier $n \geq 0$,
\begin{align*}
\left\vert \dfrac{a_n}{n!} z^n \right\vert & =\vert a_n r^n\vert \times \left\vert\dfrac{(z/r)^n}{n!}\right\vert \\
&  \leq M\times \left\vert\dfrac{(z/r)^n}{n!}\right\vert
\end{align*}
Or la série exponentielle $\Sum{n \geq 0}{} \dfrac{(z/r)^n}{n!}$ converge absolument pour tout $z\in\mathbb{C}$, donc par critère de comparaison pour des séries à termes positifs, on en déduit que $\Sum{n \geq 0}{} \dfrac{a_n}{n!} z^n$ converge aussi absolument pour tout $z\in\mathbb{C}$. Ainsi, la série $\Sum{n \geq 0}{} \dfrac{a_n}{n!} z^n$ a un rayon de convergence infini.

\begin{Exa}[\ding{80}] Déterminer le rayon de convergence $R$ de $\dis \sum_{n \geq 0} x^{n^2}$. Déterminer un équivalent simple en $R^{-}$ de la somme de cette série entière.
\end{Exa}

\begin{corr} On pose pour tout $n \geq 0$, $a_n=1$ si $n$ est un carré et $a_n =0$ si $n$ n'est pas un carré. La suite $(a_n)_{n \geq 0}$ est clairement bornée donc le rayon de convergence de la série est supérieur ou égal à $1$. Remarquons que la sous-suite $(a_{n^2})$ converge vers $1$. La série diverge donc grossièrement pour $z=1$.Finalement, le rayon de convergence de la série vaut $1$.

%Remarquons maintenant que pour $n>1$,
%$$ (n+1)^2-n^2 = 2n+1>1$$
%Autrement dit, la différence entre deux carrés consécutifs est strictement plus grande que $1$. On en déduit que pour tout $n \geq 1$, $n^2+1$ n'est pas un carré et on en déduit que la sous-suite $(a_{n^2+1})_{n \geq 1}$ converge vers $0$. L'existence de ces deux sous-suites ne convergeant pas vers le même nombre implique que $(a_n)$ diverge donc le rayon de convergence est inférieur à $1$. Finalement, le rayon de convergence de la série vaut $1$.

\medskip

\noindent Utilisons une comparaison série-intégrale pour étudier le comportement de la somme $S$ de cette série entière en $1^{-}$. Soit $x \in [0,1[$. La fonction $t \mapsto x^{t^2} = e^{t^2 \ln(x)}$ est continue sur $\mathbb{R}_+$ et décroissante sur $\mathbb{R}_+$ (car $\ln(x)<0$). Par croissance de l'intégrale, on en déduit que pour tout $k \geq 1$,
$$ \int_{k}^{k+1} x^{t^2} \dt \leq x^{k^2} \leq \int_{k-1}^{k} x^{t^2} \dt$$
Puis par sommation de $1$ à $N \geq 1$ :
$$ \int_{1}^{N+1} x^{t^2} \dt \leq \sum_{k=1}^N x^{k^2} \leq \int_{0}^{N} x^{t^2} \dt$$
Remarquons que la fonction $t \mapsto x^{t^2} = e^{t^2 \ln(x)}$ est continue sur $\mathbb{R}_+$ et sachant que $\ln(x)<0$, on a :
$$ x^{t^2} \underset{+ \infty}{=} o \left( \dfrac{1}{t^2} \right)$$
On en déduit (à bien rédiger) que la fonction $t \mapsto x^{t^2}$ est intégrable sur $\mathbb{R}_+$. Ainsi, par passage à la limite quand $N$ tend vers $+ \infty$, on a :
$$  \int_{1}^{+ \infty} x^{t^2} \dt \leq \sum_{k=1}^{+ \infty} x^{k^2} \leq \int_{0}^{+ \infty} x^{t^2} \dt$$
puis en rajoutant le terme manquant :
$$ 1+  \int_{1}^{+ \infty} x^{t^2} \dt \leq S(x) \leq 1+ \int_{0}^{+ \infty} x^{t^2} \dt$$
Or par croissance de l'intégrale et décroissance de $t \mapsto x^{t^2}$ sur $[0,1]$ , on a :
$$ \int_{0}^{1} x^{t^2} \dt \leq x^{0} =1$$
donc finalement :
$$\int_{0}^{+ \infty} x^{t^2} \dt \leq S(x) \leq 1+ \int_{0}^{+ \infty} x^{t^2} \dt$$
On utilise alors le changement de variable affine  $u : t \mapsto t \sqrt{\vert \ln(x) \vert}$ qui implique :
$$ \int_{0}^{+ \infty} x^{t^2} \dt = \dfrac{1}{\sqrt{\vert \ln(x) \vert}} \int_{0}^{+ \infty} e^{-t^2} \dt$$
On sait que $\ln(x) \sim x-1$ quand $x \rightarrow 1^{-}$ donc $\vert \ln(x) \vert \sim 1-x$ quand $x \rightarrow 1^{-}$. En posant :
$$ C =  \int_{0}^{+ \infty} e^{-t^2} \dt \left( = \dfrac{\sqrt{\pi}}{2} \right)$$
On en déduit par théorème d'encadrement (à rédiger proprement) :
$$S(x) \underset{x \rightarrow 1^{-}}{\sim} \dfrac{C}{\sqrt{1-x}}$$
\end{corr}


\medskip

\begin{center}
\textit{{ {\large Développement en série entière}}}
\end{center}

\medskip

\begin{Exa} Développer en série entière $x \mapsto \e^{-x}\sin x$.
\end{Exa}

\corr Les fonctions $x\mapsto\e^{-x}$ et $\sin$ sont développables en série entière, donc leur produit $f$ l'est aussi, et on a pour tout $x\in\mathbb{R}$ : 
\[
\begin{array}{rcl}
f(x) &= & \e^{-x}\sin x=\Im\mathrm{m}[\e^{(-1+i)x}]=\Im\mathrm{m}\left[\dsp\sum_{n=0}^{+\infty}\dfrac{(-1+i)^n}{n!}x^n\right]\\
\vspace*{-0.2cm}\\
& = & \dsp\sum_{n=0}^{+\infty}\dfrac{\Im\mathrm{m}[(\sqrt{2})^n\e^{i\frac{3n\pi}{4}}]}{n!}x^n=\sum_{n=0}^{+\infty}\dfrac{(\sqrt{2})^n}{n!}\sin\left(\dfrac{3n\pi}{4}\right)x^n
\end{array}
\]

\begin{Exa} Développer en série entière $x \mapsto \dfrac{1}{x^2-5x+6}$.
\end{Exa}

\corr Le trinôme $X^2-5X+6$ a pour racines $2$ et $3$. La fonction (que l'on note $f$) est définie sur $\mathbb{R} \setminus \lbrace 2,3 \rbrace$ donc $f$ est développable en série entière au mieux sur $]-2,2[$. Pour tout $x \in ]-2,2[$, on a :
\begin{align*}
f(x) & = \dfrac{1}{(x-2)(x-3)}\\
&  = \dfrac{(x-2)-(x-3)}{(x-2)(x-3)}  \\
& = \dfrac{1}{x-3} - \dfrac{1}{x-2} \\
& = -\dfrac{1}{3}\dfrac{1}{1-\frac{x}{3}}+\dfrac{1}{2}\dfrac{1}{1-\frac{x}{2}} \\
& =-\dfrac{1}{3}\sum_{n=0}^{+\infty}\dfrac{x^n}{3^n}+\dfrac{1}{2}\sum_{n=0}^{+\infty}\dfrac{x^n}{2^n}
\end{align*} 
car $\vert x/3 \vert <1$ et $\vert x/2 \vert <1$. Finalement, $f$ est développable en série entière sur $]-2,2[$ et pour tout $x \in ]-2,2[$, on a :
$$ f(x) = \dsp\sum_{n=0}^{+\infty}\left(  \dfrac{1}{2^{n+1}}-\dfrac{1}{3^{n+1}}\right)x^n$$

\begin{Exa} Développer en série entière $x \mapsto \dsp\int_0^x\sin(t^2)\mathrm{d}t$.
\end{Exa}

\corr Pour tout $t\in\mathbb{R}$ : 
\[
\sin(t^2)=\dsp\sum_{n=0}^{+\infty}\dfrac{(-1)^n}{(2n+1)!}t^{4n+2}
\]
Notons $f$ la fonction étudiée. Par théorème d'intégration terme à terme, il vient alors que pour tout $x\in\mathbb{R}$ : 
$$f(x)=\dsp\int_0^x\dsp\sum_{n=0}^{+\infty}\dfrac{(-1)^n}{(2n+1)!}t^{4n+2}\mathrm{d}t=\dsp\dsp\sum_{n=0}^{+\infty}\dfrac{(-1)^n}{(2n+1)!}\int_0^xt^{4n+2}\mathrm{d}t=\sum_{n=0}^{+\infty}\dfrac{(-1)^n}{(2n+1)!(4n+3)}x^{4n+3}$$

\begin{Exa} On pose $f(x)=\dfrac{1}{(x+1)^{2}(3-x)} \cdot$
\begin{enumerate}
\item Déterminer la primitive $G$ de $f$ définie sur l'intervalle $]-1,3[$ telle que $G(1)=0$.
\item Déterminer le développement en série entière en 0 de la fonction $f$ et précisez le rayon de convergence.
\item Déduire de ce développement la valeur de $G^{(3)}(0)$.
\end{enumerate}
\end{Exa}

\corr \begin{enumerate}
\item Il existe des réels $a$, $b$ et $c$ tels que pour tout $x \in \mathbb{R} \setminus \lbrace -1,3 \rbrace$,
$$ f(x) = \dfrac{a}{x+1} + \dfrac{b}{(x+1)^2} +  \dfrac{c}{3-x}$$
En mettant au même dénominateur et par identification, on en déduit que :
$$ \forall x \in \mathbb{R} \setminus \lbrace -1,3 \rbrace , \; \dis f(x)=\dfrac{1}{16}\times\dfrac{1}{x+1}+\dfrac{1}{4}\times\dfrac{1}{(x+1)^2}+\dfrac{1}{16}\times\dfrac{1}{3-x}$$
Les primitives de $f$ sur $\left]-1,3\right[$ sont donc les fonctions $F$ définies sur cet intervalle par :
$$F(x)=\dfrac{1}{16}\ln \left( \dfrac{x+1}{3-x}\right) -\dfrac{1}{4}\times\dfrac{1}{(x+1)}+C$$
où $C\in\mathbb{R}$. Or on a :
$$F(1)=0\Longleftrightarrow C=\dfrac{1}{8}$$
Finalement, la primitive $G$ cherchée est définie par :
$$ \forall x\in \left]-1,3 \right[ , \; G(x)= \dfrac{1}{16}\ln \left( \dfrac{x+1}{3-x}\right) -\dfrac{1}{4}\times\dfrac{1}{(x+1)}+\dfrac{1}{8}$$
\item On sait que pour tout $x \in ]-1,1[$,
$$\dfrac{1}{1+x}=\sum\limits_{n=0}^{+\infty}(-1)^{n}x^n$$
Par dérivation terme à terme (la somme d'une série entière est $\mathcal{C}^{\infty}$ sur son intervalle ouvert de convergence), on en déduit que pour tout $x \in ]-1,1[$,
$$\dfrac{1}{(1+x)^2}=\sum\limits_{n=1}^{+\infty}(-1)^{n+1}nx^{n-1}$$
Si $x$ un réel différent de $3$, on a :
$$\dfrac{1}{3-x}=\dfrac{1} {3\left( 1-\dfrac{x}{3}\right) }$$
Si $\vert x \vert <3$, on a donc d'après le cours :
$$\dfrac{1}{3-x}=\dfrac{1}{3}\sum\limits_{n=0}^{+\infty}\dfrac{x^n}{3^n}$$
Finalement, $f$ est développable en série entière et on a pour tout $x \in ]-1,1[$,
$$ f(x) = \dfrac{1}{16} \sum\limits_{n=0}^{+\infty}(-1)^{n}x^n
+\dfrac{1}{4}\sum\limits_{n=0}^{+\infty}(-1)^{n}(n+1)x^{n}+\dfrac{1}{16}\times \dfrac{1}{3}\displaystyle\sum\limits_{n=0}^{+\infty}\dfrac{x^n}{3^n} = \displaystyle\sum\limits_{n=0}^{+\infty}\left(\dfrac{(-1)^n}{16}+\dfrac{(-1)^n(n+1)}{4}+\dfrac{1}{16\times3^{n+1}} \right) x^n$$
Le rayon de convergence de la série est supérieure ou égal à $1$ et de plus, pour $x=1$, la série diverge grossièrement ($n+1$ tend vers $+ \infty$ quand $n$ tend vers $+ \infty$). Ainsi, le rayon de la convergence vaut $1$.
\item Notons pour tout $n \geq 0$, les coefficients $a_n$ de la série entière étudiée.  D'après le cours, on a pour tout $n \geq 0$,
$$ a_n = \dfrac{f^n(0)}{n!}$$
Ainsi,
$$G^{(3)}(0)=f^{(2)}(0)=2!a_2=2\times\left(\dfrac{1}{16} +\dfrac{3}{4}+\dfrac{1}{16\times 27}\right)$$
et donc
$$ G^{(3)}(0)=\dfrac{44}{27}$$
\end{enumerate}

\begin{Exa} \begin{enumerate}
\item Montrer que l'ensemble de définition de $g$, définie par 
$$ g(x) = \int_{0}^1 \dfrac{\ln(1+xt)}{t} \dt$$
contient $]-1,1[$.
\item Déterminer le développement en série entière de $g$ sur $]-1,1[$.
\item Montrer que $g$ est de classe $\mathcal{C}^1$ sur $]0,1[$ et déterminer $g'$ de deux manières différentes.
\end{enumerate}
\end{Exa}

\corr \begin{enumerate}
\item Soit $x \in ]-1,1[$. Pour tout $t \in ]0,1]$, $xt \in ]-1,1[$ donc $1+xt >0$ et ainsi $t \mapsto \dfrac{\ln(1+xt)}{t} $ est continue sur $]0,1]$. Si $t$ tend vers $0$, on a :
$$ \dfrac{\ln(1+xt)}{t}  \sim \dfrac{xt}{t}= x$$
donc la fonction $t \mapsto \dfrac{\ln(1+xt)}{t} $ est prolongeable par continuité en $0$ et ainsi l'intégrale définissant $g(x)$ est faussement impropre en $0$ donc convergente. Ainsi, $]-1,1[ \subset \mathcal{D}_g$.
\item On sait que pour tout $u \in ]-1,1[$,
$$ \ln(1+u) = \sum_{n=1}^{+ \infty} \dfrac{(-1)^{n-1}}{n} u^n$$
Soit $x \in ]-1,1[$. Pour tout $t \in ]0,1]$, $xt \in ]-1,1[$ donc :
$$ \ln(1+xt) = \sum_{n=1}^{+ \infty} \dfrac{(-1)^{n-1}}{n} x^n t^n$$
et donc :
$$ \dfrac{\ln(1+xt)}{t} = \sum_{n=1}^{+ \infty} \dfrac{(-1)^{n-1}}{n} x^n t^{n-1}$$
La série entière $\dis \sum_{n \geq 1} \dfrac{(-1)^{n-1}}{n} x^n t^n$ a pour rayon de convergence $+ \infty$ si $x=0$ et en utilisant le critère de D'Alembert, $\dfrac{1}{\vert x \vert}$ si $x \neq 0$. Dans tous les cas, le rayon de convergence est strictement plus grand que $1$ donc la série entière converge uniformément sur $[0,1]$ donc par intégration terme à terme, on en déduit que :
\begin{align*}
g(x) & = \int_{0}^1 \sum_{n=1}^{+ \infty} \dfrac{(-1)^{n-1}}{n} x^n t^{n-1} \\
& = \sum_{n=1}^{+ \infty} \dfrac{(-1)^{n-1}}{n} x^n \int_{0}^1 t^{n-1} \dt \\
& = \sum_{n=1}^{+ \infty} \dfrac{(-1)^{n-1}}{n^2} x^n
\end{align*} 
On a donc obtenu le développement en série entière de $g$ sur $]-1,1[$.
\item La fonction g est somme d'une série entière convergente sur $]-1,1[$ donc $g$ est $\mathcal{C}^{\infty}$ sur $]-1,1[$. Par dérivation terme à terme, on en déduit que pour tout $x \in ]0,1[$,
$$g'(x) =  \sum_{n=1}^{+ \infty} \dfrac{(-1)^{n-1}}{n^2} nx^{n-1} = \sum_{n=1}^{+ \infty} \dfrac{(-1)^{n-1}}{n} x^{n-1}$$
donc 
$$ g'(x) = \dfrac{1}{x} \sum_{n=1}^{+ \infty} \dfrac{(-1)^{n-1}}{n} x^{n} = \dfrac{\ln(1+x)}{x}$$
Retrouvons ce résultat avec le théorème de dérivation pour des intégrales à paramètres :
\begin{itemize}
\item Pour tout $x \in ]0,1[$, $t \mapsto \dfrac{\ln(1+xt)}{t}$ est continue sur $]0,1]$ et intégrable sur $]0,1]$ (on a déjà montré que l'intégrale convergeait et l'intégrande est positive).
\item Pour tout $t \in ]0,1]$, $x \mapsto \dfrac{\ln(1+xt)}{t}$ est $\mathcal{C}^1$ sur $]0,1[$ de dérivée $x \mapsto \dfrac{1}{1+xt}$ qui est une fonction continue sur $]0,1[$.
\item Pour tout $t \in ]0,1]$ et tout $x \in ]0,1]$,
$$ 0 \leq \dfrac{1}{1+xt} \leq 1$$
et la fonction $t \mapsto 1$ est intégrable sur $]0,1]$.
\end{itemize}
Par théorème de dérivation pour les intégrales à paramètres, on en déduit que pour tout $x \in ]0,1[$,
$$ g'(x) = \int_{0}^1 \dfrac{1}{1+xt} \dt$$
et ainsi :
$$ g'(x) = \dfrac{1}{x} [ \ln(1+xt) ]_0^1 = \dfrac{\ln(1+x)}{x}$$
\end{enumerate}
\medskip

\begin{center}
\textit{{ {\large Intégration terme à terme}}}
\end{center}

\medskip

\begin{Exa} Montrer que $\dis \int_{0}^1 \dfrac{\ln(1+t)}{t}\dt = \sum_{n=1}^{+ \infty} \dfrac{(-1)^{n-1}}{n^2} \cdot$
\end{Exa}

\corr Pour tout $t \in ]0,1[$, on a :
$$ \ln(1+t) = \sum_{n=0}^{+ \infty} (-1)^{n} \dfrac{x^{n+1}}{n+1}$$
donc
$$ \dfrac{\ln(1+t)}{t} = \sum_{n=0}^{+ \infty} (-1)^{n} \dfrac{x^{n}}{n+1}$$
Posons pour tout entier $n \geq 0$,
$$ f_n : t \mapsto (-1)^{n} \dfrac{x^{n}}{n+1}$$
Vérifions les hypothèses du théorème d'intégration terme à terme.

\begin{itemize}
\item Pour tout entier $n \geq 0$, $f_n$ est continue sur $]0,1[$ et prolongeable par continuité sur $[0,1]$ donc intégrable sur $]0,1[$.
\item La série de fonctions de terme général $f_n$ converge simplement vers $S: t \mapsto \dfrac{\ln(1+t)}{t}$ sur $]0,1[$ qui est continue sur $]0,1[$.
\item Pour tout entier $n \geq 0$,
\begin{align*}
\int_0^1 \vert f_n(x) \vert\dx & = \int_0^1 \dfrac{x^n}{n+1} \dx \\
& = \dfrac{1}{(n+1)^2}
\end{align*}
Ainsi, la série de terme général $\int_0^1 \vert f_n(x) \vert\dx$ converge.
\end{itemize}
D'après le théorème d'intégration terme à terme, on en déduit que $S$ est intégrable sur $]0,1[$ et que l'on a :
$$ \int_0^1 \dfrac{\ln(1+t)}{t} \dt = \sum_{n=1}^{+ \infty} \int_0^1 f_n(x) \dx = \sum_{n=0}^{+ \infty} (-1)^{n} \times \dfrac{1}{(n+1)^2} = \sum_{n=1}^{+ \infty} \dfrac{(-1)^{n-1}}{n^2}$$

\begin{Exa}
\begin{enumerate}
\item Justifier l'existence de $I= \dis \int_{0}^1 \ln(x) \ln(1-x) \dx$.
\item Montrer que $I =\dis \sum_{n=1}^{+ \infty} \dfrac{1}{n(n+1)^2}$ puis calculer $I$ sachant que $\dis \sum_{n=1}^{+ \infty} \dfrac{1}{n^2} = \dfrac{\pi^2}{6}\cdot$
\end{enumerate}
\end{Exa}


\begin{corr}
\begin{enumerate}
\item La fonction $x \mapsto \ln(x) \ln(1-x)$ est continue (et positive) sur $]0,1[$. L'intégrale est impropre en $0$ et en $1$.
\begin{itemize}
\item On a :
$$  \ln(x) \ln(1-x) \underset{0}{\sim} -x \ln(x)$$
Ainsi par théorème des croissances comparées,
$$ \lim_{x \rightarrow 0 }   \ln(x) \ln(1-x) =  \lim_{x \rightarrow 0 }  -x \ln(x) = 0$$
L'intégrale est donc faussement impropre en $0$.
\item On obtient de même que l'intégrale est faussement impropre en $1$.
%$$  \ln(x) \ln(1-x) = \ln(1+x-1) \ln(1-x)\underset{1}{\sim} (x-1) \ln(1-x)$$
%La fonction $x \mapsto (x-1) \ln(1-x)$ est prolongeable par continuité en $1$ (elle a pour limite $0$ en $1$ d'après le théorème des croissances comparées) donc $\dis \int_{1/2}^{1} (x-1) \ln(1-x)\dx$ converge. Par critère de comparaison (les intégrandes étant positives), on en déduit que $\dis \int_{1/2}^{1} \ln(x) \ln(1-x) \dx$ converge.
\end{itemize}
Ainsi, $\dis \int_{0}^1 \ln(x) \ln(1-x) \dx$ converge.

%\medskip

\noindent \textit{Remarque.} Pour étudier l'impropreté en $1$, on peut utiliser le changement de variable $u : t \mapsto 1-t$ et se ramener à l'impropreté en $0$.


\medskip
\item On utilise le théorème d'intégration terme à terme. Pour tout $x \in ]0,1[$, on a :
$$ \ln(1-x) = - \sum_{n=1}^{+ \infty} \dfrac{x^n}{n}$$
et donc :
$$ \ln(x)\ln(1-x) = \sum_{n=1}^{+ \infty} - \dfrac{x^n \ln(x)}{n}$$
Posons pour tout $n \geq 1$ et tout $x \in ]0,1[$, $f_n(x)=  - \dfrac{x^n \ln(x)}{n} \cdot$

\begin{itemize}
\item Pour tout $n \geq 1$, $f_n$ est continue sur $]0,1[$. La fonction $f_0$ est intégrable sur $]0,1[$ (fonction de référence) et pour $n \geq 1$, $f_n$ se prolonge par continuité en $0$ d'après le théorème de croissances comparées donc $f_n$ est aussi intégrable sur $]0,1[$.
\item La série de fonctions $\dis \sum_{n \geq 1} f_n$ converge simplement sur $]0,1[$ vers la fonction $x \mapsto \ln(x) \ln(1-x)$.
\item Soit $n \geq 1$. On a :
$$\int_{0}^1 \vert f_n(x) \vert dx  = \int_{0}^1  - \dfrac{x^n \ln(x)}{n} \dx$$
Les fonctions $u: x \mapsto -\dfrac{x^{n+1}}{n+1}$ et $v: x \mapsto \ln(x)$ sont de classe $\mathcal{C}^1$ sur $]0,1[$, de dérivées respectives $x \mapsto -x^n$ et $x \mapsto \dfrac{1}{x}$ et le produit $uv$ admet une limite nulle en $0$ (par théorème des croissances comparées) et une limite nulle en $1$. Par théorème d'intégration par parties, on a :
\begin{align*}
\int_{0}^1 \vert f_n(x) \vert dx & = \dfrac{1}{n}\int_{0}^1 \dfrac{x^{n+1}}{n+1} \times \dfrac{1}{x} \dx \\
& = \dfrac{1}{n(n+1)} \int_{0}^1 x^n \dx \\
& = \dfrac{1}{n(n+1)^2} 
\end{align*}
La série de terme général $\dfrac{1}{n(n+1)^2}$ est convergente (par comparaison à une série de Riemann).
\end{itemize}
Par théorème d'intégration terme à terme, on en déduit que :
$$ \int_{0}^1 \ln(x)\ln(1-x) \dx = - \sum_{n=1}^{+ \infty} \int_{0}^1 \dfrac{x^n\ln(x)}{n} \dx = \sum_{n=1}^{+\infty} \dfrac{1}{n(n+1)^2}$$
Remarquons maintenant que pour tout $n \geq 1$,
$$ \dfrac{1}{n(n+1)^2} = \dfrac{1+n - n}{n(n+1)^2} = \dfrac{1}{n(n+1)}- \dfrac{1}{(n+1)^2}$$
et on recommence cette idée :
$$ \dfrac{1}{n(n+1)} = \dfrac{1+n-n}{n(n+1)} = \dfrac{1}{n}- \dfrac{1}{n+1}$$
et ainsi :
$$ \dfrac{1}{n(n+1)^2} = \dfrac{1}{n}- \dfrac{1}{n+1} - \dfrac{1}{(n+1)^2}$$
Pour tout $N \geq 1$, on a alors :
\begin{align*}
\sum_{n=1}^{N} \dfrac{1}{n(n+1)^2} & = \sum_{n=1}^{N}\dfrac{1}{n}- \dfrac{1}{n+1} -  \sum_{n=1}^{N} \dfrac{1}{(n+1)^2} \\
& = 1 - \dfrac{1}{N+1} - \sum_{n=2}^{N+1} \dfrac{1}{n^2} 
\end{align*}
par télescopage et changement d'indice. Par passage à la limite et d'après l'indication, on a alors :
$$ I = 1 - \sum_{n=2}^{+ \infty} \dfrac{1}{n^2} = 1 - \dfrac{\pi^2}{6} +1 = 2 - \dfrac{\pi^2}{6}$$

\end{enumerate}
\end{corr}

\begin{Exa} Montrer que pour tout $x \in \mathbb{R}$, $\ch(x) \leq e^{x^2/2}$.
\end{Exa}

\corr Pour tout réel $x$,
$$ \ch(x) = \sum_{k=0}^{+ \infty} \dfrac{x^{2k}}{(2k)!} \; \hbox{ et } \; e^{x^2/2} = \sum_{k=0}^{+ \infty} \dfrac{x^{2k}}{2^k k!}$$
Il suffit de montrer que pour tout entier $k \geq 0$,
$$ \dfrac{1}{(2k)!} \leq \dfrac{1}{2^k	k!}$$
Pour $k=0$ c'est évident. Pour $k \geq 1$, il suffit de montrer que :
$$ 2^k \leq \dfrac{(2k)!}{k!} = (2k) \times (2k-1) \times \cdots \times (k+1)$$
Il y a $k$ termes dans le produit et chaque terme est supérieur ou égal à $2$ car $k \geq 1$. Le résultat est donc prouvé.

\medskip

\begin{center}
\textit{{ {\large Divers}}}
\end{center}

\medskip


\begin{Exa} On pose $a_0=1$, $a_1=1$ et pour tout $n \geq 0$, $a_{n+2} = a_{n+1} + (n+1)a_n$.
\begin{enumerate}
\item Montrer que pour tout $n \geq 0$, $\dfrac{a_n}{n!} \leq 1$.
\item Montrer que $f$, définie par $f(x) = \dis \sum_{k=0}^{+ \infty} \dfrac{a_k}{k!}x^k$, est solution d'une équation différentielle du premier ordre que l'on résoudra.
\item Exprimer pour tout $p \geq 0$, $a_{2p}$ et $a_{2p+1}$.
\end{enumerate}
\end{Exa}

\corr \begin{enumerate}
\item Montrons par récurrence double que pour tout $n \geq 0$, $\dfrac{a_n}{n!} \leq 1$.

\medskip

\noindent La propriété est clairement vraie au rang $0$ et $1$.

\medskip

\noindent Soit $n \in \mathbb{N}$ tel que $a_n \leq n!$ et $a_{n+1} \leq (n+1)!$. Montrons que :
$$ a_{n+2} \leq (n+2)!$$
On a :
\begin{align*}
a_{n+2} & = a_{n+1} + (n+1)a_n \\
& \leq (n+1)! + (n+1) n! \\
& \leq 2(n+1)! \\
& \leq (n+2)(n+1)! \\
& = (n+2)!
\end{align*}
C'est ce qu'il fallait prouver. Ainsi, pour tout $n \geq 0$, $\dfrac{a_n}{n!} \leq 1$.
\item La série entière $\dis \sum x^k $ a pour rayon de convergence $1$ et d'après la question précédente, pour tout $n \geq 0$, $0 \leq \dfrac{a_n}{n!} \leq 1$ donc par critère de comparaison, on en déduit que la série entière $\dis \sum \dfrac{a_k}{k!}x^k$ a un rayon de convergence supérieur ou égal à $1$. Ainsi, $f$ est définie et $\mathcal{C}^{\infty}$ sur (au moins) $]-1,1[$. Par dérivation terme à terme, on a pour tout $x \in ]-1,1[$,
\begin{align*}
f'(x) & = \sum_{k=1}^{+ \infty} \dfrac{a_k}{(k-1)!} x^{k-1} \\
& = a_1 + \sum_{k=2}^{+ \infty} \dfrac{a_k}{(k-1)!} x^{k-1} \\
& = 1 + \sum_{k=0}^{+ \infty} \dfrac{a_{k+2}}{(k+1)!} x^{k+1} \\
& = 1+ \sum_{k=0}^{+ \infty} \dfrac{a_{k+1}}{(k+1)!} x^{k+1}  + \sum_{k=0}^{+ \infty} \dfrac{(k+1)a_{k}}{(k+1)!} x^{k+1} \\
& = 1+ \sum_{k=1}^{+ \infty} \dfrac{a_{k}}{k!} x^{k} +  \sum_{k=0}^{+ \infty} \dfrac{a_{k}}{k!} x^{k+1} \\
& = 1+ f(x)-1  + x \sum_{k=0}^{+ \infty} \dfrac{a_{k}}{k!} x^{k} \\
& = f(x)+xf(x) \\
& = (x+1)f(x)
\end{align*} 
Ainsi, $f$ est solution de l'équation différentielle linéaire d'ordre $1$ homogène suivante :
$$ y'=(x+1)y$$
donc l'ensemble des solutions est la droite vectorielle engendrée par $g : ]-1,1[ \rightarrow \mathbb{R}$ définie par $g(x)=e^{x+x^2/2}$. Sachant que $f(0)=a_0=1$, on en déduit que $f=g$.
\item Pour tout $x \in ]-1,1[$, on a :
\begin{align*}
f(x) & = e^x \times e^{x^2/2} \\
& = \left( \sum_{k=0}^{+ \infty} \dfrac{x^k}{k!} \right) \left( \sum_{k=0}^{+ \infty} \dfrac{x^{2k}}{2^k k!} \right)  \\
& = \left( \sum_{k=0}^{+ \infty} \dfrac{x^k}{k!} \right) \left( \sum_{k=0}^{+ \infty} c_k x^k \right)  
\end{align*}
où pour tout $k \geq 0$, $c_k=0$ si $k$ est impair et 
$$ c_k = \dfrac{1}{2^{k/2} (k/2)!}$$
si $k$ est pair. Par produit de Cauchy (les deux séries entières ont un rayon de convergence infini), on en déduit que pour tout $x \in ]-1,1[$,
$$ f(x) = \sum_{n=0}^{+ \infty} \left(\sum_{k=0}^n \dfrac{c_k}{(n-k)!} \right) x^n$$
et on sait aussi que :
$$ f(x) = \sum_{n=0}^{+ \infty} \dfrac{a_n}{n!} x^n$$
donc par unicité du développement en série entière, on en déduit que pour tout $n \geq 0$,
$$\dfrac{a_n}{n!} = \sum_{k=0}^n \dfrac{c_k}{(n-k)!}$$
Soit $p \geq 0$. Alors :
\begin{align*}
a_{2p} & = (2p)! \sum_{j=0}^p \dfrac{c_{2j}}{(2p-2j)!} \\
& = (2p)! \sum_{j=0}^p \dfrac{1}{2^j j! (2p-2j)!} \\
\end{align*}
et 
\begin{align*}
a_{2p+1} & = (2p+1)! \sum_{j=0}^p \dfrac{c_{2j}}{(2p+1-2j)!} \\
& = (2p+1)! \sum_{j=0}^p \dfrac{1}{2^j j! (2p+1-2j)!} \\
\end{align*}
\end{enumerate}

\begin{Exa}[\ding{80}] Soit $(a_n)_{n \geq 0}$ une suite de réels telle que $na_n$ tend vers $0$ quand $n$ tend vers $+ \infty$.
\begin{enumerate}
\item Montrer que $\dis \sum_{n \geq 0} a_n x^n$ a un rayon de convergence supérieur ou égal à $1$.
\item Montrer que $\dis \sum_{n=0}^{+ \infty} a_n x^n = o(\ln(1-x))$ quand $x$ tend vers $1^{-}$.
\end{enumerate}
\end{Exa}

\begin{corr} 
\begin{enumerate}
\item D'après l'hypothèse,
$$ a_n \underset{+ \infty}{=} o \left(\dfrac{1}{n}\right)$$
La série entière $\dis \sum \dfrac{z^n}{n}$ a un rayon de convergence égal à $1$ (série entière usuelle) donc par critère de comparaison, la série entière $\dis \sum a_n z^n$ a une rayon de convergence supérieur ou égal à $1$.
\item Posons pour tout $x \in ]0,1[$,
$$ f(x) = \sum_{n=0}^{+ \infty} a_n x^n$$
D'après l'hypothèse, 
$$ a_n \underset{+ \infty}{=} o \left(\dfrac{1}{n}\right)$$
donc il existe une suite $(\varepsilon_n)_{n \geq 1}$ convergeant vers $0$ tel que pour tout $n \geq 1$,
$$ a_n = \dfrac{\varepsilon_n}{n}$$
Soit $\varepsilon>0$. Il existe un entier $N \geq 1$, tel que pour tout $n \geq N$, $\vert \varepsilon_n \vert \leq \varepsilon$ et ainsi :
$$ \left\vert f(x) - \sum_{k=1}^{N-1}  \dfrac{\varepsilon_n}{n}x^n \right\vert \leq \varepsilon \sum_{k=N}^{+ \infty} \dfrac{x^n}{n}$$
La majoration est licite car la deuxième somme est associée à une série convergente (rayon de convergence de la série entière égal à $1$). De plus, on a :
$$ \sum_{k=N}^{+ \infty} \dfrac{x^n}{n} \leq \sum_{k=1}^{+ \infty} \dfrac{x^n}{n} =  -\ln(1-x)$$
si $x$ tend vers $1$, $-\ln(1-x)$ tend vers $+ \infty$ tandis que 
$$ \sum_{k=1}^{N-1} \dfrac{\varepsilon_n}{n}x^n$$
tend vers une constante. Ainsi, si $x$ est assez proche de $1$, on a :
$$ \left\vert \sum_{k=1}^{N-1}  \dfrac{\varepsilon_n}{n} x^n \right\vert \leq -\varepsilon \ln(1-x)$$
On en déduit d'après l'inégalité triangulaire que pour $x$ assez proche de $1$,
\begin{align*}
\vert f(x) \vert & =  \left\vert f(x) - \sum_{k=1}^{N-1} \dfrac{\varepsilon_n}{n} x^n +  \sum_{k=1}^{N-1}  \dfrac{\varepsilon_n}{n} x^n \right\vert \\
& \leq  \left\vert f(x) - \sum_{k=1}^{N-1} \dfrac{\varepsilon_n}{n}x^n \right\vert +  \left\vert\sum_{k=1}^{N-1}  \dfrac{\varepsilon_n}{n}x^n\right\vert \\
& \leq -2 \varepsilon \ln(1-x) \\
& = 2 \varepsilon \vert \ln(1-x) \vert
\end{align*}
On en déduit que $f(x)= o(\ln(1-x))$ quand $x$ tend vers $1^{-}$.
\end{enumerate}
\end{corr}

\begin{Exa} Soit $(u_n)_{n \geq 0}$ la suite définie par $u_0 >0$ et pour tout $n \geq 0$, $u_{n+1}= \ln(1+u_n)$.
\begin{enumerate}
\item Prouver la convergence et déterminer la limite de cette suite.
\item Déterminer le rayon de convergence de $\dis \sum_{n \geq 0} u_n x^n$.
\item Soit $(v_n)_{n \geq 1}$ une suite de réels convergeant vers $\ell \in \mathbb{R}$. Montrer que $(w_n)_{n \geq 1}$ définie par :
$$ w_n = \dfrac{1}{n} \sum_{k=1}^n v_k$$
converge vers $\ell$.
\item A l'aide de la suite de terme général $\dfrac{1}{u_{n+1}} - \dfrac{1}{u_n}$, déterminer un équivalent de $u_n$ quand $n$ tend vers $+ \infty$.
\end{enumerate}
\end{Exa}

\newpage

\begin{corr} 
\begin{enumerate}
\item Une récurrence immédiate montre que la suite est strictement positive. Soit $n \geq 0$. On sait que pour tout réel $x>-1$,
$$ \ln(1+x) \leq x$$
donc pour tout $n \geq 0$,
$$ u_{n+1} = \ln(1+u_n) \leq u_n$$
Ainsi, $(u_n)_{n \geq 0}$ est décroissante et minorée par $0$ donc converge vers un réel $\ell \geq 0$. Par passage à la limite dans la relation de récurrence définissant $(u_n)_{n \geq 0}$, on sait que :
$$ \ln(1+ \ell) = \ell$$
L'étude de la fonction $x \mapsto \ln(1+x) - x$ sur $\mathbb{R}_+$ montre que la seule possibilité est $\ell=0$.
\item D'après la question précédente, $u_n$ tend vers $0$ quand $n$ tend vers $+ \infty$ donc :
$$ u_{n+1} = \ln(1+u_n) \underset{+ \infty}{\sim} u_n $$
et sachant que $(u_n)_{n \geq 0}$ est strictement positive :
$$ \lim_{n \rightarrow + \infty} \left\vert \dfrac{u_{n+1}}{u_n} \right\vert = 1$$
D'après le critère de D'Alembert, on en déduit que la série entière $\dis \sum_{n \geq 0} u_n x^n$ a pour rayon de convergence $1$.
\item C'est la preuve du Théorème de Césaro (voir TD1).
\item Soit $n \geq 1$. Sachant que $(u_n)$ tend vers $0$, on a :
\begin{align*}
\dfrac{1}{u_{n+1}} - \dfrac{1}{u_n} & = \dfrac{u_n-u_{n+1}}{u_{n+1}u_n} \\
& = \dfrac{u_n-\ln(1+u_n)}{u_{n+1}u_n} \\
& \underset{+ \infty}{=} \dfrac{u_n^2/2 + o(u_n^2)}{u_{n+1}u_n} \\
& \underset{+ \infty}{\sim} \dfrac{u_n^2/2}{u_n^2} \\
& \underset{+ \infty}{\sim} \dfrac{1}{2}
\end{align*}
Ainsi,
$$ \lim_{n \rightarrow + \infty} \dfrac{1}{u_{n+1}} - \dfrac{1}{u_n}  = \dfrac{1}{2}$$
D'après la question précédente, on en déduit que :
$$ \lim_{n \rightarrow + \infty} \dfrac{1}{n} \sum_{k=1}^n  \dfrac{1}{u_{k+1}} - \dfrac{1}{u_k} = \dfrac{1}{2}$$
donc par télescopage :
$$ \lim_{n \rightarrow + \infty} \dfrac{1}{n} \left(  \dfrac{1}{u_{n+1}} - \dfrac{1}{u_1} \right) = \dfrac{1}{2}$$
On a ainsi :
$$  \lim_{n \rightarrow + \infty} \dfrac{1}{n}   \dfrac{1}{u_{n+1}}  = \dfrac{1}{2}$$
On en déduit que :
$$ \dfrac{1}{n u_{n+1}} \underset{+ \infty}{\sim} \dfrac{1}{2}$$
puis 
$$ n u_{n+1} \underset{+ \infty}{\sim} 2$$
et sachant que $u_n \underset{+ \infty}{\sim} u_{n+1}$, on a finalement que :
$$ u_n \underset{+ \infty}{\sim} \dfrac{2}{n}$$
\end{enumerate}
\end{corr}

\begin{Exa}[\ding{80}] Soit $(u_n)_{n \geq 0}$ la suite définie par $u_0=0$ et pour tout $n \geq 0$ par 
$$ \dfrac{u_{n+1}}{u_n} = \dfrac{n+a}{n+b}$$
où $a$ et $b$ sont deux réels strictement positifs. On pose pour tout $n \geq 1$, $v_n = \ln(n^{b-a}u_n)$.
\begin{enumerate}
\item Montrer la convergence de $\Sum{}{}  v_{n+1}- v_n$ et en déduire une condition sur $a$ et $b$ pour que $\Sum{}{} u_n$ converge.
\item On considère que la condition précédente est vérifiée. Soit $f$ définie par :
$$ f(x) = \sum_{k=0}^{+ \infty} u_k x^k$$
Déterminer l'ensemble de définition de $f$ et déterminer $f(-1)$.
\end{enumerate}
\end{Exa}

\begin{corr} 
\begin{enumerate}
\item Soit $n \geq 1$. Alors :
\begin{align*}
v_{n+1}-v_n & = \ln((n+1)^{b-a} u_{n+1}) - \ln(n^{b-a} u_n) \\
& = (b-a) (\ln(n+1)- \ln(n)) + \ln \left( \dfrac{u_{n+1}}{u_n} \right) \\
& = (b-a) \ln \left( 1+\dfrac{1}{n} \right) + \ln\left( \dfrac{n+a}{n+b} \right) \\
& = (b-a) \ln \left( 1+\dfrac{1}{n} \right) + \ln\left( 1+\dfrac{a-b}{n+b} \right) 
\end{align*}
et donc :
\begin{align*}
v_{n+1}-v_n  & \underset{+ \infty}{=} \dfrac{(b-a)}{n} - \dfrac{(b-a)^2}{2n^2} +\dfrac{a-b}{n+b} - \dfrac{(a-b)^2}{2(n+b)^2} + o \left( \dfrac{1}{n^2} \right) \\
& = (b-a) \left( \dfrac{1}{n} - \dfrac{1}{n+b} \right) - \dfrac{(b-a)^2}{2n^2} - \dfrac{(a-b)^2}{2(n+b)^2} + o \left( \dfrac{1}{n^2} \right) \\
& = \dfrac{(b-a)b}{n(n+1)} - \dfrac{(b-a)^2}{2n^2} - \dfrac{(a-b)^2}{2(n+b)^2} + o \left( \dfrac{1}{n^2} \right) \\
\end{align*}
et ainsi,
$$ v_{n+1}-v_n \underset{+ \infty}{=}  O \left( \dfrac{1}{n^2} \right)$$
Par critère de comparaison, on  en déduit que la série de terme général $v_{n+1}-v_n$ converge absolument et donc converge.

\medskip

\noindent La série de terme général $v_{n+1}-v_n$ est téléscopique et convergente donc $(v_n)_{n \geq 0}$ converge donc il existe un réel $\ell$ tel que :
$$ \lim_{n \rightarrow + \infty} \ln(n^{b-a} u_n) = \ell$$
donc par continuité de l'exponentielle en $\ell$ :
$$ \lim_{n \rightarrow + \infty} n^{b-a} u_n = e^{\ell}$$
et ainsi,
$$ u_n \underset{+ \infty}{\sim} \dfrac{e^{\ell}}{n^{b-a}}$$
Remarquons que $e^{\ell} \neq 0$ et que les séries étudiées sont à termes positifs. Par comparaison à une série de Riemann, on en déduit que $\dis \sum u_n$ converge si et seulement si $b-a>1$.
\item Supposons que $b>a+1$. La suite $(u_n)_{n \geq 1}$ est à termes strictement positifs et on pour tout $n \geq 0$,
$$ \dfrac{u_{n+1}}{u_n} = \dfrac{n+a}{n+b} \underset{+ \infty}{\sim} 1$$
D'après le critère de D'Alembert, on en déduit que le rayon de convergence de la série entière étudiée vaut $1$. Cette série converge pour $x=1$ d'après la question précédente et la série converge absolument pour $x=-1$ (on revient au cas $x=1$ par convergence absolue) et donc converge. Ainsi, $f$ est définie sur $[-1,1]$.

\medskip

\noindent On sait que pour tout $n \geq 0$,
$$ (n+b) u_{n+1} = (n+a) u_n$$
donc :
$$ (n+1) u_{n+1} + (b-1) u_{n+1} = n u_n + au_n $$
puis :
$$ (n+1) u_{n+1}- n u_n = a u_n + (1-b) u_{n+1} = (1-b) u_{n+1} - (1-b) u_n +  (a-b+1) u_n$$
Sachant que $a-b+1 \neq 0$, on en déduit que :
$$ u_n = \dfrac{1}{a-b+1} (n+1) u_{n+1}- n u_n ) + (b-1) u_{n+1} - (b-1) u_n)$$
Ainsi, par télescopage, on en déduit que pour $N \geq 0$,
$$ \sum_{k=0}^N u_k = \dfrac{1}{a-b+1}  ( (N+1) u_{N+1} +(b-1) u_{N+1} - (b-1)u_0 )$$
Or $u_0=1$ donc : 
$$  \sum_{k=0}^N u_k = \dfrac{1}{a-b+1}  ( (N+1) u_{N+1} +(b-1) u_{N+1} - (b-1))$$
On sait que :
$$ u_n \underset{+ \infty}{\sim} \dfrac{e^{\ell}}{n^{b-a}}$$
et $b-a>1$ donc $(u_n)$ et $(nu_n)$ converge vers $0$. Par passage à la limite, on en déduit que :
$$ f(1) = \dfrac{1-b}{a-b+1}$$
\end{enumerate}
\end{corr}



\begin{Exa} Montrer que la fonction $x \mapsto \dfrac{\sin(x)}{x}$ se prolonge en une fonction $\mathcal{C}^{\infty}$ sur $\mathbb{R}$.
\end{Exa}

\corr Soit $f$ cette fonction. Remarquons que 
$$ \lim_{x \rightarrow 0} f(x) = 1$$
Pour tout réel $x$, on a :
$$ \sin(x) = \sum_{k=0}^{+ \infty} \dfrac{(-1)^k x^{2k+1}}{(2k+1)!}$$
donc pour $x \neq 0$,
$$ f(x) = \sum_{k=0}^{+ \infty} \dfrac{(-1)^k x^{2k}}{(2k+1)!}$$
$f$ coïncide donc avec la somme d'une série entière sur $\mathbb{R}^*$ (cette série entière a donc un rayon de converge infini) et la limite de $f$ en $0$ est égal à la somme de cette série en $0$. Or la somme d'une série entière est $\mathcal{C}^{\infty}$ sur son intervalle de convergence (qui est $\mathbb{R}$). Finalement, $x \mapsto \dfrac{\sin(x)}{x}$ se prolonge en une fonction $\mathcal{C}^{\infty}$ sur $\mathbb{R}$.

\begin{Exa} \begin{enumerate}
\item Déterminer le rayon de convergence de la série entière $\Sum{n \geq 0}{} \dfrac{x^n}{(2n)!} \cdot$

\noindent On note $S$ la somme de cette série entière.
\item Donner le développement en série entière en 0 de la fonction $x\mapsto \text{ch}(x)$  et précisez le rayon de convergence.
\item \begin{enumerate}
	\item Déterminer $S$.
	\item On considère la fonction $f$ définie sur $\mathbb{R}$ par :
	\begin{equation*}
	f(0)=1,\ \ f(x)=\text{ch}(\sqrt x)\text{ pour $x>0$},\ \ f(x)=\cos(\sqrt{-x})\text{ pour $x<0$}\ .
	\end{equation*}
	Démontrer que $f$ est de classe $\mathcal{C}^{\infty}$ sur $\mathbb{R}$.
	\end{enumerate}
\end{enumerate}
\end{Exa}

\corr \begin{enumerate}
\item Pour $x \neq 0$, posons  pour tout entier $n \geq 0$,
 $$u_n  = \left\vert \dfrac{x^n }{(2n)!} \right\vert > 0$$
Alors
$$ \dfrac{u_{n+1}}{u_n} = \dfrac{ \vert x \vert }{(2n+2)(2n+1)} \underset{n \rightarrow + \infty}{\longrightarrow} 0$$
D'après le critère de d'Alembert, on en déduit que la série $\Sum{n \geq 0}{} {\dfrac{{x^n }}{{(2n)!}}} $ converge absolument (donc converge) pour tout $x \in \mathbb{R}$ donc le rayon de convergence de cette série entière est $R =  + \infty $.
\item C'est du cours :
$$\forall x\in \mathbb{R} , \; \textrm{ch} (x) = \displaystyle\sum\limits_{n = 0}^{ + \infty } {\dfrac{{x^{2n} }}{{(2n)!}}} $$ 
et le rayon de convergence du développement en série entière de $\textrm{ch}$  est égal à $ + \infty $.
\item 
\begin{enumerate}
\item Soit $x \geq 0$. Alors $x= t^2$ (avec $t = \sqrt{x}$) et 
$$S(x) = \displaystyle\sum\limits_{n = 0}^{ + \infty } {\dfrac{{x^n }}{{(2n)!}}}  = \displaystyle\sum\limits_{n = 0}^{ + \infty } {\dfrac{{t^{2n} }}{{(2n)!}}}  = \textrm{ch} (t) = \textrm{ch}( \sqrt x) $$
Pour $x < 0$, on peut écrire $x =  - t^2 $ (avec $t=\sqrt{-x}$) et 
$$S(x) = \displaystyle\sum\limits_{n = 0}^{ + \infty } {\dfrac{{x^n }}{{(2n)!}}}  = \displaystyle\sum\limits_{n = 0}^{ + \infty } {\dfrac{{( - 1)^n t^{2n} }}{{(2n)!}}}  = \cos (t) = \cos (\sqrt { - x}) $$
\item La fonction $f$ est la fonction $S$. Or $S$ est  de classe $\mathcal{C}^\infty $ sur $\mathbb{R}$ car développable en série entière à l'origine avec un rayon de convergence égal à $+\infty$. Ainsi, $f$ est  de classe $\mathcal{C}^\infty  $ sur $\mathbb{R}$.
\end{enumerate}
\end{enumerate}
\end{document}
