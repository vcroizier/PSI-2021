\documentclass[french,11pt,twoside]{VcCours}

\newenvironment{ApplicationDirecte}{\textbf{Application directe du cours :}

}{}
\newcommand{\dx}{\text{d}x}
\newcommand{\dt}{\text{d}t}
\DeclareMathOperator{\e}{e}
\newcommand{\Sum}[2]{\sum_{#1}^{#2}}
\newcommand{\Int}[2]{\int_{#1}^{#2}}

\renewcommand{\trou}[1]{{\color{white}#1}}
%\renewcommand{\trou}[1]{{\color{blue}#1}}

\begin{document}

\Titre{PSI}{Promotion 2021--2022}{Mathématiques}{Chapitre 5 : Matrices et applications linéaires}

\tableofcontents
\separationTitre


Dans tout le chapitre, $n$, $k$, $q$ et $p$ seront des entiers naturels non nuls et $\mathbb{K}$ désignera $\mathbb{R}$ ou $\mathbb{C}$.

\section{Calcul matriciel}
\subsection{L'espace vectoriel $\mathcal{M}_{n,p}(\mathbb{K})$}

On note $\mathcal{M}_{n,p}(\mathbb{K})$ l'ensemble des matrices à $n$ lignes et $p$ colonnes à coefficients dans $\mathbb{K}$.
%
%Soient $A=(a_{i,j})_{\substack{1 \leq i \leq n \\ 1 \leq j \leq p}}$ et $B=(b_{i,j})_{\substack{1 \leq i \leq n \\ 1 \leq j \leq p}}$ deux éléments de $\mathcal{M}_{n,p}(\mathbb{K})$ (l'ensemble des matrices à $n$ lignes et $p$ colonnes à coefficients dans $\mathbb{K}$) et $\lambda \in \mathbb{K}$.
%
%\medskip
%
%\begin{itemize}
%\item La matrice $\lambda \cdot A$ (ou plus simplement $\lambda A$) est la matrice de $\mathcal{M}_{n,p}(\mathbb{K})$ définie par :
%$$ \forall i \in \iii{1}{n}, \, \forall j \in \iii{1}{p}, \quad ( \lambda \cdot A)_{i,j} = \lambda a_{i,j} $$
%\item La matrice $A+B$ est la matrice de $\mathcal{M}_{n,p}(\mathbb{K})$ définie par :
%$$ \forall i \in \iii{1}{n}, \, \forall j \in \iii{1}{p}, \quad (A+B)_{i,j} = a_{i,j} + b_{i,j}$$
%\end{itemize}
%
%\medskip
%
%On définit ainsi une loi de composition interne $+$ et une loi de composition externe $\cdot$ sur $\mathcal{M}_{n,p}(\mathbb{K})$ qui en font un $\mathbb{K}$-espace vectoriel.
%
%\medskip

\medskip

Pour tout $k\in \iii{1}{n}$ et $ l \in \iii{1}{p}$, on définit la matrice $E_{k,l}$ de $\mathcal{M}_{n,p}( \mathbb{K})$ comme étant la matrice dont tous les coefficients sont nuls exceptés le coefficient de la $k$-ième ligne et de la $l$-ième colonne qui vaut $1$. Autrement dit,
$$ E_{k,l} = ( \delta_{i,k} \delta_{j,l})_{\substack{1 \leq i \leq n \\ 1 \leq j \leq p}}$$
où $\delta_{i,j}$ est le symbole de Kronecker.

\begin{Proposition}{} La famille $(E_{i,j})_{\substack{1 \leq i \leq n \\ 1 \leq j \leq p}}$ est une base de $\mathcal{M}_{n,p}(\mathbb{K})$ et ainsi $\textrm{dim}(\mathcal{M}_{n,p}(\mathbb{K}))=np$.
\end{Proposition}


\subsection{Produit de deux matrices}

Soient $A=(a_{i,j})_{\substack{1 \leq i \leq n \\ 1 \leq j \leq p}} \in \mathcal{M}_{n,p}(\mathbb{K})$ et $B=(b_{i,j})_{\substack{1 \leq i \leq p \\ 1 \leq j \leq q}} \in \mathcal{M}_{p,q}(\mathbb{K})$. On définit la matrice $AB$ de $\mathcal{M}_{n,q}(\mathbb{K})$ par :
$$ \forall i \in \iii{1}{n}, \, \forall j \in \iii{1}{q}, \quad (AB)_{i,j} = \sum_{k=1}^p a_{i,k} b_{k,j}$$
Autrement dit, le coefficient $(AB)_{i,j}$ est le produit\footnote{Ici on triche car on définit un produit à l'aide de ce même produit... } de la $i$-ème ligne de $A$ avec la $j$-ième colonne de $B$.

\medskip

\begin{Remarques}{}
\begin{itemize}
 \item Le \emph{nombre de colonnes} de $A$ doit être égal au \emph{nombre de lignes} de $B$.
% \item Le coefficient $c_{i,j}$ est égal au produit de la $i\eme$ ligne de $A$ par la $j\eme$ colonne de $B$.
\item Il est important de bien connaitre la formule liée aux tailles : 


\begin{center}
\fbox{(Taille $n \times p$) $\times$ (Taille $p \times q$) =  Taille $n \times q$}
\end{center}
%\item Il est plus simple d'utiliser la configuration suivante pour calculer un produit de matrice : 
%\begin{center}
%\includegraphics[scale=0.5]{produitmatrice}
%\end{center}
\end{itemize}
\end{Remarques}{}

\begin{Remarque}{\warning}

 Le produit de matrices \emph{ne vérifie pas} certaines propriétés du produit de nombres :
 \begin{enumerate}
\item Le produit $AB$ peut être défini et le produit $BA$ ne pas être défini.
\item Même si les produits $AB$ et $BA$ sont définis, ils peuvent être différents.
%
%\emph{Exemple :} Soit $A=\begin{pmatrix}
% 0 & 1\\
% 0 & 0
% \end{pmatrix} \in \mathcal{M}_{2}(\mathbb{R})$ et $B=\begin{pmatrix}
% 1 & 0\\
% 0 & 0
% \end{pmatrix} \in \mathcal{M}_{2}(\mathbb{R}).$\\
% On a alors :
% \vspace{-0.5cm}
% $$AB=\begin{pmatrix}
% 0 & 0\\
% 0 & 0
% \end{pmatrix} \;\;\text{ et }\;\;BA=\begin{pmatrix}
% 0 & 1\\
% 0 & 0
% \end{pmatrix} \;\;\text{ donc }\;\; AB \neq BA.$$
\item On peut avoir $AB=0$ sans avoir $A=0$ ou $B=0$.
%
%\emph{Exemple :} Dans l'exemple précédent, $AB=0_2$ et pourtant $A \neq 0_2$ et $B \neq 0_2$.
\item On peut avoir $AC=BC$ sans avoir $A=B$.

%
% \emph{Exemple :} Soit $A=\begin{pmatrix}
% 1 & -1\\
% 2 & 1
% \end{pmatrix}$, $B=\begin{pmatrix}
% 1 & 0\\
% 2 & 1
% \end{pmatrix}$ et $C=\begin{pmatrix}
% 1 & 1\\
% 0 & 0
% \end{pmatrix}$.\\
% On a $AC=\begin{pmatrix}
% 1 & 1\\
% 2 & 2
% \end{pmatrix}=BC$ et pourtant $A \neq B$.
\end{enumerate}
\end{Remarque}

\medskip


\begin{Proposition}{}  Soient $n \geq 1$ et $(i,j,k,l) \in \iii{1}{n}^4$. On a $E_{i,j} E_{k,l} = \delta_{j,k} E_{i,l}$ (on considère des matrices carrées d'ordre $n$).
\end{Proposition}

\begin{Exemple} Soient $A$ et $B$ deux matrices de $\mathcal{M}_n(\mathbb{R})$ telles que la somme des coefficients sur chaque ligne de ces matrices est égal à $1$. Montrons que $AB$ vérifie la même propriété.



\end{Exemple}

\newpage

$\phantom{test}$

\vspace{8cm}
\subsection{Matrices et tailles particulières}

\begin{Definition}{}
\begin{itemize}
\item On appelle \emph{matrice carrée d'ordre $n$} toute matrice de taille $n \times n$. Les coefficients $(a_{i,i})_{1 \leq i \leq n}$ d'une matrice carrée $A$ sont appelés \emph{les coefficients diagonaux} de $A$ et ils constituent \emph{la diagonale} de $A$.

On note $\mathcal{M}_n(\mathbb{K})$ (au lieu de $\mathcal{M}_{n,n}( \mathbb{K})$) \emph{l'ensemble des matrices carrées d'ordre $n$ à coefficients dans $\mathbb{K}$}.
\item On dit qu'une matrice carrée est \emph{diagonale} lorsque tous ses coefficients non diagonaux sont nuls.
\item On dit qu'une matrice est \emph{triangulaire supérieure} lorsque tous ses coefficients situés en-dessous de la diagonale sont nuls.
\item On dit qu'une matrice est \emph{triangulaire inférieure} lorsque tous ses coefficients situés au dessus de la diagonale sont nuls.
\end{itemize}
\end{Definition}



\begin{Definition}{}
\begin{itemize}
\item La matrice de taille $n \times p$ dont tous les coefficients sont nuls est appelée la \emph{matrice nulle de taille $n \times p$} et est notée $0_{n,p}$ (ou $0_{n}$ si $n=p$).
\item La matrice carrée d'ordre $n$ dont tous les coefficients sont nuls sauf les coefficients diagonaux qui valent tous 1 est appelée la \emph{matrice identité d'ordre $n$} et est notée $I_n$.
\end{itemize}
\end{Definition}

\subsection{Puissances de matrices}
Si on manipule uniquement deux matrices carrés de tailles $n$ notées $A$ et $B$, alors les produits $AB$ et $BA$ \emph{sont toujours définis} (on a le même nombre de lignes et colonnes pour les deux matrices). On a par contre déjà vu qu'ils pouvaient être différents. 


\begin{Definition}{}[Puissance d'une matrice carrée]
Soit $A \in \mathcal{M}_{n}(\mathbb{K})$.\\
On pose $A^0=I_n$ et pour $p \in \mathbb{N}^*$, on appelle \emph{puissance $p$-ième de la matrice $A$} et on note $A^p$ l'élément de $\mathcal{M}_n(\mathbb{K})$ défini par :
$$A^p= \underbrace{A \times A \times \cdots \times A}_{p \text{ termes}}$$
\end{Definition}

\begin{Remarques}{}
\begin{itemize}
\item Soient $A \in \mathcal{M}_n(\mathbb{K})$ et $p,q \in \mathbb{N}$. On a $A^{p+q}=A^pA^q=A^qA^p.$
\item Il faut faire attention quand on factorise des puissances de matrices, par exemple si $A \in \mathcal{M}_{n}(\mathbb{K})$, \newline $A^2+A = A( A+I_n)$.
\end{itemize}
\end{Remarques}{}

\begin{Methode}{}
Il est souvent demandé dans les exercices l'expression explicite de $A^n$ en fonction de $n \in \mathbb{N}$. Il existe (au moins) 4 cas où il est facile de donner une expression explicite des puissances d'une matrice.
\begin{itemize}
\item Le cas où la matrice est diagonale.
\item Le cas où l'on peut conjecturer une expression puis démontrer celle-ci par récurrence.
\item Le cas où il y a une cyclicité (par exemple si $A^3=A$).
\item Le cas où $A$ est somme de deux matrices qui commutent : dans ce cas, on peut utiliser la formule du binôme de Newton.
\end{itemize}
\end{Methode}

\begin{ApplicationDirecte} Déterminer les puissances de $A = \begin{pmatrix}0&-1\\1&0\end{pmatrix}$ et $B=\begin{pmatrix} 1&-1\\-1&1\end{pmatrix}\cdot$
\end{ApplicationDirecte}

\begin{Theoreme}{Formule du binôme}
Soient $n\in \mathbb{N}$ et $A,B$ deux matrices carrées de même ordre. Si $A$ et $B$ commutent alors :
\[ (A+B)^n = \sum_{k=0}^n \binom{n}{k} A^k B^{n-k} \]
\end{Theoreme}

\begin{Exemple} Déterminons les puissances de $M = \begin{pmatrix}
2 & 0 & 1 \\
0 & 1 & 0 \\
0 & 0 & 2
\end{pmatrix}\cdot$
\vspace{\stretch{1}}

\newpage
\vspace*{10cm}
\end{Exemple}

\begin{Remarques}{}
\begin{itemize}
\item Attention, pour appliquer cette formule il faut vérifier que $A$ et $B$ commutent!
\item En particulier si l'une des deux matrices est la matrice identité (ou un multiple de celle-ci par un scalaire), elle commute avec l'autre matrice et on peut donc appliquer cette formule.
\end{itemize}
\end{Remarques}{}

\subsection{Transposée d'une matrice}
\begin{Definition}{}
 
Soit $A=(a_{i,j})_{{1 \leq i \leq n}\atop{1 \leq j \leq p}} \in 
\mathcal{M}_{n,p}(\mathbb{K})$.\\


On appelle \emph{matrice transposée de $A$} la matrice de 
$\mathcal{M}_{p,n}(\mathbb{K})$, notée $^tA$, définie par 
\linebreak $^tA=(b_{i,j})_{{1 \leq i \leq p}\atop{1 \leq j \leq n}}$ 
où pour tout  $i$ de $\iii1p$ et tout $j$ de $\iii1n$, on a $b_{i,j}= a_{j,i}.$
\end{Definition}

%\begin{Remarque}{}
%La première ligne de $A$ est la première colonne de $^tA$, la seconde ligne de $A$ est la seconde colonne de $^tA$...
%\end{Remarque}

\begin{Proposition}{}
Soient $A,B \in \mathcal{M}_{n,p}(\mathbb{K})$ et $\lambda \in \mathbb{R}$.
\begin{itemize}
\item 
$^t(^tA)= A$ 
\item 
$^t(A+B)=^tA + ^tB $ et $^t( \lambda A) = \lambda ^tA$ (l'application transposée est une application linéaire)
\item $^t(AB) =^tB \, ^tA$
\end{itemize}
\end{Proposition}


\subsection{Inversibilité d'une matrice}
\begin{TheoremeDefinition}{}
Soit $A \in \mathcal{M}_n(\mathbb{K})$.\\
On dit que la matrice $A$ est \emph{inversible} lorsqu'il existe une matrice $B$ de $\mathcal{M}_n(\mathbb{K})$ telle que
$$AB=BA=I_n$$
Dans ce cas, une telle matrice $B$ est \emph{unique} : on l'appelle \emph{l'inverse de $A$} et on la note $A^{-1}$.

On note $GL_{n}(\mathbb{K})$ l'ensemble des matrices inversibles de $\mathcal{M}_n(\mathbb{K})$.
\end{TheoremeDefinition}
%
%\begin{Demonstration}{}
% Soit $B$ et $C$ deux éléments de $\mathcal{M}_n(\mathbb{R})$ vérifiant $AB=BA=I_n$ et $AC=CA=I_n$.\\
% On a alors $CAB=(CA)B=I_nB=B$ et $CAB=C(AB)=CI_n=C$ d'où $B=C$.
%\end{Demonstration}

\begin{ApplicationDirecte} L'ensemble $GL_{n}(\mathbb{K})$ est-il un sous-espace vectoriel de $\mathcal{M}_n(\mathbb{K})$?
\end{ApplicationDirecte}


\begin{Proposition}{}
Soient $A$ et $B$ deux matrices inversibles de $\mathcal{M}_n(\mathbb{K})$. Alors : 
\begin{itemize}
\item $A^{-1}$ est inversible et $(A^{-1})^{-1}=A$.
\item $^tA$ est inversible et $(^tA)^{-1} = \ ^t(A^{-1})$.
\item $AB$ est inversible et $(AB)^{-1}=B^{-1}A^{-1}$.
\end{itemize}
\end{Proposition}




\begin{Remarque}{}
 Soient $A$, $B$ et $C$ trois matrices de $\mathcal{M}_n(\mathbb{K})$.
\begin{itemize}
\item Si $AB=0_n$ et $B$ est inversible alors $A=0_n$.\\
En effet, \phantom{on a alors $ABB^{-1}=0_nB^{-1}$ donc $AI_n=0_n$ donc $A=0_n$.}
\item Si $AC=BC$ et $C$ est inversible alors $A=B$ (preuve semblable).
%En effet, on a alors $AC-BC=0_n$ donc $(A-B)C=0_n$ avec $C$ inversible donc par la propriété précédente, $A-B=0_n$ donc $A=B$.
\end{itemize}
\end{Remarque}

\medskip

\begin{Theoreme}{}
Soient $A$ et $B$ deux matrices de $\mathcal{M}_n(\mathbb{K})$.\\
Si $AB=I_n$, alors les matrices $A$ et $B$ sont inversibles et sont inverses l'une de l'autre.
\end{Theoreme}

\begin{Remarque}{} 
L'intérêt du théorème précédent est qu'il suffit de vérifier que le produit $AB$ est égal à l'identité et non pas vérifier que les deux matrices $AB$ \emph{et} $BA$ sont égales à l'identité.
\end{Remarque}

\begin{Exemple} Soient $n \geq 1$ et $J$ la matrice carrée d'ordre $n$ dont tous les coefficients valent $1$. Posons $M=J+ I_n$. Déterminer une relation entre $M^2$, $M$ et $I_n$ et en déduire que $M$ est inversible.
        \vspace{\stretch{1}}

        \newpage
\vspace*{3cm}
\end{Exemple}

\medskip

\begin{ApplicationDirecte}
Soit $A=\begin{pmatrix}
3& 0 & 2 & 0\\
0 & -1 & 0 & 2\\
-2 & 0& -1 & 0\\
0 & -2 & 0 & 3
\end{pmatrix} \cdot$

Calculer $(A-I_4)^2$. En déduire que $A$ est inversible et déterminer $A^{-1}$. 
%
%Montrer que $A^2$ est inversible et calculer $(A^2)^{-1}$.
\end{ApplicationDirecte}

\begin{Methode}{}
Pour étudier si une matrice $A$ est inversible, on forme un système dont les coefficients sont les coefficients de $A$ et dont le second membre est une matrice colonne \emph{quelconque} de $\mathcal{M}_{n,1}(\mathbb{K}).$ On résout ce système à l'aide de la méthode du pivot de Gauss. 
 \begin{enumerate}
  \item[$\bullet$]   Si on obtient un système triangulaire $n \times n$ dont les coefficients diagonaux sont non nuls, alors la matrice $A$ est inversible et la résolution du système fournit $A^{-1}$.
  \item[$\bullet$] Sinon la matrice $A$ n'est pas inversible. 
 \end{enumerate}
\end{Methode}

\begin{Remarque}{} La méthode du pivot de Gauss doit être la méthode privilégiée pour résoudre un tel système. 
\end{Remarque}

\begin{Exemple} Soit $P = \begin{pmatrix}
        1&0&1\\-1&0&-2\\0&1&0
       \end{pmatrix} \cdot$
       
Étudions l'inversibilité de $P$. 

%Soient $(b_1,b_2,b_3) \in \mathbb{R}^3$. On résout alors le système suivant :
%\[ (S) \left\lbrace \begin{array}{rcl}
%x \; \; \, \; \; + z & = & b_1 \\
%-x  \; \; \; -2z & = & b_2 \\
%\; \, y \qquad & = & b_3 \\
%\end{array}\right. \]
%
%On a 
%\[ \begin{array}{ccll}
%(S) & \Longleftrightarrow & \left\lbrace \begin{array}{rcl}
%x \; \; \, \; \; + z & = & b_1 \\
%-x  \; \; \; -2z & = & b_2 \\
%\; \, y \qquad & = & b_3 \\
%\end{array}\right. &  \\
% & \Longleftrightarrow & \left\lbrace \begin{array}{rcl}
%x \; \; \, \; \; + z & = & b_1 \\
%\; \;  \; \; \; -z & = & b_1 + b_2 \\
%\; \, y \qquad & = & b_3 \\
%\end{array}\right. & \begin{array}{c}
% \\
% L_2 \leftarrow L_2 + L_1 \\
%  \\
%  \end{array} \\
%   & \Longleftrightarrow & \left\lbrace \begin{array}{rcl}
%x \; \; \, \; \; + z & = & b_1 \\
%  \; \, y \qquad & = & b_3 \\
%\; \;  \; \; \; -z & = & b_1 + b_2 \\
%\end{array}\right. & \begin{array}{c}
% \\
% L_2 \leftrightarrow L_3 \\
%  \\
%  \end{array} \\
%     & \Longleftrightarrow & \left\lbrace \begin{array}{rcl}
%x  & = & b_1 - (-b_1-b_2) \\
% y  & = & b_3 \\
% z & = &- b_1 - b_2 \\
%\end{array}\right. & \begin{array}{c}
% \\
% \\
%  \\
%  \end{array} \\
%       & \Longleftrightarrow & \left\lbrace \begin{array}{rcl}
%x  & = & 2b_1+b_2 \\
% y  & = &\qquad \qquad b_3 \\
% z & = &- b_1 - b_2 \\
%\end{array}\right. & \begin{array}{c}
% \\
% \\
%  \\
%  \end{array} \\
%
%
%\end{array}\]
%Ainsi, on obtient le système admet une unique solution et la matrice $P$ est donc inversible d'inverse :
%$$ P^{-1} = \begin{pmatrix}
%2 & 1 & 0 \\
%0 & 0 & 1 \\
%-1 & -1 & 0 
%\end{pmatrix}$$
\vspace{\stretch{1}}

\newpage
\vspace*{3cm}
\end{Exemple}

\begin{ApplicationDirecte} Soient $A = \begin{pmatrix} 1 & 2 & -1\\
2 & 4 & -1\\
-2 & -5 & 3
\end{pmatrix}$ et $ B = \begin{pmatrix}
                   2 & -3 & -1\\
1 & 1 & -3\\
-3 & -1 & 7
                  \end{pmatrix}\cdot$
                  
Déterminer si les matrices $A$ et $B$ sont inversibles (on donnera l'inverse le cas échéant).
\end{ApplicationDirecte}


Rappelons pour finir un résultat généralement utile dans un cadre théorique :

\begin{Theoreme}{} Soit $A \in \mathcal{M}_n(\mathbb{K})$. Les assertions suivantes sont équivalentes :

\begin{itemize}
\item $A$ est inversible.
\item L'équation $AX= 0_{n,1}$ d'inconnue $X \in \mathcal{M}_{n,1}(\mathbb{K})$ n'admet que la solution nulle.
\item Pour tout $B \in \mathcal{M}_{n,1}(\mathbb{K})$, l'équation $AX= B$ admet une unique solution.
\end{itemize}
\end{Theoreme}


\section{Matrices, vecteurs et applications linéaires}

Dans cette section, $E$ et $F$ sont deux $\mathbb{K}$-espaces vectoriels de dimensions respectives $n$ et $p$. On se donne aussi $\mathcal{B}=(e_1, \ldots, e_n)$ une base de $E$ et $\mathcal{C}=(f_1, \ldots, f_p)$ une base de $F$.

\subsection{Matrice d'un vecteur}

Tout vecteur $x$ de $E$ s'écrit sous la forme :
$$ x = \sum_{k=1}^n x_k e_k $$
où $(x_1, \ldots, x_n)$ sont les coordonnées de $x$ dans la base $\mathcal{B}$. On appelle \emph{matrice des coordonnées} de $x$ dans la base $\mathcal{B}$, et on note $\textrm{Mat}_{\mathcal{B}}(x)$, la matrice suivante :
$$ \textrm{Mat}_{\mathcal{B}}(x) = \phantom{\begin{pmatrix}
x_1 \\
x_2 \\
\vdots \\
x_n
\end{pmatrix}}$$

\subsection{Matrice d'une famille de vecteurs}

Soient $k \in \mathbb{N}^*$ et $\mathcal{F}=(x_1, x_2, \ldots, x_k)$ des vecteurs. On appelle \emph{matrice de la famille} $\mathcal{F}$ dans la base $\mathcal{B}$, et on note $\textrm{Mat}_{B}(x_1, \ldots, x_k)$ la matrice de $\mathcal{M}_{n,k}( \mathbb{K})$ dont la $j$-ième colonne (pour $1 \leq j \leq k)$ est $\textrm{Mat}_{B}(x_j)$.

\medskip

\begin{Exemple} Soient $E = \mathbb{R}_3[X]$ et $\mathcal{B}$ la base canonique de $E$. Soit $\mathcal{F}=(1,(X-1),(X-1)^2,(X-1)^3)$. La matrice de la famille $\mathcal{F}$ dans la base $\mathcal{B}$ est :
        \vspace{\stretch{1}}

        \newpage
\vspace*{5cm}
%$$ \begin{pmatrix}
%1 & -1 & 1 & - 1 \\
%0 & 1 & -2 & 3 \\
%0 & 0 & 1 & - 3 \\
%0 & 0 & 0 & 1 \\
%\end{pmatrix}$$


\end{Exemple}


\subsection{Matrice d'une application linéaire}

Soit $u \in \mathcal{L}(E,F)$. L'application linéaire $u$ est entièrement caractérisée par son action sur une base de $E$.

\begin{Definition}{} La \emph{matrice de} $u$ dans les bases $\mathcal{B}$ et $\mathcal{C}$, notée $\textrm{Mat}_{\mathcal{B}, \mathcal{C}}(u)$, est définie par :
$$ \textrm{Mat}_{\mathcal{B}, \mathcal{C}}(u)= \phantom{\textrm{Mat}_{\mathcal{C}}(u(e_1), u(e_2), \ldots, u(e_n))}$$
Si $E=F$ et $\mathcal{B}= \mathcal{C}$, on note simplement $\textrm{Mat}_{\mathcal{B}}(u)$.
\end{Definition}

Ainsi, pour déterminer cette matrice, il suffit de calculer pour tout $i \in \iii{1}{n}$, $u(e_i) \in F$ et de déterminer ses coordonnées dans la base $\mathcal{C}$.

%\newpage

Cela se schématise par :

%\begin{center}
%\[
%\begin{blockarray}{ccccc}
%  u(e_1) & u(e_2)& \dots & u(e_n)  \\
%  \begin{block}{(cccc)c}
%  a_{11} & a_{1,2} & \dots  & a_{1n} & f_{1}  \\
%  \vdots & \vdots & \ddots & \vdots &  \vdots    \\
%  a_{p1} & a_{p,2} & \dots  & a_{pn} & f_p  \\
%  \end{block}
%\end{blockarray}
%\]
%\end{center}

\vspace*{5cm}


\begin{Exemple}\label{det} Soient $n \geq 1$ et $\varphi : \mathbb{R}_n[X] \rightarrow \mathbb{R}_n[X]$ définie par $\varphi(P)=P+P'$. Justifier que $\varphi$ est un endomorphisme et donner sa matrice dans la base canonique de  $\mathbb{R}_3[X]$.

        \vspace{\stretch{1}}

        \newpage


%Soit $\varphi : \mathbb{R}_3[X] \rightarrow \mathbb{R}_3[X]$ défini par $\varphi(P)=P-P'$. Cette application est linéaire et on a :
%
%\begin{itemize}
%\item $\varphi(1) = 1$.
%\item $\varphi(X) = -1 + X$.
%\item $\varphi(X^2) = -2X+X^2$.
%\item $\varphi(X^3) = -3X^2 + X^3$.
%\end{itemize}
%En notant $\mathcal{B}$ la base canonique de $\mathbb{R}_3[X]$, on a :
%$$ \textrm{Mat}_{\mathcal{B}}(u) = \begin{pmatrix}
%1 & -1 & 0 & 0 \\
%0 & 1 & -2 & 0 \\
%0 & 0 & 1 & -3 \\
%0 & 0 & 0 & 1 \\
%\end{pmatrix}$$
\end{Exemple}

\medskip

\begin{ApplicationDirecte} Soient $(a,b,c) \in \mathbb{K}^3$ et $\varphi : \mathbb{K}_3[X] \rightarrow \mathbb{K}^3$ définie par :
$$ \varphi(P) = (P(a), P(b), P(c))$$
Déterminer la matrice de $\varphi$ dans les bases canoniques de $\mathbb{K}_3[X]$ et $\mathbb{K}^3$.
\end{ApplicationDirecte}

\begin{ApplicationDirecte} Soit $f : \mathbb{R}^3 \rightarrow \mathbb{R}^3$ définie par :
$$ f((x,y,z)) = (y+z, z+x, x+y)$$
Donner la matrice de $f$ dans la base canonique $\mathcal{B}=(e_1,e_2,e_3)$ de $\mathbb{R}^3$. 

Soit $\mathcal{B}'=(e_1, e_1+e_2, e_1+e_2+e_3)$. Montrer que $\mathcal{B}'$ est une base de $\mathbb{R}^3$ et donner la matrice de $f$  dans cette base.
\end{ApplicationDirecte}

\medskip

En fixant des bases, on peut donc associer à chaque application linéaire de $E$ dans $F$ une matrice de $\mathcal{M}_{p,n}(\mathbb{K})$. Réciproquement, si l'on fixe une matrice de $\mathcal{M}_{p,n}(\mathbb{K})$, cela définit une application linéaire de $E$ dans $F$ (car cela définit les images des vecteurs d'une base de $E$).

\begin{Theoreme}{} L'application :
$$ \begin{array}{ccccl}
\textrm{Mat}_{\mathcal{B}, \mathcal{C}} & : & \mathcal{L}(E,F) & \rightarrow & \mathcal{M}_{p,n}(\mathbb{K}) \\
& & u & \mapsto & \textrm{Mat}_{\mathcal{B}, \mathcal{C}}(u) \\
\end{array}$$
est un isomorphisme d'espace vectoriels. En particulier :

\begin{itemize}
\item $\forall (u,v) \in \mathcal{L}(E,F)^2$, $\forall \lambda \in \mathbb{K}$, $\textrm{Mat}_{\mathcal{B}, \mathcal{C}}(u + \lambda v) = \textrm{Mat}_{\mathcal{B}, \mathcal{C}}(u) + \lambda \textrm{Mat}_{\mathcal{B}, \mathcal{C}}(v)$.
\item $\textrm{dim}(\mathcal{L}(E,F)) = np$.
\end{itemize}
\end{Theoreme}

\begin{Remarque}{\warning}
 On ne parle jamais de \emph{la} matrice d'une application linéaire sans préciser les bases considérées.
\end{Remarque}

\subsection{Image d'un vecteur}
Fixons $u \in \mathcal{L}(E,F)$.

\begin{Proposition}{} Pour tout $x \in E$, on a $\textrm{Mat}_{\mathcal{C}}(u(x)) = \textrm{Mat}_{\mathcal{B}, \mathcal{C}}(u) \textrm{Mat}_{\mathcal{B}}(x)$.
\end{Proposition}

%\begin{Demonstration}{} 
%
%%Notons $(x_1, \ldots, x_n)$ les coordonnées de $x$ dans la base $\mathcal{B}$ et $\textrm{Mat}_{\mathcal{B}, \mathcal{C}}(u) = (a_{i,j})_{\substack{1 \leq i \leq p \\ 1 \leq j \leq n}}$. On a :
%%\begin{align*}
%%u(x) & = u \left( \sum_{k=1}^n x_k e_k \right) \\
%%& = \sum_{k=1}^n x_k u(e_k) \qquad \hbox{(par linéarité)} \\
%%& = \sum_{k=1}^n x_k \sum_{i=1}^p  a_{i,k} f_i \\
%%& = \sum_{i=1}^p \left( \sum_{k=1}^n  a_{i,k} x_k \right) f_i\\
%%\end{align*}
%%Ainsi :
%%$$ \textrm{Mat}_{\mathcal{C}}(u(x)) =  \left( \sum_{k=1}^n  a_{i,k} x_k \right)_{1 \leq i \leq p}$$
%%et par définition du produit matriciel, on a donc bien :
%%$$\textrm{Mat}_{\mathcal{C}}(u(x)) = \textrm{Mat}_{\mathcal{B}, \mathcal{C}}(u) \textrm{Mat}_{\mathcal{B}}(x)$$
%
%\vspace{8cm}
%\end{Demonstration}
\subsection{Matrice et composition d'applications linéaires}

\begin{Proposition}{}
Soient $E$, $F$, $G$ trois $\mathbb{K}$-espaces vectoriels de dimension finie que l'on munit de bases respectives $\mathcal{E}$, $\mathcal{F}$ et $\mathcal{G}$. Soient $u \in \mathcal{L}(E,F)$ et $v \in \mathcal{L}(F,G)$. Alors $v \circ u \in L(E,G)$ et :
$$ \textrm{Mat}_{\mathcal{E}, \mathcal{G}}(v \circ u) = \textrm{Mat}_{\mathcal{F}, \mathcal{G}}(v ) \textrm{Mat}_{\mathcal{E}, \mathcal{F}}( u)$$
\end{Proposition}

\begin{Remarque}{} En particulier, si $u$ est un endomorphisme de $E$ muni d'une base $\mathcal{B}$, on a pour tout $n \geq 0$,
$$ \textrm{Mat}_{\mathcal{B}}(u^n) = \left(\textrm{Mat}_{\mathcal{B}}(u)\right)^n$$
où $u^n$ désigne $u \circ u \circ \cdots \circ u$ (composée $n$ fois).
\end{Remarque}

\begin{Corollaire}{} Soient $E$ et $F$ deux $\mathbb{K}-$espaces vectoriels  de dimension $n$ que l'on munit de deux bases $\mathcal{E}$ et $\mathcal{F}$. Pour tout $u \in \mathcal{L}(E,F)$, on a l'équivalence suivante :
$$ u \hbox{ est un isomorphisme si et seulement si } \textrm{Mat}_{\mathcal{E}, \mathcal{F}}(u) \in GL_n (\mathbb{K})$$
et on a dans ce cas :
$$ \textrm{Mat}_{\mathcal{F}, \mathcal{E}}(u^{-1}) = \left( \textrm{Mat}_{\mathcal{E}, \mathcal{F}}(u) \right)^{-1}$$
\end{Corollaire}


\subsection{Changement de base}

Dans la suite, $\mathcal{B}=(e_1, \ldots, e_n)$ et $\mathcal{B}'= (e_1', \ldots,e_n')$ sont deux bases d'un espace vectoriel de dimension finie $n$.

\begin{Definition}{} On appelle \emph{matrice de passage} de $\mathcal{B}$ dans $\mathcal{B}'$ et on note $P_{\mathcal{B}, \mathcal{B}'}$ la matrice définie par :
$$ P_{\mathcal{B}, \mathcal{B}'} = \phantom{\textrm{Mat}_{\mathcal{B}', \mathcal{B}}(\textrm{Id})}$$
Autrement dit, les colonnes de $P_{\mathcal{B}, \mathcal{B}'}$ sont les coordonnées des vecteurs de $\mathcal{B}'$ dans la base $\mathcal{B}$.
\end{Definition}

\begin{Exemple} Soient $\mathcal{B}=(e_1,e_2,e_3)$ la base canonique de $\mathbb{R}^3$ et $\mathcal{B}'=(e_1,e_1+e_2, e_1+e_2+e_3)$. On montre facilement que $\mathcal{B}'$ est libre et de cardinal $3$ qui est la dimension de $\mathbb{R}^3$. Ainsi $\mathcal{B}'$ est une base de $\mathbb{R}^3$ et on a :

\vspace{2.7cm}
%$$ P_{\mathcal{B}, \mathcal{B}'} = \begin{pmatrix}
%1 & 1 & 1 \\
%0 & 1 & 1 \\
%0 & 0& 1 \\
%\end{pmatrix}$$
\end{Exemple}

\begin{Proposition}{} La matrice de passage de $\mathcal{B}$ dans $\mathcal{B}'$ est inversible d'inverse la matrice de passage de $\mathcal{B}'$ dans $\mathcal{B}$.
\end{Proposition}

\begin{Demonstration}{} C'est évident avec le corollaire précédent car $P_{\mathcal{B}, \mathcal{B}'} = \textrm{Mat}_{\mathcal{B}', \mathcal{B}}(Id)$.
\end{Demonstration}

\medskip

\begin{Exemple} Reprenons l'exemple précédent. On a :
$$ P_{\mathcal{B}', \mathcal{B}} = \begin{pmatrix}
1 & 1 & 1 \\
0 & 1 & 1 \\
0 & 0& 1 \\
\end{pmatrix}^{-1}$$
On peut calculer l'inverse de la matrice mais on peut aussi remarquer qu'il est facile ici d'exprimer les vecteurs de la base canonique dans la base $\mathcal{B}'$ : $e_1=e_1$, $e_2 =(e_1+e_2)-e_1$ et $e_3 = (e_1+e_2+e_3)-(e_1+e_2)$. Ainsi :

\vspace{3cm}
%$$ P_{\mathcal{B}', \mathcal{B}} = \begin{pmatrix}
%1 & -1 & 0 \\
%0 & 1 & -1 \\
%0 & 0& 1 \\
%\end{pmatrix}$$
\end{Exemple}

\begin{Proposition}{} Soit $x \in E$. Alors :
$$ \textrm{Mat}_{\mathcal{B}}(x) = P_{\mathcal{B}, \mathcal{B}'} \textrm{Mat}_{\mathcal{B}'}(x)$$
\end{Proposition}

%\begin{Demonstration}{} 
%\vspace{4cm}
%%Par définition de la matrice de passage, cela est équivalent à montrer que :
%%$$  \textrm{Mat}_{\mathcal{B}}(x) =  \textrm{Mat}_{\mathcal{B}', \mathcal{B}}(Id) \textrm{Mat}_{\mathcal{B}'}(x)$$
%%ou encore :
%%$$ \textrm{Mat}_{\mathcal{B}}(Id(x)) =  \textrm{Mat}_{\mathcal{B}', \mathcal{B}}(Id) \textrm{Mat}_{\mathcal{B}'}(x)$$
%%ce qui est vrai d'après la proposition liée à la matrice de l'image d'un vecteur par une application linéaire.
%\end{Demonstration}

\begin{Exemple} Reprenons encore l'exemple précédent. Soit $x=(1,-2,3) \in \mathbb{R}^3$. On a :
$$ \textrm{Mat}_{\mathcal{B}}(x) = \begin{pmatrix}
1 \\
-2 \\
3 \\
\end{pmatrix}$$
Déterminons les coordonnées de $x$ dans la base $\mathcal{B}'$. On a :

\vspace{4cm}
%\begin{align*}
%\textrm{Mat}_{\mathcal{B}'}(x) & = P_{\mathcal{B'}, \mathcal{B}} \textrm{Mat}_{\mathcal{B}}(x) \\
%& = \begin{pmatrix}
%1 & -1 & 0 \\
%0 & 1 & -1 \\
%0 & 0& 1 \\
%\end{pmatrix}  \begin{pmatrix}
%1 \\
%-2 \\
%3 \\ 
%\end{pmatrix} \\
%& = \begin{pmatrix}
%3 \\
%-5 \\
%-3
%\end{pmatrix}
%\end{align*}
\end{Exemple}

\begin{Theoreme}{} Soient $\mathcal{B}$, $\mathcal{B}'$ deux bases de $E$ et $\mathcal{C}$, $\mathcal{C}'$ deux bases de $F$ et $u \in L(E,F)$. Alors :
$$ \textrm{Mat}_{\mathcal{B}', \mathcal{C}'}(u) = \textrm{Mat}_{\mathcal{C}, \mathcal{C}'}(Id_F)  \textrm{Mat}_{\mathcal{B}, \mathcal{C}}(u) \textrm{Mat}_{\mathcal{B}', \mathcal{B}}(Id_E)$$
ou encore :
$$ \textrm{Mat}_{\mathcal{B}', \mathcal{C}'}(u) = P_{\mathcal{C}', \mathcal{C}}  \textrm{Mat}_{\mathcal{B}, \mathcal{C}}(u)  P_{\mathcal{B}, \mathcal{B}'}$$
\end{Theoreme}

\begin{Corollaire}{} Soient $\mathcal{B}, \mathcal{B}'$ deux bases de $E$ et $u \in L(E)$. Alors :
$$ \textrm{Mat}_{\mathcal{B}'}(u) = P_{\mathcal{B}', \mathcal{B}}  \textrm{Mat}_{\mathcal{B}}(u)  P_{\mathcal{B}, \mathcal{B}'}$$
\end{Corollaire}

\begin{Exemple} Posons :
$$ \begin{array}{ccccl}
f & : & \mathbb{R}_2[X] & \rightarrow & \mathbb{R}_2[X] \\
& & P & \rightarrow & (X+1)P' \\
\end{array} $$
Montrons que $f$ est un endomorphisme de $ \mathbb{R}_2[X]$ :

\vspace{3cm}


Déterminons la matrice de $f$ dans la base canonique $\mathcal{B}$ de $\mathbb{R}_2[X]$ :

\vspace{5cm}

Considérons la famille suivante : $\mathcal{B}'= (1, (X+1),(X+1)^2)$. Montrons que $\mathcal{B}'$ est une base de $\mathbb{R}_2[X]$ et déterminons la matrice de $f$ dans cette base :

\vspace*{6.5cm}

Déterminons à l'aide d'un changement de base les puissances de $\textrm{Mat}_{\mathcal{B}}(f)$ :

\vspace{\stretch{1}}

\newpage
\vspace*{5cm}
%\begin{itemize}
%\item L'application $f$ est bien définie (Si $P$ est dans $\mathbb{R}_2[X]$, $P'$ est dans $\mathbb{R}_1[X]$ et ainsi $f(P)$ appartient bien à $\mathbb{R}_2[X]$.
%\item L'application $f$ est linéaire.
%\item Déterminons la matrice de $f$ dans la base canonique $\mathcal{B}$ de $\mathbb{R}_2[X]$ : $f(1) =0$, $f(X) = X+1$ et $f(X^2) = 2X+2X^2$. On a ainsi :
%$$ \textrm{Mat}_{\mathcal{B}}(f) =\begin{pmatrix}
%0 & 1 & 0 \\
%0 & 1 & 2 \\
%0 & 0 & 2 \\
%\end{pmatrix}$$
%\item Considérons la famille suivante : $\mathcal{B}'= (1, (X+1),(X+1)^2)$. Cette famille est échelonnée en degré et de cardinal $3$ qui est la dimension de $\mathbb{R}_2[X]$. Ainsi $\mathcal{B}'$ est une base de $\mathbb{R}_2[X]$. Par simple calcul, on a : $f(X+1)=(X+1)$ et $f((X+1)^2)=2(X+1)^2$ et ainsi :
%$$ \textrm{Mat}_{\mathcal{B}'}(f) =\begin{pmatrix}
%0 & 0 & 0 \\
%0 & 1 & 0 \\
%0 & 0 & 2 \\
%\end{pmatrix}$$
%\item Par changement de base, on sait que :
%$$ \textrm{Mat}_{\mathcal{B}}(f) = P_{\mathcal{B}, \mathcal{B}'} \textrm{Mat}_{\mathcal{B}'}(f) P_{\mathcal{B}', \mathcal{B}}$$
%Par définition de la matrice de passage, on a :
%$$ P_{\mathcal{B}, \mathcal{B}'} = \begin{pmatrix}
%1 & 1 & 1 \\
%0 & 1 & 2 \\
%0 & 0 & 1 
%\end{pmatrix}$$
%On sait que $P_{\mathcal{B}', \mathcal{B}} = \left(P_{\mathcal{B}', \mathcal{B}} \right)^{-1}$ donc en résolvant un système (ou en remarquant que $X= -1 + (X+1)$ et $X^2 = (X+1)^2 - 2(X+1) + 1$), on a :
%$$ P_{\mathcal{B}', \mathcal{B}}  = \begin{pmatrix}
%1 & -1 & 1 \\
%0 & 1 & -2 \\
%0 & 0 & 1 
%\end{pmatrix}$$
%\item L'intérêt de changer de base est que dans la base $\mathcal{B}'$, la matrice de $f$ est diagonale ce qui permet d'obtenir facilement la matrice de $f^n$ pour tout $n \in \mathbb{N}^*$ :
%$$ \textrm{Mat}_{\mathcal{B}'}(f^n) = \textrm{Mat}_{\mathcal{B}'}(f)^n = \begin{pmatrix}
%0 & 0 & 0 \\
%0 & 1 & 0 \\
%0 & 0 & 2^n \\
%\end{pmatrix}$$
%Or, par changement de base, on a :
%$$  \textrm{Mat}_{\mathcal{B}}(f^n) = P_{\mathcal{B}, \mathcal{B}'} \textrm{Mat}_{\mathcal{B}'}(f^n) P_{\mathcal{B}', \mathcal{B}}$$
%ou encore :
%$$ \textrm{Mat}_{\mathcal{B}}(f) ^n= P_{\mathcal{B}, \mathcal{B}'} \textrm{Mat}_{\mathcal{B}'}(f)^n P_{\mathcal{B}', \mathcal{B}}$$
%Il suffit de terminer le calcul.
%\end{itemize}
\end{Exemple}

\subsection{Matrices semblables}

\begin{Proposition}{} Soient $A$, $B$ deux matrices de $\mathcal{M}_n(\mathbb{K})$. Les assertions suivantes sont équivalentes :

\begin{enumerate}
\item Il existe $P \in GL_n(\mathbb{K})$ tel que $B=P A P^{-1}$.
\item $A$ et $B$ sont les matrices du même endomorphisme $u$ d'un $\mathbb{K}-$espace vectoriel $E$ de dimension $n$ dans deux bases de $E$ : il existe $u \in \mathcal{L}(E)$ et deux bases $\mathcal{B}$ et $\mathcal{B}'$ de $E$ tels que $B= \textrm{Mat}_{\mathcal{B}'}(u)$ et $A= \textrm{Mat}_{\mathcal{B}}(u)$.
\end{enumerate}

Dans ce cas, on dit que $A$ et $B$ sont deux \emph{matrices semblables}.
\end{Proposition}

\begin{Exemple} Montrons que les matrices $A = \begin{pmatrix}
3 & 2 \\
-3 & 8 \\
\end{pmatrix}$ et $B = \begin{pmatrix}
5 & 0 \\
0 & 6 \\
\end{pmatrix}$ sont semblables.

\medskip
%
%Soit $u$ l'endomorphisme de $\mathbb{K}^2$ canoniquement associé à $A$ (autrement dit $A = \textrm{Mat}_{\mathcal{B}}(u)$ où $\mathcal{B}$ est la base canonique de $\mathbb{K}^2$ et $u : X \mapsto AX$). On cherche $\mathcal{F}=(f_1,f_2)$ une base de $\mathbb{K}^2$ tel que $B= \textrm{Mat}_{\mathcal{F}}(u)$ ce qui est équivalent à $u(f_1)=5f_1$ et $u(f_2)=6 f_2$. 
%
%\medskip On résout alors les équations $AX=5X$ et $AX=6X$ d'inconnue $X \in \mathbb{K}^2$ :
%
%\begin{itemize}
%\item Posons $X = \begin{pmatrix}
%x_1 \\
%x_2 \\
%\end{pmatrix}$, on a :
%$$ AX=5X \Longleftrightarrow \left\lbrace \begin{array}{ccl}
%3x_1 + 2x_2 & =  & 5 x_1 \\
%-3x_1 + 8 x_2 & = & 5x_2 \\
%\end{array}\right. \Longleftrightarrow x_1=x_2$$
%\item Posons $X = \begin{pmatrix}
%x_1 \\
%x_2 \\
%\end{pmatrix}$, on a :
%$$ AX=6X \Longleftrightarrow \left\lbrace \begin{array}{ccl}
%3x_1 + 2x_2 & =  & 6 x_1 \\
%-3x_1 + 8 x_2 & = & 6x_2 \\
%\end{array}\right. \Longleftrightarrow 3x_1=2x_2$$
%\end{itemize}
%
%\medskip
%
%Posons donc $f_1 = \begin{pmatrix}
%1 \\
%1 \\
%\end{pmatrix}$ et $f_2 = \begin{pmatrix}
%2 \\
%3 \\
%\end{pmatrix}$, la famille $\mathcal{F}=(f_1,f_2)$ est libre (vecteurs non colinéaires) dans un espace de dimension $2$ donc c'est une base de $\mathbb{K}^2$. Par le raisonnement précédent, on a donc $\textrm{Mat}_{\mathcal{F}}(u)= B$ et ainsi $A$ et $B$ sont semblables.

\end{Exemple}
\newpage



\section{Image, noyau et rang d'une matrice}
\subsection{Définitions}
Soit $A \in \mathcal{M}_{n,p}(\mathbb{K})$. On associe à cette matrice une application linéaire :
$$ \begin{array}{ccccl}
u_A & : & \mathcal{M}_{p,1}(\mathbb{K}) & \rightarrow & \mathcal{M}_{n,1}(\mathbb{K}) \\
& & X & \mapsto & AX
\end{array}$$

\begin{Definition}{} On transpose à la matrice $A$ certaines définitions liées à $u_A$ :
\begin{itemize}
\item Le \emph{noyau} de $A$, noté $\textrm{Ker}(A)$, est le noyau de $u_A$.
\item L'\emph{image} de $A$, notée $\textrm{Im}(A)$, est l'image de $u_A$.
\item Le \emph{rang} de $A$, noté $\textrm{rg}(A)$, est le rang de $u_A$.
\end{itemize}
\end{Definition}

\begin{Remarques}{}
\begin{itemize}
\item Les lignes de $A$ fournissent les équations définissant $\textrm{Ker}(A)$.
\item Les colonnes engendrent l'image de $A$ car en notant $(e_1, \ldots, e_p)$ la base canonique de $ \mathcal{M}_{p,1}(\mathbb{K}) $, on a :
$$ \textrm{Im}(A) = \textrm{Im}(u_A) = \textrm{Vect}(u_A(e_1), \ldots, u_A(e_p))= \textrm{Vect}(Ae_1, \ldots, Ae_p)$$
\end{itemize}
\end{Remarques}{}

\begin{Theoreme}{} Soit $A \in \mathcal{M}_{n,p}(\mathbb{K})$. On a :
$$ p = \textrm{dim}(\textrm{Ker}(A)) + \textrm{rg}(A)$$
\end{Theoreme}

\subsection{Méthodes pratiques d'obtention du noyau et de l'image}

Soit $A \in \mathcal{M}_{n,p}(\mathbb{K})$.

\medskip

$\rhd$ Dans tous les cas, on peut obtenir une base du noyau en résolvant $AX = 0_{n,1}$ d'inconnue $X \in \mathcal{M}_{p,1}(\mathbb{K})$.

\medskip

$\rhd$ L'image de $A$ est le sous-espace vectoriel engendré par les colonnes (mais bien entendu, une famille génératrice n'est pas nécessairement une base de ce sous-espace).

\medskip

$\rhd$ On utilise la proposition suivante :

\begin{Proposition}{} Les opérations élémentaires sur les colonnes ne changent pas l'image d'une matrice.
\end{Proposition}

On utilise alors les opérations élémentaires sur la matrice $A$ pour faire apparaitre une matrice triangulaire. Les colonnes nulles fourniront une base du noyau.

\medskip

\begin{Exemple} Soit $A = \begin{pmatrix}
1 & 1 & -1 \\
2 & 4 & 0 \\
-1 & 1 & 2 \\
\end{pmatrix} \cdot$

Déterminons le noyau et l'image de $A$. 

%On part de la configuration suivante :
%$$ \begin{pmatrix}
%1 & 1 & -1 \\
%2 & 4 & 0 \\
%-1 & 1 & 3 \\
%\end{pmatrix} \quad \begin{pmatrix}
%1 & 0 & 0 \\
%0 & 1 & 0 \\
%0 & 0 & 1 \\
%\end{pmatrix}$$
%On effectue : $C_2 \leftarrow C_2-C_1$ et $C_3 \leftarrow C_3+C_1$ :
%$$ \begin{pmatrix}
%1 & 0 & 0 \\
%2 & 2 & 2 \\
%-1 & 2 & 2 \\
%\end{pmatrix} \quad \begin{pmatrix}
%1 & -1 & 1 \\
%0 & 1 & 0 \\
%0 & 0 & 1 \\
%\end{pmatrix}$$
%puis $C_3 \leftarrow C_3-C_2$ :
%$$ \begin{pmatrix}
%1 & 0 & 0 \\
%2 & 2 & 0 \\
%-1 & 2 & 0 \\
%\end{pmatrix} \quad \begin{pmatrix}
%1 & -1 & 2 \\
%0 & 1 & -1 \\
%0 & 0 & 1 \\
%\end{pmatrix}$$
%La matrice est donc de rang $2$, Une base de l'image est $\left( \begin{pmatrix}
%1 \\
%2 \\
%-1 \\
%\end{pmatrix}, \begin{pmatrix}
%0 \\
%2 \\
%2 
%\end{pmatrix} \right)$ (car elle est de cardinal $2$ et génératrice) et le noyau est ainsi de dimension $1$ et une base est donnée par $\left( \begin{pmatrix}
%2 \\
%-1 \\
%1 \\
%\end{pmatrix} \right)$ (le vecteur est non nul et appartient à noyau par opérations élémentaires).
\newpage

\vspace*{13cm}
\end{Exemple}

\medskip

$\rhd$ Quelques astuces pour aller plus vite dans le cas d'une matrice $A$ carrée d'ordre $3$ :

\begin{itemize}
\item Si deux colonnes de la matrice sont non colinéaires, la matrice a un rang supérieur ou égal $2$. Si l'on trouve facilement une relation linéaire entre les colonnes, cela nous donne un élément de noyau (et donc une base car le noyau est au maximum de dimension $1$ d'après le théorème du rang).



\begin{Exemple} Soit $A = \begin{pmatrix}
1 & 1 & 2 \\
-1 & 4 & 3 \\
0 & 1 & 1 \\
\end{pmatrix} \cdot$
%
%Les deux premières colonnes sont non colinéaires donc le rang de la matrice est supérieur ou égal à $2$. La troisième colonne est la somme des deux premières colonnes. En notation $(e_1,e_2,e_3)$ la base canonique de $\mathcal{M}_{3,1}(\mathbb{K}$, cela signifie que $Ae_1 + Ae_2 = Ae_3$ ou encore $A(e_1+e_2-e_3) = O_{3}$. Ainsi $e_1+e_2-e_3$ est dans le noyau qui est au maximum de dimension $1$ (d'après le Théorème du rang) et est non nul : c'est donc une base du noyau. Le rang est donc égal à $2$ et une base de celui-ci est $(C_1,C_2)$ (les deux premières colonnes de la matrice).

\vspace{5cm}
\end{Exemple}
\item Si la matrice est de rang $1$ (ce qui se voit directement), on obtient une base du noyau en obtenant des relations linéaires sur les colonnes.

\begin{Exemple} Soit $A = \begin{pmatrix}
1 & 1 & 2 \\
-1 & -1 & -2 \\
1 & 1 & 2 \\
\end{pmatrix} \cdot$

%Notons $C_1$, $C_2$ et $C_3$ les colonnes de $A$.
%
%La matrice est de rang $1$ et une base de l'image est $(C_1)$. On remarque que $C_1=C_2$ et $2C_2=C_3$ donc en notant $(e_1,e_2,e_3)$ la base canonique de $\mathbb{K}^3$, une base du noyau est $(e_1-e_2, 2e_2-e_3)$ (c'est une famille libre car les deux vecteurs sont non colinéaires et le noyau est de dimension $2$ car le rang de la matrice est $1$).

\vspace{5cm}
\end{Exemple}
\end{itemize}

\begin{ApplicationDirecte} Déterminer l'image, le noyau et le rang des matrices suivantes :
$$ \begin{pmatrix}
-2 & 1 & 3 \\
2 & -1 & -3 \\
4 & -2 & -6 \\
\end{pmatrix} \quad \hbox{ et } \begin{pmatrix}
2 & 1 & 0 \\
-1 & 5 & 4 \\
4 & 1 & 3 \\
\end{pmatrix} $$
\end{ApplicationDirecte}

\begin{Theoreme}{} Le rang d'une matrice est égal au rang de sa transposée. 
En particulier, si l'on souhaite obtenir le rang d'une matrice, on peut utiliser les opérations élémentaires sur les lignes et sur les colonnes. 
\end{Theoreme}

\begin{Remarque}{\warning}
 Si l'on souhaite obtenir une base de l'image, on manipule uniquement les colonnes.
\end{Remarque}
\section{Trace d'une matrice, d'un endomorphisme}

\begin{Definition}{} Soit $A=(a_{i,j})_{1 \leq i,j \leq n} \in \mathcal{M}_n(\mathbb{K})$. On appelle \emph{trace} de $A$, et on note $\textrm{Tr}(A)$ la somme des coefficients diagonaux de $A$ :
$$\textrm{Tr}(A) = \sum_{k=1}^n a_{k,k}$$
\end{Definition}

\begin{Proposition}{} La trace est une forme linéaire sur $\mathcal{M}_n( \mathbb{K})$.
\end{Proposition}

\begin{Exemple} L'ensemble des matrices de trace nulle est un hyperplan de $\mathcal{M}_n(\mathbb{K})$ (donc de dimension $n^2-1$) car noyau d'une forme linéaire non nulle.
\end{Exemple}

\begin{Proposition}{} Soient $A \in \mathcal{M}_{n,p}(\mathbb{K})$ et $B \in \mathcal{M}_{p,n}(\mathbb{K})$. Alors :
$$ \textrm{Tr}(AB) = \textrm{Tr}(BA) $$
\end{Proposition}

\begin{Demonstration}{} 

\vspace{7cm}
%On a :
%\begin{align*}
%\textrm{Tr}(AB) & = \sum_{k=1}^n (AB)_{k,k} \\
%& = \sum_{k=1}^n \sum_{i=1}^p a_{k,i} b_{i,k} \\
%& = \sum_{i=1}^p \sum_{k=1}^n  b_{i,k} a_{k,i} \\
%& = \sum_{i=1}^p (BA)_{i,i} \\
%& = \textrm{Tr}(BA)  \\
%\end{align*}
\end{Demonstration}

\begin{Corollaire}{} Deux matrices semblables ont la même trace.
\end{Corollaire}

\begin{Demonstration}{} 
\vspace{3cm}
%Soient $A, B \in \mathcal{M}_n( \mathbb{K})$ et $P \in GL_n(\mathbb{K})$ tels que $B=PAP^{-1}$. On a :
%$$ \textrm{Tr}(B) = \textrm{Tr}((PA)P^{-1}) = \textrm{Tr}(P^{-1}PA) = \textrm{Tr}(I_nA)= \textrm{Tr}(A)$$
\end{Demonstration}


\begin{ApplicationDirecte} Justifier rapidement que $\begin{pmatrix}
1 & 2 \\
3 & 4 \\
\end{pmatrix}$ et $\begin{pmatrix}
-1 & 4 \\
4 & 7 \\
\end{pmatrix}$ ne sont pas semblables.
\end{ApplicationDirecte}



\begin{TheoremeDefinition}{} Soient $E$ un espace vectoriel de dimension finie et $u \in \mathcal{L}(E)$. On appelle \emph{trace} de $u$, et on note $\textrm{Tr}(u)$, la trace commune aux matrices représentant $u$.
\end{TheoremeDefinition}

\begin{Demonstration}{} Cette définition a bien un sens car deux matrices représentant $u$ sont semblables et ont donc la même trace.
\end{Demonstration}

\medskip

\begin{Exemple} La trace d'un projecteur est égal à son rang.

        \vspace{\stretch{1}}

\newpage
\vspace*{5cm}
\medskip

%Soit $p$ un projecteur d'un espace vectoriel $E$ de dimension finie $n$. On sait que $E = \textrm{Im}(p) \oplus \textrm{Ker}(p)$. Soit $\mathcal{B}$ une base de $E$ obtenu par concaténation d'une base de $\textrm{Im}(p)$ (de dimension $r= rg(p)$) et d'une base de $\textrm{Ker}(p)$. En remarquant que pour tout élément de $y \in \textrm{Im}(p)$, $p(y)=y$, on a :
%$$ \textrm{Mat}_{\mathcal{B}}(u) = \begin{pmatrix}
%I_r & 0_{r,n-r} \\
%0_{n-r,r}& 0_{n-r,n-r} \\
%\end{pmatrix}$$
%et ainsi $\textrm{Tr}(\textrm{Mat}_{\mathcal{B}}(u)) = \textrm{Tr}(I_r)= r = rg(u)$.
\end{Exemple}

\begin{ApplicationDirecte} Soit $n \in \mathbb{N}^*$. Posons pour tout $P \in \mathbb{R}_n[X]$, $\varphi(P) = ((X^2-1)P(X))''$. Justifier que $\varphi$ induit un endomorphisme de $\mathbb{R}_n[X]$ et déterminer la trace de celui-ci.
\end{ApplicationDirecte}

\section{Déterminant}

\subsection{Définition et premières propriétés}

\begin{Definition}{} Il existe une unique application $f : \mathcal{M}_n(\mathbb{K}) \rightarrow \mathbb{K}$ vérifiant les trois propriétés suivantes :

\begin{enumerate}
\item L'application $f$ est linéaire par rapport à chaque colonne de sa variable.
\item L'application $f$ est antisymétrique par rapport à chaque colonne de sa variable, c'est-à-dire : pour toute matrice $A \in \mathcal{M}_n(\mathbb{K})$, si $B$ est une matrice obtenue à partir de $A$ en permutant deux colonnes d'indices différents de celles-ci, alors $f(B)=-f(A)$.
\item On a l'égalité : $f(I_n)=1$.
\end{enumerate}

Cette application est appelée \emph{déterminant} et est notée $\textrm{det}$.
\end{Definition}

\begin{Remarque}{} On note aussi :
$$ \textrm{det} \left( \begin{pmatrix}
a_{1,1} & \ldots & a_{1,n} \\
\vdots & & \vdots \\
a_{n,1} & \ldots & a_{n,n} \\
\end{pmatrix}\right) = \left\vert \begin{array}{ccc}
a_{1,1} & \ldots & a_{1,n} \\
\vdots & & \vdots \\
a_{n,1} & \ldots & a_{n,n} \\
\end{array}\right\vert$$
\end{Remarque}

\begin{Corollaire}{} Soit $A \in \mathcal{M}_n(\mathbb{K})$.

\begin{enumerate}
\item Si $A$ a deux colonnes égales alors $\textrm{det}(A)=0$.
\item Si une colonne de $A$ est combinaison linéaire des autres colonnes de $A$ alors $\textrm{det}(A)=0$.
\end{enumerate}
\end{Corollaire}
%
%\begin{Demonstration}{}
%%\begin{enumerate}
%%\item Si l'on permute les deux colonnes de $A$ identiques, on a par antisymétrie du déterminant $\textrm{det}(A)=- \textrm{det}(A)$ donc $\textrm{det}(A)=0$.
%%\item Notons $C_1, \ldots, C_n$ les colonnes de $A$ et notons pour simplifier $\textrm{det}(C_1, \ldots, C_n)$ au lieu de $\textrm{det}(A)$. Supposons, sans perte de généralité, que $C_1$ soit combinaison linéaire des autres colonnes de $A$ : il existe $(\lambda_2, \ldots, \lambda_n) \in \mathbb{K}^{n-1}$ tel que :
%%$$ C_1 = \sum_{k=2}^n \lambda_k C_k $$
%%Ainsi :
%%$$\textrm{det}(A) = \textrm{det} \left( \sum_{k=2}^n \lambda_k C_k, C_2, \ldots, C_n \right)$$
%%puis par linéarité par rapport à la première colonne :
%%$$ \textrm{det}(A) = \sum_{k=2}^n \lambda_k \textrm{det} ( C_k, C_2, \ldots, C_n)$$
%%et chacun des déterminants est nul (car la matrice associée a à chaque fois deux colonnes identiques) et ainsi $\textrm{det}(A)=0$.
%%\end{enumerate}
%
%\vspace{5cm}
%\end{Demonstration}
%
%\newpage
%
%$\phantom{ bla }$
%\vspace{3cm}

\begin{Proposition}{} $\left\vert \begin{array}{cc}
a & b \\
c & d
\end{array} \right\vert = ad-bc$
\end{Proposition}

\begin{Proposition}{Règle de Sarrus}

$$\left\vert \begin{array}{ccc}
a_{1,1} & a_{1,2} & a_{1,3} \\
a_{2,1} & a_{2,2} & a_{2,3} \\
a_{3,1} & a_{3,2} & a_{3,3} \\
\end{array} \right\vert =  (a_{1,1}a_{2,2} a_{3,3} +  a_{2,1}a_{3,2}a_{1,3} + a_{3,1}a_{1,2}a_{2,3}) - (a_{3,1}a_{2,2}a_{1,3} + a_{2,1} a_{1,2} a_{3,3} + a_{3,3} a_{3,2} a_{1,2} ) $$
\end{Proposition}

\begin{Remarque}{} En particulier, le déterminant d'une matrice triangulaire carrée d'ordre $2$ ou $3$ est égal au produit de ses coefficients diagonaux.
\end{Remarque}

\begin{ApplicationDirecte} Calculer $\left\vert \begin{array}{ccc}
1 & j & j^2 \\
j & j^2& 1 \\
j^2 & 1 & j \\
\end{array}\right\vert$ où $j=e^{\frac{2i\pi}{3}}$.
\end{ApplicationDirecte}

\subsection{Opérations élémentaires}

Soient $A \in \mathcal{M}_n(\mathbb{K})$. Notons $C_1, \ldots, C_n$ les colonnes de $A$. Voici les opérations élémentaires liées aux déterminant :

\begin{itemize}
\item Si $B$ est obtenue à partir de $A$ en permutant deux colonnes $C_i$ et $C_j$ avec $i \neq j$ alors $\textrm{det}(B) = - \textrm{det}(A)$.
\item Si $B$ est obtenue à partir de $A$ avec l'opération $C_i \leftarrow C_i + \lambda C_j$ $i \neq j$ et $\lambda \in \mathbb{K}$ alors $\textrm{det}(B) = \textrm{det}(A)$.
\item Si $B$ est obtenue à partir de $A$ avec l'opération $C_i \leftarrow \lambda C_i$ $i \neq j$ et $\lambda \in \mathbb{K}$ alors $\textrm{det}(B) = \lambda\textrm{det}(A)$.
\item Pour tout $\lambda \in \mathbb{K}$, $\textrm{det}(\lambda A) = \lambda^n \textrm{det}(A)$.
\end{itemize}


\subsection{Propriétés classiques}

\begin{Proposition}{} Soit $A \in \mathcal{M}_n(\mathbb{K})$. Alors $A$ est inversible si et seulement si $\textrm{det}(A) \neq 0$.
\end{Proposition}

\begin{Proposition}{} Soient $A$, $B$ deux matrices de $\mathcal{M}_n(\mathbb{K})$. Alors :
$$ \textrm{det}(AB) = \textrm{det}(A) \textrm{det}(B)$$
\end{Proposition}

\begin{Corollaire}{} Soit $A \in \mathcal{M}_n(\mathbb{K})$ une matrice inversible. Alors $\textrm{det}(A^{-1}) = \dfrac{1}{\textrm{det}(A)}
\cdot$
\end{Corollaire}

\begin{Proposition}{} Deux matrices semblables ont le même déterminant.
\end{Proposition}

\begin{Demonstration}{} 
%Soient $A$, $B$ deux matrices semblables de $\mathcal{M}_n(\mathbb{K})$. Il existe une matrice $P \in GL_n(\mathbb{K})$ tel que $A=PBP^{-1}$. Ainsi :
%$$ \textrm{det}(A) = \textrm{det}(PBP^{-1}) = \textrm{det}(P) \textrm{det}(B) \textrm{det}(P^{-1}) =  \textrm{det}(P) \textrm{det}(B) \dfrac{1}{\textrm{det}(P)} = \textrm{det}(B) $$
\vspace{4cm}
\end{Demonstration}

\begin{Proposition}{} Soit $A \in \mathcal{M}_n(\mathbb{K})$. Alors $\textrm{det}(~^tA) = \textrm{det}(A)$.
\end{Proposition}

\begin{Corollaire}{} Les propriétés du déterminant par rapport aux colonnes sont également vraies par rapport aux lignes.
\end{Corollaire}

\subsection{Déterminant d'une famille de vecteurs}

Soient $E$ un espace vectoriel de dimension finie $n$ et $\mathcal{B}$ une base de $E$.

\begin{Definition}{} Soit $\mathcal{F} = (x_1, \ldots, x_n)$ une famille de vecteurs de $E$. On appelle \emph{déterminant} de $\mathcal{F}$ dans la base $\mathcal{B}$, et on note $\textrm{det}_{\mathcal{B}}(\mathcal{F})$, le déterminant de $\textrm{Mat}_{\mathcal{B}}(\mathcal{F})$.
\end{Definition}

\begin{Proposition}{} Soit $\mathcal{F} = (x_1, \ldots, x_n)$ une famille de vecteurs de $E$. Alors $\mathcal{F}$ est une base de $E$ si et seulement si \newline $\textrm{det}_{\mathcal{B}}(\mathcal{F}) \neq 0$.
\end{Proposition}

\begin{Exemple} Soit $\mathcal{B}=((1,1,1), (-1,2,0), (0,1,5))$. Montrons que $\mathcal{B}$ est une base de $\mathbb{R}^3$.

\vspace{5.5cm}
\end{Exemple}

\subsection{Déterminant d'un endomorphisme}

Soient $E$ un espace vectoriel de dimension finie $n$ et $\mathcal{B}$ une base de $E$.

\begin{TheoremeDefinition}{} Soit $u \in \mathcal{L}(E)$. Toutes les matrices représentant $u$ ont le même déterminant. On définit le \emph{déterminant} de $u$ comme le déterminant d'une de ses matrices.
\end{TheoremeDefinition}

\begin{Demonstration}{} Deux matrices de $u$ dans des bases différentes de $E$ sont semblables et donc admettent le même déterminant.
\end{Demonstration}

\medskip


\begin{Proposition}{} Les propriétés du déterminant de matrice (déterminant d'un produit, lien entre inversibilité et déterminant) se transposent au déterminant d'un endomorphisme (déterminant d'une composée, lien entre automorphisme et déterminant).
\end{Proposition} 

\begin{ApplicationDirecte} Donner le déterminant de l'endomorphisme de l'exercice d'application directe du cours $6$.
\end{ApplicationDirecte}

\subsection{Méthodes de calcul}
%\textbf{MINEUR COFACTEUR}
\subsubsection{Matrice triangulaire}

\begin{Proposition}{} Le déterminant d'une matrice triangulaire est égal au produit de ses coefficients diagonaux.
\end{Proposition}

\begin{Remarque}{} Le résultat est en particulier vrai pour une matrice diagonale.
\end{Remarque}

\subsubsection{Déterminant par blocs}

\begin{Proposition}{} Soit $A$ une matrice carrée de la forme $A = \begin{pmatrix}
B & C \\
0 & D \\
\end{pmatrix}$ où $B$ et $D$ sont des matrices carrées. Alors \newline $\textrm{det}(A)= \textrm{det}(B) \textrm{det}(D)$.
\end{Proposition}

\begin{Exemple} Soient $n \in \mathbb{N}^*$ et $A, B \in \mathcal{M}_n(\mathbb{R})$. Montrons que :
$$ \left\vert \begin{array}{cc}
A & B \\
B & A \\
\end{array}\right\vert= \textrm{det}(A+B) \textrm{det}(A-B)$$

\medskip

\vspace{8.5cm}
%
%Par opération sur les colonnes, on a : 
%$$  \left\vert \begin{array}{cc}
%A & B \\
%B & A \\
%\end{array}\right\vert =  \left\vert \begin{array}{cc}
%A+B & B \\
%B+A & A \\
%\end{array}\right\vert $$
%puis par opérations sur les lignes :
%$$  \left\vert \begin{array}{cc}
%A & B \\
%B & A \\
%\end{array}\right\vert =  \left\vert \begin{array}{cc}
%A+B & B \\
%0 & A-B \\
%\end{array}\right\vert = \textrm{det}(A+B) \textrm{det}(A-B)$$
%par calcul du déterminant par blocs.
\end{Exemple}

\subsubsection{Développement par rapport à une ligne ou à une colonne}


\begin{Proposition}{} Soit $A \in \mathcal{M}_n(\mathbb{K})$. Pour tout $(i,j) \in  \iii{1}{n}^2$, on définit la matrice $A_{i,j} \in  \mathcal{M}_{n-1}(\mathbb{K})$ comme étant la matrice obtenue en supprimant la $i$-ème ligne et la $j$-ième colonne de $A$. Alors :

\begin{itemize}
\item On peut développer par rapport à la $j$-ième colonne :
$$ \textrm{det}(A) = \sum_{i=1}^n (-1)^{i+j} a_{i,j} \textrm{det}(A_{i,j}) $$
\item On peut développer par rapport à la $i$-ième ligne :
$$ \textrm{det}(A) = \sum_{j=1}^n (-1)^{i+j} a_{i,j} \textrm{det}(A_{i,j}) $$
\end{itemize}
\end{Proposition}

\begin{Remarques}{}
\begin{itemize} 
\item Ces formules sont intéressantes quand une ligne ou une colonne a beaucoup de coefficients nuls. Si ce n'est pas le cas, on utilise des opérations élémentaires pour se ramener à ce cas.\footnote{Et on évite la méthode dite \og du gros bourrin \fg }
\item Le nombre $\textrm{det}(A_{i,j})$ est appelé \emph{mineur} d'indice $(i,j)$ et $(-1)^{i+j}\textrm{det}(A_{i,j})$ le \emph{cofacteur} d'indice $(i,j)$.
\end{itemize}
\end{Remarques}{}


\begin{Exemple} Soit $(a,b,c) \in \mathbb{R}^3$. Donner une forme factorisée de $\left\vert \begin{array}{ccc}
a & b & c \\
c & a & b \\
b & c & a \\
\end{array} \right\vert \cdot$
%
%\medskip
%
%On remarque\footnote{argument classique !} que la somme des coefficients sur les colonnes est constante. En utilisant l'opération $C_1 \leftarrow C_1+C_2+C_3$ puis par linéarité par rapport à la première colonne :
%$$ \left\vert \begin{array}{ccc}
%a & b & c \\
%c & a & b \\
%b & c & a \\
%\end{array} \right\vert = \left\vert \begin{array}{ccc}
%a+b+c & b & c \\
%a+b+c & a & b \\
%a+b+c & c & a \\
%\end{array} \right\vert = (a+b+c) \left\vert \begin{array}{ccc}
%1 & b & c \\
%1 & a & b \\
%1 & c & a \\
%\end{array} \right\vert$$
%On utilise ensuite les opérations $L_2 \leftarrow L_2-L_1$ et $L_3 \leftarrow L_3 - L_1$ :
%$$  \left\vert \begin{array}{ccc}
%a & b & c \\
%c & a & b \\
%b & c & a \\
%\end{array} \right\vert = (a+b+c)  \left\vert \begin{array}{ccc}
%1 & b & c \\
%0 & a-b & b-c \\
%0 & c -b& a-c \\
%\end{array} \right\vert$$
%et donc :
%$$ \left\vert \begin{array}{ccc}
%a & b & c \\
%c & a & b \\
%b & c & a \\
%\end{array} \right\vert = (a+b+c) ((a-b)(a-c)+(c-b)^2) = (a^2+b^2+c^2 - ab - ac-bc)$$

\vspace{9cm}
\end{Exemple}

\newpage
\begin{ApplicationDirecte} 
Calculer pour $n \geq 1$, $\left\vert \begin{array}{ccccc}
1 & 1 & 1 & \ldots & 1 \\
1 & 2 & 2 & \ldots & 2 \\
1 & 2 & 3 & \ldots & 3 \\
\vdots & \vdots & \vdots & \ddots & \vdots \\
1 & 2 & 3 & \ldots & n \\
\end{array} \right\vert \cdot$
\end{ApplicationDirecte}

\subsubsection{Déterminant de Vandermonde}

\begin{Definition}{} Soit $(a_1, \ldots, a_n) \in \mathbb{K}^n$. On pose :
$$ V_n(a_1, \ldots, a_n) = \left\vert \begin{array}{ccccc}
1 & a_1 & a_1^2 & \ldots & a_1^{n-1} \\
1 & a_2 & a_2^2 & \ldots & a_2^{n-1} \\
1 & a_3 & a_3^2 & \ldots & a_3^{n-1} \\
\vdots & \vdots & & \vdots \\
1 & a_n & a_n^2 & \ldots & a_n^{n-1} \\
\end{array}\right\vert = \textrm{det}( (a_i^{j-1})_{1 \leq i,j \leq n})$$
Ce déterminant est appelé \emph{déterminant de Vandermonde} associé à $a_1$, $\ldots$, $a_n$.
\end{Definition}

\begin{Proposition}{} En gardant les notations précédentes, pour tout $n \geq 1$, on a :
$$ V_n(a_1, \ldots, a_n) = \prod_{1 \leq i<j \leq n} (a_j-a_i) $$
Rappelons que par convention, un produit vide vaut $1$.
\end{Proposition}

\begin{Demonstration}{} On commence par déterminer une relation de récurrence de deux manières différentes. 

%\medskip
%
%$\rhd$ \emph{Première méthode} : soit $n \geq 2$. En effectuant pour tout $k \in \iii{2}{n}$ les opérations élémentaires $L_k \leftarrow L_k-L_1$, on a :
%$$ V_n(a_1, \ldots, a_n) = \left\vert \begin{array}{ccccc}
%1 & a_1 & a_1^2  & \ldots & a_1^{n-1} \\
%0 & a_2-a_1 & a_2^2 - a_1^2 & \ldots & a_2^{n-1} - a_1^{n-1} \\
%0 & a_3 -a_1& a_3^2-a_1^2 & \ldots & a_3^{n-1}- a_1^{n-1} \\
%\vdots & \vdots & & \vdots \\
%0 & a_n-a_1 & a_n^2 - a_1^2 & \ldots & a_n^{n-1} - a_1^{n-1}\\
%\end{array}\right\vert$$
%et ainsi en développant par rapport à la première colonne :
%\begin{align*}
%V_n(a_1, \ldots, a_n) & = \left\vert \begin{array}{cccc}
% a_2-a_1 & a_2^2 - a_1^2 & \ldots & a_2^{n-1} - a_1^{n-1} \\
% a_3 -a_1& a_3^2-a_1^2 & \ldots & a_3^{n-1}- a_1^{n-1} \\
% \vdots &  \vdots & & \vdots \\
% a_n-a_1 & a_n^2 - a_1^2 & \ldots & a_n^{n-1} - a_1^{n-1}\\
%\end{array}\right\vert \\
%& = \left\vert\begin{array}{cccc}
% a_2-a_1 & (a_2-a_1)(a_2+a_1)& \ldots & (a_2-a_1)(a_2^{n-2} + a_1 a_2^{n-3} + \cdots + a_1^{n-2}) \\
% a_3 -a_1& (a_3-a_1)(a_3+a_1) & \ldots & (a_3-a_1)(a_3^{n-2} + a_1 a_3^{n-3} + \cdots + a_1^{n-2}) \\
% \vdots &  \vdots & & \vdots \\
% a_n-a_1 & (a_n - a_1)(a_n+a_1) & \ldots & (a_n-a_1)(a_n^{n-2} + a_1 a_n^{n-3} + \cdots + a_1^{n-2}) \\
%\end{array}\right\vert \\
%\end{align*}
%et ainsi par linéarité par rapport à chaque ligne :
%$$ V_n(a_1, \ldots, a_n)  =  \prod_{2 \leq k \leq n} (a_k - a_1) \, \left\vert\begin{array}{cccc}
% 1 & a_2+a_1& \ldots & a_2^{n-2} + a_1 a_2^{n-3} + \cdots + a_1^{n-2} \\
% 1& a_3+a_1 & \ldots & a_3^{n-2} + a_1 a_3^{n-3} + \cdots + a_1^{n-2} \\
% \vdots &  \vdots & & \vdots \\
% 1 & a_n+a_1 & \ldots & a_n^{n-2} + a_1 a_n^{n-3} + \cdots + a_1^{n-2} \\
%\end{array}\right\vert  $$
%On effectue ensuite \emph{successivement}, pour tout $k \in \iii{2}{n-1}$, les opérations $C_k \leftarrow \Sum{i=1}{k-1} a_1^i C_i$ :
%$$ V_n(a_1, \ldots, a_n)  =  \prod_{2 \leq k \leq n} (a_k - a_1) \, \, \left\vert\begin{array}{cccc}
% 1 & a_2& \ldots & a_2^{n-2}\\
% 1& a_3 & \ldots & a_3^{n-2}  \\
% \vdots &  \vdots & & \vdots \\
% 1 & a_n & \ldots & a_n^{n-2}  \\
%\end{array}\right\vert = \prod_{2 \leq k \leq n} (a_k - a_1) V_{n-1}(a_2, \ldots, a_n) $$
%
%\medskip
%
%$\rhd$ \emph{Deuxième méthode} : Soit $n \geq 2$. Remarquons que :
%$$ V_n(X, a_2 \ldots, a_n) = \left\vert \begin{array}{ccccc}
%1 & X & X^2 & \ldots & X^{n-1} \\
%1 & a_2 & a_2^2 & \ldots & a_2^{n-1} \\
%1 & a_3 & a_3^2 & \ldots & a_3^{n-1} \\
%\vdots & \vdots & & \vdots \\
%1 & a_n & a_n^2 & \ldots & a_n^{n-1} \\
%\end{array}\right\vert$$
%est un polynôme de $\mathbb{K}_{n-1}[X]$ : en effet, il suffit de développer par rapport à la première ligne. Ce polynôme admet $a_2$, $\ldots$, $a_{n}$ comme racines (déterminant d'une matrice ayant deux lignes identiques). Ce polynôme admet donc $n-1$ racines et est degré maximum égal à $n-1$ : il existe donc $\lambda \in \mathbb{K}$ tel que :
%$$ V_n(X, a_2, \ldots, a_n) = \lambda \prod_{k=2}^{n} (X-a_k) $$
%Or $\lambda$ est le coefficient dominant du polynôme, il s'obtient en développant par rapport à la première ligne (en remarquant que seul le dernier terme dans le développement intervient pour ce coefficient dominant et sans oublier la puissance de $(-1)$) : 
%$$ \lambda =  (-1)^{n+1} \left\vert \begin{array}{ccccc}
%1 & a_2 & a_2^2 & \ldots & a_2^{n-2} \\
%1 & a_3 & a_3^2 & \ldots & a_3^{n-2} \\
%\vdots & \vdots & & \vdots \\
%1 & a_n & a_n^2 & \ldots & a_n^{n-2} \\
%\end{array}\right\vert =  (-1)^{n-1} V_{n-1}(a_2, \ldots, a_n)$$
%Finalement :
%$$ V_n(X, a_2\ldots, a_n) =  (-1)^{n-1} V_{n-1}(a_2, \ldots, a_n) \prod_{k=2}^{n} (X-a_k)$$
%et ainsi :
%$$ V_n(a_1, \ldots, a_n) =  (-1)^{n-1} V_{n-1}(a_2, \ldots, a_n) \prod_{k=2}^{n} (a_1-a_k)$$
%ou encore :
%$$ V_n(a_1, \ldots, a_n) =   V_{n-1}(a_2, \ldots, a_n) \prod_{k=2}^{n} (a_k-a_1)$$
%
%\medskip

\vspace{5cm}

\newpage

$\phantom{test}$

%\newpage
%
%\phantom{$\rhd$ Prouvons maintenant le résultat souhaitée par récurrence. }

%Pour tout $n \in \mathbb{N}^*$, on pose :
%$$ \mathcal{P}_n \, \hbox{\og} \forall (a_1, \ldots, a_n) \in \mathbb{K}^n, \, V_n(a_1, \ldots, a_n) = \prod_{1 \leq i<j \leq n} (a_j-a_i)$$
%
%\begin{itemize}
%\item Si $n=1$, pour tout $a_1 \in \mathbb{K}$, $V_1(a_1) = \textrm{det}((1)) =1 =\prod_{1 \leq i<j \leq 1} (a_j-a_i)$et ainsi $\mathcal{P}_1$ est vraie.
%\item Soit $n \in \mathbb{N}^*$ tel que $\mathcal{P}_n$ est vraie. Montrons que $\mathcal{P}_{n+1}$ est vraie : on a $n+1 \geq 2$ donc d'après la relation de récurrence, pour tout $a_1, \ldots, a_{n+1} \in \mathbb{K}$, on a :
%$$ V_n(a_1, \ldots, a_{n+1}) =   V_{n}(a_2, \ldots, a_{n+1}) \prod_{k=2}^{n+1} (a_k-a_1)$$
%et par hypothèse de récurrence :
%$$V_n(a_1, \ldots, a_{n+1}) =  \prod_{2 \leq i<j \leq n+1} (a_j-a_i) \;  \prod_{k=2}^{n+1} (a_k-a_1)  =  \prod_{1 \leq i<j \leq n+1} (a_j-a_i)$$
%et ainsi $\mathcal{P}_{n+1}$ est vraie.
%\item Par principe de récurrence, la propriété est vraie pour tout $n \geq 1$.
%\end{itemize}

\vspace*{\stretch{1}}
\end{Demonstration}

%\textbf{RAJOUTER DES EXEMPLES}
%
%\newpage
%\section{Exercices d'applications directes du cours}





%\section{Annexe 1 : méthode du pivot de Gauss}
%
%\subsection{Opérations élémentaires}
%
%\begin{Definition}{}[Opérations élémentaires]
%On appelle \emph{opération élémentaire sur les lignes} d'un système $(S)$ toute transformation du système $(S)$ en un système $(S')$ consistant en :
%\begin{itemize}
%\item \emph{L'échange de deux lignes}\\
%On permute les lignes $L_i$ et $L_j$, opération codée \phantom{$L_i \leftrightarrow L_j$.}
%\item \emph{La multiplication d'une ligne par un réel non nul}\\
%On remplace la ligne $L_i$ par $a$ fois la ligne $L_i$ avec $a \neq 0$, opération codée \phantom{$L_i \leftarrow aL_i$.}
%\item \emph{L'ajout à une ligne d'un multiple d'une autre ligne}\\
%On ajoute à la ligne $L_i$, $b$ fois la ligne $L_j$, avec $i \neq j$, opération codée \phantom{$L_i \leftarrow L_i+bL_j$.}
%\end{itemize}
%\end{Definition}
%
%\begin{Remarque}{}
%On admet aussi comme opération élémentaire l'opération obtenue en combinant les deux dernières opérations : $L_i \leftarrow aL_i+bL_j$ avec $a \neq 0$, $b \neq 0$ et $i \neq j$.
%\end{Remarque}
%
%
%\begin{Remarque}{}
%Si l'on a effectué une opération élémentaire transformant le système $(S)$ en le système $(S')$, on peut retrouver le système $(S)$ en effectuant une nouvelle opération élémentaire au système $(S')$ (appelée \emph{opération inverse}).
%\end{Remarque}
%
%\begin{Proposition}{}
%Toute opération élémentaire transforme un système $(S)$ en un système $(S')$ équivalent (c'est-à-dire un système ayant les mêmes solutions).
%\end{Proposition}
%
%\subsection{Résolution d'un système linéaire}
%
%\begin{Theoreme}{}
%Grâce à des opérations élémentaires, on peut transformer tout système linéaire $(S)$ $n \times p$ (avec $n \leq p$) en un système équivalent $(S')$ qui est échelonné, auquel s'ajoutent éventuellement des équations du type $0$=constante. Le nombres de pivots du système $(S')$ est le \emph{rang} du système linéaire.
%\end{Theoreme}
%
%Ainsi, on peut toujours se ramener à un système échelonné que l'on sait résoudre.
%
%\begin{Methode}{}[Pivot de Gauss]
%Soit $(S)$ est un système de $n$ équations à $p$ inconnues. Voici la méthode du pivot de Gauss :
% \begin{itemize}
% \item On multiplie éventuellement certaines lignes par un coefficient non nul ($L_i \leftarrow a L_i$) pour simplifier les équations.
%  \item On échange si besoin la première ligne avec une autre (opération $L_1 \leftrightarrow L_i$)  pour que le coefficient de $x_1$ de la première ligne soit non nul et \emph{le plus simple possible} pour la suite des opérations.
%  \item On effectue une opération du type $L_i \leftarrow L_i - a L_1$ pour les lignes $2,\, \ldots, n$ afin de faire \og disparaitre \fg le terme en $x_1$ dans ces lignes et se retrouver avec un système du type :
%  $$ \l\{ \begin{array}{rcl}
%           a_{1,1}x_{1}+a_{1,2}x_{2}+\ldots+a_{1,p}x_{p} &= &b_{1} \hspace{1cm}(L_1)\\
%\phantom{a_{2,1}x_{1}+}a_{2,2}x_{2}+\ldots+a_{2,p}x_{p} &= &b_{2} \hspace{1cm}(L_2)\\
%                                             &\vdots & \\
%\phantom{a_{n,1}x_{1}+}a_{n,2}x_{2}+\ldots+a_{n,p}x_{p} &= &b_{n} \hspace{1cm}(L_n)
%          \end{array}\r.$$
% \item On garde alors la première ligne et on recommence le procédé sur le sous-système constitué des lignes $L_2$ à $L_n$ et dont les inconnues sont $x_2$ à $x_n$ (si l'une des lignes $L_2$ à $L_n$ a un coefficient non nul pour $x_2$, sinon les inconnues sont $x_i$ à $x_n$ où $x_i$ est la première inconnue dont au moins un coefficient est non nul).
%\item On continue ainsi jusqu'à obtenir un système échelonné, auquel s'ajoutent éventuellement des équations du type $0$=constante.
%%\item On répète pour le sous système des lignes $L_2$ à $L_n$ et dont les inconnues sont $x_2$ à $x_n$ (si l'une des lignes $L_2$ à $L_n$ a un coefficient non nul pour $x_2$, sinon les inconnues sont $x_i$ à $x_n$ où $x_i$ est la première inconnue dont au moins un coefficient est non nul).
%%\item On recommence ainsi jusqu'à obtenir un système échelonné, auquel s'ajoutent éventuellement des équations du type 0=constante.
%\item Les équations du type $0=0$ peuvent être négligées. Si le système comporte une équation du type $c=0$ où $c$ est une constante non nulle, cela signifie que le système n'a pas de solution.
%\end{itemize}
%\end{Methode}
%
%Autrement dit, dans la première étape on utilise comme pivot \og le $x_1$ \fg de la première ligne pour \og éliminer \fg{} tous les termes en $x_1$ des autres lignes. Puis on recommence avec $x_2$ sans toucher à la première ligne, etc$\ldots$
%
%
%
%\begin{Remarque}{}
%Quand on résout un système il faut placer les inconnues en colonnes pour faciliter les calculs. A chaque étape on peut soit échanger deux lignes soit pour un $i$ fixé effectuer simultanément les opérations $L_j \leftarrow a L_j + b L_i$ pour $j>i$. On indiquera toujours les opérations effectuées.
%\end{Remarque}
%
%\subsection{Systèmes avec plus d'équations que d'inconnues}
%\begin{Methode}{}
%Soit $(n,p) \in (\mathbb{N}^*)^2$. \\
%Si un système à $n$ équations inconnues et $p$ inconnues avec $n>p$, on procède de la manière suivante :
%\begin{itemize}
%\item On ne travaille qu'avec $p$ équations, on se ramène donc à un système avec autant d'équations que d'inconnues que l'on sait résoudre (grâce à la méthode du pivot de Gauss).
%\item Si il y a des solutions, on remplace celles-ci dans les équations restantes pour voir si ces solutions le sont aussi pour ces équations.
%\end{itemize}
%\end{Methode}
%
%
%
%

\end{document}
