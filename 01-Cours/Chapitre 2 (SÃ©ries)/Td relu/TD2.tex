\documentclass[a4paper,10pt]{report}
\usepackage{Cours}
\usepackage{pifont}

\newcommand{\Sum}[2]{\ensuremath{\textstyle{\sum\limits_{#1}^{#2}}}}
\newcommand{\Int}[2]{\ensuremath{\mathchoice%
	{{\displaystyle\int_{#1}^{#2}}}
	{{\displaystyle\int_{#1}^{#2}}}
	{\int_{#1}^{#2}}
	{\int_{#1}^{#2}}
	}}


\begin{document}
\everymath{\displaystyle}
\begin{center}
\textit{{ {\huge TD 2 : Séries}}}
\end{center}
\bigskip


\begin{center}
\textit{{ {\large Nature de séries à termes positifs}}}
\end{center}


\begin{Exa} Donner la nature des séries suivantes ($a$ est un réel) :
\begin{multicols}{3}
%Termes positifs
\begin{enumerate}
\item $\Sum{n \geq 0}{} \dis\frac{n-1}{7n^2+7}$
\item $\Sum{n \geq 1}{}\dis\frac{\sqrt{n}\ln(n)}{n^2+1}$
\item $\Sum{n \geq 1}{}\dis\frac{\sqrt{n+1}}{n\ln(n)+2}$ 
\item $\Sum{n \geq 1}{} \dfrac{n!}{n^n} $
\columnbreak
% plus petit que 2/n^2 
\item $\Sum{n \geq 0}{} e^{-n^2} $
% plus petit que exp(-n)
\item $\Sum{n \geq 1}{} \dfrac{\ln(n)}{\sqrt{n}} $
% plus grand que ln(2)/sqrt(n)
\item $\Sum{n \geq 2}{} \dfrac{1}{n+(-1)^n \sqrt{n}} $
% Par équivalences
\item $\Sum{n \geq 1}{} \dfrac{1}{1+2+ \cdots + n} $
\columnbreak
% Par équivalences
\item $\Sum{n \geq 1}{} n^{-1- \frac{1}{n}} $
% Par équivalences
\item $\Sum{n \geq 1}{}  \dfrac{n^2}{(n-1)!} $
% D'alembert
\item $\Sum{n \geq 0}{} \dfrac{1}{(2n)!} \dis \prod_{k=0}^n (a+k)^2$ 
% D'alembert mais attention au cas où a est un entier relatif négatif
\item $\Sum{n \geq 1}{} \dfrac{\sqrt{n+1}-\sqrt{n}}{n^a}$ 
% Par équivalents et quantité conjugué
%\item $\dis \sum \frac{1}{d_n^2}$
\end{enumerate}
\end{multicols}

\vspace{0.05cm}

\end{Exa}


\begin{Exa} Soient $\Sum{n \geq 0}{} u_n$ et $\Sum{n \geq 0}{} v_n$ deux séries convergentes à termes strictement positifs.
\begin{enumerate}
\item Montrer que $\Sum{n \geq 0}{} \min(u_n,v_n)$ et $\Sum{n \geq 0}{} \max(u_n,v_n)$ convergent.
\item Montrer que $\Sum{n \geq 0}{} \sqrt{u_n v_n}$ et $\Sum{n \geq 0}{} \dfrac{u_n v_n}{u_n+v_n}$ convergent.
\end{enumerate}
\end{Exa}

\begin{Exa} Considérons une série de terme général $u_n$ à termes strictement positifs et notons pour tout $n\in\N,$ $S_n$ la somme partielle d'ordre $n$ associée. 

On suppose que la s\'erie $\Sum{n \geq 0}{} u_n$ converge. Quelle est la nature de $\Sum{n \geq 0}{} \dis \dfrac{u_n}{S_n}?$
\end{Exa}



\medskip

\begin{center}
\textit{{ {\large Calcul de sommes}}}
\end{center}

\medskip

\begin{Exa} Déterminer la nature de la série suivante et donner sa somme en cas de convergence : 
$$ \sum_{n \geq 1} \ln \l( \frac{(n+1)^2}{n(n+2)}\r)$$
\end{Exa}

\begin{Exa} On pose pour tout entier $n \geq 1$,
$$ a_n = \dfrac{1}{1^2+2^2 + \cdots + n^2}$$
\begin{enumerate}
\item Montrer que la série de terme général $a_n$ est convergente.
\item On pose pour tout $n \geq 1$,
$$ H_n = \sum_{k=1}^n \dfrac{1}{k}$$
Montrer que $\dis \lim_{n \rightarrow + \infty} H_{2n}-H_n = \ln(2)$.
\item Trouver $a$, $b$ et $c$ tels que pour tout $n \geq 1$,
$$ a_n = \dfrac{a}{n} + \dfrac{b}{n+1} + \dfrac{c}{2n+1}$$
\item En déduire la valeur de $\dis \sum_{n=1}^{+ \infty} a_n$.
\end{enumerate}
\end{Exa}


\begin{Exa} Déterminer la nature, et la somme le cas échéant, de $\Sum{n \geq 0}{} e^{-2n} \ch(n)$.
\end{Exa}

\begin{Exa} Déterminer la nature, et la somme le cas échéant, de $\dis \Sum{n \geq 0}{} \dfrac{\sin(n \theta)}{2^n}$ où $\theta \in \mathbb{R}$.
\end{Exa}


\begin{Exa} 
\begin{enumerate}
\item Déterminer trois réels $a$, $b$ et $c$ tels que pour tout entier $n \geq 1$,
$$ \frac{1}{n(n+1)(n+2)} = \frac{a}{n} + \frac{b}{n+1} + \frac{c}{n+2}$$
\item En déduire que $\dis \Sum{n \geq 1}{} \dfrac{1}{n(n+1)(n+2)}$ converge et donner sa somme.
\end{enumerate}
\end{Exa}


\begin{Exa} Pour tout entier naturel non nul $p$, on pose :
$$ \forall n \in \mathbb{N}^*, \quad u(n,p) = \frac{1}{n(n+1)\cdots(n+p)}$$

\begin{enumerate}
\item Montrer que $\Sum{n \geq 1}{}  u(n,p)$ converge.
\item On pose pour tout entier $p \geq 1$,
$$ \sigma(p) = \sum_{n=1}^{+ \infty} u(n,p)$$
Calculer $\sigma(1)$.
\item Pour tout entier $p \geq 2$ et $n \in \mathbb{N}^*$, exprimer $u(n,p-1)-u(n+1,p-1)$ en fonction de $p$ et $u(n,p)$.
\item En déduire la valeur de $\sigma(p)$ pour tout entier $p \geq 2$.
\end{enumerate}
\end{Exa}




\medskip

\begin{center}
\textit{{ {\large Nature de séries à termes quelconques}}}
\end{center}

\medskip

\begin{Exa} Déterminer la nature de $\Sum{n \geq 0}{} \dis\frac{\sin(e^n)}{n^3+n^2+1} \cdot$
\end{Exa}

\begin{Exa}[\ding{80}] Déterminer la nature de $\Sum{n \geq 0}{} \sin \left({\pi \sqrt {n^2 + 1}} \right)$.
\end{Exa}



\begin{Exa} Étudier la convergence de $\Sum{n \geq 1}{} \dis \ln \left( 1 + \dfrac{(-1)^n}{n^{\alpha}} \right)$ pour $\alpha>0$.
\end{Exa} 


\begin{Exa} Déterminer la nature de la série de terme général $u_n = \exp \left( (-1)^n \dfrac{\ln(n)}{n} \right)-1$.
\end{Exa}


\medskip

\begin{center}
\textit{{ {\large Comparaison série-intégrale}}}
\end{center}

\medskip

\begin{Exa} Déterminer un équivalent de $\Sum{k=1}n \dfrac{1}{3k+1}$ quand $n$ tend vers $+ \infty$.
\end{Exa}

\begin{Exa} Donner un équivalent de $\dis \Sum{k=1}n \dfrac{1}{k}$ et de $\dis \Sum{k=n}{+ \infty} \dfrac{1}{k^2}$ quand $n$ tend vers $+ \infty$.
\end{Exa} 

\medskip

\begin{center}
\textit{{ {\large Séries alternées}}}
\end{center}

\medskip

\begin{Exa} Déterminer la nature de $\Sum{n \geq 0}{} \dis\frac{(-1)^n}{\sqrt{n+2}}\cdot$
\end{Exa}

\begin{Exa}[\ding{80}] Montrer que $\Sum{n \geq 0}{} {\dfrac{( - 1)^n 8^n}{(2n)!}}$ est convergente et que sa somme est négative. \end{Exa}

\begin{Exa}[\ding{80}] Donner la nature de la série de terme général $u_n = \dfrac{(-1)^n}{\dis \Sum{k=1}{n} \dfrac{1}{k} + (-1)^n} \cdot$
\end{Exa}

\begin{Exa} Étudier la convergence et la convergence absolue de $\dis \Sum{n \geq 1}{} \dfrac{(-1)^n}{n-\sin(n)}\cdot$
\end{Exa}

\begin{Exa} Étudier la nature de $\dis \Sum{n \geq 0}{} (-1)^n u_n$ où pour tout $n \geq 0$,
\vspace{-0.4cm}
$$ u_n = \int_{0}^1 x^n e^{-x} dx$$
\end{Exa}

\begin{Exa}[\ding{80}] Étudier la nature de la série de terme général $\dfrac{(-1)^n}{\sqrt[n]{n!}} \cdot$
\end{Exa}



\medskip

\begin{center}
\textit{{ {\large Séries dont le terme général n'est pas explicite}}}
\end{center}

\medskip

\begin{Exa}  
\begin{enumerate}
\item Étudier la convergence de la suite définie par $u_0 \in \mathbb{R}_+$ et pour tout entier $n \geq 0$ par :
 \[
u_{n + 1} = \frac{e^{ - u_n}}{n + 1}
 \]
% Encadrement, tend vers 0
\item Donner la nature de la série $\Sum{n \geq 0}{} u_n$ et celle de la série $\Sum{n \geq 0}{} (-1)^n u_n$.
% un equivalent à 1/n avec la def. Un positif, Un=exp(-Un-1)/n + DL avec un equiv à 1/n donc Un = 1/n + o(1/n)
\end{enumerate}
\end{Exa}



\begin{Exa} Soit $(a_n )_{n \geq 0} $ la suite d\'efinie par $a_0  \in \mathbb{R}^{*}_+$ et pour $n \in \mathbb{N}$ par :
$$a_{n + 1}  = 1 - {\mathrm{e}}^{ - a_n } $$
\begin{enumerate}
	\item Étudier la convergence de la suite $(a_n )_{n \geq 0}$.
	
\item D\'eterminer la nature de la s\'erie de terme g\'en\'eral $( - 1)^n a_n.$
	
	\item D\'eterminer la nature de la s\'erie de terme g\'en\'eral $a_n^2 $.
	
	\item  D\'eterminer la nature de la s\'erie de terme g\'en\'eral $a_n $. On pourra commencer par étudier la s\'erie de terme général $\dis \ln \left( {\dis\frac{{a_{n + 1} }}{{a_n }}} \right) \cdot$
\end{enumerate}
\end{Exa}



\begin{Exa}[\ding{80}]
\begin{enumerate}
\item Montrer que pour tout entier $n \geq 1$, l'équation $nx^3+n^2x-2=0$ admet une unique solution réelle. On note $x_n$ cette solution.
\item Étudier la convergence de $(x_n)_{n \geq 1}$.
\item Étudier la convergence de la série de terme général $x_n$.
\end{enumerate}
\end{Exa}




\medskip

\begin{center}
\textit{{ {\large Divers}}}
\end{center}

\medskip

\begin{Exa}[\ding{80}] Soit $\alpha \in \mathbb{R}$. Pour tout $n \in \mathbb{N}^*$, on pose :
$$ u_n = \sum_{k=0}^n \frac{1}{2k+1} -  \alpha \ln(n)$$

\begin{enumerate}
\item Donner un équivalent de $u_{n+1}-u_n$ quand $n$ tend vers $+ \infty$.
\item Déterminer $\alpha$ pour que $(u_n)_{n \geq 0}$ converge.
\end{enumerate}
\end{Exa} 


\begin{Exa}[\ding{80}] Montrer que $\Sum{k=n+1}{+ \infty} \dfrac{1}{k!} \underset{ + \infty}{\sim} \dfrac{1}{(n+1)!} \cdot$ \end{Exa}













\end{document}