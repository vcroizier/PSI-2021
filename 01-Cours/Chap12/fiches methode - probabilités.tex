\documentclass[french,12pt,twoside]{VcCours}

\newcommand{\boite}[1]{\colorbox{lightgray}{#1}}

\begin{document}
\pagestyle{empty}

\begin{center}
    \large\textbf{\large Ce qu'il faut savoir (en plus du cours) sur les probabilités.}
\end{center}

\begin{itemize}
  \item \underline{Calcul de la probabilité d'une union :}
  
        
        $P(A\cup B)=\cas{P(A)+P(B)\ \textrm{si $A$ et $B$ sont incompatibles}\\
                         P(A)+P(B)-P(A\cap B)\\
                         1-P(\overline{A}\cap\overline{B})\ \textrm{éventuellement}}$

        
        $P\left(\bigcup_{i=1}^nA_i\right)=\cas{1\ \textrm{si les $A_i$ forment un système complet d'événements}\\
                                        \sum_{i=1}^nP(A_i)\ \textrm{si les $A_i$ sont 2 à 2 incompatibles}\\
                                        1-P\left(\bigcap_{i=1}^n\overline{A_i}\right)}$
                                        
      $P\left(\bigcup_{i=1}^{+\infty}A_i\right)=\cas{1\ \textrm{si les $A_i$ forment un système complet d'événements}\\
      \sum_{i=1}^{+\infty}P(A_i)\ \textrm{si les $A_i$ sont 2 à 2 incompatibles ($\sigma$-additivité)}\\
      \lim_{n\to+\infty}P(A_n)\ \textrm{si $(A_i)_i$ est une suite croissante d'événements (continuité croissante)}\\
      \lim_{n\to+\infty}P\left(\bigcup_{i=1}^n A_i\right)\\
      1-P\left(\bigcap_{i=1}^{+\infty}\overline{A_i}\right)}$
  \item \underline{Calcul de la probabilité d'une intersection :}
  
        
                $P(A\cap B)=\cas{0\ \textrm{si $A$ et $B$ sont incompatibles}\\
                         P(A)P(B)\ \textrm{si $A$ et $B$ sont indépendants}\\
                         P(A)P_A(B)\textrm{ ou }P_B(A)P(B)\\
                         P(A)+P(B)-P(A\cup B)\ \textrm{rarement utilisé}}$

        
        $P\left(\bigcap_{i=1}^nA_i\right)=\cas{\prod_{i=1}^nP(A_i)\ \textrm{si les $A_i$ sont mutuellement indépendants}\\
            P(A_1)P_{A_1}(A_2)P_{A_1\cap A_2}(A_3)\ldots P_{A_1\cap\ldots\cap A_{n-1}}(A_n)\ \textrm{formule des probabilités composées}}$

        $P\left(\bigcap_{i=1}^{+\infty}A_i\right)=\cas{\lim_{n\to+\infty}P(A_n)\ \textrm{si $(A_i)_i$ est une suite décroissante d'évts (continuité décroissante)}\\
        \lim_{n\to+\infty}P\left(\bigcap_{i=1}^nA_i\right)}$
  \item \underline{Calcul d'une probabilité conditionnelle :}
  
        En général, il vaut mieux revenir à la définition: $P_B(A)=\frac{P(A\cap B)}{P(B)}$.

  \vspace{-.5em}      
  \item \underline{Autres remarques utiles :}
  	\begin{itemize}
	  \item Il est souvent intéressant de s'intéresser à l'événement contraire: $P(A)=1-P(\overline{A})$.
	  \item Si on veut $P(A)$ alors qu'on connaît $P_?(A)$, il faut penser à la formule des probabilités totales.

              Il faut alors penser à dire qu'il y a un système complet d'événements $(B_i)_i$, ou que les $B_i$ sont 2 à 2 incompatibles et alors que $\sum P(B_i)=1$. (autrement dit : \og{} Ne me faites pas une Barnabé !\fg{})
	  \item Toutes les formules précédent sont valables si on remplace partout $P$ par $P_C$. 
	  \item Dans le cas de l'équiprobabilité, il est souvent possible de calculer tous ces probabilités à l'aide du dénombrement.
	\end{itemize}
                
  
\end{itemize}

\end{document}
