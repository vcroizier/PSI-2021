\documentclass[a4paper,10pt]{report}
\usepackage{cours}
\usepackage{pifont}

\begin{document}
\everymath{\displaystyle}

\begin{center}
\textit{{ {\huge TD 5 : Matrices et applications linéaires }}}
\end{center}

\bigskip

\noindent Dans la suite, $\mathbb{K}$ désignera $\mathbb{R}$ ou $\mathbb{C}$.

\medskip

\begin{center}
\textit{{ {\large Calcul matriciel}}}
\end{center}

\begin{Exercice}{} Soient $(a,b) \in \mathbb{R}^2$ et $A$ définie par :
$$ A = \begin{pmatrix}
a & b \\
0  & a
\end{pmatrix}$$ 
Déterminer toutes les puissances de $A$.
\end{Exercice}

\begin{Exercice}{} Soient $n \in \mathbb{N}^*$ et $M$ une matrice de $\mathcal{M}_n(\mathbb{K})$. On suppose que $M-I_n$ est nilpotente. Montrer que $M$ est inversible.
\end{Exercice}

 \begin{Exercice}{\ding{80}} Soient $n \geq 1$ et $A = (a_{ij})_{1 \leq i,j \leq n}$ une matrice de $\mathcal{M}_n(\mathbb{R})$. On suppose que pour tout $i \in \Interv{1}{n}$,
$$ \vert a_{ii} \vert > \sum_{k \neq i} \vert a_{ik} \vert $$
Montrer que $A$ est inversible.
\end{Exercice}

\begin{Exercice}{\ding{80}} Soient $n\geq 1$ et $A$, $B$ deux matrices de $\mathcal{M}_n(\mathbb{K})$. Montrer que si $I_n - AB$ est inversible alors $I_n - BA$ est inversible.
\end{Exercice} 

\begin{Exercice}{} Soient $n \geq 2$ et $D= \textrm{diag}(\lambda_1, \ldots, \lambda_n)$ ou $\lambda_1$, $\ldots$, $\lambda_n$ sont $n$ éléments de $\mathbb{K}$ deux à deux distincts.
\begin{enumerate}
\item Soit $M \in \mathcal{M}_n(\mathbb{K})$. Montrer que $M$ commute avec $D$ si et seulement si $M$ est diagonale.
\item Soit $M \in \mathcal{M}_n(\mathbb{K})$ une matrice diagonale. Montrer qu'il existe un polynôme $P$ de degré au plus $n-1$ tel que $M=P(D)$.
\end{enumerate}
\end{Exercice}


 \medskip

\begin{center}
\textit{{ {\large Matrice d'application linéaire}}}
\end{center}

\medskip





\begin{Exercice}{} Soit $E$ un espace vectoriel de dimension $n \geq 1$. On suppose que $f \in \mathcal{L}(E)$ vérifie $f^n = \tilde{0}$ et $f^{n-1} \neq \tilde{0}$. Soit $x_0 \in E$ tel que $f^{n-1}(x_0) \neq 0_E$.

\begin{enumerate}
\item Montrer que $\mathcal{B}= (x_0, f(x_0), \ldots, f^{n-1}(x_0))$ est base de $E$. 
\item Déterminer la matrice de $f$ dans $\mathcal{B}$.
\item Soit $f$ l'application canoniquement associée à la matrice $A$ définie par :
$$ A = \begin{pmatrix}
2 & 1 & 0 \\
-3 & -1 & 1 \\
1 & 0 & -1 
\end{pmatrix}$$
Montrer que $f$ vérifie l'hypothèse de l'énoncé et déterminer un $x_0$ correspondant.
\end{enumerate}
\end{Exercice}

\begin{Exercice}{} Donner les matrices, dans la base canonique, des endomorphismes suivants de $E=\mathbb{R}_n[X]$. Que peut-on en déduire concernant ces endomorphismes ?

\begin{enumerate}
\item $f : E \rightarrow E$ définie par :
$$ \forall P \in E, \; f(P) = P(X+1) $$
\item $f : E \rightarrow E$ définie par :
$$ \forall P \in E, \; f(P) = P(X+1)-P(X-1) $$
\item $f : E \rightarrow E$ définie par :
$$ \forall P \in E, \; f(P) = P-P' $$
\end{enumerate}
\end{Exercice} 


\begin{Exercice}{} On considère l'application $\varphi : \mathbb{R}_2[X] \rightarrow \mathbb{R}^3$ définie par :
\[ \varphi(P)=(P(0),P'(0),P(1)) \]

\begin{enumerate}
\item Déterminer la matrice de $\varphi$ dans les bases canoniques des espaces considérés.
\item Montrer que $\varphi$ est un isomorphisme d'espaces vectoriels.
\item Déterminer une expression de $\varphi^{-1}$.
\end{enumerate}
\end{Exercice} 




\begin{Exercice}{} Soit $E=\R^3.$

\begin{enumerate}

\item Montrer que les deux sous-espaces vectoriels suivants sont
suppl\'ementaires dans $E.$
$$F=\hbox{Vect}((1,-1,1))\quad\hbox{ et }\quad G=\{(x,y,z)\in
E,\ 2x+4y+z=0\}$$

\item D\'eterminer la matrice, dans la base canonique ${\cal B}$
de $E$, de la projection sur $G$ parallèlement à $F.$

\item En d\'eduire la matrice, dans la base ${\cal B}$, de la
sym\'etrie par rapport \`a $F$ parallèlement à $G.$
\end{enumerate}
\end{Exercice} 

\medskip

\begin{center}
\textit{{ {\large Matrices semblables}}}
\end{center}

\medskip


\begin{Exercice}{} Soit ${\cal B}=(e_1,e_2,e_3)$ la base canonique de $\R^3.$ On consid\`ere
l'endomorphisme $f$ de $E$ d\'efini par~:

\begin{center}
$\forall (x,y,z)\in\R^3,\quad f(x,y,z)=(x+y-z,x+2y+z,y+2z).$
\end{center}

\begin{enumerate}

\item Écrire la matrice $M$ de $f$ dans la base ${\cal B}.$

\item D\'eterminer une base de $\textrm{Ker}(f)$ et une base de $\textrm{Im}(f)$.

\item Montrer que la r\'eunion des bases pr\'ec\'edentes constitue une base de $E$. On note ${\cal B}'$ cette nouvelle base.

\item  D\'eterminer la matrice $M'$ de $f$ dans cette nouvelle base. Quel est le lien entre $M$ et $M'$ ?
\end{enumerate}

\end{Exercice}

\begin{Exercice}{\ding{80}} Montrer que $\begin{pmatrix}
1 & 1 & -1 \\
-3 & -3 & 3 \\
-2 & -2 & 2 \\
\end{pmatrix}$ et $\begin{pmatrix}
0 & 1 & 0 \\
0 & 0 & 0 \\
0 & 0 & 0 \\
\end{pmatrix}$ sont semblables.
\end{Exercice}

\begin{Exercice}{} Soient $n \geq 1$ et $S : \mathcal{M}_n(\mathbb{C}) \rightarrow \mathbb{C}$ définie par :
$$ S((a_{i,j})_{1 \leq i,j \leq n}) = \sum_{1 \leq i,j \leq n} a_{i,j} a_{j,i}$$

\begin{enumerate}
\item Montrer que pour tout $(A,B) \in \mathcal{M}_n(\mathbb{C})^2$, $S(AB)=S(BA)$.
\item Montrer que deux matrices semblables ont la même image par $S$.
\end{enumerate}
\end{Exercice}

\begin{Exercice}{\ding{80}} Soient $n \geq 1$ et $A$ et $B$ deux matrices \textit{réelles} semblables dans $\mathcal{M}_n(\mathbb{C})$.

\begin{enumerate}
\item Monter l'existence de deux matrices $U$, $V$ de $\mathcal{M}_n(\mathbb{R})$ telles que $AU=UB$, $AV=VB$ et $U+iV$ soit dans $GL_n(\mathbb{C})$.
\item Justifier l'existence d'un réel $a$ tel que $U+aV$ soit inversible.
\item En déduire que $A$ et $B$ sont semblables dans $\mathcal{M}_n(\mathbb{R})$.
\end{enumerate}
\end{Exercice} 



\medskip

\begin{center}
\textit{{ {\large Noyau, Rang}}}
\end{center}

\medskip


\begin{Exercice}{} Donner le noyau, le rang et l'image de $A = \begin{pmatrix}
-5 & 4 & -1 \\
-4 & 3 & - 1 \\
2 	 & -2 & 0 \\
\end{pmatrix}\cdot$
\end{Exercice}

\begin{Exercice}{} Déterminer le rang de $A = \begin{pmatrix}
x & 0 & 0& y \\
y & x & 0 & 0 \\
0 & y & x & 0 \\
0 & 0 & y & x
\end{pmatrix}$ où $(x,y) \in \mathbb{R}^2$.
\end{Exercice} 

\begin{Exercice}{} Déterminer le rang de $A=(i+j+ij)_{1\leq i,j\leq n}$ pour $n \geq 2$.
\end{Exercice} 


\begin{Exercice}{} Soient $n \geq 1$ et $H \in \mathcal{M}_n(\mathbb{C})$ une matrice de rang 1.
    \begin{enumerate}
      \item
        Montrer qu'il existe des matrices $U,V \in \mathcal{M}_{n,1}(\mathbb{C})$ telles que $H = U ~^tV$.
      \item En déduire que $H^2 = \textrm{Tr}(H)H$.
 \end{enumerate}
\end{Exercice}
 \medskip

\begin{center}
\textit{{ {\large Trace}}}
\end{center}

\medskip


\begin{Exercice}{} Soit $n \in \mathbb{N}^*$. Existe-t-il des matrices $A,B \in \mathcal{M}_n(\mathbb{C})$ vérifiant $AB - BA = I_n$ ?
\end{Exercice}


\begin{Exercice}{\ding{80}} Soient $n \geq 1$ et $A$, $B$ deux matrices de $\mathcal{M}_n(\mathbb{R})$. On suppose que pour tout $X \in \mathcal{M}_n(\mathbb{R})$, 
$$\textrm{Tr}(AX)= \textrm{Tr}(BX)$$
Montrer que $A=B$.
\end{Exercice}


\medskip

\begin{center}
\textit{{ {\large Déterminant}}}
\end{center}

\medskip


\begin{Exercice}{} Peut-on trouver une matrice $A \in \mathcal{M}_3(\mathbb{R})$ telle que $A^2= - I_3$?
\end{Exercice}

\begin{Exercice}{} Soient $n \geq 1$, $\lambda_1 ,\lambda_2 ,\ldots,\lambda_n$ des réels et $a$, $b$ deux réels distincts. On pose pour tout $x \in \mathbb{R}$,
    \[
    \Delta_n(x) =
    \begin{vmatrix}
        {\lambda_1 + x} & {a + x} & \cdots & {a + x} \\
        {b + x} & {\lambda_2 + x} & \ddots & \vdots \\
        \vdots & \ddots & \ddots & {a + x} \\
        {b + x} & \cdots & {b + x} & {\lambda_n + x} \\
    \end{vmatrix}    
    \]
    \begin{enumerate}
      \item Montrer que $\Delta_n$ est une fonction affine.
      \item Donner son expression.
    \end{enumerate}
\end{Exercice}


\begin{Exercice}{} Soient $n \geq 2$ et $f \in \mathcal{L}(\mathcal{M}_n(\mathbb{R}))$ définie par $f(M) = \dfrac{M+~^tM}{2}\cdot$

\begin{enumerate}
\item Donner, sans aucun calcul, le déterminant de $f$.
\item Déterminer le noyau, l'image et le rang de $f$. 
\end{enumerate}
\end{Exercice}


\begin{Exercice}{} Déterminer pour tout entier $n \geq 1$, le déterminant $D_n$ de taille $n$ suivant :

\[
    D_n =
    \begin{vmatrix}
        2 & 1 & {} & {(0)} \\
        1 & \ddots & \ddots & {} \\
        {} & \ddots & \ddots & 1 \\
        {(0)} & {} & 1 & 2 \\
    \end{vmatrix}_{[n]}
    \]
\end{Exercice} 


\begin{Exercice}{} Soient $a_1 ,a_2 , \ldots ,a_n \in \mathbb{C}$. Calculer :
    \[
    \begin{vmatrix}
        {a_1} & {a_2} & \cdots & {a_n} \\
        {} & \ddots & \ddots & \vdots \\
        {} & {} & \ddots & {a_2} \\
        {(a_1)} & {} & {} & {a_1} \\
    \end{vmatrix}
    \]
\end{Exercice}




\end{document}