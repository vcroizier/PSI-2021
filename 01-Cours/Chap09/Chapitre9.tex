\documentclass[french,11pt,twoside]{VcCours}
\newcommand{\dx}{\text{d}x}
\newcommand{\dt}{\text{d}t}
\DeclareMathOperator{\e}{e}
\newcommand{\Sum}[2]{\sum_{#1}^{#2}}
\newcommand{\Int}[2]{\int_{#1}^{#2}}

\renewcommand{\trou}[1]{{\color{white}#1}}
%\renewcommand{\trou}[1]{{\color{blue}#1}}

\begin{document}

\Titre{PSI}{Promotion 2021--2022}{Mathématiques}{Chapitre 9 : Intégrales généralisées}

\tableofcontents
\separationTitre

\newpage
Dans la suite $\mathbb{K}= \mathbb{R}$ ou $\mathbb{C}$. Si rien n'est précisé, une fonction définie sur un intervalle est à valeurs complexes.

\section{Intégrale sur un segment de fonctions continues par morceaux}

\subsection{Fonctions continues par morceaux}

\begin{Definition}{}
Soient $(a,b) \in \mathbb{R}^2$ tel que $a \leq b$ et $f$ une fonction définie sur $[a,b]$.

La fonction $f$ est dite \emph{continue par morceaux} sur le segment $[a,b]$ s'il existe une subdivision
\[a_0 = a < a_1 < a_2 < \dots < a_n =b\]
telle que les restrictions de $f$ sur chaque intervalle ouvert $]a_i,a_{i+1}[$ (où $i \in \iii0{n-1}$) admettent un prolongement continu à l'intervalle fermé $[a_i,a_{i+1}]$.
\end{Definition}

\begin{Remarque}{} La subdivision $(a_0, \ldots, a_n)$ est dite adaptée à $f$. Elle n'est pas unique.
\end{Remarque}

\begin{center}
\emph{Exemple d'une courbe d'une fonction continue par morceaux à valeurs réelles}
\end{center}

\pgfarrowsdeclare{ouvert}{ouvert}
{
\arrowsize=0.2pt
\advance\arrowsize by .5\pgflinewidth
\pgfarrowsleftextend{-5\arrowsize-.5\pgflinewidth}
\pgfarrowsrightextend{5\arrowsize}
}
{
\arrowsize=0.2pt
\advance\arrowsize by .5\pgflinewidth
\pgfsetdash{}{0pt} % do not dash
\pgfsetroundjoin % fix join
\pgfsetroundcap % fix cap
\pgfpathmoveto{\pgfpoint{5\arrowsize}{0}}
\pgfpathlineto{\pgfpoint{1\arrowsize}{0}}
\pgfsetstrokecolor{white}
\pgfsetlinewidth{2\arrowsize}
\pgfusepathqstroke
\pgfpathmoveto{\pgfpoint{5\arrowsize}{5\arrowsize}}
\pgfpatharc{90}{270}{5\arrowsize}
\pgfsetstrokecolor{black}
\pgfsetlinewidth{\arrowsize}
\pgfusepathqstroke
}
\pgfarrowsdeclare{fermé}{fermé}
{
\arrowsize=0.2pt
\advance\arrowsize by .5\pgflinewidth
\pgfarrowsleftextend{-5\arrowsize-.5\pgflinewidth}
\pgfarrowsrightextend{5\arrowsize}
}
{
\arrowsize=0.2pt
\advance\arrowsize by .5\pgflinewidth
\pgfsetdash{}{0pt} % do not dash
\pgfsetroundjoin % fix join
\pgfsetroundcap % fix cap
\pgfsetplotmarksize{5\arrowsize}
\pgfplothandlermark{\pgfuseplotmark{*}}
\pgfplotstreamstart
\pgfplotstreampoint{\pgfpoint{5\arrowsize}{0}}
\pgfplotstreamend
\pgfusepathqstroke
}
% \begin{tikzpicture}
% \draw[help lines] (-4,-2) grid (1,1);
% \draw[line width=1pt,ouvert-ouvert] (-3,0) -- (-1.5,1) -- (0,0);
% \draw[line width=1pt,fermé-fermé] (-3,-1) -- (0,-1);
% %\draw[line width=2pt,red] (-3,0) -- (0,0);
% \end{tikzpicture}

\begin{center}
 \begin{tikzpicture}[>=stealth,scale=1]
\draw[->,color=black] (-0.6,0) -- (6.6,0) node[below] {$x$};
\foreach \x in {1,2,...,6}
\draw[shift={(\x,0)},color=black] (0pt,2pt) -- (0pt,-2pt) node[below] {\footnotesize $\x$};
\draw[->,color=black] (0,-0.6) -- (0,4.6) ;
\foreach \y in {1,2,...,4}
\draw[shift={(0,\y)},color=black] (2pt,0pt) -- (-2pt,0pt) node[left] {\footnotesize $\y$};
\draw[color=black] (0pt,-10pt) node[left] {\footnotesize $0$};
\clip(-0.6,-0.6) rectangle (6.6,4.6);
\draw[smooth,samples=100,domain=0:2,fermé-ouvert] plot(\x,{sin(180*\x)+3.5-\x});
\draw (2,0) node {} (2,4) node {$\bullet$} (6,3) node {$\bullet$};
\draw[smooth,samples=100,domain=2:5,ouvert-fermé] plot(\x,{\x-.5-((\x-3.5)/1.5)^2});
\draw[ouvert-ouvert] (5,2)--(6,2);
\end{tikzpicture}
\end{center}

\medskip
 
\begin{center}
\emph{Exemple d'une courbe d'une fonction non continue par morceaux à valeurs réelles}
\end{center}
% 
%\begin{cex} La fonction $h$ définie sur $[0,6]$ et dont la courbe est 
 \begin{center}
  \begin{tikzpicture}[>=stealth,scale=0.8]
\draw[->,color=black] (-0.6,0) -- (6.6,0) node[below] {$x$};
\foreach \x in {1,2,...,6}
\draw[shift={(\x,0)},color=black] (0pt,2pt) -- (0pt,-2pt) node[below] {\footnotesize $\x$};
\draw[->,color=black] (0,-0.6) -- (0,4.6);
\foreach \y in {1,2,...,4}
\draw[shift={(0,\y)},color=black] (2pt,0pt) -- (-2pt,0pt) node[left] {\footnotesize $\y$};
\draw[color=black] (0pt,-10pt) node[left] {\footnotesize $0$};
\clip(-0.6,-0.6) rectangle (6.6,4.6);
\draw[smooth,samples=100,domain=0:1.9] plot(\x,{1/(2-\x)-\x});
\draw (0,0.5) node {$\bullet$};
\draw (2,0) node {$\bullet$};
\draw[smooth,samples=100,domain=2:6] plot(\x,{0.5*(\x-4)^3-(\x-6)}) node {$\bullet$};
\end{tikzpicture}
\end{center}

%\end{cex}
%
%\begin{exems} 
%\item la fonction $f$ définie sur $[0,5]$ par $f(x) = \begin{cases} x+1 &\text{si } x \leq 2 \\ \ln x & \text{si } x > 2 \end{cases}$ est continue par morceaux sur $[0,5]$.
%\item La fonction $f$ définie sur $[-1,1]$ par $f(x) = \begin{cases} 1 &\text{si } x \leq 0 \\ \dfrac{1}{x} & \text{si } x >0 \end{cases}$ n'est pas continue par morceaux sur $[-1,1]$.
%\end{exems}

\begin{Definition}{} Une fonction $f$ est \emph{continue par morceaux} sur un intervalle $I$ si elle est continue par morceaux sur tout segment $[a,b]$ inclus dans $I$.
\end{Definition}

\begin{Remarque}{} La définition est cohérente dans le cas où $I$ est un segment car dans ce cas, $f$ est bien continue par morceaux sur tout sous-segment.
\end{Remarque}

\begin{Proposition}{} L'ensemble des fonctions continues par morceaux sur $I$ à valeurs dans $\mathbb{K}$, notée $\mathcal{CM}(I,\mathbb{K})$ est un sous-espace vectoriel de l'ensemble des fonctions de $I$ à valeurs dans $\mathbb{K}$.
\end{Proposition}

\begin{Demonstration}{} On utilise la méthode des trois points :

\begin{itemize}
\item $\mathcal{CM}(I,\mathbb{K}) \subset \mathcal{F}(I, \mathbb{K})$.
\item L'application nulle est continue par morceaux sur $I$.
\item Soient $f$, $g$ deux fonctions continues par morceaux sur $I$ et $\alpha \in \mathbb{K}$. Fixons un segment $[a,b]$ de $I$. Remarquons d'abord que toute subdivision adaptée à $f$ sur $[a,b]$ est aussi adaptée à $\alpha f$. Donnons nous deux subdivisions $(a_0, a_1, \ldots, a_n)$ et $(b_0,b_1, \ldots, b_p)$ de $f$ et $g$ sur $[a,b]$(avec $n,p \in \mathbb{N}^*$). On crée alors l'union de ces deux subdivisions (en rangeant dans l'ordre croissant ces $n+p$ nombres et en enlevant les répétitions). Cette nouvelle subdivision est alors une subdivision adaptée à $\alpha f + g$ sur $[a,b]$. Pour cela il suffit de remarquer que si $c$ est un point de la nouvelle subdivision alors soit $c$ est un point des deux subdivisions précédentes et il n'y a rien à vérifier et si $c$ est un point de la subdivision de l'une des deux uniquement alors $c$ est un point où l'autre est continue. Ainsi $\alpha f+g$ est continue par morceaux sur tout segment de $I$ donc sur $I$.
\end{itemize}\vspace{-2em}
\end{Demonstration}

\begin{Proposition}{} Toute fonction continue par morceaux sur un segment $[a,b]$ est bornée sur ce segment.
\end{Proposition}



\subsection{Intégrale sur un segment d'une fonction continue par morceaux}

\begin{TheoremeDefinition}{} 
Soient $(a,b) \in \mathbb{R}^2$ tel que $a \leq b$ et $f$ une fonction continue par morceaux sur $[a,b]$.\\
Soit $a_0 = a < a_1 < a_2 < \dots < a_n =b$ une subdivision adaptée à $f$. Pour tout $i \in \iii{0}{n-1}$, on note $\tilde{f}_i$ le prolongement continu de $f$ sur $[a_i,a_{i+1}]$. On définit l'intégrale de $f$ sur $[a,b]$ par :
$$ \int_{a}^b f(t) \dt = \int_{a_0}^{a_1} \tilde{f}_0(t) \dt + \int_{a_1}^{a_2} \tilde{f}_1(t) \dt + \cdots + \int_{a_{n-1}}^{a_n} \tilde{f}_{n-1}(t) \dt$$
Cette définition ne dépend pas de la subdivision choisie.
\end{TheoremeDefinition}

\begin{Demonstration}{} Donnons juste l'idée : si l'on considère une subdivision adaptée à $f$ sur $[a,b]$, on obtient une autre subdivision adapté à $f$ en \og enlevant \fg les points de la première subdivision où $f$ est continue (qui, en un sens, sont inutiles). En faisant cela, on se ramène à une subdivision \og minimale \fg et on peut alors vérifier que la définition ne dépend pas de la subdivision choisie.
\end{Demonstration}

\subsection{Propriétés}

Les propriétés des intégrales des fonctions continues sur un segment se généralisent, pour la plupart, aux intégrales des fonctions continues par morceaux. Nous ne donnons pas les preuves de celles-ci. Dans toute la suite, $(a,b)$ est un couple de réels tels que $a \leq b$.

\begin{Proposition}{Linéarité} Soient $f,g$ deux fonctions continues par morceaux sur $[a,b]$ et à valeurs dans $\mathbb{K}$ et $\lambda \in \mathbb{K}$. Alors :
$$ \int_{a}^b (\lambda f(t) + g(t)) \dt = \lambda \int_{a}^b  f(t) \dt +  \int_{a}^b  g(t) \dt$$
\end{Proposition}

\begin{Proposition}{Relation de Chasles} Soient $f$ une fonction continue par morceaux sur $[a,b]$ et à valeurs dans $\mathbb{K}$ et $c \in [a,b]$. Alors :
$$ \int_{a}^b f(t) \dt = \int_{a}^c f(t) \dt + \int_{c}^b f(t) \dt $$
\end{Proposition}

\begin{Proposition}{Positivité} Soient $f$ une fonction continue par morceaux sur $[a,b]$ et à valeurs réelles \emph{positives}. Alors $\Int{a}{b} f(t) \dt \geq 0$.
\end{Proposition}

\begin{Proposition}{Croissance} Soient $f,g$ deux fonctions continues par morceaux sur $[a,b]$ à valeurs réelles. Si pour tout $t \in [a,b]$, $f(t) \leq g(t)$ alors :
$$\int_{a}^{b} f(t) \dt  \leq \int_{a}^{b} g(t) \dt $$
\end{Proposition}

\begin{Proposition}{Stricte positivité}  Soit $f$ une fonction continue sur $[a,b]$ et à valeurs réelles \emph{positives}. 

\begin{itemize}
\item S'il existe $x_0 \in [a,b]$ tel que $f(x_0)>0$ alors $\Int{a}{b} f(t) \dt>0$.
\item Si $\Int{a}{b} f(t) \dt =0$ alors $f$ est la fonction nulle sur $[a,b]$.
\end{itemize}
\end{Proposition}

\begin{Remarque}[\alerte]{} Cette proposition est fausse en toute généralité si la fonction est continue par morceaux : il suffit de penser à la fonction $f$ définie sur $[0,1]$ par $f(x)=0$ si $x \neq \frac{1}{2}$ et par $f(\frac{1}{2})=1$.
\end{Remarque}

\begin{Proposition}{} Soit $f$ une fonction continue par morceaux sur $[a,b]$ à valeurs dans $\mathbb{K}$. Alors $\vert f \vert$ est continue par morceaux sur $[a,b]$ et à valeurs positives et :
$$ \left\vert \int_{a}^b f(t) \dt \right\vert \leq \int_{a}^b \vert f(t) \vert \dt $$
\end{Proposition}

\begin{Proposition}{} Soient $f$, $g$ deux fonctions continues par morceaux sur $[a,b]$ et à valeurs dans $\mathbb{K}$. 
	
Si $f$ et $g$ coïncident sauf en un nombre fini de points alors :
$$ \int_{a}^b f(t) \dt = \int_{a}^b g(t) \dt $$
\end{Proposition}

\subsection{Quelques méthodes de calcul}
\subsubsection{Méthodes du chapitre 8}

\begin{itemize}
\item En déterminant une primitive.
\item Par linéarisation.
\item Par intégration par parties.
\item Par changement de variable.
\item En utilisant une fonction complexe.
\end{itemize}

Il est très important de retravailler sérieusement les exercices du chapitre 8. 

\subsubsection{Changement de variable}

\begin{Proposition}{} Soient $I$ et $J$ deux intervalles de $\mathbb{R}$, $f$ une fonction continue par morceaux sur $I$ et $\varphi : J \rightarrow I$ une fonction de classe $\mathcal{C}^1$ et strictement monotone. Pour tout $(\alpha, \beta) \in J^2$, on a :
$$ \int_{\varphi(\alpha)}^{\varphi(\beta)} f(t) \dt = \int_{\alpha}^{\beta} f( \varphi(x)) \varphi'(x) \dx$$
\end{Proposition}

\begin{Remarque}{} Dans le cas où $f$ est continue sur $I$, l'hypothèse de stricte monotonie n'est pas utile et on retrouve la formule utilisée en Sup.
\end{Remarque}

\newpage
\begin{Exemple} Calculons $I= \int_{0}^1 \dfrac{1}{ch(t)} \dt$.

\vspace*{7cm}
\end{Exemple}

\begin{Exemple} Calculons $I= \int_{0}^1 \sqrt{1-t^2} \dt$.

\vspace*{7cm}
\end{Exemple}


\subsubsection{Fractions rationnelles}

On appelle \emph{fraction rationnelle} tout élément de la forme $\dfrac{A(X)}{B(X)}$ où $(A,B) \in \mathbb{R}[X]^2$ avec $B$ non nul. Pour calculer l'intégrale d'une fonction associée à une fraction rationnelle, on procède de la manière suivante : 

\bigskip

$\rhd$ On effectue la division euclidienne de $A$ par $B$ : il existe un unique couple $(Q,R) \in \mathbb{R}[X]^2$ tel que :
$$ A(X) = Q(X)B(X) + R(X)$$
avec $\textrm{deg}(R)< \textrm{deg}(B)$. On a ainsi :
$$ \dfrac{A(X)}{B(X)} = Q(X) + \dfrac{R(X)}{B(X)}$$
$Q$ est un polynôme, il suffit donc de savoir calculer l'intégrale de la fonction associée à $\dfrac{R(X)}{B(X)} \cdot$

$\rhd$ On utilise le résultat suivant :

\begin{Theoreme}{} Si la décomposition de $B$ en produit de facteurs irréductibles de $\mathbb{R}[X]$ est :
$$ B(X) = \lambda \prod_{k=1}^n (X-a_k)^{\alpha_k} \times \prod_{k=1}^m (X^2+ c_k X + d_k)^{\beta_k} $$
Alors il existe une unique décomposition de $\dfrac{R(X)}{B(X)}$ sous la forme :
$$ \sum_{k=1}^n \sum_{i=1}^{\alpha_k} \frac{\alpha_{k,i}}{(X-a_k)^{i}} + \sum_{k=1}^m \sum_{i=1}^{\beta_k} \dfrac{\mu_{k,i} X + \nu_{k,i}}{(X^2+ c_k X + d_k)^i} $$
où les $\alpha_{k,i}$, $\mu_{k,i}$ et $\nu_{k,i}$ sont des réels.
\end{Theoreme}

\begin{Exemple} Il existe des réels $a,b,c$ et $d$ tels que :
$$ \dfrac{1}{(X-1)^2 (X^2+1)} = \frac{a}{(X-1)} + \frac{b}{(X-1)^2} + \frac{cX+d}{X^2+1}$$
Déterminons ces réels.

\vspace{11cm}
\end{Exemple}

\medskip 

$\rhd$ Il reste à savoir déterminer des primitives de fonctions dont l'expression est de la forme :
$$ \frac{1}{(X-a)^m} \quad \hbox{ ou }  \quad\frac{ax+b}{(x^2+px+q)^m} \hbox{ avec } p^2-4q<0$$


\textbf{1.} Une primitive de $x \mapsto \dfrac{1}{(x-a)}$ sur $]-\infty,a[$ ou $]a, + \infty[$ est $x \mapsto \ln( \vert x-a \vert)$.

\textbf{2.} Si $n$ est un entier naturel supérieur ou égal à $2$, une primitive de 
$$x \mapsto \dfrac{1}{(x-a)^n}= (x-a)^{-n}$$ sur $]-\infty,a[$ ou $]a, + \infty[$ est 
$$x \mapsto \dfrac{(x-a)^{-n+1}}{-n+1} = \dfrac{1}{(1-n)(x-a)^{1-n}} $$

\textbf{3.} Pour déterminer une primitive d'une fonction dont l'expression est $\dfrac{1}{x^2+px+q}$ où $p^2-4q<0$, on écrit sous forme canonique le dénominateur et on sa ramène à une expression de la forme :
$$ \hbox{constante} \times \frac{1}{(\ldots)^2+1}$$
ce qui permet de déterminer une primitive à l'aide de la fonction $\arctan$. 

\begin{Exemple} Pour tout $x \in \mathbb{R}$,

\medskip
\quad \qquad $\frac{1}{x^2+x+1} =$
 
 \vspace{10cm}
% 
% 
% 
% \frac{1}{\left( x + \frac{1}{2}\right)^2- \frac{1}{4}+1}  \\
% & = \frac{1}{\left( x + \frac{1}{2}\right)^2 + \frac{3}{4}} \\
% &  = \frac{4}{3} \frac{1}{\left(\frac{2}{\sqrt{3}}\left( x + \frac{1}{2}\right)\right)^2 +1} \\
%   &  = \frac{4}{3} \frac{1}{\left(\frac{2x+1}{\sqrt{3}} \right)^2 +1} \\
%    & = \frac{4}{3} \times \frac{\sqrt{3}}{2} \times  \frac{\frac{2}{\sqrt{3}}}{\left(\frac{2x+1}{\sqrt{3}} \right)^2 +1} \\
%  & =\frac{2}{\sqrt{3}}  \times  \frac{\frac{2}{\sqrt{3}}}{\left(\frac{2x+1}{\sqrt{3}} \right)^2 +1} 
%\end{align*}
Une primitive est ainsi donnée par $x \mapsto \phantom{\frac{2}{\sqrt{3}} \arctan \left( \frac{2x+1}{\sqrt{3}} \right)\cdot}$
\end{Exemple}

\begin{ApplicationDirecte}{} Déterminer une primitive sur $\mathbb{R}$ de :
$$ x \mapsto \frac{1}{x^2+5x+7}$$
\end{ApplicationDirecte}


\textbf{4.} Pour déterminer une primitive d'une fonction dont l'expression est $\dfrac{ax+b}{x^2+px+q}$, on fait apparaitre la dérivée du dénominateur au numérateur (on détermine une primitive à l'aide de la fonction logarithme népérien), le terme restant se traitant comme au cas précédent.

\begin{Exemple} Pour tout $x \in \mathbb{R}$,
$$ \frac{3x-1}{x^2+x+1} = \; \phantom{\frac{3}{2}} \times \frac{2x+1}{x^2+x+1} - \phantom{\frac{5}{2} \times \frac{1}{x^2+x+1}}$$
Une primitive est ainsi donnée par :

\vspace{3cm}
%\begin{align*}
% x & \mapsto  \dfrac{3}{2} \ln(\vert x^2+x+1 \vert) - \frac{5}{2} \times  \frac{2}{\sqrt{3}} \arctan \left( \frac{2x+1}{\sqrt{3}} \right)\\
%& =  \dfrac{3}{2} \ln( x^2+x+1 ) -  \frac{5}{\sqrt{3}} \arctan \left( \frac{2x+1}{\sqrt{3}} \right)
%\end{align*}
\end{Exemple}

\begin{ApplicationDirecte}{} Déterminer une primitive sur $\mathbb{R}$ de :
$$ x \mapsto \frac{x+3}{x^2+5x+7}$$
\end{ApplicationDirecte}

\textbf{5.} Pour déterminer une primitive d'une fonction dont l'expression est $\dfrac{ax+b}{(x^2+px+q)^n}$ avec $n$ un entier naturel supérieur ou égal à $2$, on on fait apparaitre la dérivée du dénominateur au numérateur (on détermine une primitive à l'aide de la forme $u'u^{-n}$), le terme restant se ramenant avec un changement de variable à déterminer la primitive suivante :
$$ \int \frac{1}{(1+x^2)^n} \dx$$
Celle-ci se déterminant par intégration par parties successives en partant de $\Int{}{} \dfrac{1}{1+x^2} \dx$

\begin{Exemple} Déterminons $\Int{a}{b} \dfrac{1}{(x^2+1)^2} \dx$.
%
%Les fonctions $x \mapsto x$ et $x \mapsto  \dfrac{1}{1+x^2}$ sont de classe $\mathcal{C}^1$ sur $[a,b]$ et de dérivées respectives $x \mapsto 1$ et $x \mapsto \frac{-2x}{(1+x^2)^2}$, on a donc par intégration par parties :
%\begin{align*}
%\int_{a}^b \dfrac{1}{1+x^2} \dx & = \left[ \frac{x}{1+x^2} \right]_a^b - \int_{a}^b x \left( \frac{-2x}{(1+x^2)^2} \right) \dx \\
%& = \left[ \frac{x}{1+x^2} \right]_a^b +2 \int_{a}^b \frac{x^2}{(1+x^2)^2} \dx \\
%& = \left[ \frac{x}{1+x^2} \right]_a^b + 2 \int_{a}^b \frac{1}{1+x^2} \dx - \frac{1}{(1+x^2)^2} \dx \\
%\end{align*}
%Finalement, en isolant l'intégrale cherchée, on obtient :
%$$ \int_{a}^b \frac{1}{(1+x^2)^2} \dx = \frac{1}{2} \int_{a}^b \frac{1}{1+x^2} + \frac{1}{2}\left[ \frac{x}{1+x^2} \right]_a^b $$

\vspace{7cm}

\end{Exemple}

\newpage
\begin{Exemple} Déterminons $\Int{0}{1} \dfrac{3x-1}{(x^2+x+1)^2} \dx$.
%
%Pour tout $x \in [0,1]$,
%$$ \frac{3x-1}{(x^2+x+1)^2} = \frac{3}{2} \times \frac{2x+1}{(x^2+x+1)^2} - \frac{5}{2} \times \frac{1}{(x^2+x+1)^2} $$
%Une primitive de $x \mapsto \dfrac{2x+1}{(x^2+x+1)^2}= (2x+1)(x^2+x+1)^{-2}$ est donnée par :
%$$ x \mapsto \frac{(x^2+x+1)^{-1}}{-1} = - \frac{1}{x^2+x+1}$$
%On sait de plus que pour tout $x \in [0,1]$,
%$$  \frac{1}{(x^2+x+1)^2} = \frac{1}{\left(\left( x + \frac{1}{2}\right)^2 + \frac{3}{4}\right)^2} = \frac{16}{9}  \frac{1}{\left(\left(\frac{2}{\sqrt{3}}\left( x + \frac{1}{2}\right)\right)^2 + 1 \right)^2}$$
%Ainsi par changement de variable affine $u= \dfrac{2}{\sqrt{3}}\left( x + \frac{1}{2}\right)$, on a :
%$$ \int_{0}^1  \frac{1}{(x^2+x+1)^2} \dx = \frac{16}{9} \frac{\sqrt{3}}{2} \int_{\frac{1}{\sqrt{3}}}^{\sqrt{3}} \frac{1}{(u^2+1)^2} \textrm{d}u$$
%Cette dernière intégrale se calcule à l'aide de l'exemple précédent.

\vspace*{10cm}
\vspace*{\stretch{1}}
\end{Exemple}

%\begin{ApplicationDirecte}{} Exprimer en fonction de $\arctan$ la valeur de :
%$$ I =  \int_{0}^{1} \frac{x}{(x^2+5x+7)^2} \dx$$
%\end{ApplicationDirecte}

\section{Intégrales généralisées sur un intervalle quelconque}
\subsection{Intégrales généralisées sur \texorpdfstring{$[a,+\infty[$}{[a,+∞[}}

\begin{Definition}{}
Soit $f$ une fonction continue par morceaux sur $[a,+\infty[$ à valeurs réelles ou complexes.
\begin{itemize}
\item On dit que l'intégrale $\int_a^{+\infty} f(t)\dt$ \emph{converge} lorsque la fonction $x \mapsto \int_a^x f(t) \dt$ admet une limite finie lorsque $x$ tend vers $+\infty$. Dans ce cas, on pose :
\[ \int_a^{+\infty} f(t) \dt = \phantom{\lim_{x\to +\infty} \int_a^x f(t)\dt} \]
\item Dans le cas contraire, on dit que $\int_a^{+\infty} f(t)\dt$ \emph{diverge}.
\end{itemize}
\end{Definition}

Une intégrale de la forme précédente est dite \emph{impropre} (ou \emph{généralisée}) en $+ \infty$.

\begin{Remarque}{} 
Pour tout $x \in [a, + \infty[$, l'intégrale $\int_{a}^x f(t) dt$ existe bien car $f$ est continue par morceaux sur $[a,x]$. On peut donc bien étudier la limite de cette expression quand $x$ tend vers $+\infty$.
\end{Remarque}

\newpage
\begin{Exemple}
Étudions la convergence de l'intégrale suivante : $I = \int_{0}^{+ \infty} e^{-t} \dt$. 

%\begin{itemize}
%\item La fonction $t \mapsto e^{-t}$ est continue sur $\mathbb{R}_+$.
%\item L'intégrale est impropre en $+ \infty$.
%\item Pour tout $x \in \mathbb{R}_+$, on a :
%$$ \int_{0}^x e^{-t} \dt = -e^{-x}+1 \underset{x \rightarrow + \infty}{\rightarrow} 1$$
%\end{itemize}
%Ainsi, $I$ converge et $I = 1$.

\vspace{6cm}
\end{Exemple} 

\begin{ApplicationDirecte}{} Étudier la convergence puis donner la valeur de $\int_{0}^{+ \infty} t e^{-t} \dt$.
\end{ApplicationDirecte}

\begin{Proposition}{} Soit $f$ une fonction continue par morceaux sur $[a,+\infty[$ à valeurs réelles \emph{positives}. Les assertions suivantes sont équivalentes :
\begin{enumerate}
\item L'intégrale $\int_a^{+\infty} f(t) \dt$ converge.
\item La fonction $x \mapsto \int_{a}^x f(t) dt$ est majorée sur $[a,+\infty[$.
\end{enumerate}
\end{Proposition}

\begin{Demonstration}{} 

	\vspace{4cm}
	\vspace{\stretch{1}}
%
%La fonction $x \mapsto \int_{a}^x f(t) dt$ est croissante sur $[a,+ \infty[$ car $f$ est positive sur cet intervalle donc cette fonction admet une limite en $+ \infty$ si et seulement si elle est majorée sur $[a, + \infty[$.
\end{Demonstration}

\begin{Remarque}{} On remarquera l'analogie entre la proposition précédente et le fait qu'une série à termes positifs est convergente si et seulement si ses sommes partielles sont majorées.
\end{Remarque}

\newpage
\subsection{Intégrales généralisées sur un intervalle quelconque}
Dans la suite, $a$ et $b$ désigneront deux éléments de $\overline{\mathbb{R}} = \mathbb{R} \cup \lbrace - \infty, + \infty \rbrace$, tels que $a<b$ (on se doute bien ce que cela signifie si $a$ et/ou $b$ sont infinis) et $I$ sera un intervalle de la forme suivante : $[a,b]$ ($a$, $b$ finis), $[a,b[$ ($a$ fini), $]a,b]$ ($b$ fini) ou $]a,b[$. Les notations $a^+$ et $b^{-}$ dans le cas où $a= - \infty$ et $b = + \infty$ signifient juste $a$ et $b$.

\begin{Definition}{} Soit $f : I \rightarrow \mathbb{K}$ une fonction continue par morceaux.

\begin{itemize}
\item Si $I=[a,b[$, on dit que $\int_a^b f(t)\dt$ \emph{converge} lorsque la fonction $x \mapsto \int_a^x f(t) \dt$ admet une limite finie lorsque $x$ tend vers $b^{-}$. Dans ce cas, on pose :
\[ \int_a^b f(t) \dt = \phantom{\lim_{x\to b^{-}} \int_a^x f(t)\dt} \]
Dans le cas contraire, on dit que $\int_a^b f(t)\dt$ \emph{diverge}.
\item Si $I=]a,b]$, on dit que $\int_a^b f(t)\dt$ \emph{converge} lorsque la fonction $x \mapsto \int_x^b f(t) \dt$ admet une limite finie lorsque $x$ tend vers $a^+$. Dans ce cas, on pose :
\[ \int_a^b f(t) \dt = \phantom{\lim_{x\to a^+} \int_x^b f(t)\dt} \]
Dans le cas contraire, on dit que $\int_a^b f(t)\dt$ \emph{diverge}.
\item Si $I=]a,b[$, on dit que $\int_a^b f(t)\dt$ \emph{converge} s'il existe un réel $c \in ]a,b[$ tel que $\int_a^c f(t)\dt$ et $\int_c^b f(t)\dt$ convergent. Dans ce cas, on pose :
$$ \int_a^b f(t)\dt = \int_a^c f(t)\dt + \int_c^b f(t)\dt $$
Le choix est indépendant du réel $c$ (conséquence de la relation de Chasles).

Dans le cas contraire, on dit que $\int_a^b f(t)\dt$ \emph{diverge}.
\end{itemize}
\end{Definition}

\begin{Remarque}{}
Étudier la \emph{nature} d'une intégrale, c'est étudier sa convergence.
\end{Remarque}

\begin{Exemple} Étudions la nature de $I=\int_{0}^1 \ln(t) \dt$.

%\begin{itemize}
%\item La fonction $\ln$ est continue sur $]0,1]$.
%\item L'intégrale est impropre en $0$.
%\item Pour tout $x \in ]0,1]$,
%$$ \int_{x}^1 \ln(t) \dt = \left[ t \ln(t)-t \right]_{x}^1 = -1 -x \ln(x)+x \underset{x \rightarrow + \infty}{\rightarrow} -1$$
%d'après le théorème des croissances comparées.
%\end{itemize}
%Ainsi, $I$ converge et $I=-1$.

\newpage
%\vspace*{3cm}
\end{Exemple}

\begin{Exemple} Étudions la nature de $I=\int_{- \infty}^{+ \infty} e^{-\vert t \vert} \dt$.

	\vspace*{5cm}

\end{Exemple}
\begin{ApplicationDirecte}{} Étudier la nature de $ I= \int_{0}^1 \frac{1}{\sqrt{1-t}} \dt$.
\end{ApplicationDirecte} 

\subsection{Intégrales de références}

\begin{Proposition}{Intégrales de Riemann}
Soit $\alpha \in \mathbb{R}$.
\begin{itemize}
\item L'intégrale $\int_{1}^{+ \infty} \dfrac{\dt}{t^{\alpha}}$ converge si et seulement si $\alpha>1$.
\item L'intégrale $\int_{0}^{1} \dfrac{\dt}{t^{\alpha}}$ converge si et seulement si $\alpha<1$.
\end{itemize}
\end{Proposition} 

\begin{Remarque}{} Dans les deux cas, on peut remplacer $1$ par n'importe quel nombre strictement positif.
\end{Remarque}

\begin{center}
\emph{Bernhard Riemann (1826-1866)}

\includegraphics[scale=0.4]{Riemann}
\end{center}

\begin{Demonstration}{} Montrons le premier point.

%\begin{itemize}
%\item La fonction $t \mapsto \dfrac{1}{t^{\alpha}}$ est continue sur $[1, +  \infty[$.
%\item L'intégrale est impropre en $+ \infty$.
%\item On distingue deux cas :
%\begin{enumerate}
%\item[$\star$] Si $\alpha=1$, on a pour tout réel $x \geq 1$,
%$$ \int_{1}^x \dfrac{\dt}{t} = \ln(x) \underset{x \rightarrow + \infty}{\rightarrow} + \infty$$
%et ainsi, l'intégrale diverge.
%\item[$\star$] Si $\alpha \neq 1$, on a pour tout réel $x \geq 1$,
%$$ \int_{1}^x \dfrac{\dt}{t^{\alpha}} = \left[ \frac{1}{(1- \alpha) t^{\alpha -1}} \right]_1^x =  \frac{1}{(1- \alpha) x^{\alpha -1}} + \frac{1}{\alpha-1} $$
%Si $\alpha>1$, on a :
%$$ \lim_{x \rightarrow + \infty} \frac{1}{(1- \alpha) x^{\alpha -1}} + \frac{1}{\alpha-1} = \frac{1}{\alpha-1}$$
%et ainsi, l'intégrale converge (et l'on connait la valeur).
%
%Si $\alpha<1$, on a :
%$$ \lim_{x \rightarrow + \infty} \frac{1}{(1- \alpha) x^{\alpha -1}} + \frac{1}{\alpha-1} = + \infty$$
%et ainsi, l'intégrale diverge.
%\end{enumerate}
%\end{itemize}
%Finalement, on a bien montré que l'intégrale converge si et seulement si $\alpha>1$.

\newpage
\vspace*{5cm}
\end{Demonstration}

\begin{ApplicationDirecte}{} Montrer le deuxième point de la proposition précédente.
\end{ApplicationDirecte}

\begin{Proposition}{} L'intégrale $\int_{0}^1 \ln(t) \dt$ converge.
\end{Proposition}

\begin{Proposition}{}
Soit $\lambda \in \mathbb{R}$. L'intégrale $\int_0^{+\infty} e^{-\lambda t} \dt$ converge si et seulement si $\lambda > 0$.
\end{Proposition}

\begin{Demonstration}{} 
%\begin{itemize}
%\item La fonction $t \mapsto e^{- \lambda t}$ est continue sur $[0, +  \infty[$.
%\item L'intégrale est impropre en $+ \infty$.
%\item On distingue deux cas :
%\begin{enumerate}
%\item[$\star$] Si $\lambda=0$, on a pour tout réel $x \geq 0$,
%$$ \int_{0}^x 1 \dt = x \underset{x \rightarrow + \infty}{\rightarrow} + \infty$$
%\item[$\star$] Si $\lambda \neq 0$, on a pour tout réel $x \geq 0$,
%$$ \int_{0}^xe^{-\lambda t} \dt = \left[- \frac{e^{-\lambda t}}{\lambda} \right]_{0}^x = - \frac{e^{-\lambda x}}{\lambda} + \frac{1}{\lambda}$$
%Si $\lambda>0$, on a alors :
%$$ \lim_{x \rightarrow + \infty}  - \frac{e^{-\lambda x}}{\lambda} + \frac{1}{\lambda} = \frac{1}{\lambda}$$
%et ainsi, l'intégrale converge.
%Si $\lambda<0$, on a alors :
%$$ \lim_{x \rightarrow + \infty}  - \frac{e^{-\lambda x}}{\lambda} + \frac{1}{\lambda} = + \infty$$
%et ainsi, l'intégrale diverge.
%\end{enumerate}
%Finalement, on a bien montré que l'intégrale converge si et seulement si $\lam>0$.
%\end{itemize}

\vspace{\stretch{1}}
% \vspace*{10cm}
\end{Demonstration}
\newpage

\subsection{Lien avec l'intégrale \og usuelle \fg}

\begin{Proposition}{} Soient $a$, $b$ deux réels tels que $a < b$ et $f$ une fonction continue par morceaux sur $[a,b]$ à valeurs réelles ou complexes. Alors les intégrales de $f$ sur $]a,b]$, $[a,b[$ et $]a,b[$ sont convergentes et leurs valeurs est la valeur de l'intégrale usuelle de $f$ sur $[a,b]$.
\end{Proposition}

\begin{Demonstration}{} Montrons le résultat dans le premier cas. La fonction $f$ est continue par morceaux sur $]a,b]$ donc elle est bornée sur $[a,b]$. On a pour tout $x \in ]a,b]$,

%\begin{align*}
%\left\vert \int_{x}^b f(t) \dt - \int_{a}^b f(t) \dt \right\vert & = \left\vert \int_{a}^x f(t) \dt \right\vert \\
%& \leq (x-a) \sup_{t \in [a,b]} \vert f(t) \vert \\
%\end{align*}
%et on a :
%$$ \lim_{x \rightarrow a^{+}}  (x-a) \sup_{t \in [a,b]} \vert f(t) \vert = 0$$
%ce qui donne le résultat avec le théorème d'encadrement.

\vspace{6cm}
\end{Demonstration}

\begin{Proposition}{} Soient $a$ un réel et $f : ]a,b] \rightarrow \mathbb{K}$ une fonction continue par morceaux. Si la fonction $f$ admet une limite dans $\mathbb{K}$ en $a^{+}$ alors $\int_{a}^b f(t) \dt$ est convergente.

On dit que l'intégrale est \emph{faussement impropre} en $a$ (ou qu'il y a un \emph{faux problème} en $a$).
\end{Proposition}

\begin{Demonstration}{} La fonction $f$ est prolongeable par continuité en $a^+$ en une fonction $\tilde{f}$ continue par morceaux sur $[a,b]$. Pour tout $x \in ]a,b]$, on a alors :
$$ \int_{x}^b f(t) \dt = \int_{x}^b \tilde{f}(t) \dt \underset{t \rightarrow a^+}{\longrightarrow} \int_{a}^b \tilde{f}(t) \dt$$
d'après la proposition précédente. Ainsi, l'intégrale est convergente.
\end{Demonstration}

\begin{Exemple} Étudions la convergence de $I = \int_{0}^1 t \ln(t) \dt$.

%La fonction $t \mapsto t \ln(t)$ est continue sur $]0,1]$ et est prolongeable 
%par continuité en $0$ car :
%$$ \lim_{t \rightarrow 0} t \ln(t) = 0 $$
%d'après le théorème des croissances comparées. 
%Ainsi l'intégrale $I$ est faussement impropre en $0$ donc convergente.

%\vspace{3cm}
\end{Exemple}
\newpage

\begin{ApplicationDirecte}{} Étudier la convergence de $I = \int_{0}^{1} \dfrac{\sin(t)}{t} \dt$.
\end{ApplicationDirecte}

\begin{Remarque}[\alerte]{} Un faux problème ne peut avoir un sens qu'en une extrémité de $I$ \emph{fini} : un prolongement par continuité en $ \pm \infty$ n'a aucun sens.
\end{Remarque}


\subsection{Quelques propriétés}
Dans la suite, $a$ et $b$ sont deux éléments de $\overline{\mathbb{R}}$ dans \og le bon ordre \fg .

\begin{Proposition}{Linéarité de l'intégration} 
Soient $f,g$ deux fonctions continues par morceaux sur $]a,b]$ à valeurs dans $\mathbb{K}$ et $\lambda \in \mathbb{K}$. Si $\int_{a}^b f(t) \dt$ et $\int_{a}^b g(t) \dt$ convergent alors $\int_{a}^b \lambda f(t) + g(t) \dt$ converge et on a :
$$ \int_{a}^b ()\lambda f(t) + g(t)) \dt = \lambda \int_{a}^b f(t) \dt + \int_{a}^b g(t) \dt$$
\end{Proposition}

\begin{Proposition}{Positivité, croissance}
Soient $f,g$ deux fonctions continues par morceaux sur $]a,b]$ (avec $a<b$) à valeurs dans $\mathbb{R}$.

\begin{itemize}
\item Si $f$ est positive sur $]a,b]$ et que $\int_{a}^b f(t) \dt$ converge 
alors $\int_{a}^b f(t) \dt \geq 0$.
\item Si $f \leq g$ sur $]a,b]$ et que $\int_{a}^b f(t) \dt$ et $\int_{a}^b g(t) \dt$ convergent alors $\int_{a}^b f(t) \dt \leq \int_{a}^b g(t) \dt$.
\end{itemize}
\end{Proposition}

\begin{Proposition}{} Soient $f$ une fonctions continue par morceaux sur $]a,b]$ à valeurs dans $\mathbb{C}$. Les assertions suivantes sont équivalentes :

\begin{enumerate}
\item $\int_{a}^b f(t) \dt$ converge.
\item $\int_{a}^b \Re e(f(t)) \dt$ et $\int_{a}^b \Im m(f(t)) \dt$ convergent.
\end{enumerate}
Si ces conditions sont vérifiées, on a alors :
$$ \int_{a}^b f(t) \dt = \int_{a}^b \Re e(f(t)) \dt + i \int_{a}^b \Im m(f(t)) \dt$$
\end{Proposition}

\newpage
\begin{Remarques}{}
\begin{itemize}
\item On adapte les trois résultats précédents pour les intervalles du type $[a,b[$ et $]a,b[$.
\item Pour prouver les trois résultats précédents, il suffit d'écrire le résultat analogue sur $[x,b]$ (dans le cas \og usuel \fg) puis d'obtenir le résultat par passage à la limite quand $x$ tend vers $a^{+}$.
\end{itemize}
\end{Remarques}{}

\section{Méthodes de calcul d'intégrales généralisées}

\subsection{Utilisation d'une primitive}
Si $f$ est une fonction continue sur $]a,b]$ et à valeurs dans $\mathbb{K}$ et si $F$ est une primitive de $f$ sur $]a,b]$ alors pour tout $x \in ]a,b]$,
$$ \int_{x}^b f(t) \dt = F(b)-F(x)$$
Donc l'intégrale $\int_{a}^b f(t) \dt$ converge si et seulement si $F$ admet une limite dans $\mathbb{K}$ en $a^{+}$.

On adapte évidemment cette méthode sur les intervalles du type $[a,b[$ et $]a,b[$.
\subsection{Intégration par parties}

Il faut se méfier de la formule d'intégration par parties dans le cas des intégrales généralisées : une intégrale convergente peut s'écrire comme somme de deux termes qui divergent.

\begin{Theoreme}{} Soient $f,g$ deux fonctions de classe $\mathcal{C}^1$ sur $]a,b]$. Si la fonction $fg$ a une limite dans $\mathbb{K}$ en $a^+$ alors les intégrales 
$$ \int_{a}^{b} f'(t) g(t) \dt \quad \hbox{ et } \quad \int_{a}^{b} f(t) g'(t) \dt $$
sont de même nature. En notant :
$$ [f(t) g(t)]_{a}^{b} = f(b)g(b) - \lim_{t \rightarrow a^+} f(t)g(t)$$
On a alors en cas de convergence :
$$  \int_{a}^{b} f'(t) g(t) \dt = [f(t) g(t)]_{a}^{b} - \int_{a}^b f(t) g'(t) \dt$$
\end{Theoreme}

\begin{Remarque}{} Le résultat s'adapte très facilement sur les intervalles du type $[a,b[$ et $]a,b[$ (dans ce cas, le crochet est constitué de deux limites).
\end{Remarque}

\begin{Demonstration}{} Pour tout $x \in ]a,b]$, on a par intégration par parties \og usuelle \fg :
$$ \int_{x}^{b} f'(t) g(t) \dt = f(b)g(b)-f(x)g(x) - \int_{x}^b f(t) g'(t) \dt$$
La fonction $x \mapsto f(b)g(b)-f(x)g(x)$ admet une limite finie en $a^+$ ce qui donne le résultat.
\end{Demonstration}

\begin{Remarque}{} On peut procéder de deux manières pour utiliser la formule d'intégration par parties :

\begin{itemize}
\item Utiliser le théorème précédent : dans ce cas, il faut très précis sur la rédaction et sur les hypothèses.
\item Utiliser la formule d'intégration par parties usuelle sur un intervalle $[x,b]$ ou $[a,x]$ puis utiliser un passage à la limite.
\end{itemize}
\end{Remarque}


\begin{Exemple} Montrons que $\int_{0}^1  \dfrac{\ln(t)}{(1+t)^2} \dt$ converge.

%\begin{itemize}
%\item La fonction $t \mapsto \dfrac{\ln(t)}{(1+t)^2}$ est continue sur $]0,1]$. L'intégrale est donc impropre en $0$.
%\item Posons pour tout $t \in ]0,1]$,
%$$ u(t) = \frac{-1}{1+t} \quad \hbox{ et } v(t) = \ln(t)$$
%Alors $u$ et $v$ sont $\mathcal{C}^1$ sur $]0,1]$ et on a pour tout $t \in ]0,1]$,
%$$ u'(t) = \frac{1}{(1+t)^2} \quad \hbox{ et } v'(t) = \frac{1}{t}$$
%Pour tout $x \in ]0,1]$, on a par intégration par parties :
%
%\begin{align*}
%\int_{x}^1  \dfrac{\ln(t)}{(1+t)^2} \dt & = \left[ \frac{-\ln(t)}{1+t} \right]_{x}^1 + \int_{x}^1 \frac{1}{t(t+1)} \dt \\
%& =  \left[ \frac{-\ln(t)}{1+t} \right]_{x}^1 + \int_{x}^1 \frac{1+t-t}{t(t+1)} \dt \\
%&=  \left[ \frac{-\ln(t)}{1+t} \right]_{x}^1 + \int_{x}^1 \frac{1}{t} - \frac{1}{t+1} \dt \\
%& = \left[ \frac{-\ln(t)}{1+t} + \ln(t) - \ln(t+1) \right]_{x}^1 \\
%& = \dfrac{\ln(x)}{1+x}  - \ln(x)+ \ln(x+1) - \ln(2) \\
%& = \ln(x) \left( \frac{1}{1+x} - 1\right) + \ln(x+1)- \ln(2) \\
%& = \frac{-x\ln(x)}{1+x} + \ln(x+1)- \ln(2) \\
%\end{align*}
%et par théorème des croissances comparées, on a :
%$$ \lim_{x \rightarrow 0^+}  \frac{-x\ln(x)}{1+x} + \ln(x+1)- \ln(2) = - \ln(2)$$
%\end{itemize}
%Ainsi, l'intégrale converge et l'on a :
%$$ \int_{0}^1  \dfrac{\ln(t)}{(1+t)^2} \dt = - \ln(2)$$

\vspace{12cm}
\vspace{\stretch{1}}
\end{Exemple}

\begin{ApplicationDirecte}{} Déterminer, pour tout $\lambda>0$, la valeur de :
$$ I(\lambda) = \int_{0}^{+ \infty} \lambda t e^{-\lambda t} \dt$$
\end{ApplicationDirecte}

\subsection{Changement de variable}

\begin{Theoreme}{} Soient $\varphi : ]\alpha, \beta[ \rightarrow ]a,b[= \varphi(]\alpha,\beta[)$ une bijection strictement monotone de classe $\mathcal{C}^1$ et $f : [a,b] \rightarrow \mathbb{K}$ une fonction continue par morceaux. Alors les intégrales :
$$  \int_{\alpha}^{\beta} (f \circ \varphi)(u) \varphi'(u) du \quad \hbox{ et } \quad \int_{a}^b f(t) \dt$$
sont de même nature et en cas de convergence, on a :
$$ \int_{a}^b f(t) \dt= \int_{\alpha}^{\beta} (f \circ \varphi)(u) \vert \varphi'(u) \vert du$$
\end{Theoreme}

\begin{Remarque}{} Le terme $\vert \varphi'(u) \vert$ permet de compenser l'inversion des bornes quand $\varphi$ est décroissante.
\end{Remarque}

\begin{Demonstration}{} Prouvons le résultat quand $\varphi$ est strictement croissante. Soient $(x,y) \in ]a,b[^2$ et $(w,z) \in ]\alpha, \beta[^2$. 

En utilisant la forme de changement de variable \og usuelle \fg, on a :
$$ \int_{\varphi(w)}^{\varphi(t)} f(t) \dt = \int_{w}^{z} f (\varphi(u)) \varphi'(u) du \quad \hbox{ et } \int_{x}^{y} f(t) \dt = \int_{\varphi^{-1}(x)}^{\varphi^{-1}(y)} f (\varphi(u)) \varphi'(u) du$$
La fonction $\varphi$ est strictement croissante donc :
$$ \varphi(w) \underset{w \rightarrow \alpha^+}{\rightarrow} a^+, \; \varphi(z) \underset{z \rightarrow \beta^-}{\rightarrow} b^-, \; \varphi^{-1}(x) \underset{x \rightarrow a^+}{\rightarrow} \alpha^+, \; \varphi^{-1}(y) \underset{y \rightarrow b^-}{\rightarrow} \beta^-$$
Ainsi les deux intégrales sont simultanément convergentes et sont égales en cas de convergence.

%\vspace{6cm}
\end{Demonstration} 

%\begin{Theoreme}{} Soit $\varphi : ]\alpha, \beta[ \rightarrow ]a,b[$ une bijection strictement décroissante de classe $\mathcal{C}$ et $f : [a,b] \rightarrow \mathbb{K}$ une fonction continue par morceaux. Alors les intégrales :
%$$ \int_{\alpha}^{\beta} (f \circ \varphi)(u) \varphi'(u) du \quad \hbox{ et } \int_{a}^b f(t) \dt$$
%sont de même nature et en cas de convergence, on a : 
%$$ \int_{\alpha}^{\beta} (f \circ \varphi)(u) \varphi'(u) du =-  \int_{a}^b f(t) \dt$$
%\end{Theoreme}
%
%\begin{Demonstration}{} Similaire à la précédente mais les bornes sont échangées par décroissance de $\varphi$.
%\end{Demonstration} 

\begin{Exemple} Calculons l'intégrale $I=\int_{0}^1 \frac{1+t^2}{1+t^4} \dt$ à l'aide du changement de variable $t=e^{-u}$.

\vspace{7cm}
%\medskip
%
%\begin{itemize}
%\item La fonction $t \mapsto \frac{1+t^2}{1+t^4}$ est continue sur $[0,1]$ (son intégrale est donc convergente sur $[0,1]$ et celle-ci coïncide avec l'intégrale sur $]0,1]$).
%\item La fonction $t \mapsto e^{-t}$ est une bijection strictement décroissante de classe $\mathcal{C}^1$ de $\mathbb{R}_+$ vers $]0,1]$. Ainsi, d'après le théorème de changement de variable assure $I$ est de même nature (c'est-à-dire convergente) que l'intégrale suivante :
%$$ J = \int_{0}^{+\infty} \frac{1+e^{-2t}}{1+e^{-4t}} (-e^{-t}) \dt =- \int_{0}^{+\infty} \frac{e^t+e^{-t}}{e^{2t}+e^{-2t}} = -\int_{0}^{+\infty} \frac{\ch(t)}{\ch(2t)} \dt $$
%Or, pour tout $t \in \mathbb{R}_+$, $\ch(2t) = \ch(t)^2+ \sh(t)^2 = 1 + 2 \sh(t)^2$ et on a donc :
%\begin{align*}
%\frac{\ch(t)}{\ch(2t)} & = \frac{\ch(t)}{1 + 2 \sh(t)^2} \\
%& = \frac{1}{\sqrt{2}} \frac{\sqrt{2}\ch(t)}{1 + (\sqrt{2} \sh(t))^2} \\
%\end{align*}
%On reconnait la formule $\dfrac{U'}{1+U^2}$ donc :
%$$ J = -\left[ \frac{1}{\sqrt{2}} \arctan (\sqrt{2} t) \right]_{0}^{+\infty} =- \frac{\pi}{2\sqrt{2}}$$
%Or le changement de variable était strictement décroissant donc $I=-J = \dfrac{\pi}{2\sqrt{2}} \cdot$
%\end{itemize}
\end{Exemple}

\newpage

$\phantom{test}$

\vspace{2cm}
%\section{Critères de comparaison d'intégrales de fonctions positives}
%
%\begin{Theoreme}{} Soient $f : [a,b[ \rightarrow \mathbb{K}$ et $g : [a,b[ \rightarrow \mathbb{K}$ deux fonctions \emph{positives} et continues par morceaux sur $I$. 
%
%\begin{itemize}
%\item Supposons l'existence d'un réel $c \in [a,b[$ tel que pour tout $t \in [c,b[$, $f(t) \leq g(t)$. Alors :
%
%\begin{enumerate}
%\item La convergence de $\int_{a}^b g(t) \dt$ implique la convergence de  $\int_{a}^b f(t) \dt$. 
%\item La divergence de $\int_{a}^b f(t) \dt$ implique la divergence de  $\int_{a}^b g(t) \dt$. 
%\end{enumerate}
%\item Si $f(t) \underset{ t \rightarrow b^{-}}{=} O(g(t))$ alors la convergence de $\int_{a}^b g(t) \dt$ implique la convergence de  $\int_{a}^b f(t) \dt$. 
%\item Si $f(t) \underset{ t \rightarrow b^{-}}{=} o(g(t))$ alors la convergence de $\int_{a}^b g(t) \dt$ implique la convergence de  $\int_{a}^b f(t) \dt$. 
%\item Si $f(t) \underset{ t \rightarrow b^{-}}{\sim}g(t)$ alors $\int_{a}^b f(t) \dt$ et $\int_{a}^b g(t) \dt$ sont de même nature.
%\end{itemize}
%\end{Theoreme}
%
%\begin{Demonstration}{} A taper en modifiant preuves pour intégrabilité !!
%\end{Demonstration}
%
%\begin{Exemple} Déterminons la nature de $I= \int_{0}^{+ \infty} \frac{e^{-t}}{\sqrt{t+1}} \dt$.
%
%\medskip
%
%\begin{itemize}
%\item La fonction $t \mapsto \dfrac{e^{-t}}{\sqrt{t+1}}$ est continue sur $[0, + \infty[$. 
%\item L'intégrale est impropre en $+ \infty$.
%\item Pour tout $t \in \mathbb{R}_+$,
%$$ 0 \leq \frac{e^{-t}}{\sqrt{t+1}} \leq e^{-t} $$
%Or l'intégrale $\int_{0}^{+ \infty} e^{-t} \dt$ converge (intégrale de référence) donc par comparaison de fonctions positives, $I$ converge.
%\end{itemize}
%\end{Exemple}
%
%\begin{Exemple} Déterminons la nature de $I= \int_{1}^{+ \infty} t+2 - \sqrt{t^2+4t+1} \dt$.
%
%\medskip
%
%
%\begin{itemize}
%\item La fonction $t \mapsto t+2 - \sqrt{t^2+4t+1} $ est continue sur $[1, + \infty[$. 
%\item L'intégrale est impropre en $+ \infty$.
%\item Si $t \rightarrow + \infty$, on a :
%$$ t+2 - \sqrt{t^2+4t+1}  = t+2 - t \sqrt{1 + \frac{4}{t} + \frac{1}{t^2}}$$
%Or si $t \rightarrow + \infty$, $\frac{4}{t} + \frac{1}{t^2} \rightarrow 0$ donc :
%\begin{align*}
%t+2 - \sqrt{t^2+4t+1}  & = t+2 - t  \left( 1 + \frac{1}{2} \left(\frac{4}{t} + \frac{1}{t^2} \right) - \frac{1}{8} \left(\frac{4}{t} + \frac{1}{t^2} \right) + o \left( \frac{1}{t}^2 \right) \right) \\
%& = t+2 - t  \left( 1 + \frac{2}{t} + \frac{1}{2t^2} - \frac{2}{t^2} + o \left( \frac{1}{t^2} \right) \right) \\
%& =  \frac{3}{2t} + o \left( \frac{1}{t} \right) \\
%\end{align*}
%Ainsi :
%$$ t+2 - \sqrt{t^2+4t+1} \underset{ + \infty}{\sim} \frac{3}{2t} $$
%Les fonctions sont continues et positives sur $[1, + \infty[$ (petite inéquation simple pour la première) et l'intégrale $\int_{1}^{+ \infty} \frac{3}{2t} \dt$ diverge (intégrale de Riemann) donc par comparaison de fonctions positives, $I$ diverge.
%\end{itemize}
%\end{Exemple}

\section{Absolue convergence et intégrabilité}

\subsection{Absolue convergence}

Dans la suite, $I$ est un intervalle de $\mathbb{R}$ d'extrémités $a$ et $b$ (éléments de $\overline{\mathbb{R}}$).

\begin{Definition}{} Soit $f : I \rightarrow \mathbb{K}$ une fonction continue par morceaux. On dit que l'intégrale $\int_{a}^b f(x) \dx$ est \emph{absolument convergente} si $\int_{a}^b \vert f(x) \vert \dx$ est convergente.
\end{Definition}

\begin{Proposition}{} Soit $f : I \rightarrow \mathbb{K}$ une fonction continue par morceaux. Si l'intégrale $\int_{a}^b f(x) \dx$ est absolument convergente alors elle est convergente et on a :
$$ \left\vert \int_{a}^b  f(x)  \dx \right\vert \leq \int_{a}^b \vert f(x) \vert \dx $$
\end{Proposition}

\begin{Demonstration}{} Faisons la preuve dans le cas où $I = ]a,b]$. 

%Posons $h = \Re e(f)$, $h^+ = \max \lbrace 0,h \rbrace$ et $h^{-}= \max \lbrace 0, -h \rbrace$. On a $h=h^+ - h^{-}$ et $\vert h = \vert h^+ + h^{-}$ et ainsi :
%$$ h^+ = \frac{1}{2} ( \vert h \vert + h) \quad \hbox{ et } \frac{1}{2} ( \vert h \vert -h ) $$
%Les fonctions $h^+$ et $h^{-}$ sont ainsi continues par morceaux sur $I$ et on a :
%$$ 0 \leq h^+ \leq \vert h \vert  = \vert \Re (f) \vert \leq \sqrt{Re (f)^2 + \Im m(f)^2} = \vert f \vert $$
%et de même :
%$$ 0 \leq h^{-} \leq \vert h \vert \leq \vert f \vert $$
%Pour tout $x \in ]a,b]$ et par croissance de l'intégrale, on a :
%$$ \int_{x}^b h^+(t) \dt \leq \int_{x}^b \vert f(t) \vert \dt $$
%Or la fonction $t \mapsto \vert f(t) \vert$ est positive et sont intégrale sur $]a,b]$ est convergente par hypothèse donc la fonction :
%$$ x \mapsto \int_{x}^b \vert f(t) \vert \dt$$
%est majorée sur $I$. Ainsi la fonction :
%$$ x \mapsto \int_{x}^b h^+(t) \dt$$ 
%est aussi majorée sur $I$ et comme $t \mapsto h^+ (t) $ est positive, on en déduite que $\int_{a}^b h^+(t) \dt$ converge. De même, on montre que   $\int_{a}^b h^{-}(t) \dt$ converge. Sachant que $h=h^+-h^{-}$, on en déduit que $\int_{a}^b h(t) \dt$. On procède de la même manière pour la partie imaginaire de $f$ ce qui implique que  $\int_{a}^b f(t) \dt$ converge.
%
%\medskip
%
%Pour obtenir l'inégalité, il suffit d'utiliser l'inégalité triangulaire \og usuelle \fg sur les segments $[x,b]$ puis de passer à la limite quand $x$ tend vers $a^+$.


\vspace{11cm}
\end{Demonstration}

\begin{Definition}{} Soit $f : I \rightarrow \mathbb{K}$ une fonction continue par morceaux. On dit que $f$ est \emph{intégrable} sur $I$ si l'intégrale $\int_{a}^b f(x) \dx$ est absolument convergente. Dans ce cas, cette intégrale converge et on pourra noter la valeur de cette intégrale par l'une des notations suivantes :
$$ \int_{a}^b f(x) \dx, \quad \int_{I} f(x) \dx, \quad \int_I f$$
\end{Definition}

\begin{Remarques}{}
\begin{itemize}
\item Si une fonction est intégrable sur $I$ alors $\int_{a}^b f(x) \dx$ converge. La réciproque est \emph{fausse} : il existe des fonctions dont l'intégrale converge mais que ne sont pas intégrables sur l'intervalle considéré : voir TD.
\item Si une fonction continue par morceaux sur $I$ est à valeurs réelles et de signe \emph{constant} alors :
$$ f \hbox{ est intégrable sur I} \Longleftrightarrow \int_{a}^b f(x) \dx \hbox{ converge}$$
\item Si une fonction $f$ définie sur $I$ est à valeurs complexes, alors :
\begin{align*}
	&\left(\int_{a}^b f(x) \dx \text{ converge absolument}\right)\\
	&\phantom{\int_{a}^b f(x) \dx}\Longleftrightarrow \left(\int_{a}^b \Re e(f)(x) \dx \et  
	\int_{a}^b \Im m(f)(x) \dx \text{ convergent absolument}\right)
\end{align*}
\end{itemize}
\end{Remarques}

\begin{Proposition}{} Si une fonction $f$ est intégrable sur un intervalle $I$, elle l'est sur tout sous-intervalle de $I$.
\end{Proposition}

\begin{Proposition}{} Si $f$ est continue par morceaux sur $I=[a,b]$ ($a$, $b$ deux réels tels que $a <b$)  alors $f$ est intégrable sur $[a,b]$.
\end{Proposition}

\begin{Demonstration}{}
La fonction $\vert f \vert$ est continue par morceaux sur $[a,b]$ donc l'intégrale existe (au sens usuel) donc converge.
\end{Demonstration}

\newpage
\subsection{Critères de comparaison}

\begin{Theoreme}{} Soient $f : [a,b[ \rightarrow \mathbb{K}$ et $g : [a,b[ \rightarrow \mathbb{K}$ deux fonctions continues par morceaux.

\begin{itemize}
\item Si $\vert f \vert \leq \vert g \vert$ sur $[a,b[$ et si $g$ est intégrable sur $[a,b[$ alors $f$ est intégrable sur $[a,b[$.
\item Si $f(t) \underset{ t \rightarrow b^{-}}{=} O(g(t))  $ et si $g$ est intégrable sur $[a,b[$ alors $f$ est intégrable sur $[a,b[$.
\item Si $f(t) \underset{ t \rightarrow b^{-}}{=} o(g(t))  $ et si $g$ est intégrable sur $[a,b[$ alors $f$ est intégrable sur $[a,b[$.
\item Si $f(t) \underset{ t \rightarrow b^{-}}{\sim}g(t)$ alors $f$ est intégrable sur $[a,b[$ si et seulement si $g$ est intégrable sur $[a,b[$.
\end{itemize}
\end{Theoreme}

\begin{Remarque}{} On adapte les énoncés dans le cas d'un intervalle de la forme $]a,b]$.
\end{Remarque}

\begin{Demonstration}{}
%
%\begin{itemize}
%\item Pour tout $x \in [a,b[$ et par croissance de l'intégrale, on a :
%$$ \int_{a}^x \vert f(t) \vert \dt \leq \int_{a}^x \vert g(t) \vert \dt$$
%Or la fonction $t \mapsto \vert g(t) \vert$ est positive et $g$ est intégrable sur $[a,b[$ donc la fonction
%$$ x \mapsto \int_{a}^x \vert g(t) \vert \dt$$
%est majorée sur $[a,b[$. Ainsi, la fonction 
%$$  x \mapsto \int_{a}^x \vert f(t) \vert \dt$$
%est majorée sur $[a,b[$ et la fonction $t \mapsto \vert f(t) \vert$ est positive sur $[a,b[$ donc $f$ est intégrable sur $[a,b[$.
%\item Supposons que $f(t) \underset{ t \rightarrow b^{-}}{=} O(g(t))$. Il existe alors un réel $K>0$ et $c \in [a,b[$ tel que pour tout $t \in [c,b[$, 
%$$ \vert f(t) \vert \leq K \vert g(t) \vert $$
%La fonction $g$ est intégrable sur $[a,b[$ donc elle est intégrable sur $[a,c[$ (JUSTIFIER AVANT) donc d'après le cas précédent, la fonction $f$ est intégrable sur $[c,b[$. De plus, $f$ est continue sur $[a,c]$ donc elle est intégrable sur cette intervalle (JUSTIFIER AVANT) donc par la relation de Chasles, $f$ est intégrable sur $[a,b[$.
%\item Si $f(t) \underset{ t \rightarrow b^{-}}{=} o(g(t))$ alors $f(t) \underset{ t \rightarrow b^{-}}{=} O(g(t))$ et on sa ramène au cas précédent.
%\item Si $f(t) \underset{ t \rightarrow b^{-}}{\sim}g(t)$ alors $f(t) \underset{ t \rightarrow b^{-}}{=} O(g(t))$ et $g(t) \underset{ t \rightarrow b^{-}}{=} (f(t))$ donc on sa ramène au deuxième cas.
%\end{itemize}

\vspace{10cm}
\vspace{\stretch{1}}

%$\phantom{test}$
%
%\vspace{5cm}
\end{Demonstration}
\newpage

\begin{Exemple} Montrons que $x \mapsto \dfrac{\cos(x)}{x^2}$ est intégrable sur $[1,+ \infty[$.

\vspace{4cm}
\vspace{\stretch{1}}
\end{Exemple}

\begin{ApplicationDirecte}{} Montrer que $x \mapsto \dfrac{1}{x} \sin \left( \frac{1}{x} \right)$ est intégrable sur $[1,+ \infty[$.
\end{ApplicationDirecte} 

\textbf{Remarque importante dans la pratique :} Si la fonction $f$ est positive sur $[a,b[$, alors $f$ est intégrable si et seulement si son intégrale sur $[a,b[$ converge ce qui donne le critère de comparaison suivant :

\begin{Theoreme}{Critère de comparaison} Soient $f$ et $g$ deux fonctions continues par morceaux sur $[a,b[$ à \emph{valeurs positives}.

\begin{itemize}
\item Si pour tout $t \in [a,b[$, $f(t) \leq g(t)$ alors :

\begin{enumerate}
\item Si $ \int_{a}^{b} g(t) \dt$ converge alors $ \int_{a}^{b} f(t) \dt$ converge.
\item Si $ \int_{a}^{b} f(t) \dt$ diverge alors $ \int_{a}^{b} g(t) \dt$ diverge.
\end{enumerate}
\item Si $f(t) \underset{ t \rightarrow b^{-}}{=} O(g(t))$ et si $ \int_{a}^{b} g(t) \dt$ converge alors $ \int_{a}^{b} f(t) \dt$ converge.
\item Si $f(t) \underset{ t \rightarrow b^{-}}{=} o(g(t))  $ et si $ \int_{a}^{b} g(t) \dt$ converge alors $ \int_{a}^{b} f(t) \dt$ converge.
\item Si $f(t) \underset{ t \rightarrow b^{-}}{\sim}g(t)$ alors $\int_{a}^{b} f(t) \dt$ et $\int_{a}^{b} g(t) \dt$ sont de même nature.
\end{itemize}
\end{Theoreme}

\begin{Remarques}{}
\begin{itemize}
\item Pour la première propriété, il suffit que $f \leq g$ sur un intervalle de la forme $[c,b[$ (sur $[a,c]$ la fonction est continue par morceaux donc son intégrale sur $[a,c]$ converge).
\item Si $f$ est continue par morceaux sur $[a,b[$ et positive alors :
$$ x \mapsto \int_{a}^x f(t) \dt$$
est croissante donc si l'intégrale $\int_{a}^{b} f(t) \dt$ diverge cela signifie que :
$$ \lim_{x \rightarrow b} \int_{a}^x f(t) \dt = + \infty$$
\end{itemize}
\end{Remarques}{}

\subsection{Des exemples}

\textbf{Exemple 1 : Fraction rationnelle}

Étudions la nature de $\int_{0}^{+ \infty} \frac{2x}{x^2+x+1} \dx$.

\vspace{5cm}

\begin{ApplicationDirecte}{} Étudier la nature de $\int_{0}^{+ \infty} \frac{x}{x^3+x^2+1} \dx$.
\end{ApplicationDirecte}

\textbf{Exemple 2 : Intégrales de Bertrand}

Posons pour tout $(\alpha, \beta) \in \mathbb{R}^2$, 
$$ I(\alpha, \beta) = \int_{2}^{+ \infty} \frac{\dx}{x^{\alpha} \ln(x)^{\beta}}$$

\begin{itemize}
\item La fonction $f: x \mapsto \frac{1}{x^{\alpha} \ln(x)^{\beta}}$ est continue sur $[2, + \infty[$.
\item Si $\alpha>1$, on montre que $f(x)=o \left( \frac{1}{x^{\gamma}} \right)$ où $1 < \gamma < \alpha$ et on montre que l'intégrale converge par critère de comparaison.
\item Si $\alpha<1$, on montre que $xf(x) \geq 1$ à partir d'un certain rang en calculant la limite en $+ \infty$ et on montre que l'intégrale diverge par critère de comparaison.
\item Si $\alpha=1$, l'intégrale $I(1, \beta)$ converge si et seulement si $\beta>1$. On revient pour cela à la définition en calculant \og l'intégrale partielle \fg .
\end{itemize}

\newpage
Étudions la nature de $I=\int_{1}^{+ \infty} \frac{\ln(t)}{1+t^2} \dt$.

\vspace{5cm}

Étudions la nature de $J=\int_{0}^{+ \infty} \frac{\ln(t)}{1+t^2} \dt$.

\vspace{5cm}

Étudions la nature de $K=\int_{1}^{+ \infty} \frac{1}{t \ln(t)^3} \dt$.

\vspace{5cm}

Étudions la nature de $L=\int_{1}^{+ \infty} \frac{\ln(t)}{\sqrt{t}} \dt$.

\vspace{5cm}

\begin{Remarque}[\alerte]{} La méthode précédente fonctionne quand l'intégrale est impropre en $+ \infty$.
\end{Remarque}

\begin{ApplicationDirecte}{} Donner la nature de $\int_{0}^{+ \infty} \frac{\ln(t)}{t^3+t^2+1} \dt$.
\end{ApplicationDirecte}

%\medskip
%
%\textbf{Exemple 3 : Règle du \og $t^{\alpha}f(t)$ \fg }
%
%Parfois, étudier la limite de $t^{\alpha}f(t)$ quand $t \rightarrow + \infty$ où $t \rightarrow 0$ avec un bon choix de $\alpha$ permet de montrer que l'intégrale de $f$ converge ou diverge sur l'intervalle considéré.
%
%\medskip
%
%Étudions la nature de $\int_{0}^{+ \infty} \frac{\cos(t)}{1+t^2} \dt$.
%
%\vspace{5cm}
%
%Étudions la nature de $\int_{1}^{+ \infty} \frac{1}{\ln(1+t)} \dt$.
%
%\vspace{5cm}
%
%
%\begin{ApplicationDirecte}{} Étudier la nature de $\int_{0}^{+ \infty} P(t) e^{-t^2} \dt$ où $P$ est une fonction polynômiale non nulle.
%\end{ApplicationDirecte}

\textbf{Exemple 3 : avec une exponentielle ...}

Étudions la nature de $\int_{0}^{+ \infty} P(t) e^{-t^2} \dt$ où $P$ est une fonction polynomiale non nulle.

\vspace{6cm}


%\textbf{Exemple 4 : Intégrales de Riemann translatées}
%
%Voir TD.

\subsection{Quelques propriétés}

\begin{Theoreme}{} L'ensemble des fonctions continues par morceaux intégrables 
	sur un intervalle $I$ et à valeurs dans $\mathbb{K}$ est un $\mathbb{K}$-espace vectoriel.
\end{Theoreme}

\begin{Proposition}{} Soit $f$ une fonction \emph{continue} et intégrable sur un intervalle $I$. Alors :
$$ \int_{I} \vert f(t) \vert \dt=0 \Longrightarrow f \hbox{ est nulle sur } I $$
\end{Proposition}

\section{Suites et séries de fonctions intégrables}

\begin{Theoreme}{convergence dominée}
Soit $(f_n)_{n \geq 0}$ une suite de fonctions continues par morceaux sur $I$ 
et à valeurs dans $\mathbb{K}$. Si :
\begin{enumerate}
\item La suite $(f_n)_{n \geq 0}$ converge simplement vers une fonction $f$ continue par morceaux sur $I$.
\item Il existe une fonction $\varphi : I \rightarrow \mathbb{K}$, continue par morceaux et intégrable sur $I$, telle que pour tout $n \geq 0$, $\vert f_n \vert \leq \varphi$.
\end{enumerate}
Alors :
\begin{itemize}
\item Pour tout $n \geq 0$, $f_n$ est intégrable sur $I$.
\item $f$ est intégrable sur $I$.
\item $\int_{I} f_n(t) \dt \underset{n \rightarrow + \infty}{ \longrightarrow} \int_{I} f(t) \dt$
\end{itemize}
\end{Theoreme}

\begin{Remarque}{} Le fait que chaque fonction $f_n$ soit intégrable provient du fait que $\vert f_n \vert \leq \varphi$ et que $\varphi$ soit intégrable.
\end{Remarque}

\begin{Exemple} Déterminons $\lim_{n \rightarrow + \infty} \int_{0}^{+ \infty} \frac{e^{- \frac{x}{n}}}{1+x^2} \dx$.

%\medskip
%
%Posons pour tout $n \geq 1$, $f_n$ la fonction définie sur $\mathbb{R}_+$ par :
%$$ \forall x \in \mathbb{R}_+, \; f_n(x) = \frac{e^{- \frac{x}{n}}}{1+x^2}$$
%
%\begin{itemize}
%\item Pour tout $n \geq 1$, $f_n$ est continue sur $\mathbb{R}_+$.
%\item Pour tout $x \in \mathbb{R}$,
%$$ -\frac{x}{n} \underset{n \rightarrow + \infty}{\longrightarrow} 0$$
%et donc par continuité de l'exponentielle, on en déduit que :
%$$ f_n(x) \underset{n \rightarrow + \infty}{\longrightarrow} \frac{1}{1+x^2}$$
%Ainsi, $(f_n)_{n \geq 1}$ converge simplement vers $f$ sur $\mathbb{R}_+$ où pour tout $x \in \mathbb{R}_+$,
%$$ f(x) = \frac{1}{1+x^2}$$
%La fonction $f$ est continue sur $\mathbb{R}_+$.
%\item Pour tout $n \geq 1$ et tout $x \in \mathbb{R}_+$, on a :
%$$ \vert f_n(x) \vert \leq \frac{1}{1+x^2} = f(x)$$
%La fonction $f$ est intégrable sur $\mathbb{R}^+$ car pour tout $A>0$,
%$$ \int_{0}^{A} \vert f(x) \vert \dx  = \arctan(A) \underset{n \rightarrow + \infty}{\longrightarrow} \frac{\pi}{2}$$
%\end{itemize}
%Ainsi, par théorème de convergence dominée, on en déduit que pour tout $n \geq 1$, $f_n$ est intégrable sur $\mathbb{R}_+$, et que :
%$$ \lim_{n \rightarrow + \infty} \int_{0}^{+ \infty} f_n(x) \dx = \int_{0}^{+ \infty} f(x) \dx = \frac{\pi}{2}$$

\vspace{11.5cm}
\end{Exemple}


\begin{Exemple} Déterminons $\lim_{n \rightarrow + \infty} \int_{0}^{\frac{\pi}{4}} \tan(t)^n \dt$.

%Posons pour tout $n \geq 1$, $f_n$ la fonction définie sur $[0, \frac{\pi}{4}]$ par :
%$$ \forall x \in [0, \frac{\pi}{4}], \; f_n(x) = \tan(x)^n$$
%
%\begin{itemize}
%\item Pour tout $n \geq 1$, $f_n$ est continue sur $[0, \frac{\pi}{4}]$.
%\item Pour tout $x \in [0, \frac{\pi}{4}[$, $\tan(x) \in [0,1[$ donc :
%$$ f_n(x) \underset{n \rightarrow + \infty}{\longrightarrow}0$$
%et si $x= \frac{\pi}{4}$,
%$$ f_n(\frac{\pi}{4})  \underset{n \rightarrow + \infty}{\longrightarrow}1$$
%Ainsi, $(f_n)_{n \geq 1}$ converge simplement vers $f$ sur $[0, \frac{\pi}{4}]$ où pour tout $x \in  [0, \frac{\pi}{4}]$, $f(x)=0$ si $x< \frac{\pi}{4}$ et $f(\frac{\pi}{4})=1$. La fonction $f$ est continue par morceaux sur $[0, \frac{\pi}{4}]$.
%\item Pour tout $n \geq 1$ et tout $x \in [0, \frac{\pi}{4}]$, on a :
%$$ \vert f_n(x) \vert \leq 1$$
%et $x \mapsto 1$ est intégrable sur $[0, \frac{\pi}{4}]$.
%\end{itemize}
%Ainsi, par théorème de convergence dominée, on en déduit que pour tout $n \geq 1$, $f_n$ est intégrable sur $\mathbb{R}_+$, que $f$ aussi (même si c'est clair) et que :
%$$\lim_{n \rightarrow + \infty} \int_{0}^{\frac{\pi}{4}} f_n(x) \dx = \int_{0}^{\frac{\pi}{4}} f(x) \dx = 0$$
\end{Exemple}

\newpage

\vspace*{4cm}

\begin{Remarque}{} Dans l'exemple précédent, il n'y a pas convergence uniforme (limite simple non continue sur $[0, \pi/4]$) donc on ne peut pas appliquer le théorème d'interversion limite-intégrale.
\end{Remarque}

%\begin{Exemple} Déterminer $\lim_{n \rightarrow + \infty} \int_{1}^n \dfrac{1}{x} \left(1 - \frac{x}{n} \right)^n \dx$.
%
%Préciser qu'il faut poser $f_n$ selon l'intervalle...
%\end{Exemple}

\begin{Theoreme}{Intégration terme à terme}
Soit $\Sum{n \geq 0}{} f_n$ une série de fonctions définies sur $I$. Supposons que :
\begin{enumerate}
\item Pour tout entier $n \geq 0$, $f_n$ est continue par morceaux et intégrable sur $I$.
\item $\Sum{n \geq 0}{} f_n$ converge simplement sur $I$ vers une fonction somme $S=\Sum{n=0}{+\infty} f_n$ continue par morceaux sur $I$.
\item La série numérique $ \Sum{n \geq 0}{} \int_{I} \vert f_n(t) \vert \dt$ converge.
\end{enumerate}
Alors :
\begin{itemize}
\item $S$ est intégrable sur $I$.
\item $\int_{I} S(t) \dt =  \Sum{n=0}{+\infty} \int_{I} f_n(t) \dt$.
\end{itemize}
\end{Theoreme}

\begin{Exemple} Pour tout entier naturel $n \geq 1$, on définit la fonction $f_n : \mathbb{R}_+ \rightarrow \mathbb{R}$ par :
$$ \forall x \in \mathbb{R}_+, \; f_n(x)=xe^{-nx}$$

Utilisons le théorème d'intégration terme à terme pour obtenir une jolie égalité :
%
%\begin{enumerate}
%\item $f_n$ continue et intégrable sur $\mathbb{R}_+$.
%\item $S(x)= \dfrac{x}{e^{x}-1}$ si $x \neq 0$ et $0$ si $x=0$ donc continue par morceaux sur $\mathbb{R}_+$.
%\item l'intégrale de la valeur absolue de $f_n$ est égal à simplement l'intégrale qui donne $\dfrac{1}{n^2}$ (intégration par parties). Donc la série des intégrales des valeurs absolue converge ...
%\item Conclusion avec l'intégrale de $S$ égale à zeta(2).
%\end{enumerate}

\newpage

\vspace*{5cm}
\end{Exemple}


\section{Comparaison série-intégrale}

\begin{Theoreme}{} Soit $f$ une fonction continue par morceaux sur $[0,+ \infty[$, positive et décroissante. Les assertions suivantes sont équivalentes :

\begin{enumerate}
\item La série $\Sum{n \geq 0}{} f(n)$ converge.
\item $f$ est intégrable sur $[0, + \infty[$
\end{enumerate}
\end{Theoreme}

\begin{Remarque}{} Le résultat s'adapte facilement si $f$ est définie sur $[m, + \infty[$ avec $m$ entier.
\end{Remarque} 

\begin{Demonstration}{}

\newpage

\vspace*{8cm}
\end{Demonstration}

\begin{ApplicationDirecte}{} Montrer que $\Sum{n \geq 2}{} \dfrac{1}{n \sqrt{\ln(n)}}$ diverge.
\end{ApplicationDirecte}
\end{document}



