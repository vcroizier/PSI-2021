\documentclass[a4paper,10pt]{report}
\usepackage{cours}
\newcommand{\pmatrice}[1]{\begin{pmatrix}#1\end{pmatrix}}

\begin{document}
\everymath{\displaystyle}


\begin{center}
 \shadowbox{{\huge TD 9 : Réduction}}
\end{center}

\bigskip

\noindent Dans la suite, $\mathbb{K} = \mathbb{R}$ ou $\mathbb{C}$ et $n \in \mathbb{N}^*$.

\medskip

\begin{Exercice}{} Soit $A$ la matrice de $\mathcal{M}_n(\mathbb{R})$ avec des coefficients égaux à $1$, sauf ceux de la diagonale valant $0$.

\begin{enumerate}
%\item La matrice est-elle diagonalisable ?
\item Calculer $(A+I_n)^2$.
\item Montrer que si $P$ est un polynôme annulateur de $A$ alors toute valeur propre de $A$ est racine de $P$.
\item La matrice $A$est-elle diagonalisable ? Déterminer ses valeurs propres de $A$. 
\end{enumerate}
\end{Exercice}

\corr 
\begin{enumerate}
%\item $A$ est symétrique réelle.
\item Par simple calcul, $(A+I_n)^2 = n (A+I_n)$.
\item Preuve du cours.
\item D'après la question $1$, sachant que $A$ et $I_n$ commutent :
$$ A^2 +2A+I_n -nA-nI_n=0_n$$
ou encore $P(A)=0_n$ où :
$$ P(X)=X^2+(2-n)X+(1-n) = (X+1)(X+1-n)$$
Sachant que $n \geq 1$, $P$ est un polynôme annulateur de $A$ scindé à racines simples donc $A$ est diagonalisable. Les valeurs propres possibles de $A$ sont $-1$ et $n-1$. $A$ est diagonalisable donc admet au moins une valeur propre et si $A$ admettait une unique valeur propre $\lambda$, sachant qu'elle est diagonalisable, elle serait semblable à $\lambda I_n$ donc égale à $\lambda I_n$. Ainsi, $A$ admet au moins et au plus deux valeurs propres donc les valeurs propres de $A$ sont $-1$ et $n-1$.
\end{enumerate}

\begin{Exercice}{} Soient $a \in \mathbb{C}$ et $A$ la matrice de $\mathcal{M}_n(\mathbb{C})$ définie par $A=( a^{i+j-2})_{1 \leq i,j \leq n}$.

\begin{enumerate}
%\item Dans le cas où $a \in \mathbb{R}$, montrer que $A$ est diagonalisable.
\item Donner une valeur propre de $a$.
\item Déterminer une condition nécessaire sur $a$ pour que $A$ soit diagonalisable.
\end{enumerate}
\end{Exercice}

\corr 
\begin{enumerate}
%\item $A$ est symétrique réelle.
\item La matrice est de rang $1$ car le coefficient $a_{1,1}$ est non nul et que si l'on note $C_1$, $\ldots$, $C_n$, les colonnes de $A$, on a $C_2= aC_1$, $C_3=a^2 C_1$, $\ldots$, $C_n=a^{n-1} C_1$. Le noyau de $A$ est donc de dimension $n-1$ et ainsi $0$ est valeur propre de $A$.
\item Une matrice de rang $1$ est diagonalisable si et seulement si sa trace est non nulle (voir l'exercice correspondant).
\end{enumerate}

\begin{Exercice}{} Soient $A \in \mathcal{M}_n(\mathbb{R})$ et $f_A : \mathcal{M}_n(\mathbb{R}) \rightarrow \mathcal{M}_n(\mathbb{R})$ définie $f_A(M)=AM$.

\begin{enumerate}
\item Montrer que $f_A$ est un endomorphisme de $\mathcal{M}_n(\mathbb{R})$.
\item Montrer que si $A^2=A$ alors $f_A$ est un projecteur.
\item Montrer que $A$ est diagonalisable si et seulement $f_A$ l'est.
\end{enumerate}
\end{Exercice}

\corr 

\begin{enumerate}
\item Pour tout $M \in \mathcal{M}_n(\mathbb{R})$, $AM \in \mathcal{M}_n(\mathbb{R})$ donc $f(M) \in \mathbb{M}_n(\mathbb{R})$. Soient $\lambda \in \mathbb{R}$ et $(M,N) \in \mathcal{M}_n(\mathbb{R})$. Alors :
\begin{align*}
f(\lambda M +N) & = A(\lambda M +N) \\
& = \lambda AM + AN \\
& = \lambda f(M)+f(N)
\end{align*}
Ainsi, $f \in \mathcal{L}(\mathcal{M}_n(\mathbb{R}))$.
\item Soit $M \in \mathcal{M}_n(\mathbb{R}))$. Alors :
\begin{align*}
f^2(M) & = f(f(M)) \\
& = Af(M) \\
& = A(AM) \\
& = A^2M \\
& = AM \\
& = f(M)
\end{align*}
Ainsi, $f^2=f$ donc $f$ est un projecteur.
\item Soit $P \in \mathbb{R}[X]$. Avec le même procédé que dans la question précédente, on montre que pour tout entier $k \geq 0$ et pour tout $M \in \mathcal{M}_n(\mathbb{R})$,
$$ f^k(M) = A^k M$$
et on en déduit que $P(f)(M)= P(A) M$.
\begin{itemize}
\item Si $A$ est diagonalisable, il existe un polynôme $P \in \mathbb{R}[X]$ scindé à racines simples tel que $P(A)=0_n$. On en déduit que pour tout $M \in \mathcal{M}_n(\mathbb{R})$, $P(f)(M)= 0_n$ et ainsi $P(f)$ est l'endomorphisme nul. Finalement, $P$ est un polynôme annulateur de $f$, scindé à racines simples donc $f$ est diagonalisable.
\item Si $f$ est diagonalisable, il existe un polynôme $P \in \mathbb{R}[X]$ scindé à racines simples tel que $P(f)$ est l'endomorphisme nul. On en déduit que pour tout $M \in \mathcal{M}_n(\mathbb{R})$, $P(f)(M)= P(A)M=0_n$. En particulier pour $M=I_n$, on en déduit que $P(A)=0_n$. Finalement, $P$ est un polynôme annulateur de $A$, scindé à racines simples donc $A$ est diagonalisable.
\end{itemize}
Ainsi, $A$ est diagonalisable si et seulement $f_A$ l'est.
\end{enumerate}

\begin{Exercice}{}Soit $A= \begin{pmatrix}
1 & 2 & 0 \\
0 & 3 & 0 \\
4 & 0 & -1 \\
\end{pmatrix}\cdot$
\begin{enumerate}
\item Montrer que $A$ est diagonalisable et donner ses valeurs propres.
\item Soit $D$ la matrice diagonale portant les valeurs propres de $A$. Montrer que si une matrice $M \in \mathcal{M}_3(\mathbb{R})$ commute avec $D$ alors elle est diagonale.
\item Soit $P(X)=X^7+ 4X^3+1$. Trouver toutes les matrices $M$ de $\mathcal{M}_3(\mathbb{R})$ telles que $P(M)=A$.
\end{enumerate}
\end{Exercice}

\corr 

\begin{enumerate}
\item Par simple calcul, on a :
$$ \chi_A(X)=(X+1)(X-1)(X-3)$$
Ainsi $A$ a trois valeurs propres distinctes et appartient à $\mathcal{M}_3(\mathbb{R})$ donc $A$ est diagonalisable. Même si ce n'est pas demandé, donnons les sous-espaces propres :
$$ E_{-1}(A) = \textrm{Vect} \left( \begin{pmatrix}
0 \\
0 \\
1
\end{pmatrix} \right), \; E_{1}(A) = \textrm{Vect} \left( \begin{pmatrix}
1 \\
0 \\
2
\end{pmatrix} \right) \hbox{ et } E_{3}(A) = \textrm{Vect} \left( \begin{pmatrix}
1 \\
1 \\
1
\end{pmatrix} \right) $$
\item On a (sans savoir l'ordre des valeurs propres car ce n'est pas donné dans l'énoncé...) :
$$ D = \begin{pmatrix}
-1 & 0 & 0 \\
0 & 1& 0 \\
0 & 0 & 3
\end{pmatrix}$$
Pour toute matrice $M = \begin{pmatrix}
a & b & c \\
d & e & f \\
g & h & i 
\end{pmatrix}$, on a :
\begin{align*}
MD=DM & =\begin{pmatrix}
-a & b & 3c \\
-d & e & 3f \\
-g & h & 3i 
\end{pmatrix} = \begin{pmatrix}
-a &- b & -c \\
d & e & f \\
3g & 3h & 3i 
\end{pmatrix} \\
& \Longleftrightarrow b=c=f=d=g=h=0 \\
& \Longleftrightarrow M= \textrm{diag}(a,e,i) 
\end{align*}
Ainsi si matrice $M \in \mathcal{M}_3(\mathbb{R})$ commute avec $D$ alors elle est diagonale (et la réciproque est vraie).
\item Soit $M \in \mathcal{M}_3(\mathbb{R})$. Il existe une matrice inversible $P \in \mathcal{M}_3(\mathbb{R})$ tel que $A=PDP^{-1}$. Alors :
\begin{align*}
P(M) =A & \Longleftrightarrow M^7+4M^3 + I_3 = PDP^{-1} \\
& \Longleftrightarrow (P^{-1}MP)^7 + 4(P^{-1}MP)^3+ I_3 = D 
\end{align*}
On cherche donc toutes les matrices $N \in \mathcal{M}_3(\mathbb{R})$ ($N=P^{-1}MP$) telles que :
$$ N^7+4N^3+I_3=D$$
Une matrice $N$ vérifiant cette égalité est nécessairement diagonale car commute avec $N^7+4N^3+I_3$ donc avec $D$ d'après la question précédente. En posant $N= \textrm{diag}(a,b,c)$, $N$ vérifie l'égalité précédente si et seulement si :
$$ a^7+ 4a^3+1= -1, \; b^7+ 4b^3+1= 1 \hbox{ et } c^7+ 4c^3+1= 3 $$
ou encore :
$$ a^7+ 4a^3= -2, \; b^7+ 4b^3= 0 \hbox{ et } c^7+ 4c^3= 2 $$
La fonction polynômiale $x \mapsto x^7+4x^3$ est strictement croissante et continue sur $\mathbb{R}$ (par somme). D'après le théorème de bijection monotone, elle est bijective de $\mathbb{R}$ sur $\mathbb{R}$ (les limites en $+ \infty$ et $- \infty$ sont évidentes). Il existe donc un unique triplet de réels $(a,b,c)$ vérifiant les trois égalités précédentes. Il est évident que $b=0$ et que $c=-a$ (par imparité). Trouver une valeur pour $a$ ne semble pas évident ... Une méthode de dichotomie donne $a \approx -0.771557$. En se rappelant que $N=P^{-1}MP$, il existe donc une unique matrice $M$ de $\mathcal{M}_3(\mathbb{R})$ telle que $P(M)=A$ et cette matrice est définie par :
$$ M = P \textrm{diag}(a,0,-a) P^{-1}$$
\end{enumerate}


\begin{Exercice}{} Soit $A= \begin{pmatrix}
1& 2 \\
2 & 1 \\
\end{pmatrix}\cdot$
\begin{enumerate}
\item $A$ est-elle diagonalisable ? $A$ est-elle inversible ? Donner les éléments propres de $A$.
\item  On définit $B = \begin{pmatrix}
A & A \\
A & A 
\end{pmatrix}\cdot$

$B$ est-elle diagonalisable ? Donner les éléments propres de $B$.
\end{enumerate}
\end{Exercice}

\corr \begin{enumerate}
\item Le déterminant de $A$ est non nul donc $A$ est inversible. Le calcul du polynôme caractéristique donne :
$$ \chi_A(X) = (X+1)(X-3)$$
Donc $A$ admet deux valeurs propres distinctes, $-1$ et $3$, et appartient à $\mathcal{M}_2(\mathbb{R})$ donc elle est diagonalisable. En calculant $A+I_2$ et $A-3I_2$, on obtient :
$$ E_{-1}(A) = \textrm{Vect} \left( \begin{pmatrix}
1 \\
-1 \\
\end{pmatrix}\right) \hbox{ et } E_{3}(A) = \textrm{Vect} \left( \begin{pmatrix}
1 \\
1 \\
\end{pmatrix}\right)$$
\item La première colonne de $B$ est égale à la troisième et la deuxième colonne est égale à la quatrième donc les vecteurs :
$$ \begin{pmatrix}
1 \\
0 \\
-1 \\
0
\end{pmatrix} \hbox{ et } \begin{pmatrix}
0 \\
1 \\
0 \\
-1
\end{pmatrix}$$
appartiennent à $\textrm{Ker}(B)= E_0(B)$ et sont non colinéaires donc $\textrm{dim}(E_0(B)) \geq 2$. La forme de la matrice $B$ incite à un calcul par blocs. Si $X \in \mathcal{M}_{2,1}(\mathbb{R})$ est un vecteur propre de $A$ associé à la valeur propre $\lambda$ alors le vecteur $\tilde{X} \in \mathcal{M}_{4,1}(\mathbb{R})$ défini (abusivement) par :
$$ \tilde{X} = \begin{pmatrix}
X \\
X \\
\end{pmatrix}$$
vérifie :
$$ B \tilde{X} = \begin{pmatrix}
A & A \\
A & A 
\end{pmatrix} \begin{pmatrix}
X \\
X \\
\end{pmatrix} = \begin{pmatrix}
2 AX \\
2AX
\end{pmatrix} = 2 \lambda \begin{pmatrix}
X \\
X \\
\end{pmatrix} = 2 \lambda \tilde{X}$$
Ainsi $\tilde{X}$ (non nul car $X$ l'est) est un vecteur propre de $B$ associé à la valeur propre $2 \lambda$. Ainsi, les vecteurs 
$$ \begin{pmatrix}
1 \\
-1 \\
1 \\
-1
\end{pmatrix} \hbox{ et } \begin{pmatrix}
1 \\
1 \\
1 \\
1 \\
\end{pmatrix}$$
sont des vecteurs propres de $B$ associés aux valeurs propres $-2$ et $6$. On a alors que :
$$ \textrm{dim}(E_0(B)) + \textrm{dim}(E_{-2}(B)) + \textrm{dim}(E_6(B))  \geq 4 = \textrm{dim}(\mathcal{M}_{4,1}(\mathbb{R}))$$
Or d'après le cours :
$$ \textrm{dim}(E_0(B)) + \textrm{dim}(E_{-2}(B)) + \textrm{dim}(E_6(B))  \leq \textrm{dim}(\mathcal{M}_{4,1}(\mathbb{R}))$$
et ainsi :
$$ \textrm{dim}(E_0(B)) + \textrm{dim}(E_{-2}(B)) + \textrm{dim}(E_6(B))  = \textrm{dim}(\mathcal{M}_{4,1}(\mathbb{R}))$$
On en déduit que $B$ est diagonalisable et que $-2$ et $6$ sont valeurs propres simples et que $E_0(B)$ est de dimension $2$ donc :
$$ E_0(B) = \textrm{Vect} \left(\begin{pmatrix}
1 \\
0 \\
-1 \\
0
\end{pmatrix},\begin{pmatrix}
0 \\
1 \\
0 \\
-1
\end{pmatrix}\right), \;  E_{-2}(B) = \textrm{Vect} \left( \begin{pmatrix}
1 \\
-1 \\
1 \\
-1
\end{pmatrix}\right) \hbox{ et } E_{6}(B) = \textrm{Vect} \left(\begin{pmatrix}
1 \\
1 \\
1 \\
1
\end{pmatrix}\right)$$
\end{enumerate}

\begin{Exercice}{}
\begin{enumerate}
\item Deux matrices carrées $A$ et $B$ d'ordre $n \geq 2$ possèdent le même spectre. Sachant que leurs valeurs propres sont deux à deux distinctes, montrer que $A$ et $B$ sont semblables.
\item Donner deux matrices possédant les mêmes valeurs propres mais qui ne soient pas semblables.
\end{enumerate}
\end{Exercice} 

\corr \begin{enumerate}
\item Les valeurs propres de $A$ et $B$ sont les mêmes et deux à deux distinctes donc ces deux matrices sont diagonalisables et semblables à la même matrice diagonale. La relation \og être semblable à \fg étant une relation d'équivalence alors $A$ et $B$ sont semblables par transitivité.
\item Posons :
$$ A = I_2 \hbox{ et } B = \begin{pmatrix}
1 & 1 \\
0 & 1 
\end{pmatrix}$$
Les matrices $A$ et $B$ ont le même spectre (unique valeur propre égale à $1$) mais si $B$ était semblable à $A$ alors $B$ serait égale à $A$ (l'unique matrice semblable à $I_2$ est $I_2$) ce qui est faux.
\end{enumerate}

\begin{Exercice}{} \begin{enumerate}
\item Soit $E$ un $\mathbb{K}$-espace vectoriel de dimension $3$. Montrer que $H$, plan de $E$, est stable par $u \in \mathcal{L}(E)$ si et seulement si il existe $\lambda \in \mathbb{K}$ tel que $\textrm{Im}(u- \lambda \textrm{Id}) \subset H$.
\item Trouver tous les sous-espaces de $\mathcal{M}_{3,1}(\mathbb{R})$ stables par l'application linéaire canoniquement associée à 
$$A= \begin{pmatrix}
-1 & 2 & -3 \\
-2 & -5 & -2 \\
-2 & 2 & 0 \\
\end{pmatrix} $$
\end{enumerate}
\end{Exercice}

\corr \begin{enumerate}
\item Supposons l'existence d'un scalaire $\lambda$ tel que $\textrm{Im}(u- \lambda \textrm{Id}) \subset H$. Montrons que $H$ est stable par $u$, c'est-à-dire :
$$ \forall h \in H, \; u(h) \in H$$
On sait que $\textrm{Im}(u- \lambda \textrm{Id}) \subset H$ donc :
$$ \forall x \in E, \; u(x)- \lambda x \in H$$
Soit $h \in H$. On a :
$$ u(h) = u(h)- \lambda h + \lambda h$$
On sait que $u(h) - \lambda h \in H$ et $\lambda h \in H$ donc sachant que $H$ est un espace vectoriel, on obtient que $u(h) \in H$ ce qui donne le résultat.

\medskip

\noindent Supposons maintenant que $H$ est stable par $u$ :
$$ \forall h \in H, \; u(h) \in H$$
$H$ admet un supplémentaire engendré par un vecteur non nul $a$. Soit $x \in E$. Il existe un unique vecteur $h \in H$ et un unique scalaire $\lambda$ tel que $x=h + \alpha a$. Par linéarité de $u$, on a :
$$ u(x) = u(h) + \alpha u(a)$$
Remarquons que $u(h)$ appartient à $H$ par hypothèse. Le vecteur $u(a)$ se décompose de manière unique sous la forme :
$$ u(a) = h' + \beta  a$$
avec $h' \in H$ et $\beta \in \mathbb{K}$. Ainsi,
\begin{align*}
u(x) & = u(h)+ \alpha (h'+ \beta a) \\
& = u(h) + \alpha h' + \beta (\alpha a) \\
& = u(h) + \alpha h' + \beta(x-h) \\
& = u(h)+ \alpha h' - \beta h + \beta x
\end{align*}
Ainsi,
$$ u(x)- \beta x = u(h)+ \alpha h' - \beta h \in H$$
car $u(h)$, $h'$ et $h$ appartient à $H$ qui est un espace vectoriel. Finalement, $\textrm{Im}(u- \beta \textrm{Id}) \subset H$.
\item Notons $f \in \mathcal{L}(\mathcal{M}_{3,1}(\mathbb{K}))$ l'application linéaire canoniquement associée à $A$. Les sous-espaces stables triviaux de $f$ sont $\lbrace 0_{\mathcal{M}_{3,1}(\mathbb{K})} \rbrace$ et $\mathcal{M}_{3,1}(\mathbb{K})$. Il reste à déterminer les sous-espaces stables de $f$ de dimension $1$ et $2$. D'après le cours, les droites vectorielles stables par $f$ sont les droites engendrées par des vecteurs propres. Le calcul du polynôme caractéristique donne :
$$ \chi_A(x) = (X-2)(X^2+8X+23)$$
La seule valeur propre réelle de $A$ est $2$ donc la seule droite vectorielle de $f$ est $E_2(f)$, c'est-à-dire (par la méthode habituelle) :
$$ E_2(f) = \textrm{Vect} \left( \begin{pmatrix}
1 \\
0 \\
-1 
\end{pmatrix} \right)$$
Si $H$ est un plan stable par $f$, il existe d'après la question précédente un réel $\lambda$ tel que $\textrm{Im}(f- \lambda \textrm{Id}) \subset H$. Nécessairement, $\textrm{Im}(f- \lambda \textrm{Id})$ est de dimension au plus $2$ ce qui implique d'après le théorème du rang que $E_{\lambda}(f)$ est de dimension au moins égale à $1$. La seule possibilité est $\lambda=2$. On sait que $E_2(f)$ est de dimension $1$ donc $\textrm{Im}(f- 2 \textrm{Id})$ est de dimension $2$. Sachant que $f$ et $f- 2 \textrm{Id}$ commutent, on en déduit que $\textrm{Im}(f- 2 \textrm{Id})$ est bien stable par $f$. Sachant que :
$$ A - 2I_3 = \begin{pmatrix}
-3 & 2 & -3 \\
-2 & -7 & -2 \\
-2 & 2 & -2 \\
\end{pmatrix} $$
L'image de $A-2I_3$ est de dimension $2$ et ses deux premières colonnes sont non colinéaires donc finalement, le seul plan stable par $f$ est :
$$ \textrm{Vect} \left( \begin{pmatrix}
-3 \\
-2 \\
-2 \\
\end{pmatrix}, \begin{pmatrix}
2 \\
-7 \\
2 \\
\end{pmatrix}\right)$$
\end{enumerate}

\begin{Exercice}{} Soient $A$ et $B$ deux matrices complexes inversibles d'ordres $n \geq 1$. On pose :
$$ N = \begin{pmatrix}
0 & B \\
A & 0
\end{pmatrix}$$
\begin{enumerate}
\item Montrer que $N$ est inversible et déterminer $N^{-1}$.
\item Calculer $N^2$ et $P(N^2)$ pour $P \in \mathbb{C}[X]$.
\item Si $N$ est diagonalisable, $AB$ l'est-elle ? Étudier la réciproque.
\end{enumerate}
\end{Exercice}

\corr \begin{enumerate}
\item Posons :
$$ M = \begin{pmatrix}
0 & A^{-1} \\
B^{-1} & 0& 
\end{pmatrix}$$
Un calcul par blocs montre que $NM=I_{2n}$. Ainsi, $N$ est inversible est :
$$ N^{-1} = \begin{pmatrix}
0 & A^{-1} \\
B^{-1} & 0& 
\end{pmatrix}$$
\item Un calcul par blocs montre que :
$$ N^{2} = \begin{pmatrix}
BA & 0 \\
0 & AB
\end{pmatrix}$$
Une démonstration par récurrence montre que pour tout $k \geq 0$,
$$ (N^2)^k =  \begin{pmatrix}
(BA)^k & 0 \\
0 & (AB)^k
\end{pmatrix}$$
et ainsi pour $P \in \mathbb{C}[X]$, 
$$ P(N^2) = \begin{pmatrix}
P(BA) & 0 \\
0 & P(AB) 
\end{pmatrix}$$
\item Supposons que $N$ est diagonalisable. Alors $N^2$ est diagonalisable et il existe donc un polynôme annulateur scindé à racines simples $P$ tel que $P(N^2)=0_{2n}$ et donc d'après la question précédente, $P(AB)=0_n$. Ainsi, $AB$ a un polynôme annulateur scindé à racines simples et est donc diagonalisable.

\medskip

\noindent On pourrait se passer de la question précédente en remarquant que si $f$ est l'endomorphisme de $\mathbb{R}^{2n}$ canoniquement associé à $N^2$ alors $f$ est diagonalisable et $AB$ est la matrice de l'endomorphisme induit par $f$ sur l'espace engendré par $n$ derniers vecteurs de la base canonique de $\mathbb{R}^{2n}$ donc $AB$ est aussi diagonalisable.

\medskip

\noindent Supposons maintenant que $AB$ est diagonalisable. Sachant que $A$ est inversible, on a :
$$ AB=A(BA)A^{-1}$$
Donc $AB$ et $BA$ sont semblables et ainsi $BA$ est aussi diagonalisable et ces deux matrices sont semblables à une matrice diagonale $D$ : il existe deux matrices $P$, $Q$, d'ordre $n$ et inversibles tels que $AB=PDP^{-1}$ et $BA= QDQ^{-1}$. Un petit calcul par blocs prouve que :
$$ N^2 = \begin{pmatrix}
Q & 0 \\
0 & P
\end{pmatrix} \begin{pmatrix}
D & 0 \\
0 & D
\end{pmatrix} \begin{pmatrix}
Q^{-1} & 0 \\
0 & P^{-1}
\end{pmatrix}$$
et sachant que :
$$ \begin{pmatrix}
Q & 0 \\
0 & P
\end{pmatrix} \begin{pmatrix}
Q^{-1} & 0 \\
0 & P^{-1}
\end{pmatrix} = I_{2n}$$
on en déduit que $N^2$ est diagonalisable. Notons $\lambda_1$, $\ldots$, $\lambda_k$ les valeurs propres de $N^2$ ($k \geq 1$). D'après le cours, on sait que :
$$ P(X) = \prod_{j=1}^k X- \lambda_j$$
est un polynôme annulateur de $N^2$. Ainsi :
$$ \prod_{j=1} ^k (N^2- \lambda_j I_{2n}) = 0_{2n}$$
Or $N^2$ est inversible car $N$ l'est donc toutes les valeurs propres de $N^2$ sont non nulles. Ainsi, chaque valeur propre $\lambda_j$ admet deux racines carrées distinctes $a_j$ et $b_j$. On en déduit que :
$$ \prod_{j=1} ^k (N- a_j I_{2n})(N - b_j I_{2n}) = 0_{2n}$$
Ainsi le polynôme :
$$ Q(X) = \prod_{j=1}^k (X-a_j)(X- b_j)$$
est un polynôme annulateur scindé à racines simples de $N$ donc $N$ est diagonalisable.
\end{enumerate}

\begin{Exercice}{} Soit $A \in \mathcal{M}_n(\mathbb{R})$, non nulle, vérifiant $A^3+9A=0_n$.
\begin{enumerate}
\item Montrer que les valeurs propres éventuelles de $A$ sont $0$, $3i$ et $-3i$.
\item $A$ est-elle diagonalisable dans $\mathbb{C}$? Dans $\mathbb{R}$?
\item Montrer que $A$ n'est pas inversible si $n$ est impair. 
\end{enumerate}
\end{Exercice}

\begin{enumerate}
\item On peut juste dire que $X^3+9X=X(X^2+9)$ est un polynôme annulateur de $A$ et utiliser le résultat hors-programme du cours mais je pense qu'il faut le reprouver ici...

\medskip

\noindent Soit $\lambda$ une valeur propre éventuelle de $A$ et $X$ un vecteur propre associé. Remarquons que :
$$ A^3 X= A^2(AX)= \lambda A^2X = \lambda A(AX) = \lambda^2 AX = \lambda^3 X$$
donc
$$ (A^3+9A)X = \lambda ^3 X+ 9 \lambda X = (\lambda^3+9)X$$
Or $ A^3+9A=0_n$ donc $(\lambda^3+9)X= 0_{n,1}$ et $X$ étant non nul, $\lambda^3+9=0$. Ainsi, $\lambda$ est racine de $Y^3+9Y$ dont les racines sont $0$, $3i$ et $-3i$.
\item Le polynôme $X(X-3i)(X+3i)$ est scindé à racines simples dans $\mathbb{C}$ et est un polynôme annulateur de $A$ donc $A$ est diagonalisable dans $\mathbb{C}$.

\medskip

\noindent La seule valeur propre réelle éventuelle de $A$ est $0$. Si $A$ est diagonalisable dans $\mathbb{R}$ alors $A$ est semblable à la matrice nulle donc est nulle ce qui faux d'après l'énoncé donc $A$ n'est pas diagonalisable dans $\mathbb{R}$.

\item Soit $n$ un entier impair. Supposons par l'absurde que $A$ est inversible. On sait que :
$$ A^3 = -9A$$
donc par multilinéarité du déterminant, on a :
$$ \textrm{det}(A^3) = (-9)^n \textrm{det}(A)$$
et ainsi :
$$ \textrm{det}(A)^3 = (-9)^n \textrm{det}(A)$$
Le déterminant de $A$ est non nul car $A$ est inversible donc :
$$ \textrm{det}(A)^2 = (-9)^n $$
Sachant que $n$ est impair, $(-9)^n$ est strictement négatif alors que $\textrm{det}(A)^2$ est un réel au carré (la matrice $A$ est réelle) donc est positif ou nul. Par l'absurde, on a bien montré que $A$ n'est pas inversible si $n$ est impair. 
\end{enumerate}

\begin{Exercice}{} Soit $A \in GL_n(\mathbb{R})$. Montrer que $f$, définie pour tout $M \in \mathcal{M}_n(\mathbb{R})$ par $f(M)= 2 \textrm{Tr}(M) A$, est un endomorphisme de $\mathcal{M}_n(\mathbb{R})$. Est-il diagonalisable ?
\end{Exercice}

\corr Il est clair que pour tout $M \in \mathcal{M}_n(\mathbb{R})$, $f(M) \in \mathcal{M}_n(\mathbb{R})$. Soient $(M,N) \in \mathcal{M}_n(\mathbb{R})^2$ et $\lambda \in \mathbb{R}$. On a :
\begin{align*}
f(\lambda M + N) & = 2 \textrm{Tr}(\lambda M + N) A \\
& = 2 (\lambda \textrm{Tr}(M) +  \textrm{Tr}(N))A \quad \hbox{(linéarité de la trace)} \\
& =\lambda (2 \textrm{Tr}(M)A) + 2 \textrm{Tr}(N)A \\
&= \lambda f(M) + f(N)
\end{align*}
Ainsi, $f$ est un endomorphisme de $\mathcal{M}_n(\mathbb{R})$.

\medskip
%
%\noindent 

\noindent Pour tout $M \in \mathcal{M}_n(\mathbb{R})$, on a :
\begin{align*}
f(f(M)) & =  2\textrm{Tr}(f(M)) A \\
& = 2\textrm{Tr}(2\textrm{Tr}(M) A) A \\
& = 4 \textrm{Tr}(M) \textrm{Tr}(A) A \quad \hbox{(linéarité de la trace)} \\
& = 2 \textrm{Tr}(A) f(M)
\end{align*}
Ainsi, $X^2-2\textrm{Tr}(A)= X(X- 2 \textrm{Tr}(A))$ est un polynôme annulateur de $f$. Si la trace de $A$ est non nulle, ce polynôme est scindé à racines simples donc $f$ est diagonalisable. Si la trace de $A$ est nulle alors $X^2$ est un polynôme annulateur de $f$ donc la seule valeur propre possible de $f$ est $0$ donc si $f$ était diagonalisable, $f$ serait nulle (sa matrice dans n'importe quelle base serait semblable à la matrice nulle) ce qui est faux car $A$ est non nulle (car inversible). Finalement, $f$ est diagonalisable si et seulement si la trace de $A$ est non nulle.

\medskip

\noindent \textit{Remarque.} Supposons que la trace de $A$ soit non nulle. Soit $M \in \mathcal{M}_n(\mathbb{R})$. Alors $M$ appartient au noyau de $f$ si et seulement si sa trace est nulle ($A$ est non nulle car inversible). Or la trace est une forme linéaire donc son noyau est un hyperplan de $\mathcal{M}_n(\mathbb{R})$ donc de dimension $n^2-1$. On remarque facilement que :
$$ f(A) = 2 \textrm{Tr}(A) A$$
donc $A$ est un vecteur propre de $f$ associée à la valeur propre $2 \textrm{Tr}(A)$. La dimension du sous-espace propre associée à $2 \textrm{Tr}(A)$ est $1$ (car le noyau est de dimension $n^2-1$ et que la somme des dimensions des sous-espaces est égale à $n^2$ car $f$ est diagonalisable). Ainsi, $E_{2\textrm{Tr}(A)}(f) = \textrm{Vect}(A)$.




\begin{Exercice}{} Soit $A = \begin{pmatrix}
1 & 1 & 0 \\
-1 & 0 & 0 \\
2 & 0 & -1 
\end{pmatrix}\cdot$
\begin{enumerate}
\item Montrer qu'il n'existe pas de polynôme annulateur non nul et de degré inférieur ou égal à deux de $A$.
\item Trouver un polynôme annulateur de $A$ et en déduire $A^{-1}$.
\item Quels sont les polynômes annulateurs de $A$?
\end{enumerate}
\end{Exercice}

\corr \begin{enumerate}
\item Il n'existe pas de polynôme constant annulateur de $A$ ni de polynôme de degré $1$ ($A$ n'est pas multiple de $I_3$). Supposons l'existence d'un polynôme de degré $2$ annulateur de $A$. On peut supposer celui-ci unitaire (en divisant par le coefficient dominant). Il existe donc deux réels $\alpha$ et $\beta$ tels que :
$$ A^2 + \alpha A + \beta I_3 = 0_3$$
On a :
$$ A^2 = \begin{pmatrix}
0 & 1 & 0 \\
-1 & -1 & 0 \\
0 & 2 & 1 
\end{pmatrix}$$
et donc :
$$ A^2+ \alpha A + \beta I_3 = \begin{pmatrix}
\alpha+ \beta & 1+ \alpha & 0 \\
-1- \alpha & -1+ \beta & 0 \\
2 \alpha & 2 & 1  - \alpha + \beta
\end{pmatrix}$$
et ainsi $1+ \alpha = 0$ donc $\alpha =-1$ mais $2 \alpha = 0$ donc $\alpha = 0$ : c'est absurde ! Ainsi, il n'existe pas de polynôme annulateur non nul et de degré inférieur ou égal à deux de $A$.
\item D'après le théorème de Cayley-Hamilton, le polynôme caractéristique de $A$ est un polynôme annulateur de $A$. Par simple calcul, on a $\chi_A(X) = X^3+1$. On en en particulier :
$$ (-A^2)A= -A^3 = I_3$$
et donc $A$ est inversible d'inverse $-A^2$.
\item On sait que toute valeur propre de $A$ est racine de tout polynôme annulateur de $A$. Or :
$$ X^3+1 = (X+1) (X+j)(X+ \overline{j})$$
Ainsi tout polynôme annulateur $P$ de $A$ admet $-1$, $-j$ et $- \overline{j}$ est racine de $P$ donc $P$ est de la forme :
$$ P(X)= Q(X) (X^3+1)$$
où $Q$ est un polynôme de $\mathbb{R}[X]$. Reciproquement, si $P$ est de cette forme, alors :
$$ P(A) = Q(A)(A^3+1) = Q(A) 0_3 = 0_3$$
Finalement, l'ensemble des polynômes qui sont annulateurs de $A$ est l'ensemble des multiples de $\chi_A$.
\end{enumerate}

\begin{Exercice}{} Soient $A \in \mathcal{M}_{3,2}(\mathbb{R})$ et $B \in \mathcal{M}_{2,3}(\mathbb{R})$ telles que :
$$ AB = \begin{pmatrix}
1 & 0 & x \\
0 & 1 & 0 \\
1 & 0 & 1 
\end{pmatrix}$$
où $x \in \mathbb{R}$.
\begin{enumerate}
\item La matrice $AB$ est-elle inversible ? Quelles sont les valeurs possibles de $x$ ?
\item La matrice $BA$ est-elle diagonalisable?
\item Montrer que $\mathbb{R}^3 = \textrm{Im}(A) \oplus \textrm{Ker}(B)$.
\item Montrer qu'il existe une infinité de couples $(A,B)$ vérifiant l'hypothèse donnée dans l'énoncé.
\end{enumerate}
\end{Exercice}

\corr 
\begin{enumerate}
\item Le déterminant de la matrice vaut (après l'opération $L_3 \leftarrow L_3- L_1$) $1-x$ donc la matrice est inversible si et seulement si $x \neq 1$. Si la matrice est inversible alors le rang de $AB$ vaut $3$. Or le rang de $AB$ est inférieur ou égal au rang de $A$ car l'image de $AB$ est inclus dans l'image de $A$. On en déduit alors que la matrice $A$ est au moins de rang $3$ ce qui est absurde car $A$ est de taille $3 \times 2$ donc au maximum de rang $2$. Ainsi, $A$ ne peut pas être inversible et nécessairement $x$ vaut $1$.
%\item On sait que la trace de $BA$ est égal à la trace de $AB$ donc $3$. La matrice $BA$ est carrée d'ordre deux donc son polynôme caractéristique est de la forme :
%$$ P(X)=X^2-3X + \lambda$$
%où $\lambda \in \mathbb{R}$. 

\item Le polynôme caractéristique de $AB$ vaut (après calculs) $X(X-1)(X-2)$ (donc $AB$ a $3$ valeurs propres distinctes et est d'ordre $3$ donc est diagonalisable) et on obtient facilement que :
$$ E_{0}(AB) = \textrm{Vect}(X_0), \; E_{1}(AB) = \textrm{Vect}(X_1) \hbox{ et } E_{2}(AB) = \textrm{Vect}(X_2)$$
où 
$$ X_0 = \begin{pmatrix}
1 \\
0 \\
-1
\end{pmatrix},  \; X_1 = \begin{pmatrix}
0 \\
1 \\
0
\end{pmatrix} \hbox{ et } X_2 = \begin{pmatrix}
1 \\
0 \\
1
\end{pmatrix}$$
Remarquons alors que $ABX_1=X_1$ donc $BABX_1=BX_1$ donc $BX_1$ appartient au sous-espace propre de $BA$ associé à la valeur propre $1$. Remarquons que $BX_1$ n'est pas nul sinon $X_1$ l'est ce qui est faux donc $BX_1$ est un vecteur propre de $BA$ associé à la valeur propre $1$. De même, $ABX_2=2X_2$ donc $BABX_2=BX_2$ donc $BX_2$ appartient au sous-espace propre de $BA$ associé à la valeur propre $2$. Remarquons que $BX_2$ n'est pas nul sinon $X_2$ l'est ce qui est faux donc $BX_2$ est un vecteur propre de $BA$ associé à la valeur propre $2$.  Ainsi $BA$ admet deux valeurs propres distinctes et est carrée d'ordre deux donc elle est diagonalisable.
\item Soit $X \in \textrm{Im}(A) \cap \textrm{Ker}(B)$. Alors $BX=0$ et il existe $Z \in \mathcal{M}_{2,1}(\mathbb{R})$ tel que $X=AZ$. Alors $BAZ=0$ donc $Z$ est dans le noyau de $BA$. Or $0$ n'est pas valeur propre de $BA$ donc $Z$ est nul et $X=AZ$ aussi. Ainsi, $\textrm{Im}(A)$ et $\textrm{Ker}(B)$ sont en somme directe. La matrice $A$ est de taille $3 \times 2$. D'après le théorème du rang,
$$ 2 = \textrm{dim}(\textrm{Im}(A)) + \textrm{dim}(\textrm{Ker}(A))$$
Or le noyau de $A$ est réduit au vecteur nul sinon $BA$ aurait un noyau de dimension au moins égal à $1$ ce qui est faux car $0$ n'est pas valeur propre. Ainsi,
$$\textrm{dim}(\textrm{Im}(A))  = 2$$
La matrice $B$ est de taille $2 \time 3$ donc a un rang inférieur ou égal à $2$ donc d'après le théorème du rang,
$$ \textrm{dim}(\textrm{Ker}(B)) = 3- \textrm{dim}(\textrm{Im}(B)) \geq 3-2 = 1$$
Ainsi :
$$ \textrm{dim}(\textrm{Im}(A)) + \textrm{dim}(\textrm{Ker}(B)) \geq 3$$
et l'autre inégalité est vérifiée car l'image de $A$ et le noyau de $B$ sont en somme directe. Finalement, $\mathbb{R}^3 = \textrm{Im}(A) \oplus \textrm{Ker}(B)$.
\item Posons :
$$ A  = \begin{pmatrix}
1 & 0 \\
0 & 1\\
1 & 0 \\
\end{pmatrix} 
\hbox{ et } B = \begin{pmatrix}
1 & 0 & 1 \\
0 & 1 & 0
\end{pmatrix}$$
Alors :
$$ AB = \begin{pmatrix}
1 & 0 & 1 \\
0 & 1 & 0 \\
1 & 0 & 1 
\end{pmatrix}$$
et tout couple de la forme $(yA, 1/y \times B)$ où $y \neq 0$ est encore solution donc il existe une infinité de couples $(A,B)$ vérifiant l'hypothèse donnée dans l'énoncé.
\end{enumerate}

\begin{Exercice}{}
Soit $A$ la matrice de $\mathcal{M}_{3}\left(\mathbb{R}\right)$
d\'efinie par
\[
A=\left(\begin{array}{ccc}
1 & -1 & -1\\
-1 & 1 & -1\\
-1 & -1 & 1
\end{array}\right)
\]

\begin{enumerate}
\item Montrer que la matrice $A$ est diagonalisable. D\'eterminer ses valeurs propres et ses sous-espaces propres.
\item D\'eterminer une relation entre $A^{2}$, $A$ et $I_{n}$.\\
En d\'eduire une relation entre $A^{n+1}$, $A^{n}$ et $A^{n-1}$ pour
tout entier $n\geq1$.
\item Montrer par r\'ecurrence qu'il existe deux suites $\left(u_{n}\right)_{n\in\mathbb{N}}$
et $\left(v_{n}\right)_{n\in\mathbb{N}}$ telles que
\[
\forall n\in\mathbb{N},\quad A^{n}=\left(\begin{array}{ccc}
u_{n} & v_{n} & v_{n}\\
v_{n} & u_{n} & v_{n}\\
v_{n} & v_{n} & u_{n}
\end{array}\right)
\]
qui v\'erifient la relation de r\'ecurrence :
\[
\forall n\geq1,\quad\left\{ \begin{array}{ccc}
u_{n+1} & = & u_{n}+2u_{n-1}\\
v_{n+1} & = & v_{n}+2v_{n-1}
\end{array}\right.
\]
\item D\'eterminer, pour tout entier naturel $n$, l'expression de $u_{n}$
et $v_{n}$ en fonction de $n$.
\end{enumerate}
\end{Exercice}

\corr 
\begin{enumerate}
\item Calculons le polynôme caractéristique de $A$ :
\begin{align*}
\chi_A(X) & =\det \left\vert  \begin{array}{*{20}{c}}
{X - 1}&1&1\\
1&{X - 1}&1\\
1&1&{X - 1}
\end{array}  \right\vert \\
&  \underset{{C_1} \leftarrow {C_1} + {C_2} + {C_3}}{=} \left\vert \begin{array}{*{20}{c}}
{X + 1}&1&1\\
{X + 1}&{X - 1}&1\\
{X + 1}&1&{X - 1} 
\end{array} \right\vert  \\
&  \underset{{L_k} \leftarrow {L_k} - {L_1}, k=2,3}{=}  \left\vert \begin{array}{*{20}{c}}
{X + 1}&1&1\\
0&{X - 2}&0\\
0&0&{X - 2}
\end{array} \right\vert \\
& = \left( {X + 1} \right){\left( {X - 2} \right)^2} \\
\end{align*}
Ainsi $-1$ est valeur propre simple donc $\textrm{dim}(E_{-1}(A))=1$. On a :
$$ A-2I_3 = \left(\begin{array}{ccc}
-1 & -1 & -1\\
-1 & -1 & -1\\
-1 & -1 & -1
\end{array}\right)$$
Clairement $A-2I_3$ est de rang $1$ donc le noyau de $A-2I_3$ est de dimension $2$ par théorème du rang. Ainsi,
$$ \textrm{dim}(E_{-1}(A)) + \textrm{dim}(E_2(A)) = 1+2 = 3 = \textrm{dim}(\mathcal{M}_{3,1}(\mathbb{R}))$$
et donc $A$ est diagonalisable. $A$. a pour valeur propre $-1$ et $2$. On sait que :
$$ x_1= \begin{pmatrix}
1 \\
 -1 \\ 0 
\end{pmatrix} \et x_2=\begin{pmatrix}
1 \\
0 \\
-1
\end{pmatrix}$$
appartiennent à $E_2(A)$ qui est de dimension $2$ et sont non colinéaires. Ainsi, la famille $(x_1,x_2)$ est une base de $E_2(A)$. On a aussi :
$$ A+I_3  = \left(\begin{array}{ccc}
2 & -1 & -1\\
-1 & 2 & -1\\
-1 & -1 & 2
\end{array}\right)$$
Ainsi le vecteur 
$$ x_3 = \begin{pmatrix}
1 \\
1 \\
1
\end{pmatrix}$$
appartient à $E_{-1}(A)$ qui est de dimension $1$ et est non nul. Ainsi, la famille $(x_3)$ est une base de $E_{-1}(A)$.
\item Notons :
 \[D = \left( {\begin{array}{*{20}{c}}
{ - 1}&0&0\\
0&2&0\\
0&0&2
\end{array}} \right)\]
On a alors :
 \[{D^2} = \left( {\begin{array}{*{20}{c}}
1&0&0\\
0&4&0\\
0&0&4
\end{array}} \right) = \left( {\begin{array}{*{20}{c}}
{ - 1}&0&0\\
0&2&0\\
0&0&2
\end{array}} \right) + 2\left( {\begin{array}{*{20}{c}}
1&0&0\\
0&1&0\\
0&0&1
\end{array}} \right) = D +2 {I_3}\]
Or d'après la question précédente, la matrice $A$ est semblable à $D$ donc il existe une matrice inversible $P$ telle que $A=PDP^{-1}$. On a ainsi :
$$ A^2 = (PDP^{-1})^2 = PD^2P^{-1} = P(D+2I_3)P^{-1} = PDP^{-1} + 2I_3 = A_2+2 I_3$$
Ainsi, $A^2 = A + 2 I_3$. 

\medskip

\noindent \textit{Remarque}. Sinon on calcule tout simplement $A^2$... Mais c'est moins drôle !

\medskip

\noindent En multipliant, pour $n \geq 1$, l'égalité précédente par $A^{n-1}$, on obtient :
$$A^{n+1} = A^{n} + 2 A^{n-1}$$

\item On raisonne à l'aide d'une démonstration par récurrence. Pour tout entier $k \geq 0$, on pose $$ \mathcal{P}_k : \hbox{\og} \  \exists (u_k, v_k) \in \mathbb{R}^2 \, \vert \;  A^k  = \left( {\begin{array}{*{20}{c}}
{{u_k}}&{{v_k}}&{{v_k}}\\
{{v_k}}&{{u_k}}&{{v_k}}\\
{{v_k}}&{{v_k}}&{{u_k}}
\end{array}} \right) \hbox{\fg}$$
\begin{enumerate}
\item [$\bullet$] $\mathcal{P}_0$ et $\mathcal{P}_1$ sont vraies : il suffit de prendre $u_0=1$ et $v_0=0$ puis $u_1 = 1$, $v_1=-1$.
\item [$\bullet$] Soit $n \in \mathbb{N}^\star$ tel que $\mathcal{P}_{n-1}$ et $\mathcal{P}_n$ soient vraies. Montrons alors que $\mathcal{P}_{n+1}$ l'est aussi. On sait que $A^{n+1} = A^n + 2 A^{n-1}$ donc \[\left( {\begin{array}{*{20}{c}}
{{u_n}}&{{v_n}}&{{v_n}}\\
{{v_n}}&{{u_n}}&{{v_n}}\\
{{v_n}}&{{v_n}}&{{u_n}}
\end{array}} \right) + 2\left( {\begin{array}{*{20}{c}}
{{u_{n - 1}}}&{{v_{n - 1}}}&{{v_{n - 1}}}\\
{{v_{n - 1}}}&{{u_{n - 1}}}&{{v_{n - 1}}}\\
{{v_{n - 1}}}&{{v_{n - 1}}}&{{u_{n - 1}}}
\end{array}} \right) = \left( {\begin{array}{*{20}{c}}
{{u_n} + 2{u_{n - 1}}}&{{v_n} + 2{v_{n - 1}}}&{{v_n} + 2{v_{n - 1}}}\\
{{v_n} + 2{v_{n - 1}}}&{{u_n} + 2{u_{n - 1}}}&{{v_n} + 2{v_{n - 1}}}\\
{{v_n} + 2{v_{n - 1}}}&{{v_n} + 2{v_{n - 1}}}&{{u_n} + 2{u_{n - 1}}}
\end{array}} \right)\]
est de la forme voulue en posant $u_{n+1} = u_n + 2 u_{n-1}$ et $v_{n+1}= v_n + 2 v_{n-1}$.
\item [$\bullet$] Par principe de récurrence double, la propriété est donc vraie pour tout $n \geq 0$.
\end{enumerate}
Ainsi, il existe deux suites $\left(u_{n}\right)_{n\in\mathbb{N}}$
et $\left(v_{n}\right)_{n\in\mathbb{N}}$ telles que
\[
\forall n\in\mathbb{N},\quad A^{n}=\left(\begin{array}{ccc}
u_{n} & v_{n} & v_{n}\\
v_{n} & u_{n} & v_{n}\\
v_{n} & v_{n} & u_{n}
\end{array}\right)
\]
qui v\'erifient la relation de r\'ecurrence
\[
\forall n\geq 1,\quad\left\{ \begin{array}{ccc}
u_{n+1} & = & u_{n}+2u_{n-1}\\
v_{n+1} & = & v_{n}+2v_{n-1}
\end{array}\right.
\]
\item Les suites $(u_n)_{n \geq 0}$ et $(v_n)_{n \geq 0}$ sont des suites récurrentes linéaires d'ordre deux (avec la même relation). L'équation caractéristique associée admet deux racines distinctes qui sont $-1$ et $2$. Il existe donc quatre r\'eels $\alpha, \beta, \gamma, \delta$ tels que pour tout $n \in \mathbb{N}$,
$$ u _n = \alpha 2^n + \beta (-1)^n \et v_n  = \gamma 2^n + \delta (-1)^n$$
La prise en compte des conditions initiales $u_0=u_1=1$ permet de conclure que 
$$\forall n \geq 0, \; u_{n} = \frac{1}{3}\left(2^{n+1} + (-1)^n\right)$$ 
et les conditions $v_0=0, \; v_1=-1$ permettent d'avoir :
$$ \forall n \geq 0, \; v_{n} = \frac{1}{3} \left( (-1)^n - 2^{n} \right) $$
\end{enumerate}

\begin{Exercice}{} Soient $(u_k)_{k\in\N}$ et $(v_k)_{k\in\N}$ deux suites \`a termes r\'eels d\'efinies par
$$\left\{\begin{array}{l}u_0=1\\ v_0=2\end{array}\right.\qquad\hbox{et}\qquad \forall k\in\N,\;\left\{\begin{array}{l}u_{k+1}=4u_k-2v_k\\
v_{k+1}=u_k+v_k\end{array}\right.$$
Déterminer pour tout entier $k \geq 0$, l'expression de $u_k$ et $v_k$ en fonction de $k$.
\end{Exercice}

\corr Pour tout entier $k \geq 0$, posons :
$$ X_k = \begin{pmatrix}
u_k \\
v_k \\
\end{pmatrix}$$
et
$$ A=\pmatrice{4&-2\\1&1} $$
On montre par récurrence immédiate que pour tout entier $k \geq 0$,
$$ X_k=A^k X_0$$
Le calcul du polynôme caractéristique de $A$ donne :
$$ \chi_A(X) = (X-2)(X-3)$$
Ainsi, $A$ appartient à $\mathcal{M}_2(\mathbb{R})$ et a deux valeurs propres distinctes donc elle est diagonalisable. On a :
$$ A - 2I_2 = \pmatrice{2&-2\\1&-1}$$
LA somme des deux colonnes est nulle donc $V_1=\begin{pmatrix}
1 \\
1 \\
\end{pmatrix}$
est non nul et appartient à $\ker(A-2I_2)$ qui est de dimension $1$ car $2$ est valeur propre simple de $A$. On a aussi :
$$A-3I_2=\pmatrice{1&-2\\1&-2}$$ 
donc $V_2=\begin{pmatrix}
2 \\
1 \\
\end{pmatrix}$ est non nul et appartient à $\ker(A-3I_2)$ qui est de dimension $1$ car $2$ est valeur propre simple de $A$. La famille $(V_1,V_2)$ est donc une base de vecteurs propres de $A$ et si l'on note 
$$P=\pmatrice{1&2 \\ 1&1}$$
la matrice de passage de la base canonique \`a cette base de vecteurs propres , on a donc $A=PDP^{-1}$ avec 
$$D=\pmatrice{2&0\\0&3}$$
Pour tout entier $k \geq 0$, on a (d'après la formule de changement de base appliquée à $u^k$ où $u$ est l'endomorphisme canoniquement associé à $A$) :
$$A^k=PD^kP^{-1}$$
Par calcul, on a : 
$$P^{-1}=\pmatrice{-1&2\\1&-1}$$
puis sachant que $D$ est diagonale : 
$$D^k=\pmatrice{2^k&0\\0&3^k}$$ 
et enfin 
$$A^k=\pmatrice{1&2 \\ 1&1}\times\pmatrice{2^k&0\\0&3^k}\times \pmatrice{-1&2\\1&-1} = \pmatrice{2 \times 3^k-2^k & -2 \times 3^k+2^{k+1}\\[0.15cm]3^k-2^k& -3^k+2^k}$$
On sait que pour tout $k \geq 0$,
$$X_k=A^kX_0$$
donc :
$$\pmatrice{u_k\\v_k}=A^k\pmatrice{1\\2}$$
d'o\`u finalement :
\[
u_k=3 \times 2^k-2 \times 3^k \quad \textrm{ et } \quad v_k=3 \times 2^k-3^k 
\]

\begin{Exercice}{} Soient $E = \mathbb{K}[X]$ et $\varphi \in \mathcal{L}(E)$ défini par $\varphi(P)=XP'$. Déterminer les éléments propres de $\varphi$.
\end{Exercice}

\corr Raisonnons par analyse-synthèse. 

\medskip

\noindent \textit{Analyse.} Soit $\lambda \in \mathbb{K}$ une valeur propre de $\varphi$. Il existe donc un polynôme non nul de $E$ tel que $\varphi(P)=\lambda P$ ou encore $XP' = \lambda P$. On a donc pour tout réel $x$ :
$$ x P'(x) = \lambda P(x)$$
En particulier, $P$ est solution sur $]0,1[$ de l'équation différentielle linéaire homogène du premier ordre suivant :
$$ y' - \dfrac{\lambda}{x} y = 0$$
donc les solutions sont de la forme :
$$ x \mapsto K e^{\lambda \ln(x)} = K x^{\lambda}$$
Si $\lambda$ n'est pas un entier naturel, toutes les dérivées de $x \mapsto x^{\lambda}$ sont non nulles car pour tout entier $n \geq 0$, la dérivée de $x \mapsto x^{\lambda}$ a pour expression :
$$ \lambda (\lambda-1) \times \cdots (\lambda-n+1) x^{\lambda -n}$$
ce qui est impossible car $P$ est un polynôme donc à partir d'un certains rang, toutes ses dérivées sont nulles. Si $\lambda$ est un entier naturel, on en déduit que $P= K X^n$.

\medskip

\noindent \textit{Synthèse.} Pour tout entier $n \geq 0$,
$$ \varphi(X^n) = X(X^n)'=X(nX^{n-1})=nX^n$$
Sachant que $X^n$ est non nul, on en déduit que c'est un vecteur propre de $\varphi$ associé à la valeur propre de $n$.

\medskip

\noindent D'après l'analyse-synthèse, on sait que le spectre de $\varphi$ est $\mathbb{N}$ et que pour tout entier $n \geq 0$,
$$ E_{n}(\varphi) = \textrm{Vect}(X^n)$$

\begin{Exercice}{} Soient $f$ un endomorphisme d'un $\mathbb{K}$-espace vectoriel de dimension finie et $n \in \mathbb{N}^{*}$. On suppose que $0$ est dans le spectre de $f^n$. Montrer que $0$ est aussi dans le spectre de $f$.
\end{Exercice}

\corr Supposons que $0$ est dans le spectre de $f^n$. Alors le noyau de $f^n$ n'est pas réduit au vecteur nul donc $f^n$ n'est pas inversible donc son déterminant est nul. Or :
$$ \textrm{det}(f^n) = \textrm{det}(f)^n$$
On en déduit que $\textrm{det}(f)$ est nul donc $f$ n'est pas inversible et ainsi $f$ n'est pas injective (l'espace vectoriel est de dimension finie) et $0$ est valeur propre de $f$.

\begin{Exercice}{} Soient $n \geq 3$ et $A \in \mathcal{M}_{n}(\mathbb{C})$ une matrice de rang $2$, de trace nulle et telle que $A^n$ soit non nulle. Montrer que $A$ est diagonalisable.
\end{Exercice}

\corr La matrice est de rang $2$ donc son noyau est de dimension $n-2$. La multiplicité de $0$ en tant que valeur propre est donc supérieure ou égale à $n-2$. La matrice $A$ appartenant à $\mathcal{M}_{n}(\mathbb{C})$, son polynôme caractéristique est scindé, unitaire et de degré $n$. On sait aussi que $X^{n-2}$ divise ce polynôme donc il existe deux complexes $a$ et $b$ tels que :
$$ \chi_A(X)= X^{n-2} (X-a)(X-b)$$
On sait que :
$$ \textrm{Tr}(A) = \sum_{\lambda \in \textrm{Sp}(A)} \lambda$$
donc $a+b=0$ et ainsi $b=-a$. On en déduit que :
$$ \chi_A(X)= X^{n-2} (X-a)(X+a)$$
Si par l'absurde, $a$ était nul, alors $\chi_A(X)=X^n$ et d'après le théorème de Cayley-Hamilton, $A^n= 0_n$, ce qui est absurde d'après l'hypothèse. Ainsi $a$ et $-a$ sont deux valeurs propres distinctes de $A$ et la dimension des sous-espaces propres associés est supérieure ou égale à $1$. Ainsi,
$$ \textrm{dim}(E_0(A)) + \textrm{dim}(E_a(A)) + \textrm{dim}(E_{-a}(A)) \geq (n-2)+1+1 = n = \textrm{dim}(\mathcal{M}_n(\mathbb{C})$$
On sait d'après le cours que :
$$   \textrm{dim}(E_0(A)) + \textrm{dim}(E_a(A)) + \textrm{dim}(E_{-a}(A)) \leq \textrm{dim}(\mathcal{M}_n(\mathbb{C})$$
et ainsi :
$$   \textrm{dim}(E_0(A)) + \textrm{dim}(E_a(A)) + \textrm{dim}(E_{-a}(A)) \leq \textrm{dim}(\mathcal{M}_n(\mathbb{C})$$
On en déduit que $A$ est diagonalisable.

\begin{Exercice}{} Soient $A,B \in \mathcal{M}_{n}(\mathbb{K})$. Le but de l'exercice est de montrer que $\chi_{AB} = \chi_{BA}$.
\begin{enumerate}
\item Montrer le résultat quand $A$ est inversible.
\item En déduire le résultat dans ce cas général. \textit{On pourra utiliser, pour $x \in \mathbb{K}$ fixé, les applications $f : \mathbb{K} \rightarrow \mathbb{K}$ et $g : \mathbb{K} \rightarrow \mathbb{K}$ définies par :}
$$ f(t) = \textrm{det}((A-t I_n)B -x I_n) \; \hbox{ et } g(t) = \textrm{det}(B(A-tI_n)-xI_n)$$
\end{enumerate}
\end{Exercice}

\corr 

\begin{enumerate}
\item Pour tout scalaire $x$,
\begin{align*}
\chi_{AB}(x) & = \textrm{det}(AB-xI_n) \\
& = \textrm{det}(ABAA^{-1}-xA A^{-1}) \\
& = \textrm{det}(A(BA-xI_n) A^{-1}) \\
& = \textrm{det}(A)\textrm{det}(BA-xI_n) A^{-1})\textrm{det}(A^{-1}) \\
& = \textrm{det}(A)\textrm{det}(BA-xI_n) A^{-1}) \dfrac{1}{\textrm{det}(A)} \\
& = \textrm{det}(BA-xI_n) A^{-1}) \\
& = \chi_{BA}(x)
\end{align*}
Ainsi, $\chi_{AB}= \chi_{BA}$.

\medskip

\noindent \textit{Remarque.} On peut aussi remarquer que :
$$ AB=  A(BA)A^{-1}$$
Ainsi, $AB$ et $BA$ sont semblables donc ont le même polynôme caractéristique.
\item Soit $x \in \mathbb{K}$ fixé. La matrice $A$ a un nombre fini de valeurs propres donc pour une infinité de scalaire $t$, $A-t I_n$ est inversible et donc d'après la question précédente :
$$ f(t)=g(t)$$
Or, les applications $f$ et $g$ sont polynômiales (en $t$) car leurs coefficients le sont (et par développement). Elle coïncident en une infinité de scalaires dont sont égales. En particulier, $f(0)=g(0)$ et donc :
$$ \textrm{det}(AB-xI_n)= \textrm{det}(BA-xI_n)$$
Ceci est vrai pour tout scalaire $x$. On en déduit que $\chi_{AB}= \chi_{BA}$.
\end{enumerate}

\begin{Exercice}{}
  \begin{enumerate}
  \item  Soit $A \in \mathcal{M}_{n}(\mathbb{C})$ nilpotente. Déterminer $\chi_{A}$.
  \item Même question avec $A \in \mathcal{M}_{n}(\mathbb{R})$.
  \end{enumerate}
\end{Exercice} 

\corr 

\begin{enumerate}
\item La matrice $A$ est nilpotente : il existe un entier $k \geq 1$ tel que $A^k = 0_n$. Le polynôme $X^k$ est donc annulateur de $A$. Les valeurs propres de $A$ appartiennent à l'ensemble des racines de tout polynôme annulateur de $A$ (hors-programme à connaitre) donc la seule valeur propre possible de $A$ est $0$. Le polynôme $\chi_A$ est unitaire, de degré $n$ et scindé (tout polynôme de degré supérieur ou égal à $1$ l'est dans $\mathbb{C}[X]$) donc nécessairement $\chi_A(X)=X^n$.

\medskip

\noindent \textit{Remarque.} En particulier, d'après le théorème de Cayley-Hamilton, on en déduit que $A^n=0_n$. Par exemple une matrice nilpotente de $\mathcal{M}_5(\mathbb{C})$ vérifie nécessairement $M^5=0_5$.
\item Si $A \in \mathcal{M}_n(\mathbb{R})$, $A \in \mathcal{M}_n(\mathbb{C})$ donc d'après la question précédente (la définition de $\chi_A$ ne dépend pas du corps), $\chi_A(X)=X^n$.
\end{enumerate}

\begin{Exercice}{} Pour tout $\begin{pmatrix}
a & b \\
c & d 
\end{pmatrix} \in \mathcal{M}_2(\mathbb{R})$, on pose :
$$ f \left( \begin{pmatrix}
a & b \\
c & d 
\end{pmatrix} \right) = \begin{pmatrix}
-a & c \\
b & -d 
\end{pmatrix}$$

\begin{enumerate}
\item Montrer que $f$ est un endomorphisme de $\mathcal{M}_2(\mathbb{R})$.
\item $f$ est-il diagonalisable ? inversible ?
\item Déterminer les éléments de propres de $f$.
\end{enumerate}
\end{Exercice}

\corr 

\begin{enumerate}
\item Soient $\lambda \in \mathbb{R}$, $M = \begin{pmatrix}
a & b \\
c & d 
\end{pmatrix}$, $N = \begin{pmatrix}
e & f \\
g & h 
\end{pmatrix}$. Alors :
\begin{align*}
f( \lambda M+N) & = f \left( \begin{pmatrix}
\lambda a + e &\lambda b + f \\
 \lambda c + g &\lambda d + h
\end{pmatrix} \right) \\
& = \begin{pmatrix}
-\lambda a - e &\lambda c + g \\
  \lambda b + f&-\lambda d - h
\end{pmatrix} \\
& =  \lambda  \begin{pmatrix}
-a & c \\
b & -d 
\end{pmatrix} + \begin{pmatrix}
-e & g \\
f & -h 
\end{pmatrix} \\
& = \lambda f(M)+f(N)
\end{align*}
De plus, pour tout $M \in \mathcal{M}_2(\mathbb{R})$, $f(M) \in \mathcal{M}_2(\mathbb{R})$. Ainsi, $f$ est un endomorphisme de $\mathcal{M}_2(\mathbb{R})$.
\item Il est clair que $f^2 = \textrm{Id}$. Ainsi, $X^2-1= (X-1)(X+1)$ est un polynôme annulateur scindé à racines simples de $f$ donc $f$ est diagonalisable. Les valeurs propres de $f$ sont nécessairement racines de ce polynôme (hors-programme à connaitre) donc les valeurs propres possibles de $f$ sont $-1$ et $1$. Ainsi, $0$ n'est pas valeur propre de $f$ donc $f$ est bijective.
\item Les valeurs propres possibles de $f$ sont $-1$ et $1$.
\begin{itemize}
\item Soit $M= \begin{pmatrix}
a&b \\
c & d \\
\end{pmatrix} \in \mathcal{M}_2(\mathbb{R})$. Alors :
\begin{align*}
M \in E_1(f) & \Longleftrightarrow f(M)=M \\
& \Longleftrightarrow \begin{pmatrix}
-a&c \\
b & -d \\
\end{pmatrix} = \begin{pmatrix}
a&b \\
c & d \\
\end{pmatrix} \\
& \Longleftrightarrow a=d=0 \; \hbox{ et } \; c=b \\
& \Longleftrightarrow M= b \begin{pmatrix}
0 & 1 \\
1 & 0
\end{pmatrix} 
\end{align*}
Ainsi, $E_1(M) = \textrm{Vect}(M_1)$ où :
$$ M_1 = \begin{pmatrix}
0 & 1 \\
1 & 0
\end{pmatrix}$$
\item Soit $M= \begin{pmatrix}
a&b \\
c & d \\
\end{pmatrix} \in \mathcal{M}_2(\mathbb{R})$. Alors :
\begin{align*}
M \in E_{-1}(f) & \Longleftrightarrow f(M)=-M \\
& \Longleftrightarrow \begin{pmatrix}
-a&c \\
b & -d \\
\end{pmatrix} = \begin{pmatrix}
-a&-b \\
-c & -d \\
\end{pmatrix} \\
& \Longleftrightarrow c=-b\\
& \Longleftrightarrow M=a \begin{pmatrix} 1 & 0 \\
0 & 0 \\ 
\end{pmatrix} + b \begin{pmatrix}
0 & 1 \\
-1 & 0
\end{pmatrix} + c \begin{pmatrix} 0 & 0 \\
0 & 1 \\ 
\end{pmatrix} 
\end{align*}
Ainsi, $E_1(M) = \textrm{Vect}(M_2,M_3,M_4)$ où :
$$ M_2 = \begin{pmatrix} 1 & 0 \\
0 & 0 \\ 
\end{pmatrix} , \; M_3= \begin{pmatrix}
0 & 1 \\
-1 & 0
\end{pmatrix} \; \hbox{ et } M_4 = \begin{pmatrix} 0 & 0 \\
0 & 1 \\ 
\end{pmatrix}  $$
\end{itemize}
\end{enumerate}

\begin{Exercice}{} Soit $A \in \mathcal{M}_{n}(\mathbb{K})$ de rang $1$. Montrer que $A$ est diagonalisable si et seulement si $\textrm{Tr}(A) \neq 0$ \end{Exercice}

\corr La matrice $A$ est de rang $1$ donc son noyau est de dimension $n-1$. Ainsi, $0$ est valeur propre de $A$ avec une multiplicité (en tant que racine de $\chi_A$) supérieure ou égale à $n-1$. Or, $\chi_A$ est unitaire et de degré $n$ donc il existe un scalaire $\lambda$ tel que :
$$ \chi_A(X)=X^{n-1}(X- \lambda)$$
$\chi_A$ est scindé donc la somme des valeurs propres vaut la trace de $A$ et ainsi $\lambda = \textrm{Tr}(A)$. Finalement,
$$ \chi_A(X)= X^{n-1}(X- \textrm{Tr}(A))$$
\begin{itemize}
\item Si la trace de $A$ est non nulle alors $\textrm{Tr}(A)$ est une valeur propre non nulle et la dimension du sous-espace propre associé est supérieure ou égale à $1$. Ainsi,
$$ \textrm{dim}(E_0(A))+ \textrm{dim}(E_{\textrm{Tr}(A)}(A)) \geq (n-1)+1 = n = \textrm{dim}(\mathcal{M}_{n,1}(\mathbb{R}))$$
D'après le cours, on sait aussi que :
$$ \textrm{dim}(E_0(A))+ \textrm{dim}(E_{\textrm{Tr}(A)}(A)) \leq \textrm{dim}(\mathcal{M}_{n,1}(\mathbb{R}))$$
et ainsi :
$$ \textrm{dim}(E_0(A))+ \textrm{dim}(E_{\textrm{Tr}(A)}(A)) = \textrm{dim}(\mathcal{M}_{n,1}(\mathbb{R}))$$
La matrice $A$ est donc diagonalisable.
\item Montrons que si $A$ est diagonalisable alors la trace de $A$ est non nulle. Par contraposée, supposons que la trace de $A$ est nulle. Alors $\chi_A(X)=X^n$. La seule valeur propre de $A$ est $0$. La dimension de $E_0(A)$ est $n-1$ et la multiplicité de $0$ en tant que racine de $\chi_A$ est $n-1$ donc $A$ n'est pas diagonalisable.
\end{itemize}
Ainsi, $A$ est diagonalisable si et seulement si la trace de $A$ est non nulle.


\begin{Exercice}{} Diagonaliser $\begin{pmatrix}
0 & 2 & -1 \\
3 & -2 & 0 \\
-2 & 2 & 1 \\
\end{pmatrix}\cdot$
\end{Exercice}

\corr Notons $A$ cette matrice. On trouve :
$$ \chi_A(X) = (X-2)(X-1)(X+4)$$
La matrice $A$ a trois valeurs propres distinctes et $A$ appartient à $\mathcal{M}_3(\mathbb{R})$ donc $A$ est diagonalisable.
\begin{itemize}
\item Base de $E_1(A)$ : $((1,1,1))$.
\item Base de $E_2(A)$ : $((4,3,-2))$.
\item Base de $E_{-4}(A)$ : $((2,-3,2))$.
\end{itemize}
D'après la formule de changement de base, on a $A = PDP^{-1}$ avec :
$$ P = \begin{pmatrix}
1 & 4 & 2 \\
1 & 3 & -3 \\
1 & -2 & 2
\end{pmatrix}, \; D=\textrm{diag}(1,2,-4)$$
et par la méthode usuelle pour inverser une matrice :
$$ P^{-1} = \dfrac{1}{30} \begin{pmatrix}
0 & 12 & 18 \\
5 & 0 & -5 \\
5 & -6 & 1 
\end{pmatrix}$$

\begin{Exercice}{} Diagonaliser $\begin{pmatrix}
1 & 0& 1 \\
-1 & 2 & 1 \\
0 & 0 & 2 \\
\end{pmatrix}\cdot$
\end{Exercice}

\corr Notons $A$ cette matrice. On trouve :
$$ \chi_A(X) = (X-2)^2(X-1)$$
\begin{itemize}
\item On a : 
$$ A- I_3 = \begin{pmatrix}
0 & 0& 1 \\
-1 & 1 & 1 \\
0 & 0 & 1 \\
\end{pmatrix}$$
Notons $C_1$, $C_2$ et $C_3$ les colonnes de cette matrice. Cette matrice est de rang $2$ car $C_1=C_2 \neq \tilde{0}$ et $C_2$ et $C_3$ sont non colinéaires. Ainsi $\textrm{Ker}(A-I_3)$ est de dimension $1$ d'après le théorème du rang. Sachant que $C_1=C_2$, on en déduit que le vecteur non $\begin{pmatrix}
1 \\
-1 \\
0 \\
\end{pmatrix}$ forme une base de $E_1(A)$.
\item On a : 
$$ A- 2I_3 = \begin{pmatrix}
-1 & 0& 1 \\
-1 & 0 & 1 \\
0 & 0 & 0 \\
\end{pmatrix}$$
Notons $C_1$, $C_2$ et $C_3$ les colonnes de cette matrice. Cette matrice est de rang $1$ car $C_2$ est nulle et que $C_1=C_3 \neq \tilde{0}$. On en déduit que $\textrm{Ker}(A-I_3)$ est de dimension $2$ d'après le théorème du rang. Sachant que $C_1=C_3$ et $C_2= \tilde{0}$, on en déduit que les vecteurs non colinéaires $\begin{pmatrix}
1 \\
0 \\
-1 \\
\end{pmatrix}$ et $\begin{pmatrix}
0 \\
1 \\
0
\end{pmatrix}$ forment une base de $E_2(A)$. 
\end{itemize}
Ainsi,
$$ \textrm{dim}(E_1(A))+ \textrm{dim}(E_2(A)) = 3 = \textrm{dim}(\mathcal{M}_{3,1}(\mathbb{R}))$$
On en déduit que $A$ est diagonalisable. D'après la formule de changement de base, on a $A = PDP^{-1}$ avec :
$$ P = \begin{pmatrix}
1 & 1 & 0\\
-1 & 0 & 1 \\
0 & -1 & 0 \\
\end{pmatrix}, \; D=\textrm{diag}(1,2,2)$$
et par la méthode usuelle pour inverser une matrice :
$$ P^{-1} =  \begin{pmatrix}
1 & 0 & 1 \\
0 & 0 & -1 \\
1 & 1 & 1 
\end{pmatrix}$$

\begin{Exercice}{} Soit $A= \begin{pmatrix}
1 & -3 & 4 \\
4 & -7 & 8 \\
6 & -7 & 7 \\
\end{pmatrix}\cdot$

\begin{enumerate}
\item La matrice $A$ est-elle diagonalisable ? Trigonalisable ?
\item Trigonaliser la matrice $A$. 
\end{enumerate}
\end{Exercice}

\corr 

\begin{enumerate}
\item Par simple calcul, on trouve :
$$ \chi_A(X) = (X+1)^2(X-3)$$
On a :
$$ A+I_3 =  \begin{pmatrix}
2 & -3 & 4 \\
4 & -6 & 8 \\
6 & -7 & 8 \\
\end{pmatrix}$$
En notant $C_1$, $C_2$ et $C_3$, les colonnes de cette matrice, on a :
$$ C_1+2C_2+C_3= \tilde{0}$$
Donc le vecteur non nul $\begin{pmatrix}
1 \\
2 \\
1 \\
\end{pmatrix}$ appartient à $E_{-1}(A)$. Or $C_2$ et $C_3$ sont non colinéaires donc le rang de $A+I_3$ est supérieur ou égal à $2$ et donc égal à $2$ d'après le théorème du rang. Ainsi $E_{-1}(A)$ est de dimension $1$ alors que la multiplicité de $1$ en tant que racine de $\chi_A$ vaut $2$. Ainsi, $A$ n'est pas diagonalisable.

\medskip

\noindent La matrice $A$ est trigonalisable car $\chi_A$ est scindé.
\item Déterminons une base de $E_3(A)$ (qui est de dimension $1$ car $3$ est une valeur propre simple de $A$) :
$$ A- 3I_3 = \begin{pmatrix}
-2 & -3 & 4 \\
4 & -10 & 8 \\
6 & -7 & 4 \\
\end{pmatrix}$$
En notant $C_1$, $C_2$ et $C_3$, les colonnes de cette matrice, on a :
$$ C_1+2C_2+2C_3= \tilde{0}$$
Donc le vecteur non nul $\begin{pmatrix}
1 \\
2 \\
1 \\
\end{pmatrix}$ appartient à $E_{-1}(A)$, qui est de dimension $1$, donc forme une base de cet espace. 

\medskip

\noindent Posons :
$$ X_1 = \begin{pmatrix}
1 \\
2 \\
1 \\
\end{pmatrix}, \; X_2 = \begin{pmatrix}
1 \\
2 \\
2 \\
\end{pmatrix} \; \hbox{ et } \; X_3=\begin{pmatrix}
1 \\
0 \\
0 \\
\end{pmatrix}$$
Le déterminant de la famille $(X_1, X_2,X_3)$ dans la base canonique $\mathcal{B}$ de $\mathcal{M}_{3,1}(\mathbb{R})$ vaut :
$$ \left\vert \begin{array}{ccc}
1 & 1 & 1 \\
2 & 2 & 0 \\
1 & 2 & 0 \\
\end{array}\right\vert = \left\vert \begin{array}{cc}
2 & 2 \\
1 & 2  \\
\end{array}\right\vert = 2 \neq 0$$
Ainsi, $\mathcal{B}'=(X_1, X_2, X_3)$ est une base de $\mathcal{M}_{3,1}(\mathbb{R})$. Soit $f$ l'endomorphisme $\mathcal{M}_{3,1}(\mathbb{R})$ canoniquement associé à $A$. On sait que $f(X_1)=-X_1$, $f(X_2)=3X_3$. On cherche deux réels $\alpha$ et $\beta$ tels que\footnote{On sait que le coefficient de $X_3$ sera $-1$ car $-1$ a pour multiplicité $2$.} :
$$ f(X_3) = \alpha X_1 + \beta X_2 - X_3$$
ou encore :
$$ \begin{pmatrix}
1 \\
4 \\
6 \\
\end{pmatrix} = \alpha \begin{pmatrix}
1 \\
2 \\
1
\end{pmatrix} + \beta \begin{pmatrix}
1 \\
2 \\
2
\end{pmatrix} -  \begin{pmatrix}
1 \\
0 \\
0 \\
\end{pmatrix}$$
ce qui est équivalent à $\alpha+ \beta =2$ et $\alpha+2 \beta =6$ ou encore $\alpha=-2$ et $\beta =4$. On en déduit que :
$$ \textrm{Mat}_{\mathcal{B}'}(f) = \begin{pmatrix}
-1 & 0 & -2 \\
0 & 3 & 4 \\
0 & 0 & -1
\end{pmatrix}$$
Les matrices $A$ et $\textrm{Mat}_{\mathcal{B}'}(f)$ représentent le même endomorphisme de $\mathcal{M}_{3,1}(\mathbb{R})$ donc sont semblables. On en déduit que :
$$ A = P \textrm{Mat}_{\mathcal{B}'}(f) P^{-1}$$
où 
$$P=P_{\mathcal{B}, \mathcal{B}'} = \begin{pmatrix}
1 & 1 & 1 \\
2 & 2 & 0 \\
1 & 2 & 0 \\
\end{pmatrix}$$
La méthode usuelle pour inverser une matrice permet d'obtenir :
$$ P^{-1} = \begin{pmatrix}
0 & 1 & -1 \\
0 & -1/2 & 1 \\
1 & -1/2 & 0 \\
\end{pmatrix}$$

\end{enumerate}

\begin{Exercice}{}  Pour tout $M \in \mathcal{M}_n(\mathbb{R})$, on pose :
$$ \varphi(M) = M +2 \, {}^t M$$
\begin{enumerate}
\item Trouver un polynôme annulateur de $\varphi$.
\item Qu'en déduit-on ?
\item Déterminer les sous-espaces propres ainsi que leurs dimensions.
\item Donner la trace de $\varphi$.
\end{enumerate}
\end{Exercice}

\corr 

\begin{enumerate}
\item L'application $\varphi$ est un endomorphisme de $\mathcal{M}_n(\mathbb{R})$. Pour tout $M \in \mathcal{M}_n(\mathbb{R})$,
\begin{align*}
\varphi^2(M) & = \varphi(\varphi(M)) \\
& =  \varphi(M +2 \, {}^t M) \\
& = M +2 \, {}^t M +2 \, {}^t (M +2 \, {}^t M) \\
& = M +2 \, {}^t M +2 \, {}^t M +4M \\
& = 5M + 4{}^t M \\
& = 3M + 2(M +2{}^t M) \\
& = 3M + 2\varphi(M)
\end{align*}
Ainsi, $X^2-2X-3=(X+1)(X-3)$ est un polynôme annulateur de $\varphi$.
\item Le polynôme annulateur obtenu est scindé à racines simples donc $\varphi$ est diagonalisable.
\item Les valeurs propres possibles de $\varphi$ sont racines du polynôme annulateur obtenu donc sont $-1$ et $3$.
\begin{itemize}
\item Soit $M \in \mathcal{M}_n(\mathbb{R})$. Alors :
\begin{align*}
M \in E_{-1}(\varphi) & \Longleftrightarrow \varphi(M)=-M \\
& \Longleftrightarrow M +2{}^t M = -M \\
& \Longleftrightarrow {}^t M= - M 
\end{align*}
Ainsi, $E_{-1}(\varphi) = \mathcal{A}_n(\mathbb{R})$ dont la dimension est $\dfrac{n(n-1)}{2} \cdot$
\item Soit $M \in \mathcal{M}_n(\mathbb{R})$. Alors :
\begin{align*}
M \in E_{3}(\varphi) & \Longleftrightarrow \varphi(M)=3M \\
& \Longleftrightarrow M +2{}^t M = 3M \\
& \Longleftrightarrow {}^t M=  M 
\end{align*}
Ainsi, $E_{3}(\varphi) = \mathcal{S}_n(\mathbb{R})$ dont la dimension est $\dfrac{n(n+1)}{2}\cdot$
\end{itemize}

\item $\varphi$ est diagonalisable donc la trace de $\varphi$ est égale à la somme des valeurs propres comptées avec les multiplicités (qui sont aussi les dimensions des sous-espaces propres associés) donc :
\begin{align*}
\textrm{Tr}(\varphi)& = \dfrac{n(n-1)}{2} \times (-1) + \dfrac{n(n+1)}{2} \times 3  \\
& = \dfrac{n(-(n-1)+3(n+1))}{2} \\
& = \dfrac{n(2n+4)}{2} \\
& = n(2n+1) 
\end{align*}
\end{enumerate}

\begin{Exercice}{} Soit $A= \begin{pmatrix}
0 & a \\
b & 0
\end{pmatrix} \in \mathcal{M}_n(\mathbb{R})$ où $(a,b) \in \mathbb{R}^2$. 

Donner une condition nécessaire et suffisante sur $(a,b)$ pour que $A$ soit diagonalisable.
\end{Exercice}

\corr On a :
$$ \chi_A(X) = \left\vert \begin{array}{cc}
X & -a \\
-b & X \\
\end{array}\right\vert = X^2-ab$$
\begin{itemize}
\item Si $ab<0$ alors $-ab>0$ donc $\chi_A$ n'est pas scindé sur $\mathbb{R}$ et $A$ n'est pas diagonalisable.
\item Si $ab>0$ alors :
$$ \chi_A(X) = (X- \sqrt{ab}) (X+ \sqrt{ab})$$
donc $A$ est carrée d'ordre deux et admet deux valeurs propres distinctes donc $A$ est diagonalisable.
\item Si $ab=0$, la seule valeur propre de $A$ est $0$ donc $A$ est diagonalisable si et seulement si elle est semblable à la matrice nulle ou encore si et seulement si elle est la matrice nulle. Ainsi, sous l'hypothèse $ab=0$, $A$ est diagonalisable si et seulement si $a$ et $b$ sont nuls.
\end{itemize}
On en déduit que $A$ est diagonalisable si et seulement si $ab>0$ ou $a=0$ et $b=0$.


\begin{Exercice}{} Posons :
    \[
    A =
    \begin{pmatrix}
      1 & 1 & 1 & 1 \\
      2 & 2 & 2 & 2 \\
      3 & 3 & 3 & 3 \\
      4 & 4 & 4 & 4
    \end{pmatrix}
    \]
Déterminer simplement les valeurs propres de $A$. La matrice $A$ est-elle diagonalisable?
\end{Exercice}

\corr La matrice $A$ est de rang $1$ car toutes les colonnes sont égales et non nulles. Ainsi, $E_0(A)$ a pour dimension $3$ donc $0$ est valeur propre de $A$ avec une multiplicité supérieure ou égale à $3$. Or $\chi_A$ est unitaire et de degré $4$ donc il existe un réel $\lambda$ tel que :
$$ \chi_A(X) = X^3(X-\lambda)$$
Le polynôme $\chi_A$ est scindé dans $\mathbb{R}$ donc la somme des valeurs propres de $A$ (comptées avec multiplicité) est égal à la trace de $A$. On en déduit que $\lambda = 10$. Ainsi,
$$ \chi_A(X)=X^3(X- 10)$$
Le réel $10$ est donc valeur propre simple de $A$ donc :
$$ \textrm{dim}(E_0(A))+ \textrm{dim}(E_{10}(A)) = 3+1 = 4 = \textrm{dim}(\mathcal{M}_{4,1}(\mathbb{R}))$$
La matrice $A$ est donc diagonalisable.

\begin{Exercice}{} Soit $A \in \mathcal{M}_{n}(\mathbb{R})$ telle que :
  \[
  A^{3} - A^{2} + A - I_n = 0_n
  \]
Montrer que $\textrm{det}(A) = 1$.
\end{Exercice}

\corr Le polynôme $X^3-X^2+X-1$ est un polynôme annulateur de $A$. On a :
\begin{align*}
X^3-X^2+X-1 & = X^2(X-1)+X-1 \\
& = (X^2+1)(X-1) \\
& = (X+i)(X-i)(X-1)
\end{align*}
Les valeurs propres complexes éventuelles de $A$ sont donc $-i$, $i$ et $1$. Alors on a :
$$ \chi_A(X)= (X+i)^{m(-i)} (X+i)^{m(i)} (X-i)^{m(1)}$$
avec $m(k)$ la multiplicité (éventuellement nulle) de $k \in \lbrace -i, i ,1 \rbrace \cdot$

\medskip

\noindent La matrice $A$ est à coefficients réels donc $\chi_A$ est un polynôme de $\mathbb{R}[X]$ donc $m(-i)=m(i)$. Ainsi,
$$ \chi_A(X) = (X+i)^{m(i)} (X+i)^{m(i)} (X-i)^{m(1)}$$
Sachant que $\chi_A$ est scindé sur $\mathbb{C}$ d'après le cours, on sait que :
$$ \textrm{det}(A) = (-i)^{m(i)} (i)^{m(i)} 1^{m(1)} = (-i^2)^{m(i)} = 1$$

\begin{Exercice}{}
\begin{enumerate}
\item R\'eduire la matrice $A= \left(\begin{array}{rrr}  2 & 2 & 1\\
1 & 3 & 1\\
1 & 2 & 2 \end{array}\right) \cdot $\\
\item Déterminer alors le commutant de $A$, noté $\mathcal{C}(A)$ et définie par : 
$${\cal C}(A)=\{M\in{\cal M}_3(\R) \, \vert \, AM=MA\}$$
\end{enumerate}
\end{Exercice} 

\corr 

\begin{enumerate}
\item Par simple calcul :
$$ \chi_A(X)= (X-1)^2 (X-5)$$
La méthode usuelle donne :
\begin{itemize}
\item $((1,0,-1),(0,1,-2)$ est une base de $E_1(A)$.
\item $((1,1,1)$ est une base de $E_5(A)$.
\end{itemize}
La formule de changement de base implique que $A=PDP^{-1}$ où 
$$ P = \begin{pmatrix}
1& 0 & 1 \\
0& 1 & 1 \\
-1 & -2 & 1 \\
\end{pmatrix}, \; D= \textrm{diag}(1,1,5)$$
et la méthode usuelle pour inverser une matrice donne :
$$ P^{-1} = \dfrac{1}{4} \begin{pmatrix}
3 & -2 & -1 \\
-1 & 2 & -1 \\
1 & 2 & 1 \\
\end{pmatrix}$$
\item Soit $M \in \mathcal{M}_3(\mathbb{R})$. Alors :
\begin{align*}
M \in \mathcal{C}(A) & \Longleftrightarrow AM=MA \\
& \Longleftrightarrow PDP^{-1} M = M PDP^{-1} \\
& \Longleftrightarrow DP^{-1}MP = P^{-1}MPD  \\
& \Longleftrightarrow P^{-1}MP \in \mathcal{C}(D)
\end{align*}
Déterminons alors $\mathcal{C}(D)$. Pour toute matrice $N = \begin{pmatrix}
a & b & c \\
d & e & f \\
g & h & i \\
\end{pmatrix}$, on a :
$$ ND = \begin{pmatrix}
a & b & 5c \\
d & e & 5f \\
g & h & 5i \\
\end{pmatrix} \; \hbox{ et } \; DN = \begin{pmatrix}
a & b & c \\
d & e & f \\
5g & 5h & 5i \\
\end{pmatrix}$$
On en déduit que $N$ appartient à $\mathcal{C}(D)$ si et seulement si $c=f=g=h=0$. Ainsi,
\begin{align*}
\mathcal{C}(D) & = \left\lbrace \begin{pmatrix}
a & b & 0 \\
d & e & 0 \\
0 & 0 & i \\
\end{pmatrix}, \, (a,b,d,e,i) \in \mathbb{R}^5 \right\rbrace \\
& = \textrm{Vect}(E_{1,1}, E_{1,2}, E_{2,1}, E_{2,2},E_{3,3})
\end{align*}
Pour tout $M \in \mathcal{M}_n(\mathbb{R})$,
\begin{align*}
M \in \mathcal{C}(A) & \Longleftrightarrow P^{-1}MP \in \mathcal{C}(D) \\
& \Longleftrightarrow  P^{-1}MP \in \textrm{Vect}(E_{1,1}, E_{1,2}, E_{2,1}, E_{2,2},E_{3,3}) \\
& \Longleftrightarrow \exists (a,b,c,d,e) \in \mathbb{R}^5 \, \vert \, P^{-1}MP = aE_{1,1}+bE_{1,2}+c E_{2,1}+d E_{2,2}+eE_{3,3} \\
& \Longleftrightarrow \exists (a,b,c,d,e) \in \mathbb{R}^5 \, \vert \, M = aPE_{1,1}P^{-1}+bPE_{1,2}P^{-1}+c PE_{2,1}P^{-1}+d PE_{2,2}P^{-1}+ePE_{3,3}P^{-1} \\
\end{align*}
Ainsi,
$$ \mathcal{C}(A) = \textrm{Vect}(PE_{1,1}P^{-1},PE_{1,2}P^{-1},PE_{2,1}P^{-1},PE_{2,2}P^{-1},PE_{3,3}P^{-1}) $$
La famille $(E_{1,1}, E_{1,2}, E_{2,1}, E_{2,2}, E_{3,3})$ est libre (sous-famille de la base canonique de $\mathcal{M}_n(\mathbb{R})$) donc par injectivité de l'endomorphisme de $\mathcal{M}_n(\mathbb{R})$ définie par $M \mapsto PMP^{-1}$, la famille :
$$(PE_{1,1}P^{-1},PE_{1,2}P^{-1},PE_{2,1}P^{-1},PE_{2,2}P^{-1},PE_{3,3}P^{-1})$$
est libre ce qui implique que $\mathcal{C}(A)$ est de dimension $5$.

\end{enumerate}

\begin{Exercice}{} Soit $A= \begin{pmatrix}
2 & 3 & 1 \\
0 & -4 & -2 \\
4 & 12 & 5 \\
\end{pmatrix}\cdot$

\begin{enumerate}
\item Diagonaliser $A$.
\item Montrer que si $B \in \mathcal{M}_n(\mathbb{C})$ est telle que $B^2=A$ alors $A$ et $B$ commutent.
\item Déterminer $\lbrace B \in \mathcal{M}_n(\mathbb{C}) \, \vert \, B^2=A \rbrace \cdot$
\end{enumerate}
\end{Exercice}

\corr 

\begin{enumerate}
\item Par simple calcul, on a :
$$ \chi_A(X) = X(X-1)(X-2)$$
La matrice $A$ est diagonalisable car $A$ appartient à $\mathcal{M}_3(\mathbb{R})$ et a trois valeurs propres distinctes. La méthode usuelle donne :
\begin{itemize}
\item $((1,-2,4))$ est une base de $E_0(A)$.
\item $((1,-2,5)$ est une base de $E_1(A)$.
\item $((3,-4,12))$ est une base de $E_2(A)$.
\end{itemize}
D'après la formule de changement de base, $A=PDP^{-1}$ où :
$$ P = \begin{pmatrix}
1 & 1 & 3 \\
-2 & -2 & -4 \\
4 & 5 & 12 \\
\end{pmatrix}, \; D= \textrm{diag}(0,1,2)$$
et la méthode permettant d'obtenir l'inverse d'une matrice donne :
$$ P^{-1} = \begin{pmatrix}
2 & -3/2 & -1 \\
-4 & 0 & 1 \\
1 & 1/2 & 0 \\
\end{pmatrix}$$
\item Si $B^2=A$ alors $AB=B^2B=B^3=BB^2=BA$.
\item Pour tout $B \in \mathcal{M}_3(\mathbb{C})$,
\begin{align*}
B^2 = A & \Longleftrightarrow B^2 = PDP^{-1} \\
& \Longleftrightarrow P^{-1}B^2 P = D \\
& \Longleftrightarrow P^{-1}BPP^{-1}P = D \\
& \Longleftrightarrow (P^{-1}BP)^2 = D
\end{align*}
Déterminons donc les matrices $N= \begin{pmatrix}
a & b & c \\
d & e & f \\
g & h & i \\
\end{pmatrix}$ de $\mathcal{M}_3(\mathbb{R})$ vérifiant $N^2=D$. Soit $N$ une telle matrice. D'après la question précédente, $ND=DN$. Or on a :
$$ ND = \begin{pmatrix}
0 & b & 2c \\
0 &  e & 2f \\
0 & h & 2i \\
\end{pmatrix} \; \hbox{ et } DN = \begin{pmatrix}
0 & 0 & 0 \\
d & e & f \\
2g & 2h & 2i 
\end{pmatrix}$$
Ainsi, $b=c=d=g=f=h=0$ donc :
$$ N = \textrm{diag}(a,b,c)$$
Or $N^2= D=\textrm{diag}(0,1,2)$ donc $a=0$, $b= -1$ ou $1$ et $c= - \sqrt{2}$ ou $\sqrt{2}$. Réciproquement, toute matrice de cette forme vérifie bien $N^2=D$.

\medskip

\noindent Avec le raisonnement précédent, on en déduit que $B^2=A$ si et seulement si :
$$ P^{-1} B P = \textrm{diag}(0, \pm 1, \pm \sqrt{2})$$
ou encore :
$$ B = P \textrm{diag}(0, \pm 1, \pm \sqrt{2}) P^{-1}$$
ce qui donne $4$ solutions car $N \mapsto P^{-1}MP$ est un endomorphisme injectif de $\mathcal{M}_3(\mathbb{R})$.
\end{enumerate}

\begin{Exercice}{} Soit $A \in \mathcal{M}_{n}(\mathbb{R})$ dont des valeurs propres réelles sont positives. Montrer que le déterminant de $A$ est positif ou nul.
\end{Exercice} 

\corr La matrice $A$ a des coefficients réels donc son polynôme caractéristique appartient à $\mathbb{R}[X]$. Notons $\textrm{Sp}_{\mathbb{R}}(A)$ l'ensemble des valeurs propres réelles de $A$. Remarquons que si $\lambda \in \mathbb{C} \setminus \mathbb{R}$ est une valeur propre de $A$, alors $\overline{\lambda}$ l'est aussi car $\chi_A \in \mathbb{R}[X]$ (et la multiplicité est la même). En notant $\mathcal{S}$ l'ensemble des valeurs propres non réelles de $A$ ayant une partie imaginaire strictement positive, on a (en autorisant les répétitions) en sachant que le déterminant est le produit des valeurs propres :
\begin{align*}
 \textrm{det}(A) & = \left(\prod_{ \lambda \in \textrm{Sp}_{\mathbb{R}}(A)} \lambda \right) \times \left(\prod_{ \lambda \in \mathcal{S}} \lambda  \times \overline{\lambda }\right)  \\
&  = \left(\prod_{ \lambda \in \textrm{Sp}_{\mathbb{R}}(A)} \lambda \right) \times \left(\prod_{ \lambda \in \mathcal{S}} \vert \lambda \vert ^2\right) \geq 0 
 \end{align*}
  car les valeurs propres réelles sont positives.

 
 
\begin{Exercice}{} Déterminer un polynôme annulateur de degré $2$ de :
$$ A = \begin{pmatrix}
1 & 2 & 2 \\
2 & 1 & 2 \\
2 & 2 & 1 \\
\end{pmatrix}$$
En déduire les puissances de $A$. 
\end{Exercice}

\corr On a :
$$ A= 2J-I_3$$
où $J$ est la matrice de $\mathcal{M}_3(\mathbb{R})$ dont tous les coefficients sont égaux à $1$. On a $J^2=3J$. Sachant que $J$ et $I$ commutent, on a :
\begin{align*}
A^2 & = (2J-I_3)^2 \\
& = 4J^2 - 4J+ I_3 \\
& = 12J-4J+I_3 \\
& = 8J+I_3 \\
& = 4(A+I_3) + I_3 \\
& = 4A +5I_3
\end{align*}
Ainsi, $X^2-4X-5=(X+1)(X-5)$ est un polynôme annulateur de $A$.

\medskip

\noindent Pour tout entier $n \geq 1$, d'après le théorème de division euclidienne, il existe un unique couple $(Q_n,R_n)$ de $\mathbb{R}[X]^2$ tel que :
$$ X^n = Q_n(X) (X+1)(X-5) + R_n(X)$$
et tel que $\textrm{deg}(R_n)<\textrm{deg}((X+1)(X-2))=2$. Ainsi, il existe deux réels $a_n$ et $b_n$ tels que :
$$ R_n(X)= a_n X + b$$
On a donc :
$$ X^n = Q_n(X) (X+1)(X-5) + a_n X + b_n$$
En évaluant en $-1$ et en $5$, on obtient :
$$ \left\lbrace \begin{array}{ccl}
-a_n + b_n & = & (-1)^n \\
5a_n + b_n & = & 5^n
\end{array}\right.$$
L'opération $L_2 \leftarrow L_2- L_1$ implique que : $6 a_n = 5^n - (-1)^n$ donc :
$$ a_n = \dfrac{5^n-(-1)^n}{6}$$
et ainsi en injectant dans $L_1$ :
$$ b_n = (-1)^n + \dfrac{5^n-(-1)^n}{6}$$
On en déduit que pour tout entier $n \geq 1$,
$$ X^n = Q_n(X) (X+1)(X-5) +\dfrac{5^n-(-1)^n}{6}  X + (-1)^n + \dfrac{5^n-(-1)^n}{6}$$
et ainsi :
$$ A^n = \dfrac{5^n-(-1)^n}{6}  A + \left((-1)^n + \dfrac{5^n-(-1)^n}{6}\right) I_3$$


\begin{Exercice}{} Soient $A = \begin{pmatrix} -1 &\phantom-2 & 0 \cr 2 &2 &-3 \cr -2 &2 &1 \cr \end{pmatrix}$ et $\varphi$ l'endomorphisme de $\R^3$ canoniquement associé à $A$.

\begin{enumerate}
  \item Vérifier que $A$ n'est pas diagonalisable.
    
  \item Chercher deux vecteurs propres de $A$ linéairement indépendants.
    
  \item Compléter ces vecteurs en une base de $\R^3$.
    
  \item \'Ecrire la matrice de $\varphi$ dans cette base.
        
\end{enumerate}
\end{Exercice} 

\corr

\begin{enumerate}
\item Par simple calcul, on a :
$$ \chi_A(X)=X(X-1)^2$$
On a :
$$ A-I_3 = \begin{pmatrix} -2 &\phantom-2 & 0 \cr 2 &1 &-3 \cr -2 &2 &0 \cr \end{pmatrix}$$
Les deux dernières colonnes sont non colinéaires donc le rang de $A-I_3$ est supérieur ou égal à $2$. Or la somme des trois colonnes est nulle donc le rang est aussi inférieur ou égal à $2$ donc il vaut $2$. D'après le théorème du rang, on en déduit que $E_1(A)$ est de dimension $1$ alors que $1$ a pour multiplicité $2$ en tant que racine de $\chi_A$. Ainsi, $A$ n'est pas diagonalisable.
\item D'après la question précédente, 
$$ X_1 = \begin{pmatrix}
1 \\
1 \\
1 \\
\end{pmatrix}$$ 
est un vecteur propre non nul associé à la valeur propre $1$.
On remarque aussi que :
$$ X_2 = \begin{pmatrix}
1 \\
2 \\
1 \\
\end{pmatrix}$$ 
est un vecteur propre non nul associé à la valeur propre $0$. Les vecteurs $X_1$ et $X_2$ sont non colinéaires donc $(X_1,X_2)$ est libre.
\item Posons :
$$ X_3 = \begin{pmatrix}
1 \\
0 \\
0  \\
\end{pmatrix}$$
et $\mathcal{B}'=(X_1,X_2,X_3)$. Le déterminant de la famille $\mathcal{B}'$ dans la base canonique $\mathcal{B}$ vaut :
$$ \left\vert \begin{array}{ccc}
1 & 1 & 1 \\
1 & 2 & 0 \\
1 & 1 & 0 \\
\end{array}\right\vert = \left\vert \begin{array}{cc}
1 & 2 \\
1 & 1 \\
\end{array}\right\vert = - 1 \neq 0$$
donc $\mathcal{B}'$ est une base de $\mathbb{R}^3$.
\item On sait que $\varphi(X_1)=X_1$ et $\varphi(X_2) = \tilde{0}$. Déterminons deux réels $\alpha$ et $\beta$ tels que :
$$ \varphi(X_3) = \alpha X_1 + \beta X_2 + X_3$$
ou encore :
$$ \begin{pmatrix}
-1 \\
2 \\
-2 \\
\end{pmatrix} = \alpha \begin{pmatrix}
1 \\
1 \\
1 \\
\end{pmatrix} + \beta \begin{pmatrix}
1 \\
2 \\
1 \\
\end{pmatrix} + \begin{pmatrix}
1 \\
0\\
0\\
\end{pmatrix}$$
ce qui est équivalent à :
$$  \begin{pmatrix}
-2 \\
2 \\
-2 \\
\end{pmatrix} = \alpha \begin{pmatrix}
1 \\
1 \\
1 \\
\end{pmatrix} + \beta \begin{pmatrix}
1 \\
2 \\
1 \\
\end{pmatrix} $$
et donc à $\alpha+ \beta =-2$ et $\alpha+2 \beta = 2$. On obtient $\beta =4$ et $\alpha = -6$. Ainsi :
$$ \varphi(X_3) = -6 X_1 + 4 X_2 + X_3$$
et ainsi :
$$ \textrm{Mat}_{\mathcal{B}'}(\varphi) = \begin{pmatrix}
1 & 0 & -6 \\
0 & 0 & 4 \\
0 & 0 & 1 \\
\end{pmatrix}$$
\end{enumerate}

\begin{Exercice}{}  Soient $f$ et $g$ deux endomorphismes diagonalisables d'un espace vectoriel $E$ de dimension finie. Montrer que si $f$ et $g$ commutent alors il existe une base diagonalisante commune à $f$ et $g$.
\end{Exercice}

\corr $f$ est diagonalisable donc :
$$ E = \bigoplus_{\lambda \in \textrm{Sp}(f)} E_{\lambda}(f)$$
Les endomorphismes $f$ et $g$ commutent donc les sous-espaces propres de $f$ sont stables par $g$. Pour toute valeur propre $\lambda$ de $f$, l'endomorphisme induit par $g$ sur $E_{\lambda}(f)$ est diagonalisable car $g$ l'est. Il existe donc une base $\mathcal{B}_{\lambda}$ de $E_{\lambda}(f)$ formée de vecteurs propres de $g$. Ainsi, par concaténation,
$$ \mathcal{B} = \bigcup_{\lambda  \in \textrm{Sp}(f)} \mathcal{B}_{\lambda}$$
est une base de $E$ formée de vecteurs propres de $f$ et de $g$ donc il existe une base diagonalisante commune à $f$ et $g$.

\begin{Exercice}{} Posons, pour $n \geq 1$, la matrice $A_n$ d'ordre $2n$ suivante :
$$ A_n = \begin{pmatrix}
1 & 2n & 1 & \cdots & 2n \\
2 & 2n-1 & 2 & \cdots & 2n-1 \\
3 & 2n-2 & 3 & \cdots & 2n-2 \\
\vdots & \vdots & \vdots &  & \vdots \\
2n & 1 & 2n & \cdots & 1
\end{pmatrix}$$

\begin{enumerate}
\item Déterminer le rang de $A_n$.
\item Montrer que $A_n$ est diagonalisable.
%\item Déterminer ses éléments propres.
\end{enumerate}
\end{Exercice}

\corr 

\begin{enumerate}
\item La matrice $A_n$ est de rang $2$ car elle est de la forme :
$$A_n =[C_1 \, \vert \, C_2 \, \vert \, C_1 \, \vert \, C_2 \, \vert \, \cdots C_1 \, \vert \, C_2]$$
où $C_1$ et $C_2$ sont non nulles et non colinéaires ($1 \times (2n-1)-2\times 2n \neq 0$).
\item La matrice $A_n$ est de rang $2$ donc d'après le théorème du rang, le noyau de $A_n$ est de dimension $2n-2$. 

\medskip

\noindent La somme des termes d'une ligne est constante égale à :
$$ \lambda = n(2n+1)$$
On a ainsi :
$$ A_n \begin{pmatrix}
1 \\
1 \\
\vdots \\
1 \\
\end{pmatrix} = \lambda \begin{pmatrix}
1 \\
1 \\
\vdots \\
1 \\
\end{pmatrix}$$
donc $X = \begin{pmatrix}
1 \\
1 \\
\vdots \\
1 \\
\end{pmatrix}$ (étant non nul) est un vecteur propre de $A_n$ associé à la valeur propre $\lambda$.

\medskip

\noindent Le polynôme caractéristique de $A_n$ est unitaire et de degré $2n$. Il est divisible par $X^{2n-2}$ et par $(X- n(2n+1))$ donc il existe un réel $\mu$ tel que :
$$ \chi_{A_n}(X) = X^{2n-2} (X- n(2n+1)) (X- \mu)$$
Ce polynôme est donc scindé donc d'après le cours, la trace de $A_n$ est égale à la somme des valeurs propres. On en déduit que :
\begin{align*}
 n(2n+1) + \mu &  2 \sum_{k=1}^n (2k-1) \\
 & = 4 \sum_{k=1}^n k - 2 \sum_{k=1}^n 1 \\
 & = 2n(n+1) -2n 
 \end{align*}
 et ainsi :
 $$ \mu = 2n(n+1)-2n-n(2n+1) = n(2n+2-2-2n-1) = -n$$
 On a donc :
 $$  \chi_{A_n}(X) = X^{2n-2} (X- n(2n+1)) (X+n)$$
 Les valeurs propres $n(2n+1)$ et $-n$ sont simples donc la dimension des sous-espaces propres associé est égale à $1$. Ainsi,
 $$ \textrm{dim}(E_0(A_n))+ \textrm{dim}(E_{n(2n+1)}(A_n))+ \textrm{dim}(E_{-n}(A_n)) = 2n-2+1+1=2n = \textrm{dim}(\mathcal{M}_{2n,1}(\mathbb{R}))$$
Ainsi, $A_n$ est diagonalisable.
%\item Le noyau de $A_n$ est de dimension $2n-2$. Notons $C_1$, $\ldots$, $C_{2n-2}$ les colonnes de $A_n$. On a :
%$$ C_{2k-1}-C_{2j-1}= \tilde{0}$$
%pour tout $1 \leq k <j \leq n$ ce qui donne 
%\begin{align*}
%\sum_{1 \leq k <j \leq n} 1 & = \sum_{k=1}^n \sum_{j=k+1}^{n} 1 \\
%& = \sum_{k=1}^n n-k \\
%& = (n-1) + (n-2) + \cdots + 1 + 0 \\
%& = \dfrac{(n-1)n}{2}
%\end{align*}
%vecteurs propres.
\end{enumerate}

i
%\exo Soient $u$, $v$ deux endomorphismes d'un $\mathbb{K}$-espace vectoriel $E$.
%  \begin{enumerate}
%  \item
%    Si $\lambda \neq 0$ est valeur propre de $u \circ v$, montrer que $\lambda$ est aussi valeur propre de $v \circ u$.
%  \item
%    Pour $P \in E = \R[X]$, on pose $u(P) = P'$ et en confondant polynôme et fonction polynômiale, on pose :
%$$\forall x \in \mathbb{R}, \; v(P) = \int_{0}^{x} P(t) \dt$$
%Cela définit des endomorphismes de $E$. Déterminer :
%    \[
%    \textrm{Ker}(u \circ v) \et \textrm{Ker}(v \circ u)
%    \]
%  \item
%    Montrer que la propriété de la première question reste valable pour $\lambda = 0$ si l'espace $E$ est de dimension finie.
%  \end{enumerate}
%
%\medskip
%\begin{center}
%{\large \textit{\underline{Diagonalisation}}}
%\end{center}
%
%\exo Déterminer le polynôme caractéristique de :
%$$ A = \begin{pmatrix}
%0 & 2 & -1 \\
%3 & -2 & 0 \\
%-2 & 2 & 1 \\
%\end{pmatrix}$$
%En déduire le spectre de $A$. La matrice $A$ est-elle diagonalisable ? Si oui, diagonaliser la. Même question pour :
%$$ B = \begin{pmatrix}
%1 & 0 & 0 \\
%0 & 1 & 0 \\
%1 & -1 & 2 \\
%\end{pmatrix}$$
%
%\medskip
%
%\exo La matrice 
%$$ A = \begin{pmatrix}
%1&2&3 \\
%2 & 3 & 1 \\
%3& 1 & 2 \\
%\end{pmatrix}$$
%est-elle diagonalisable ? Déterminer ses sous-espaces propres.
%
%\medskip
%
%\exo Soit $a \in \mathbb{C}$. La matrice 
%$$ A = \begin{pmatrix}
%-1 & a & a^2 \\
%0 & 0 & -a \\
%0 & 0 & 1
%\end{pmatrix}$$
%est-elle diagonalisable ? Déterminer ses sous-espaces propres.
%
%\medskip
%
%
%
%
%\newpage
%
%\exo Considérons la matrice suivante :
%$$ A = \begin{pmatrix}
%0 & 3 & 0 \\
%1 & -2& 4 \\
%1 & 1 & 1 \\
%\end{pmatrix}$$
%
%\begin{enumerate}
%\item Déterminer le polynôme caractéristique de $A$ et en déduire le spectre de $A$.
%\item Déterminer les sous-espaces propres de $A$.
%\item Diagonaliser $A$.
%\item Posons :
%$$ u = \begin{pmatrix}
%1 \\
%-1 \\
%0
%\end{pmatrix}, \; v = \begin{pmatrix}
%1 \\
%1 \\
%1
%\end{pmatrix}, \; u = \begin{pmatrix}
%3 \\
%-1 \\
%-1
%\end{pmatrix}$$
%Montrer que $(u,v,w)$ est une base de $\mathcal{M}_{3,1}(\mathbb{R})$. Déterminer la matrice de $f$, application linéaire canoniquement associée à $A$, dans cette base.
%\item Déterminer $A^{-1}$ (on pourra utiliser $A^2$ et $A$).
%\end{enumerate}
% 
%\medskip
%
%\exo Soit $n \geq 1$. Posons :
%  \[
%  A_{n} =
%  \begin{pmatrix}
%    0 & 1 &  & (0)\\
%    1 & \ddots & \ddots &  \\
%    & \ddots & \ddots & 1 \\
%    (0) &  & 1 & 0
%  \end{pmatrix}
%  \in \mathcal{M}_{n}(\mathbb{C})
% \]
%Notons $P_n= \chi_{A_n}$.
%
%  \begin{enumerate}
%  \item
%    Montrer que pour tout $x \in \mathbb{R}$,
%    \[
%    P_{n}(x) = xP_{n - 1}(x) - P_{n - 2}(x)
%    \]
%Calculer $P_{1}(x)$ et $P_{2}(x)$.
%  \item
%    Pour tout $x \in ] - 2,2[$, on pose $x = 2\cos (\alpha)$ avec $\alpha \in ]0,\pi[$. Montrer que :
%    \[
%    P_{n}(x) = \frac{\sin((n + 1)\alpha)}{\sin (\alpha)}
%    \]
%  \item En déduire que $P_{n}$ admet $n$ racines.
%  \item $A_{n}$ est-elle est diagonalisable?
%  \end{enumerate}
%

%\exo Prouver le théorème de Cayley-Hamilton dans le cas d'une matrice diagonale.
%

%\exo 
%
%\medskip
%
%\exo Déterminer les puissances de $A= \begin{pmatrix}
%-3 & -8 & -6 \\
%1/2 & 3/2 & 1 \\
%3/2 & 7/2& 3\\
%\end{pmatrix}\cdot$

%
%\exo D\'eterminer les suites
%r\'eelles $(u_n)_{n\in\N}, (v_n)_{n\in\N}$ et $(w_n)_{n\in\N}$
%v\'erifiant
%$$\left\{\begin{array}{l}
%u_0=1, v_0=-1, w_0=2\\
%\forall n\in\N, \left\{\begin{array}{l}
%u_{n+1}=-u_n-2w_n\\
%v_{n+1}=2u_n+v_n+2w_n\\
%w_{n+1}=2u_n+3w_n\end{array}\right.
%\end{array}\right.$$
%
%\medskip
%
%
%\exo D\'eterminer l'ensemble des suites r\'eelles
%$(u_n)_{n\in\N}$ qui v\'erifient
%$$\forall n\in\N,\quad 4u_{n+3}+8u_{n+2}+5u_{n+1}+u_n=0.$$
%Quelles sont celles qui convergent?
%
%
%
%\bigskip
%
%
%\begin{center}
%{\large \textit{\underline{Approfondissement}}}
%\end{center}
%
%\exo 
%
%\medskip
%
%\
%\exo Pour tout $n \in \mathbb{N}$, posons : 
%$$\forall x \in \mathbb{R}, \;\ P_n(x) = x^n - \sum_{i=0}^{n-1} \alpha_ix^i$$
%où $\alpha_0 > 0$ et $\alpha_i \geq 0$ pour $1 \leq i \leq n-1$.
%
%\begin{enumerate}
%  \item Montrer qu'il existe une unique racine dans $\R^{+*}$ pour $P_n$.
%    \item Soit $A = \begin{pmatrix} 1       &1      &0      &\dots  &0      \cr
%                        2       &0      &1      &\ddots &\vdots \cr
%                        \vdots  &\vdots &\ddots &\ddots &0      \cr
%                        \vdots  &\vdots &       &\ddots &1      \cr
%                        n       &0      &\dots  &\dots  &0      \cr\end{pmatrix} \cdot$
%
%\noindent Montrer que $A$ admet une unique valeur propre réelle strictement positive.
%    
%\end{enumerate}


\end{document}