\documentclass[a4paper,10pt]{report}
\newcommand{\pmatrice}[1]{\begin{pmatrix}#1\end{pmatrix}}
\usepackage{cours}
\usepackage{pifont}

\begin{document}
\everymath{\displaystyle}
\begin{center}
\textit{{ {\huge TD 10 : Réduction}}}
\end{center}

\bigskip


\noindent Dans la suite, $\mathbb{K} = \mathbb{R}$ ou $\mathbb{C}$ et $n \in \mathbb{N}^*$.

\medskip

\begin{center}
\textit{{ {\large Éléments propres}}}
\end{center}

\medskip

\begin{Exercice}{\ding{80}] Soient $E = \mathbb{R}[X}$ et $\varphi \in \mathcal{L}(E)$ défini par $\varphi(P)=XP'$. Déterminer les éléments propres de $\varphi$.
\end{Exercice}



\begin{Exercice}{} Soient $f$ un endomorphisme d'un $\mathbb{K}$-espace vectoriel de dimension finie et $n \in \mathbb{N}^{*}$. On suppose que $0$ est dans le spectre de $f^n$. Montrer que $0$ est aussi dans le spectre de $f$.
\end{Exercice}


\medskip

\begin{center}
\textit{{ {\large Diagonalisation}}}
\end{center}

\medskip

\begin{Exercice}{} Diagonaliser $\begin{pmatrix}
0 & 2 & -1 \\
3 & -2 & 0 \\
-2 & 2 & 1 \\
\end{pmatrix}\cdot$
\end{Exercice}



\begin{Exercice}{} Diagonaliser $\begin{pmatrix}
1 & 0& 1 \\
-1 & 2 & 1 \\
0 & 0 & 2 \\
\end{pmatrix}\cdot$
\end{Exercice}



\begin{Exercice}{} Soit $A= \begin{pmatrix}
0 & a \\
b & 0
\end{pmatrix} \in \mathcal{M}_n(\mathbb{R})$ où $(a,b) \in \mathbb{R}^2$. 

Donner une condition nécessaire et suffisante sur $(a,b)$ pour que $A$ soit diagonalisable.
\end{Exercice}


\begin{Exercice}{} Posons :
    \[
    A =
    \begin{pmatrix}
      1 & 1 & 1 & 1 \\
      2 & 2 & 2 & 2 \\
      3 & 3 & 3 & 3 \\
      4 & 4 & 4 & 4
    \end{pmatrix}
    \]
Déterminer simplement les valeurs propres de $A$. La matrice $A$ est-elle diagonalisable?
\end{Exercice}



\begin{Exercice}{}\label{Rang} Soit $A \in \mathcal{M}_{n}(\mathbb{K})$ de rang $1$. Montrer que $A$ est diagonalisable si et seulement si $\textrm{Tr}(A) \neq 0$ \end{Exercice}


\begin{Exercice}{} Soient $a \in \mathbb{C}^*$ et $A$ la matrice de $\mathcal{M}_n(\mathbb{C})$ définie par $A=( a^{i+j-2})_{1 \leq i,j \leq n}$.

\begin{enumerate}
%\item Dans le cas où $a \in \mathbb{R}$, montrer que $A$ est diagonalisable.
\item Donner une valeur propre de $a$.
\item Déterminer une condition nécessaire sur $a$ pour que $A$ soit diagonalisable.
\end{enumerate}
\end{Exercice}



\begin{Exercice}{\ding{80}} Posons, pour $n \geq 1$, la matrice $A_n$ d'ordre $2n$ suivante :
$$ A_n = \begin{pmatrix}
1 & 2n & 1 & \cdots & 2n \\
2 & 2n-1 & 2 & \cdots & 2n-1 \\
3 & 2n-2 & 3 & \cdots & 2n-2 \\
\vdots & \vdots & \vdots &  & \vdots \\
2n & 1 & 2n & \cdots & 1
\end{pmatrix}$$

\begin{enumerate}
\item Déterminer le rang de $A_n$.
\item Montrer que $A_n$ est diagonalisable.
%\item Déterminer ses éléments propres.
\end{enumerate}
\end{Exercice}



\medskip

\begin{center}
\textit{{ {\large Polynômes annulateurs}}}
\end{center}

\medskip

\begin{Exercice}{} Soit $A$ la matrice de $\mathcal{M}_n(\mathbb{R})$ avec des coefficients égaux à $1$, sauf ceux de la diagonale valant $0$.

\begin{enumerate}
%\item La matrice est-elle diagonalisable ?
\item Calculer $(A+I_n)^2$.
\item Montrer que si $P$ est un polynôme annulateur de $A$ alors toute valeur propre de $A$ est racine de $P$.
\item La matrice $A$ est-elle diagonalisable ? Déterminer ses valeurs propres de $A$. 
\end{enumerate}
\end{Exercice}


\begin{Exercice}{} Soit $A \in \mathcal{M}_n(\mathbb{R})$, non nulle, vérifiant $A^3+9A=0_n$.
\begin{enumerate}
\item Montrer que les valeurs propres éventuelles de $A$ sont $0$, $3i$ et $-3i$.
\item $A$ est-elle diagonalisable dans $\mathbb{C}$? Dans $\mathbb{R}$?
\item Montrer que $A$ n'est pas inversible si $n$ est impair. 
\end{enumerate}
\end{Exercice}



\begin{Exercice}{} Soit $A = \begin{pmatrix}
1 & 1 & 0 \\
-1 & 0 & 0 \\
2 & 0 & -1 
\end{pmatrix}\cdot$
\begin{enumerate}
\item Montrer qu'il n'existe pas de polynôme annulateur non nul et de degré inférieur ou égal à deux de $A$.
\item Trouver un polynôme annulateur de $A$ et en déduire $A^{-1}$.
\item Quels sont les polynômes annulateurs de $A$?
\end{enumerate}
\end{Exercice}


\begin{Exercice}{}
  \begin{enumerate}
  \item  Soit $A \in \mathcal{M}_{n}(\mathbb{C})$ nilpotente. Déterminer $\chi_{A}$.
  \item Même question avec $A \in \mathcal{M}_{n}(\mathbb{R})$.
  \end{enumerate}
\end{Exercice} 


\begin{Exercice}{\ding{80}} Soient $n \geq 3$ et $A \in \mathcal{M}_{n}(\mathbb{C})$ une matrice de rang $2$, de trace nulle et telle que $A^n$ soit non nulle. Montrer que $A$ est diagonalisable.
\end{Exercice}


\begin{Exercice}{\ding{80}} Soit $A \in \mathcal{M}_{n}(\mathbb{R})$ telle que :
  \[
  A^{3} - A^{2} + A - I_n = 0_n
  \]
Montrer que $\textrm{det}(A) = 1$.
\end{Exercice}


\begin{Exercice}{} Pour tout $\begin{pmatrix}
a & b \\
c & d 
\end{pmatrix} \in \mathcal{M}_2(\mathbb{R})$, on pose :
$$ f \left( \begin{pmatrix}
a & b \\
c & d 
\end{pmatrix} \right) = \begin{pmatrix}
-a & c \\
b & -d 
\end{pmatrix}$$

\begin{enumerate}
\item Montrer que $f$ est un endomorphisme de $\mathcal{M}_2(\mathbb{R})$.
\item $f$ est-il diagonalisable ? inversible ?
\item Déterminer les éléments de propres de $f$.
\end{enumerate}
\end{Exercice}



\begin{Exercice}{} Soient $A \in \mathcal{M}_n(\mathbb{R})$ et $f_A : \mathcal{M}_n(\mathbb{R}) \rightarrow \mathcal{M}_n(\mathbb{R})$ définie $f_A(M)=AM$.

\begin{enumerate}
\item Montrer que $f_A$ est un endomorphisme de $\mathcal{M}_n(\mathbb{R})$.
\item Montrer que si $A^2=A$ alors $f_A$ est un projecteur.
\item Montrer que $A$ est diagonalisable si et seulement $f_A$ l'est.
\end{enumerate}
\end{Exercice}



\begin{Exercice}{} Soit $A \in GL_n(\mathbb{R})$. Montrer que $f$, définie pour tout $M \in \mathcal{M}_n(\mathbb{R})$ par $f(M)= 2 \textrm{Tr}(M) A$, est un endomorphisme de $\mathcal{M}_n(\mathbb{R})$. Est-il diagonalisable ?
\end{Exercice}


\begin{Exercice}{}  Pour tout $M \in \mathcal{M}_n(\mathbb{R})$, on pose :
$$ \varphi(M) = M +2 \, {}^t M$$
\begin{enumerate}
\item Trouver un polynôme annulateur de $\varphi$.
\item Qu'en déduit-on ?
\item Déterminer les sous-espaces propres ainsi que leurs dimensions.
\item Donner la trace de $\varphi$.
\end{enumerate}
\end{Exercice}



\medskip

\begin{center}
\textit{{ {\large Trigonalisation}}}
\end{center}

\medskip

\begin{Exercice}{} Soit $A= \begin{pmatrix}
1 & -3 & 4 \\
4 & -7 & 8 \\
6 & -7 & 7 \\
\end{pmatrix}\cdot$

\begin{enumerate}
\item La matrice $A$ est-elle diagonalisable ? Trigonalisable ?
\item Trigonaliser la matrice $A$. 
\end{enumerate}
\end{Exercice}


\begin{Exercice}{} Soient $A = \begin{pmatrix} -1 &\phantom-2 & 0 \cr 2 &2 &-3 \cr -2 &2 &1 \cr \end{pmatrix}$ et $\varphi$ l'endomorphisme de $\R^3$ canoniquement associé à $A$.

\begin{enumerate}
  \item Vérifier que $A$ n'est pas diagonalisable.
    
  \item Chercher deux vecteurs propres de $A$ linéairement indépendants.
    
  \item Compléter ces vecteurs en une base de $\R^3$.
    
  \item \'Ecrire la matrice de $\varphi$ dans cette base.
        
\end{enumerate}
\end{Exercice} 


\medskip

\begin{center}
\textit{{ {\large Applications de la réduction}}}
\end{center}

\medskip

\begin{Exercice}{} Déterminer un polynôme annulateur de degré $2$ de :
$$ A = \begin{pmatrix}
1 & 2 & 2 \\
2 & 1 & 2 \\
2 & 2 & 1 \\
\end{pmatrix}$$
En déduire les puissances de $A$. 
\end{Exercice}


\begin{Exercice}{}Soit $A= \begin{pmatrix}
1 & 2 & 0 \\
0 & 3 & 0 \\
4 & 0 & -1 \\
\end{pmatrix}\cdot$
\begin{enumerate}
\item Montrer que $A$ est diagonalisable et donner ses valeurs propres.
\item Soit $D$ la matrice diagonale portant les valeurs propres de $A$. Montrer que si une matrice $M \in \mathcal{M}_3(\mathbb{R})$ commute avec $D$ alors elle est diagonale.
\item Soit $P(X)=X^7+ 4X^3+1$. Trouver toutes les matrices $M$ de $\mathcal{M}_3(\mathbb{R})$ telles que $P(M)=A$.
\end{enumerate}
\end{Exercice}


\begin{Exercice}{}
\begin{enumerate}
\item R\'eduire la matrice $A= \left(\begin{array}{rrr}  2 & 2 & 1\\
1 & 3 & 1\\
1 & 2 & 2 \end{array}\right) \cdot $\\
\item Déterminer alors le commutant de $A$, noté $\mathcal{C}(A)$ et défini par : 
$${\cal C}(A)=\{M\in{\cal M}_3(\R) \, \vert \, AM=MA\}$$
\end{enumerate}
\end{Exercice} 



\begin{Exercice}{} Soit $A= \begin{pmatrix}
2 & 3 & 1 \\
0 & -4 & -2 \\
4 & 12 & 5 \\
\end{pmatrix}\cdot$

\begin{enumerate}
\item Diagonaliser $A$.
\item Montrer que si $B \in \mathcal{M}_n(\mathbb{C})$ est telle que $B^2=A$ alors $A$ et $B$ commutent.
\item Déterminer $\lbrace B \in \mathcal{M}_n(\mathbb{C}) \, \vert \, B^2=A \rbrace \cdot$
\end{enumerate}
\end{Exercice} 


\begin{Exercice}{}
Soit $A$ la matrice de $\mathcal{M}_{3}\left(\mathbb{R}\right)$
d\'efinie par
\[
A=\left(\begin{array}{ccc}
1 & -1 & -1\\
-1 & 1 & -1\\
-1 & -1 & 1
\end{array}\right)
\]

\begin{enumerate}
\item Montrer que la matrice $A$ est diagonalisable. D\'eterminer ses valeurs propres et ses sous-espaces propres.
\item D\'eterminer une relation entre $A^{2}$, $A$ et $I_{n}$.\\
En d\'eduire une relation entre $A^{n+1}$, $A^{n}$ et $A^{n-1}$ pour
tout entier $n\geq1$.
\item Montrer par r\'ecurrence qu'il existe deux suites $\left(u_{n}\right)_{n\in\mathbb{N}}$
et $\left(v_{n}\right)_{n\in\mathbb{N}}$ telles que
\[
\forall n\in\mathbb{N},\quad A^{n}=\left(\begin{array}{ccc}
u_{n} & v_{n} & v_{n}\\
v_{n} & u_{n} & v_{n}\\
v_{n} & v_{n} & u_{n}
\end{array}\right)
\]
qui v\'erifient la relation de r\'ecurrence :
\[
\forall n\geq1,\quad\left\{ \begin{array}{ccc}
u_{n+1} & = & u_{n}+2u_{n-1}\\
v_{n+1} & = & v_{n}+2v_{n-1}
\end{array}\right.
\]
\item D\'eterminer, pour tout entier naturel $n$, l'expression de $u_{n}$
et $v_{n}$ en fonction de $n$.
\end{enumerate}
\end{Exercice}




\begin{Exercice}{} Soient $(u_k)_{k \geq 0}$ et $(v_k)_{k \geq 0}$ deux suites \`a termes r\'eels d\'efinies par
$$\left\{\begin{array}{l}u_0=1\\ v_0=2\end{array}\right.\qquad\hbox{et}\qquad \forall k\in\N,\;\left\{\begin{array}{l}u_{k+1}=4u_k-2v_k\\
v_{k+1}=u_k+v_k\end{array}\right.$$
Déterminer pour tout entier $k \geq 0$, l'expression de $u_k$ et $v_k$ en fonction de $k$.
\end{Exercice}


\medskip

\begin{center}
\textit{{ {\large Matrices définies par blocs}}}
\end{center}

\medskip

\begin{Exercice}{\ding{80}} Soit $A= \begin{pmatrix}
1& 2 \\
2 & 1 \\
\end{pmatrix}\cdot$
\begin{enumerate}
\item $A$ est-elle diagonalisable ? $A$ est-elle inversible ? Donner les éléments propres de $A$.
\item  On définit $B = \begin{pmatrix}
A & A \\
A & A 
\end{pmatrix}\cdot$

$B$ est-elle diagonalisable ? Donner les éléments propres de $B$.
\end{enumerate}
\end{Exercice}


\begin{Exercice}{\ding{80}} Soient $A$ et $B$ deux matrices complexes inversibles d'ordres $n \geq 1$. On pose :
$$ N = \begin{pmatrix}
0 & B \\
A & 0
\end{pmatrix}$$
\begin{enumerate}
\item Montrer que $N$ est inversible et déterminer $N^{-1}$.
\item Calculer $N^2$ et $P(N^2)$ pour $P \in \mathbb{C}[X]$.
\item Si $N$ est diagonalisable, $AB$ l'est-elle ? Étudier la réciproque.
\end{enumerate}
\end{Exercice}



\medskip

\begin{center}
\textit{{ {\large Divers}}}
\end{center}

\medskip

\begin{Exercice}{} Soit $A \in \mathcal{M}_{n}(\mathbb{R})$ dont des valeurs propres réelles sont positives. Montrer que le déterminant de $A$ est positif ou nul.
\end{Exercice} 




\begin{Exercice}{}
\begin{enumerate}
\item Deux matrices carrées $A$ et $B$ d'ordre $n \geq 2$ possèdent le même spectre. Sachant que leurs valeurs propres sont deux à deux distinctes, montrer que $A$ et $B$ sont semblables.
\item Donner deux matrices possédant les mêmes valeurs propres mais qui ne soient pas semblables.
\end{enumerate}
\end{Exercice} 





\begin{Exercice}{\ding{80}} \begin{enumerate}
\item Soit $E$ un $\mathbb{K}$-espace vectoriel de dimension $3$. Montrer que $H$, plan de $E$, est stable par $u \in \mathcal{L}(E)$ si et seulement si il existe $\lambda \in \mathbb{K}$ tel que $\textrm{Im}(u- \lambda \textrm{Id}) \subset H$.
\item Trouver tous les sous-espaces de $\mathcal{M}_{3,1}(\mathbb{R})$ stables par l'application linéaire canoniquement associée à 
$$A= \begin{pmatrix}
-1 & 2 & -3 \\
-2 & -5 & -2 \\
-2 & 2 & 0 \\
\end{pmatrix} $$
\end{enumerate}
\end{Exercice}



\begin{Exercice}{\ding{80}} Soient $A \in \mathcal{M}_{3,2}(\mathbb{R})$ et $B \in \mathcal{M}_{2,3}(\mathbb{R})$ telles que :
$$ AB = \begin{pmatrix}
1 & 0 & x \\
0 & 1 & 0 \\
1 & 0 & 1 
\end{pmatrix}$$
où $x \in \mathbb{R}$.
\begin{enumerate}
\item La matrice $AB$ est-elle inversible ? Quelles sont les valeurs possibles de $x$ ?
\item La matrice $BA$ est-elle diagonalisable?
\item Montrer que $\mathbb{R}^3 = \textrm{Im}(A) \oplus \textrm{Ker}(B)$.
\item Montrer qu'il existe une infinité de couples $(A,B)$ vérifiant l'hypothèse donnée dans l'énoncé.
\end{enumerate}
\end{Exercice}



\begin{Exercice}{} Soient $A,B \in \mathcal{M}_{n}(\mathbb{K})$. Le but de l'exercice est de montrer que $\chi_{AB} = \chi_{BA}$.
\begin{enumerate}
\item Montrer le résultat quand $A$ est inversible.
\item En déduire le résultat dans ce cas général. \textit{On pourra utiliser, pour $x \in \mathbb{K}$ fixé, les applications $f : \mathbb{K} \rightarrow \mathbb{K}$ et $g : \mathbb{K} \rightarrow \mathbb{K}$ définies par :}
$$ f(t) = \textrm{det}((A-t I_n)B -x I_n) \; \hbox{ et } g(t) = \textrm{det}(B(A-tI_n)-xI_n)$$
\end{enumerate}
\end{Exercice}


\begin{Exercice}{\ding{80}}  Soient $f$ et $g$ deux endomorphismes diagonalisables d'un espace vectoriel $E$ de dimension finie. Montrer que si $f$ et $g$ commutent alors il existe une base diagonalisante commune à $f$ et $g$.
\end{Exercice}



\end{document}