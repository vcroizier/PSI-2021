\documentclass[a4paper,10pt]{report}
\usepackage{cours}
\usepackage{pifont}

\begin{document}
\everymath{\displaystyle}

\begin{center}
\textit{{ {\huge TD 13 : Fonctions définies par une intégrale}}}
\end{center}


\bigskip

\begin{center}
\textit{{ {\large Théorème fondamental de l'analyse}}}
\end{center}

\medskip

\begin{Exercice}{} Soit $f$ définie par 
$$f(x) = \int_{1/x}^{3x}\frac{1}{\sqrt{t^6+1}}\dt$$ 
\begin{enumerate}
\item Déterminer l'ensemble de définition $\mathcal{D}$ de $f$.
\item Étudier la dérivabilité de $f$ sur $\mathcal{D}$ et donner l'expression de sa dérivée.
\end{enumerate}
\end{Exercice} 



\begin{Exercice}{} Justifier que la fonction $H$ définie par 
$$H(x) = \int_{-x}^{x^2} \ln(1+t^4) \dt$$
est de classe $\mathcal{C}^1$ sur $\mathbb{R}$ et donner l'expression de sa dérivée.
\end{Exercice}


\begin{Exercice}{} Trouver toutes les fonctions $f$ continues sur $\mathbb{R}$ telles que pour tout $x \in \mathbb{R}$,
$$ f(x) + \int_{0}^x (x-t)f(t) \dt =1$$
\end{Exercice} 


\begin{Exercice}{} Soit $g : \mathbb{R} \rightarrow \mathbb{R}$ une fonction continue. On pose, pour tout $x \in \mathbb{R}$,
    \[
    f(x) = \int_0^x {\sin(x - t)g(t) \dt}
    \]
    \begin{enumerate}
      \item Montrer que $f$ est dérivable et déterminer sa dérivée.
      \item Montrer que $f$ est solution de l'équation différentielle $y'' + y = g(x)$.
      \item Résoudre cette équation différentielle.
    \end{enumerate}
\end{Exercice}





\bigskip

\begin{center}
\textit{{ {\large Intégrales à paramètres}}}
\end{center}

\medskip

\begin{Exercice}{} Étudier la continuité sur $\mathbb{R}$ de $\dis x \mapsto \int_{0}^1 \sin(xt^2) \dt$.
\end{Exercice}



\begin{Exercice}{} Déterminer l'ensemble de définition de la fonction $f$ définie par :
$$ f(x) = \int_{0}^{+ \infty} \dfrac{\dt}{t^x(1+t)}$$
Étudier ensuite la continuité de $f$ sur cet ensemble de définition.
\end{Exercice}




\begin{Exercice}{} 
\begin{enumerate}
\item Montrer que l'ensemble de définition de $g$, définie par 
$$ g(x) = \int_{0}^1 \dfrac{\ln(1+xt)}{t} \dt$$
contient $]-1,1[$.
\item Montrer que $g$ est de classe $\mathcal{C}^1$ sur $]0,1[$ et déterminer $g'$.
\end{enumerate}
\end{Exercice}



\begin{Exercice}{\ding{80}} Soit $f : \dis x \mapsto \int_{0}^{+ \infty} \dfrac{\arctan(xt)}{t(1+t^2)} \dt$.
\begin{enumerate}
\item Déterminer l'ensemble de définition de $f$, notée $\mathcal{D}$.
\item Montrer que $f$ est de classe $\mathcal{C}^1$ sur $\mathcal{D}$.
\item Donner l'expression de $f$ à l'aide de fonctions usuelles.
\end{enumerate}
\end{Exercice}




\begin{Exercice}{}
\begin{enumerate}
\item Déterminer l'ensemble de définition de la fonction $f$ définie par :
$$ f(x) = \int_{0}^{+ \infty} e^{-t^2} \cos(xt) \dt$$
\item Déterminer $f$ en vérifiant que celle-ci est solution d'une équation différentielle d'ordre $1$.
\end{enumerate}
\end{Exercice}



\begin{Exercice}{} Soit $f$ la fonction d\'efinie par 
$$f(x)=\int_0^1 \frac{t^x}{1+t} \dt$$
	\begin{enumerate}
	\item Montrer que $f$ est bien d\'efinie et continue sur $I=]-1, + \infty[$.

	\item Montrer que :
$$\forall x\in I,\quad f(x+1)+f(x)=\frac1{x+1}$$

	\item Montrer que :
$$\forall x\in I,\quad 0\leq f(x) \leq \frac1{x+1}$$

	\item Montrer que la fonction $f$ est d\'ecroissante sur $I$.

	\item En d\'eduire que :
$$\forall x\in]0,+\infty[,\quad \frac1{2(x+1)} \leq f(x) \leq \frac1{2x}$$

	\item Donner, en le justifiant, un \'equivalent simple de $f(x)$ au voisinage de $+\infty$.
	\end{enumerate}
\end{Exercice}



\begin{Exercice}{\ding{80}} On considère la fonction $f : x \mapsto \int_0^{+ \infty} \dfrac{e^{-xt}}{\sqrt{1+t}} \dt$.
\begin{enumerate}
\item Déterminer l'ensemble de définition de $f$. On le notera $\mathcal{D}$.
\item Montrer que $f$ est de classe $\mathcal{C}^1$ sur $\mathcal{D}$ et est solution d'équation différentielle linéaire que l'on précisera.
\item Déterminer les limites de $f$ aux bornes de $\mathcal{D}$.
\end{enumerate}
\end{Exercice}



\begin{Exercice}{} Pour tout $x \in \mathbb{R}_+$, on pose :
 $$ F(x) = \int_{0}^1 \frac{e^{-x^2(t^2+1)}}{t^2+1} \dt, \; G(x) = \left( \int_{0}^x e^{-t^2} \dt \right)^2, \; H(x)=F(x)+G(x)$$
 
 \begin{enumerate}
 \item Montrer que $F$ est de classe $\mathcal{C}^1$ sur $\mathbb{R}_+$ et donner une expression sous forme d'une intégrale.
 \item Montrer que $G$ est de classe $\mathcal{C}^1$ sur $\mathbb{R}_+$ et donner une expression sous forme d'une intégrale.
 \item Justifier que $H$ est une fonction constante et déterminer celle-ci.
 \item En déduire la valeur de l'intégrale de Gauss $\dis \int_{0}^{+ \infty} e^{-t^2} \dt$.
 \end{enumerate}
\end{Exercice}






%%\exo
%%Soit $f$ la fonction définie sur $\mathbb{R}$ par $\displaystyle{f(x)=\int_x^{2x} e^{-t^2}\;\text{d}t}$.
%%\begin{enumerate}
%%\item Montrer que $f$ est une fonction impaire.\\
%%\textit{Indication :} On pourra faire le changement de variable $u=-t$.
%%\item Montrer que $f$ est de classe $\mathcal{C}^1$ sur $\mathbb{R}$ et calculer $f'$.
%%\item Étudier les variations de $f$.
%%\item En encadrant $f(x)$, déterminer $\displaystyle{\lim_{x \rightarrow +\infty} f(x)}$.
%%\end{enumerate}
%
%\exo
%Soit $f$ la fonction définie par $\displaystyle{f(x)=\frac 1 {x-1}\int_1^x \frac{t^2}{t^8+1} \;\text{d}t.}$
%\begin{enumerate}
%\item Montrer que $f$ est définie et continue sur $\mathbb{R} \setminus \{1\}$.
%%\item Soit $\varphi : x \mapsto \int_1^x \frac{t^2}{t^8+1} \;\text{d}t$. Montrer que $\varphi$ est de classe $\mathcal C^1$ sur $\R$ donner son développement limité en 1.
%\item Montrer que $f$ est prolongeable par continuité en 1.
%\end{enumerate}
%% Sur DEPUY
%\exo Étudier la fonction $x \mapsto \int_{x}^{2x} \dfrac{\dt}{\sqrt{1+t^2+t^4}} \dt$ sur $\mathbb{R}$ : parité, dérivabilité et variations, tangente au point d'abscisse $0$, limite en $+ \infty$.
%
%\exo Trouver toutes les fonctions $f$ continues sur $\mathbb{R}$ telles que pour tout $x \in \mathbb{R}$,
%$$ f(x) + \int_{0}^x (x-t)f(t) \dt =1$$
%

%\exo \begin{enumerate}
%  \item On pose pour tout $x>0$ :
%    \[
%    f(x) = \int_{0}^{1} \frac{t^{x - 1}}{1 + t} \dt
%    \]
%    Justifier que $f$ est bien définie sur $\mathbb{R}_+^{*}$.
%  \item
%    Justifier la continuité de $f$ sur son ensemble de définition.
%  \item
%    Calculer $f(x) + f(x + 1)$ pour $x > 0$.
%  \item
%    Donner un équivalent de $f(x)$ quand $x \rightarrow 0^{+} $ et la limite de $f$ en $ + \infty$.
%  \end{enumerate}
%  
%\exo Soit $a,b$ deux réels strictement positifs.
%  \begin{enumerate}
%  \item
%    Justifier l'existence pour tout $x \in \R$ de
%    \[
%    F(x) = \int_{0}^{ + \infty} \frac{\e^{ - at} - \e^{ - bt}}{t}\cos(xt) \dt
%    \]
%  \item
%    Justifier que $F$ est de classe $\mathcal{C}^{1}$ sur $\R$ et calculer $F'(x)$.
%  \item Donner une expression simple de $F(x)$ pour $x \in \mathbb{R}$.
%  \end{enumerate}
%  
%  \newpage
%  
%  \exo Pour $x > 0$, on pose
%  \[
%  F(x) = \int_{0}^{\pi / 2} \ln \bigl( \cos^{2}(t) + x^{2} \sin^{2}(t) \bigr) \dt
%  \]
%  \begin{enumerate}
%  \item
%    Justifier que $F$ est définie et de classe $\mathcal{C}^{1}$ sur $]0, + \infty[$.
%  \item Calculer $F'(x)$ pour tout réel $x>0$.
%  \item En déduire un expression simple de $F(x)$ pour $x>0$.
%  \end{enumerate}
%  
%  \exo 
%  \begin{enumerate}
%  \item
%    Justifier la convergence de l'intégrale
%    \[
%    I = \int_{0}^{ + \infty} \frac{\sin t}{t} \dt
%    \]
%  \item
%    Pour tout $x \geq 0$, on pose :
%    \[
%    F(x) = \int_{0}^{ + \infty} \frac{\e^{ - xt} \sin t}{t} \dt
%    \]
%    Déterminer la limite de $F$ en $ + \infty$.
%  \item
%    Justifier que $F$ est dérivable sur $]0, + \infty[$ et calculer $F'$
%  \item En admettant la continuité de $F$ en 0 déterminer la valeur de $I$.
%  \end{enumerate}
%  
% 
% 
% \exo Montrer que $f : x \mapsto \int_{0}^{+ \infty} \dfrac{e^{-tx}}{1+t^2} \dt$ est de classe $\mathcal{C}^{\infty}$ sur $]0, + \infty[$ et montrer qu'elle est solution d'une équation différentielle d'ordre deux.
\end{document}