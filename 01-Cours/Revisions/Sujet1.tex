\documentclass[french,11pt,twoside]{VcCours}
\newcommand{\dt}{\text{d}t}
\newcommand{\dx}{\text{d}x}

\renewcommand{\trou}[1]{{\color{white}#1}}
%\renewcommand{\trou}[1]{{\color{blue}#1}}

\begin{document}

\Titre{PSI}{Promotion 2021--2022}{Mathématiques}{Oral Blanc CCINP - sujet 1}

% \tableofcontents
% \separationTitre

\begin{Exercice}
  Soit $J_n$ la matrice de $\mathcal{M}_n(\mathbb{R})$ formée uniquement de $1$. On pose $A_{n+1} = \begin{pmatrix}
    J_n & 0_{n,1} \\
    0_{1,n} & n \\
    \end{pmatrix}\cdot$
    
    \begin{enumerate}
    \item Justifier que $A_{n+1}$ est diagonalisable.
    \item Diagonaliser $J_n$.
    \item Trouver un élément du noyau de $A_{n+1}$ grâce à un élément de celui de $J_n$.
    \item Diagonaliser $A_{n+1}$.
    \end{enumerate}
    
\end{Exercice}


\begin{Exercice}
  On considère l'endomorphisme $u : \mathcal{M}_n(\mathbb{C}) \rightarrow \mathcal{M}_n(\mathbb{C})$ définie par :
  $$ u(M) = \dfrac{1}{2} (2M- {}^t M)$$
  \begin{enumerate}
  \item Rechercher un polynôme annulateur de $u$.
  \item Montrer que $u$ est diagonalisable.
  \item Déterminer la trace et le déterminant de $u$.
  \end{enumerate}
\end{Exercice}

\end{document}
