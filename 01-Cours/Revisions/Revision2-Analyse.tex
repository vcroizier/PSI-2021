\documentclass[french,11pt,twoside]{VcCours}
\newcommand{\dt}{\text{d}t}
\newcommand{\dx}{\text{d}x}

\renewcommand{\trou}[1]{{\color{white}#1}}
%\renewcommand{\trou}[1]{{\color{blue}#1}}

\begin{document}

\Titre{PSI}{Promotion 2021--2022}{Mathématiques}{Révisions d'analyse}

% \tableofcontents
% \separationTitre

\begin{Exercice}
  Déterminer le DSE des fonctions suivantes. Vous donnerez également 
  l'intervalle $]-r;r[$ sur lequel l'égalité a lieu entre
  la fonction et son DSE.
  \begin{multicols}{3}
  \begin{enumerate}
  \item $\frac{1}{(1-x)^2}$
  \item $\frac1{a-x}$
  \item $\frac {e^x}{1-x}$
  \item $\ln(1+2x^2)$
  \item $\ln(1+x-2x^2)$
  \item $\ln(1+x+x^2)$
  \end{enumerate}
  \end{multicols}
\end{Exercice}


\begin{Exercice}
  Soit $\theta\in\R^*$. Résoudre le système différentiel $X'=AX$ avec 
  $A=\matrice{0&\theta\\-\theta&0}$.
\end{Exercice}


\begin{Exercice}
  Posons
  \[I=\int_0^{+\infty}\frac1{1+t^4}dt
  \qquad\et\qquad 
  J=\int_0^{+\infty}\frac{t^2}{1+t^4}dt\]
  \begin{enumerate}
    \item Montrer que $I$ et $J$ convergent.
    \item Montrer que $I = J$ en effectuant le changement de variable $x=\frac1t$.
    \item Calculer $I+J$ en effectuant le changement de variable $x=t-\frac1t$.
    En déduire la valeur de $I$.
  \end{enumerate}
\end{Exercice}


\begin{Exercice}
  Soit $f$ la fonction :
  \[f(x)=\int_0^{+\infty}e^{-t}\frac{\sin(xt)}{t}dt\]
  \begin{enumerate}
    \item Quelle est le domaine de définition de $f$ ?
    \item Montrer que $f$ est de classe $C^1$ sur $\R$ puis calculer $f'$.
    \item En déduire la valeur de $f(x)$.
\end{enumerate}
\end{Exercice}


\begin{Exercice}
  Considérons la courbe :
  \[\sys{x(t)=a\cos(t)(1+\cos(t))\\
      y(t)=a\sin(t)(1+\cos(t))
    }
  \]
  pour $t\in[0;2\pi]$.
\begin{enumerate}
  \item Étudier et tracer la courbe.
  \item Déterminer la longueur de la courbe.
\end{enumerate}
\end{Exercice}


\begin{Exercice}
  Déterminer le rayon de convergence de la série $\sum a_nz^n$ si la suite $(a_n)$ est périodique, et non identique nulle.
\end{Exercice}


\begin{Exercice}
  Considérons les fonctions $f_n$ définie de $\R$ dans $\R$ par :
  \[fn(x)=\frac{x}{1+nx^2}\]
  \begin{enumerate}
    \item Montrer que $f_n$ converge uniformément sur $\R$. Identifier la limite $f$.
    \item Montrer que $f_n'$ converge simplement sur $\R$. Identifier la limite $g$. 
    En déduire qu'il n'y a pas convergence uniforme.
\end{enumerate}
\end{Exercice}


\begin{Exercice}
  Montrer que l'équation différentielle :
  \[x\ln(x)y' + y = x\]
  admet une unique solution sur $]0;+\infty[$.
\end{Exercice}


\begin{Exercice}
  Montrer que la série $\sum\frac{[\sqrt{n+1}]-[\sqrt{n}]}{n}$
    est convergente. Déterminer sa limite.
\end{Exercice}


\begin{Exercice}
  Soient $a$ et $b$ deux réels tels que $0<a<b$. Justifier la convergence puis calculer :
  \[I=\int_a^b\frac1{\sqrt{(b-x)(x-a)}}dx\]
  Pour calculer l'intégrale on pourra mettre $(b-x)(x-a)$ sous forme canonique puis effectuer un changement de variable
  adéquate.
\end{Exercice}


\begin{Exercice}
  Considérons l'application $f$ définie par :
  \[f(x,y)=\frac{xy(x^2-y^2)}{x^2+y^2}\]
  \begin{enumerate}
    \item Montrer que $f$ est prolongeable par continuité en $(0,0)$. 
    On note encore $f$ ce prolongement.
    \item Montrer que $f$ est de classe $C^1$ sur $\R^2$.
    \item Montrer que $f$ n'est pas de classe $C^2$ sur $\R$.
\end{enumerate}
\end{Exercice}


\begin{Exercice}
  Déterminer le rayon de convergence de la série entière
  \[\sum_{n\geq0}\frac{3n}{n+2}x^n\]
  puis calculer sa somme.
\end{Exercice}


\begin{Exercice}
  Déterminer l'ensemble de définition et de continuité de la fonction :
  \[F(x)=\int_0^{+\infty}\frac{e^{-tx}{1+t^2}}dt\]
  Préciser $F(0)$ et $\lim_{x\to+\infty}F(x)$.
\end{Exercice}


\begin{Exercice}
  Soit $k$ un réel positif. On considère le système différentiel :
  \[\systeme{
      x''_1(t)=k(x_2(t)-x_1(t))-kx_1(t)\\
      x''_2(t)=k(x_3(t)-x_2(t))+k(x_1(t)-x_2(t))\\
      x''_3(t)=k(x_2(t)-x_3(t))-kx_3(t)
    }
  \]
  Soit $X=\vecteur{x_1\\x_2\\x_3}$ et $A$ la matrice telle que $X''=AX$.
  \begin{enumerate}
    \item Trouver les valeurs propres de $A$
    et la matrice $P$ telle que $P^{-1}AP$ est diagonale.
    \item Résoudre ce système différentiel.
\end{enumerate}
\end{Exercice}


\begin{Exercice}
  Étudier et tracer la courbe :
  \[\systeme{x(t)=cos(2t)\\y(t)=\tan(t)\cos(2t)}\]
\end{Exercice}


\begin{Exercice}
  Soit $(an)$ une suite décroissante de $\R$ tels que la série $\sum a_n$ converge.
  \begin{enumerate}
    \item Notons $(R_n)$ la suite des restes de cette série, c'est-à-dire $R_n =
    \sum_{k=n+1}^{+\infty}a_k$. Montrer que $R_n\tend_{n\to+\infty}0$.
    \item Montrer que pour tout $n$ de $\N$, on a :
  \[\systeme{0\leq(2n)a_{2n}\leq2R_n\\0\leq(2n+1)a_{2n+1}\leq2R_n}\]
  \item En déduire que $na_n\tend_{n\to+\infty}0$.
  \item En considérant la suite $a_n=\frac1n$ si $n$ est un carré et $u_n = 0$ sinon, 
  montrer que l'hypothèse $(a_n)$ décroissante est indispensable.
\end{enumerate}
\end{Exercice}


\begin{Exercice}
  Déterminer le rayon de convergence de la série entière $\sum a_nz^n$ où $(an)$ 
  est la suite déterminée par $a_0 = \alpha$, $a_1 = \beta$ et
  $a_{n+2} = 2a_{n+1}-a_n$ pour tout $n$ dans $\N$.
\end{Exercice}


\begin{Exercice}
  Considérons l'intégrale :
  \[I=\in_0^1\frac{\ln(1-t)\ln(t)}{t}dt\]
  \begin{enumerate}
    \item Montrer que la série $\sum\frac{x^n}{n+1}$ converge simplement vers $-\frac{\ln(1-x)}{x}$.
    \item En déduire que : $I=\sum_{n=1}^{+\infty}\frac{1}{n^3}$.
\end{enumerate}
\end{Exercice}


\begin{Exercice}
  Soit $(f_n)$ une suite croissante de fonctions continues convergeant vers $f$ continue, sur un segment $[a,b]$ de $\R$. 
  On se propose de montrer que la convergence est uniforme.
  \begin{enumerate}
    \item Soit $(F_n)$ une suite décroissante de fermés de $[a,b]$ telle que chaque $F_n$ soit non vide.
    \begin{enumerate}
      \item Montrer que $F_n$ est compact pour tout $n$ de $\N$. En déduire que $u_n = \max(F_n)$ existe.
      \item Montrer que $(u_n)$ est décroissante. En déduire que $(u_n)$ est convergente. 
      Notons $l$ sa limite.
      \item Montrer que $l\in\bigcap_{n\in\N}F_n$ et donc que $\bigcap_{n\in\N}F_n$ est non vide.
\end{enumerate}
\item Soit $\epsilon$ positif. Considérons les ensembles : $Fn = \ensemble{x\in X}{f(x)-f_n(x) \geq \epsilon}$.
  \begin{enumerate}
    \item Montrer que $(F_n)$ est une suite décroissante de fermés.
  \item Montrer que l'intersection des $F_n$ est vide.
  \item En déduire : $\exists n\in\N/\ \forall x\in X,\ f(x)-f_n(x) \leq \epsilon$.
  \item Monter que la convergence est uniforme.
\end{enumerate}
\item Montrer en étudiant la suite $1-x^n$ sur $[0;1]$ que la continuité de $f$ est indispensable.
\end{enumerate}
\end{Exercice}


\begin{Exercice}
  Notons
  \[I=\int_0^{+\infty}\frac{e^{-x}-e^{-2x}}{x}dx\]
  \begin{enumerate}
    \item Justifier l'existence de $I$.
    \item Montrer que $\forall t\in\R_+^*$, $\int_t^{+\infty}\frac{e^{-x}-e^{-2x}}{x}dx
    =\int_{t}^{2t}\frac{e^{-y}}{y}dy$
    \item En déduire la valeur de $I$. On pourra utiliser $\int_{t}^{2t}\frac{1}{x}dx$
\end{enumerate}
\end{Exercice}


\begin{Exercice}
  Soit $f$ l'application de $\R^2$ dans $\R$ définie par :
  $f(x,y) = x^2 + y^2-2x-4y$
  \begin{enumerate}
    \item Déterminer les extrema de la fonction.
    \item $f$ admet-elle un maximum global sur $\R^2$ ? Préciser le minimum global sur $\R^2$.
    \item Soit \[D = \ensemble{(x,y)\in\R^2}{0 \leq x \leq 2\et0 \leq y \leq x}\]
    Représenter l'ensemble $D$.
    \item Quels sont les extrema globaux de $f$ sur $D$ ?
\end{enumerate}
\end{Exercice}


\begin{Exercice}
  Dans les $3$ cas suivants, on prolonge la fonction en $(0;0)$ par $0$. Déterminer si ces fonctions sont continues :
  \[f_1(x,y)=\frac{xy^2-3x^3}{x^2+2y^2}\qquad f_2(x,y)=\frac{xy}{x^2+y^2}
  \qquad f_3(x,y)=\frac{1-cos(xy)}{xy^2}\]
\end{Exercice}


\begin{Exercice}
  Soit $\sum a_nx^n$ une série entière dont le rayon de convergence est strictement positif. 
  On note $f$ sa somme sur $]-R;R[$.
    \begin{enumerate}
      \item Trouver des conditions nécessaires et suffisantes portant sur les coefficients $(a_n)$ pour que $f$ soit solution de
    l'équation différentielle :
    \[xy'' + 2y' + xy = 0\]
    \item On suppose ces conditions vérifiées. Déterminer $a_n$ lorsque $a_0 = 1$.
    \item Quelle est la fonction $f$ obtenue ? Quelle est le rayon de convergence $R$ ?
  \end{enumerate}
\end{Exercice}


\begin{Exercice}
  Notons $H_n=1+\frac12+\frac13+\ldots+\frac1n$ et $a\in[0;1[$. 
  Déterminer les valeurs de $a$ pour lesquelles la série $\sum a^{H_n}$ est convergente.
\end{Exercice}


\begin{Exercice}
  Calculer l'intégrale de :
  \[I=\int_0^1\frac{\ln(x)}{1-x}dx\]
\end{Exercice}


% \begin{Exercice}
%   Soit $f_n$ l'application définie par :
%   \[f_n(x)=\sum_{k=1}^n\frac{[kx]}{k3^}\]
%   où $[\ldots]$ représente la partie entière.
%   \begin{enumerate}
%     \item Montrer que $(f_n)$ converge normalement sur tout segment de $\R$. On note $f$ sa limite.
%     \item Montrer que $f$ est continue en chaque réel irrationnel.
%     \item Soit $x=\frac{p}{q}$ un élément de $\Q$. Montrer que pour tout $\epsilon\in\R_+^*$ :
%       \[f(x)-f(x-\epsilon)\geq\frac{1}{q^3}\]
%       En déduire que $f$ n'est continue en aucun élément de $\Q$.
% \end{enumerate}
% \end{Exercice}


\begin{Exercice}
  Trouver toutes les applications $f$ deux fois dérivables de $\R$ dans $\R$ vérifiant :
  \[\forall(x,y)\in\R^2,\ f(x+y)+f(x-y)=2f(x)f(y)\]
\end{Exercice}

\begin{Exercice}
Soit la surface d'équation $xyz = 1$.
\begin{enumerate}
  \item Montrer que la surface est régulière, c'est-à-dire qu'elle n'a pas de point critique.
  \item Déterminer l'équation du plan tangent en un point $M_0 = (x_0,y_0,z_0)$ quelconque de la surface.
  \item Notons $A$, $B$ et $C$ les points d'intersection du plan tangent en $M_0$ avec les axes $(Ox)$, $(Oy)$ et $(Oz)$. Montrer que
l'aire du tétraèdre $(O,A,B,C)$ ne dépend pas du point choisi. On rappelle que le volume du tétraèdre $(O,A,B,C)$
est : $\frac16\det(\vect{OA},\vect{OB},\vect{OC})$.
\end{enumerate}
\end{Exercice}

\end{document}
