\documentclass[a4paper,10pt]{report}
\usepackage{cours}
\usepackage{tkz-tab}
\usepackage{pifont}

\begin{document}
\everymath{\displaystyle}

%\begin{center}
% \shadowbox{{\huge TD 1 : Rappels d'analyse}}
%\end{center}

\begin{center}
\textit{{ {\huge TD 1 : Rappels d'analyse}}}
\end{center}
\bigskip


\begin{center}
\textit{{ {\large Étude de convergence}}}
\end{center}



\setlength{\shadowsize}{2pt} 



\begin{Exa}
On pose pour tout $n \in \mathbb{N}^*$, 
    \[
    u_n = \sum_{k = 1}^n {\frac{1}{\sqrt k}} - 2\sqrt n \; \hbox{ et } \; v_n = \sum_{k = 1}^n {\frac{1}{\sqrt k}} - 2\sqrt {n + 1}
    \]
 Montrer que les suites $(u_n)_{n \geq 1}$ et $(v_n)_{n \geq 1}$ sont convergentes.
 \end{Exa}
 
 
 
\begin{Exa} Soit $x \in \mathbb{R}$. On pose pour tout $n \geq 1$, 
$$\dis u_n = \frac{1}{n^2} \sum_{k=1}^n E(kx)$$
Étudier la convergence de la suite $(u_n)_{n \geq 1}$.
\end{Exa}



\begin{Exa} Étudier la suite définie par $u_0=1$ et pour tout $n \in \mathbb{N}$ par $u_{n+1}=\ln(1+u_n)$.
\end{Exa} 


\begin{Exa} Déterminer les limites suivantes. 

\begin{multicols}{2}
\begin{enumerate}
\item $\dis \lim_{n \rightarrow + \infty} \left(1+ \frac{1}{n}\right)^n$
\item $\dis \lim_{n \rightarrow + \infty} \left(1+ \frac{1}{n}\right)^{n^2}$
\columnbreak
\item $\dis \lim_{n \rightarrow + \infty} \left(1+ \frac{1}{n}\right)^{-n^2}$
\item $\dis \lim_{n \rightarrow + \infty} \left(1+ \frac{1}{n}\right)^{\ln(n)}$
\end{enumerate}
\end{multicols}
\vspace{0.1cm}
\end{Exa} 



\medskip

\begin{center}
\textit{{ {\large Suites usuelles}}}
\end{center}

\medskip


\begin{Exa} Déterminer le terme général de la suite définie par $u_0=2$ et pour tout $n \geq 0$ par $u_{n+1}=3u_n-2$.
\end{Exa} 



\begin{Exa} Déterminer le terme général des suites définies de la manière suivante :
    \begin{enumerate}
      \item
        $(u_n)_{n \geq 0}$ définie par $u_0 = 1,u_1 = 0$ et pour tout $n \in \N$, $u_{n + 2} = 4u_{n + 1} - 4u_n$.
      \item
        $(u_n)_{n \geq 0}$ définie par $u_0 = 1,u_1 = - 1$ et pour tout $n \in \N$, $2u_{n + 2} = 3u_{n + 1} - u_n$.
        \item
        $(u_n)_{n \geq 0}$ définie par $u_0 = 0,u_1 = \sqrt{3}$ et pour tout $n \in \N$, $u_{n + 2} = -u_{n+1}- u_n$.

    \end{enumerate}
\end{Exa}    


\begin{Exa} Déterminer les suites $(v_n)_{n \geq 0}$ telles que $v_0>0$, $v_1>0$  et vérifiant pour tout $n \in \mathbb{N}$, 
$$v_{n+2}=\left(\dfrac{v_{n+1}}{v_n}\right)^4$$
\end{Exa}


\newpage


\begin{center}
\textit{{ {\large Développements limités}}}
\end{center}

\medskip



\begin{Exa}
Déterminer les développements limités suivants :

\begin{multicols}{2}
\begin{enumerate}
\item $x \mapsto \sqrt{1+\sin(x)}$ à l'ordre $3$ en $0$.
\item $x \mapsto \exp \left( \frac{1}{1+x}\right)$ à l'ordre $3$ en $0$.
\item $x \mapsto (1+x)^{1/x}$ à l'ordre $3$ en $0$.
\item $x \mapsto \dfrac{\ln(x)}{x}$ à l'ordre $3$ en $2$.
\end{enumerate}
\end{multicols}

\vspace{0.1cm}
\end{Exa} 
 



\begin{Exa} Déterminer $\dis \lim_{x \rightarrow 0} \frac{1}{x^2} - \frac{1}{\tan(x)^2} \cdot$
\end{Exa} 



\begin{Exa} Déterminer $\dis \lim_{x \rightarrow + \infty}  \cos \left( \frac{1}{x} \right)^{x^2} \cdot$
\end{Exa}



\begin{Exa}[\ding{80}] Soit $f$ définie sur $\mathbb{R}^*$ par $f(x) = x^3 \cos \left( \frac{1}{x} \right) \cdot$

\begin{enumerate}
\item Montrer que $f$ est prolongeable par continuité en $0$. \textit{On notera encore abusivement $f$ ce prolongement.}
\item Montrer que $f$ admet un développement limité à l'ordre $2$ en $0$. Qu'en déduit-on en terme de dérivabilité ?
\item Montrer que la fonction $\sin$ n'a pas de limite en $+ \infty$. 
\item 
\begin{enumerate}
\item Justifier que $f$ est dérivable en tout point $x \in \mathbb{R}^*$ et donner $f'(x)$.
\item Montrer que $f$ n'est pas deux fois dérivable en $0$.
\end{enumerate}
\item Quelle est la moralité de cet exercice ?
\end{enumerate}
\end{Exa} 



\medskip

\begin{center}
\textit{{ {\large Équivalents}}}
\end{center}

\medskip


\begin{Exa} D\'{e}terminer le signe, au voisinage de l'infini, de 
$$u_{n}=\text{sh}\left( \dfrac{1}{n}\right) -\tan \left( \dfrac{1}{n}\right)$$
\end{Exa}




\begin{Exa} Soit $k \in \mathbb{N}$. Déterminer un équivalent de $\binom{n}{k}$ quand $n$ tend vers $+ \infty$. 
\end{Exa}


\begin{Exa}[\ding{80}] Donner un équivalent simple de $\dis \sum_{k=1}^n k!$ quand $n \rightarrow + \infty$.
\end{Exa} 


\begin{Exa} Soient $(u_n)_{n \geq 0}$ et $(v_n)_{n \geq 0}$ deux suites réelles strictement positives. On suppose que $(v_n)_{n \geq 0}$ admet une limite $\ell$ où $\ell \in \mathbb{R}_+\setminus \lbrace 1 \rbrace$ ou $\ell= + \infty$. Montrer que :
$$ u_n \underset{+ \infty}{\sim} v_n \; \Longrightarrow  \; \ln(u_n) \underset{+ \infty}{\sim} \ln(v_n)$$
\end{Exa}


\begin{Exa}[\ding{80}] Donner un équivalent de $\sqrt[n+1]{n+1} - \sqrt[n]{n}$ quand $n$ tend vers $+ \infty$.
\end{Exa} 


\begin{Exa}[\ding{80}] Soit $(u_n)_{n \geq 0}$ une suite réelle. On souhaite étudier la propriété $(\mathcal{P})$ suivante :
$$ u_n \underset{+ \infty}{\sim} \dfrac{1}{n} \; \Longleftrightarrow \; u_n + u_{n-1} \underset{+ \infty}{\sim} \dfrac{2}{n}$$
\begin{enumerate}
\item Montrer que l'implication directe est toujours vraie.
\item On suppose que $(u_n)_{n \geq 0}$ est monotone. Montrer que $(\mathcal{P})$ est alors vraie.
\item Si on retire cette hypothèse, le résultat est-il vrai ?
\end{enumerate}
\end{Exa}


\medskip

\begin{center}
\textit{{ {\large Suites implicites}}}
\end{center}

\medskip

\begin{Exa} Soient $n$ un entier naturel et $E_n $ l'équation $x + \tan x = n$ d'inconnue $x \in \left] { - \pi  / 2,\pi  / 2} \right[$.

\begin{enumerate}
\item Montrer que l'équation $E_n$ possède une solution unique notée $x_n$.
\item Montrer que la suite $(x_n)_{n \geq 0}$ converge et déterminer sa limite.
\end{enumerate}
\end{Exa} 



\begin{Exa}
\begin{enumerate}
\item Montrer, pour tout entier $n \geq 2$,  l'existence d'un unique solution à l'équation $x^n-x-1$ sur $]1, + \infty[$. On note cette solution $x_n$.
\item Déterminer le sens de variation de la suite $(x_n)_{n \geq 2}$.
\item Montrer que  $(x_n)_{n \geq 2}$ converge et déterminer sa limite $\ell$.
\item Déterminer un équivalent de $x_n - \ell$ quand $n$ tend vers $+ \infty$.
\end{enumerate}
\end{Exa}




\begin{Exa}[\ding{80}] \begin{enumerate}
\item Montrer que pour tout $n \geq 3$, $e^x=nx$ admet deux solutions $x_n$ et $y_n$ tels que $0 \leq x_n <y_n$.
\item Étudier la monotonie des suites $(x_n)$ et $(y_n)$ et en déduire qu'elles admettent une limite à déterminer.
\item Montrer que $x_n \underset{+ \infty}{\sim} \dfrac{1}{n}$, trouver un équivalent de $x_n - \dfrac{1}{n}$ quand $n$ tend vers $+ \infty$ et donner un développement asymptotique de $x_n$ à deux termes.
\item Soit $\varepsilon >0$. Montrer qu'à partir d'un certain rang, $y_n \leq (1+ \varepsilon) \ln(n)$. En déduire un équivalent de $y_n$ quand $n$ tend vers $+ \infty$.
\end{enumerate}
\end{Exa}



\medskip

\begin{center}
\textit{{ {\large Divers}}}
\end{center}

\medskip

\begin{Exa} Soit $(u_n)_{n \geq 0}$ une suite complexe telle que $(u_{2n})_{n \geq 0},(u_{2n + 1})_{n \geq 0}{\text{ et }}(u_{3n})_{n \geq 0}$ convergent. Montrer que $(u_n)_{n \geq 0}$ converge.
\end{Exa} 


\begin{Exa} Soient $\ell>0$ et $(u_n)_{n \geq 0}$ une suite réelle ou complexe (avec des termes non nuls) vérifiant la propriété suivante :
$$\exists N \in \mathbb{N}, \, \forall n \in \mathbb{N}, \, n \geq N \Longrightarrow \left\vert \frac{u_{n+1}}{u_n} \right\vert   \leq \ell$$

\begin{enumerate}
\item Supposons que $\ell<1$. Montrer que $(u_n)_{n \geq 0}$ converge vers $0$.
\item Le résultat précédent est-il vrai si $\ell = 1$ ?
\item Étudier la convergence de la suite définie pour tout $n \geq 0$ par $u_n = \dfrac{z^n}{n!}$ où $z \in \mathbb{C}$.
\end{enumerate}
\end{Exa}

\begin{Exa}[\ding{80}] Déterminer toutes les fonctions $f$ continues en $1$ vérifiant :
$$ \forall x \in \mathbb{R}, \, \, f \left( \frac{x+1}{2} \right) =f(x)$$
\end{Exa}


\begin{Exa}[Lemme de Césaro \ding{80}] Soit $(u_n)_{n \geq 0}$ une suite réelle de limite $\ell \in \mathbb{R}$. Montrer que :
$$ \lim_{n \rightarrow + \infty} \frac{1}{n} \sum_{k=1}^n u_k = \ell$$
\end{Exa}





\end{document}