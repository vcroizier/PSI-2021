\documentclass[french,11pt,twoside]{VcCours}

\newenvironment{ApplicationDirecte}{\textbf{Application directe du cours :}

}{}
\newcommand{\dx}{\text{d}x}
\DeclareMathOperator{\e}{e}
\newcommand{\Sum}[2]{\ensuremath{\textstyle{\sum\limits_{#1}^{#2}}}}
\newcommand{\Int}[2]{\ensuremath{\mathchoice%
	{{\displaystyle\int_{#1}^{#2}}}
	{{\displaystyle\int_{#1}^{#2}}}
	{\int_{#1}^{#2}}
	{\int_{#1}^{#2}}
}}


\begin{document}

\Titre{PSI}{Promotion 2021--2022}{Mathématiques}{Chapitre 1 : Rappels d'analyse}

\tableofcontents
\separationTitre

\newpage
\section{Suites}

Dans toute cette section, $\mathbb{K}= \mathbb{R}$ ou $\mathbb{C}$. Rappelons qu'une suite $(u_n)_{n \geq 0}$ d'éléments de $\mathbb{K}$ est simplement une application de $\mathbb{N}$ dans $\mathbb{K}$. Nous considérerons des suites définies à partir du rang $0$ : tout s'adapte facilement si la suite est définie à partir d'un autre rang. Nous noterons $\overline{\mathbb{R}}$ l'ensemble $\mathbb{R} \cup \lbrace - \infty, + \infty \rbrace\cdot$

\subsection{Convergence : définitions et résultats principaux}

Dans cette partie, $(u_n)_{ n \geq 0}$ est une suite d'éléments de $\mathbb{K}$.

\begin{TheoremeDefinition}{} 
\begin{itemize}
\item Soit $\ell \in \mathbb{K}$. On dit que $(u_n)_{n \geq 0}$ \emph{converge}
vers $\ell$ (ou que $u_n$ tend vers $\ell$ quand $n$ tend vers $+ \infty$) si :
$$ \forall \varepsilon>0, \, \exists N \in \mathbb{N}, \, \forall n \in \mathbb{N},
\, n \geq N \Rightarrow \vert u_n - \ell \vert \leq \varepsilon$$ 
\item On dit que $(u_n)_{n \geq 0}$ est \emph{convergente} si il existe 
$\ell \in \mathbb{K}$ tel que $(u_n)_{n \geq 0}$ converge vers $\ell$. 
Dans ce cas, $\ell$ est unique, on l'appelle \emph{limite} de $(u_n)_{n \geq 0}$, 
et on note :
$$ \lim_{ n \rightarrow + \infty} u_n = \ell \quad \hbox{ ou } \quad u_n 
\underset{n \rightarrow + \infty}{\longrightarrow} \ell$$
\end{itemize}
\end{TheoremeDefinition}{}

\begin{Demonstration}{}

\vspace{10cm}
%Montrons l'unicité de la limite. Soient $(\ell, \ell') \in \mathbb{K}^2$ tels que $u_n \underset{n \rightarrow + \infty}{\longrightarrow} \ell$ et $u_n \underset{n \rightarrow + \infty}{\longrightarrow} \ell'$.
%
%\medskip
%
%Soit $\varepsilon >0$. Il existe deux entiers naturels $N_1$ et $N_2$ tels que pour tout $n \in \mathbb{N}$,
%$$ n \geq N_1 \Rightarrow \vert u_n - \ell \vert \leq \varepsilon \quad \hbox{ et } \quad n \geq N_2 \Rightarrow \vert u_n - \ell' \vert \leq \varepsilon $$
%
%Posons $N = \max(N_1,N_2)$. Pour tout entier $n \geq N$, on a alors :
%$$ \vert u_n - \ell \vert \leq \varepsilon \quad \hbox{ et } \quad  \vert u_n - \ell' \vert \leq \varepsilon $$
%et d'après l'inégalité triangulaire :
%$$ \vert \ell - \ell ' \vert = \vert \ell - u_n +u_n - \ell' \vert \leq \vert \ell  - u_n \vert + \vert u_n - \ell' \vert \leq 2 \varepsilon$$
%Ainsi, pour tout $\varepsilon>0$,
%\begin{equation}\label{unicite}
%\vert \ell- \ell' \vert \leq 2 \varepsilon
%\end{equation}
%Par l'absurde, si $\ell \neq \ell'$, $\vert \ell- \ell' \vert >0$ et donc par (\ref{unicite}) en posant $\varepsilon = \dfrac{\vert \ell- \ell' \vert}{2}$, on a :
%$$ \vert \ell- \ell' \vert <  \vert \ell- \ell' \vert$$
%ce qui est absurde. Ainsi $\ell = \ell'$ et on a bien l'unicité de la limite.

\end{Demonstration}

\begin{Definition}{}
	Si une suite n'est pas convergente, on dit qu'elle est \emph{divergente}.
\end{Definition}

Il y a deux situations possibles pour une suite \emph{réelle} divergente :
\begin{itemize}
 \item la suite peut tendre vers $+\infty$ ou $-\infty$.
 \item la suite peut ne pas admettre de limite (finie ou infinie) lorsque $n$ tend vers $+\infty$.
\end{itemize}


\begin{Definition}{}
Soit $(u_n)_{n \geq 0}$ une suite \emph{réelle}.
\begin{itemize}
\item On dit que la suite $(u_n)_{n \geq 0}$ \emph{diverge vers} $+\infty$ lorsque pour tout réel $A$, on peut trouver un rang à partir duquel tous les termes de la suite sont supérieurs à $A$. Autrement dit si :
$$\forall A \in \R, \exists N \in \N \, \vert \, \forall n \ge N, u_n \geq A $$
On dit également que \emph{$u_n$ tend vers $+\infty$ lorsque $n$ tend vers $+\infty$} et on note :
$$\displaystyle{\lim_{n \rightarrow +\infty}u_n=+\infty} \quad \hbox{ ou } \quad u_n \underset{n \rightarrow + \infty}{\longrightarrow} + \infty $$
\item On définit de la même manière la divergence vers $- \infty$.
\end{itemize}
\end{Definition}

\begin{Remarque}{} Il y a aussi unicité des limites dans le cas des limites infinies.
\end{Remarque}

\begin{Theoreme}{} Toute suite convergente est bornée. La réciproque est fausse.
\end{Theoreme}

\begin{Demonstration}{}
\vspace*{9.5cm}

\vspace{\stretch{1}}
\end{Demonstration}
%\subsection{Opérations sur les limites}
%
%Voici les règles usuelles liées aux opérations viables sur les limites de suites réelles. Dans cette partie, $\ell_1$ et $\ell_2$ désignent deux réels et $k$ est un réel.
%
%
%\begin{center}
%\textbf{Limite d'une somme de deux suites}
%\end{center}
%
%$$\renewcommand{\arraystretch}{2.2}
%\begin{array}{|c||*{6}{c|}}
%\hline 
% \dsp \lim_{n\to +\inf} u_n &\ell_1&\ell_1&\ell_1&+\inf&+\inf&-\inf\\
% \hline 
% \dsp \lim_{n\to +\inf} v_n &\ell_2&+\inf&-\inf&+\inf&-\inf&-\inf\\
% \hline
% \dsp \lim_{n\to +\inf} (u_n+v_n) & \phantom{ bla bla } & \qquad& \qquad& \qquad& \qquad& \qquad\\
% \hline
%\end{array}$$
%
%\vspace{0.4cm}
%\pagebreak
%%\vspace{0.4cm}
%
%%\begin{center}
%%\textbf{Limite du produit d'une suite par un réel}
%%\end{center}
%%
%%$$\renewcommand{\arraystretch}{2.2}
%%\begin{array}{|c||c|c|c|c|}
%%\hline 
%% & \dsp \lim_{n\to +\inf} u_n &\ell_1&+\inf&-\inf\\
%% \hline 
%% k>0 & \dsp \lim_{n\to +\inf} k u_n & \qquad &\qquad&\qquad\\
%% \hline
%% k<0 & \dsp \lim_{n\to +\inf} k u_n  & \phantom{ bla bla } & \qquad& \qquad \\
%% \hline
%%\end{array}$$
%
%%\medskip
%
%\begin{center}
%\textbf{Limite d'un produit de deux suites}
%\end{center}
%
%
%$$\renewcommand{\arraystretch}{2.2}
%\begin{array}{|c||*{4}{c|}}
%\hline 
% \dsp \lim_{n\to +\inf} u_n &\ell_1&\ell_1\ne 0&\pm \infty&\pm \infty\\
% \hline 
% \dsp \lim_{n\to +\inf} v_n &\ell_2&\pm \infty&\pm \infty&0\\
% \hline
% \dsp \lim_{n\to +\inf} (u_n\times v_n) & \qquad & \qquad & \qquad&  \qquad \\
% \hline
%\end{array}$$
%
%\bigskip
%
%Pour déterminer la limite de l'inverse d'une suite réelle qui tend vers 0, on a besoin de supposer que le signe des termes de la suite soit constant à partir d'un certain rang.
%
%\vspace{0.2cm}
% 
%\begin{itemize}
% \item Si $\dsp \lim_{n\to +\inf} u_n = 0$ et $u_n > 0$ à partir d'un certain rang, on notera $\ell = 0^+$.
% \item Si $\dsp \lim_{n\to +\inf} u_n = 0$ et $u_n < 0$ à partir d'un certain rang, on notera $\ell = 0^-$.
%\end{itemize}
%
%\begin{center}
%\textbf{Limite de l'inverse d'une suite}
%\end{center}
%
%$$\renewcommand{\arraystretch}{2.2}
%\begin{array}{|c||*{4}{c|}}
%\hline 
% \dsp \lim_{n\to +\inf} u_n &\ell\ne 0&\pm \infty &\ell = 0^+&\ell = 0^-\\
% \hline 
% \dsp \lim_{n\to +\inf} \dfrac 1{u_n} & \qquad & \qquad & \qquad &  \qquad \\[1ex]
% \hline
%\end{array}$$
%
%\medskip
%
%\begin{Remarque}{}
%Pour un quotient, il suffit de se ramener à produit avec un inverse.
%\end{Remarque}



\begin{Proposition}{Limites et inégalités larges}
Soit $(u_n)_{n \geq 0}$ une suite \emph{réelle} convergeant vers $\ell \in \mathbb{R}$ et soient $a$ et $b$ deux réels.
\begin{itemize}
\item Si pour tout $n \in \mathbb{N}$, $u_n \geq a$ alors $\ell \geq a$.
\item Si pour tout $n \in \mathbb{N}$, $u_n \leq b$ alors $\ell \leq b$.
\item Si pour tout $n \in \mathbb{N}$, $a \leq u_n \leq b$ alors $a \leq \ell \leq b$.
\end{itemize}
\end{Proposition}

\begin{Remarque}{} En toute généralité, les inégalités strictes ne sont pas conservées par passage à la limite. Par exemple, pour tout $n \geq 1$, $1/n > 0$ et pourtant $\lim_{n \rightarrow + \infty}1/n= 0.$
\end{Remarque}


\subsection{Théorèmes de convergence pour des suites réelles}

\begin{Theoreme}{de la limite par comparaison}
Soient $(u_n)_{n \geq 0}$ et $(v_n)_{n \geq 0}$ deux suites \emph{réelles}. On suppose que pour tout $n \in \mathbb{N}$, $u_n \leq v_n$.
\begin{itemize}
 \item Si la suite $(u_n)_{n \geq 0}$ diverge vers $+\infty$ alors  $(v_n)_{\geq 0}$ aussi. 
 \item Si la suite $(v_n)_{n \geq 0}$ diverge vers $-\infty$ alors $(u_n)_{n \geq 0}$ aussi.
 \end{itemize}
\end{Theoreme}

\begin{Theoreme}{de la limite par encadrement}
Soient $(u_n)_{n \geq 0}$, $(a_n)_{n \geq 0}$ et $(b_n)_{n \geq 0}$ trois suites \emph{réelles} et $\ell \in \mathbb{R}$.

\medskip

On suppose que :

\begin{itemize}
\item Pour tout $n \in \mathbb{N}$,  $a_n \leq u_n \leq b_n$.
\item Les suites $(a_n)_{n \geq 0}$ et $(b_n)_{n \geq 0}$ convergent vers $\ell$.
\end{itemize}

Alors la suite $(u_n)_{n \geq 0}$ converge et $\lim_{n \rightarrow + \infty} u_n = \ell$.
\end{Theoreme}

\begin{Exemple} Étudier la convergence de la suite dont le terme général est donné pour tout $n \geq 1$ par :
$$ u_n = \sum_{k=1}^n \frac{1}{\sqrt{k}}$$

\vspace{5cm}
\end{Exemple}

\newpage
\begin{ApplicationDirecte} Étudier la convergence de la suite $(u_n)_{n \geq 0}$ dont le terme général est donné par :
$$ u_n = \int_0^1 e^x x^n \dx$$
%Soient $(u_n)_{n \geq 0}$ une suite réelle bornée et $(v_n)_{n \geq 0}$ et une suite réelle divergeant vers $+ \infty$. Que peut-on dire de la convergence de $(u_n+v_n)_{n \geq 0}$?
\end{ApplicationDirecte}

\begin{Theoreme}{de la limite monotone} Soit $(u_n)_{n \geq 0}$ une suite \emph{réelle}.


\begin{itemize}
\item Si $(u_n)_{n \geq 0}$ est croissante et majorée alors elle converge et :
$$\lim_{n \rightarrow + \infty} u_n = \sup \lbrace u_n \, \vert \, n \in \mathbb{N} \rbrace $$
\item Si $(u_n)_{n \geq 0}$ est croissante et non majorée alors elle diverge vers $+ \infty$.
\item Si $(u_n)_{n \geq 0}$ est décroissante et minorée alors elle converge et :
$$\lim_{n \rightarrow + \infty} u_n = \inf \lbrace u_n \, \vert \, n \in \mathbb{N} \rbrace $$
\item Si $(u_n)_{n \geq 0}$ est décroissante et non minorée alors elle diverge vers $- \infty$.
\end{itemize}
\end{Theoreme}

\begin{Definition}{}[Suites adjacentes]
Deux suites \emph{réelles} $(u_n)_{n \geq 0}$ et $(v_n)_{n \geq 0}$ sont dites \emph{adjacentes} lorsque :
\begin{itemize}
\item
\item 
\end{itemize}
% \begin{itemize}
%  \item \phantom{$(u_n)_{n \geq 0}$ est croissante et $(v_n)_{n \geq 0}$ est décroissante (ou le contraire).}
%  \item \phantom{la suite $(v_n-u_n)_{n \geq 0}$ converge vers $0$.}
% \end{itemize}
\end{Definition}

\begin{Theoreme}{}
Deux suites adjacentes sont convergentes et elles convergent vers la même limite. De plus, si $\ell$ est la limite et que $(u_n)_{n \geq 0}$ est croissante et $(v_n)_{n \geq 0}$ est décroissante, alors pour tout $n \geq 0$,
$$ u_n \leq \ell \leq v_n $$
\end{Theoreme}

\begin{ApplicationDirecte}
Montrer que les deux suites $(u_n)_{n \geq 1}$ et $(v_n)_{n \geq 1}$ définies pour tout entier $n \geq 1$ par :
\[ u_n = \sum_{k=1}^n \frac{1}{k!} \; \hbox{ et } \; v_n = \sum_{k=1}^n \frac{1}{k!} + \frac{1}{n n!} \]
sont adjacentes. On admet que ces deux suites convergent vers $e$. Comment obtenir une approximation de $e$ à $10^{-3}$ près ?
\end{ApplicationDirecte}

\newpage
\subsection{Comment étudier la convergence d'une suite complexe ?}

\begin{Proposition}{} Soit $(u_n)_{n \geq 0}$ une suite complexe. Les assertions suivantes sont équivalentes :

\begin{enumerate}
\item La suite $(u_n)_{n \geq 0}$ est convergente.
\item Les suites $(\Re e(u_n))_{n \geq 0}$ et $(\Im m(u_n))_{n \geq 0}$ convergent.
\end{enumerate}

Dans ce cas, on a : $\lim_{n \rightarrow + \infty} u_n = \lim_{n \rightarrow + \infty} \Re e(u_n) + i \lim_{n \rightarrow + \infty} \Im m(u_n)$.
\end{Proposition}

\begin{Remarque}{} Si l'on a une idée de la limite éventuelle $\ell$ de la suite, on peut essayer de majorer $\vert u_n- \ell \vert$ (cela permet de se ramener à une suite réelle).
\end{Remarque}

\begin{Exemple} Étudier la convergence de la suite dont le terme général est donné pour tout entier $n \geq 1$ par : 
$$ z_n = \frac{e^{\frac{i n \pi}{3}} + e^{\frac{i n \pi}{4}}}{n} $$ 

\vspace{4cm}
\end{Exemple}

\subsection{Suites extraites}

\begin{Definition}{} 
On appelle \emph{suite extraite} (ou \emph{sous-suite}) d'une suite 
$(u_n)_{n \geq 0}$ toute suite de la forme $(u_{\varphi(n)})_{n \geq 0}$ 
où $\varphi : \mathbb{N} \rightarrow \mathbb{N}$ est une suite 
strictement croissante.
\end{Definition}

\begin{Remarque}{} Une sous-suite d'une suite est une \og sélection \fg d'une infinité de termes de cette suite.
\end{Remarque}

\begin{Exemple} Les suites $(u_{2n})_{n \geq 0}$, $(u_{2n+1})_{n \geq 0}$ et $(u_{6n})_{n \geq 0}$ sont des sous-suites de $(u_n)_{n \geq 0}$. La suite $(u_{6n})_{n \geq 0}$ est aussi une sous-suite de $(u_{2n})_{n \geq 0}$.
\end{Exemple}

\begin{Proposition}{} Si une suite $(u_n)_{n \geq 0}$ tend vers $\ell \in \overline{\mathbb{R}}$ (ou $\mathbb{C}$) quand $n$ tend vers $+ \infty$, il en est de même pour toute suite extraite de $(u_n)_{n \geq 0}$.
\end{Proposition}

\begin{Demonstration}{} 

%Soit $\varphi : \mathbb{N} \rightarrow \mathbb{N}$ est une suite strictement croissante. On montre par une récurrence aisée que pour tout $n \in \mathbb{N}$, $\varphi(n) \geq n$. Montrons le résultat dans le cas où $\ell \in \mathbb{R}$. Soit $\varepsilon>0$. Il existe $N \in \mathbb{N}$ tel que pour tout entier $n \geq N$, $\vert u_n - \ell \vert \leq \varepsilon$.

%\medskip
%
%Pour tout entier $n \geq N$, $\varphi(n) \geq n \geq N$ et ainsi $\vert u_{\varphi(n)} -\ell \vert \leq \varepsilon$.
%
%\medskip 
%
%Ainsi, pour tout $\varepsilon >0$, il existe un entier naturel $N$ tel que pour tout entier $n \geq N$, $\vert u_{\varphi(n)} -\ell \vert \leq \varepsilon$. La suite $(u_{\varphi(n)})_{n \geq 0}$ est donc convergente et converge vers $\ell$.

\vspace*{7cm}
\end{Demonstration}

\begin{Exemple} La suite $((-1)^n)_{n \geq 0}$ ne converge pas car $ u_{2n}= 1 \underset{n \rightarrow + \infty}{\longrightarrow} 1$ et $ u_{2n+1} = -1 \underset{n \rightarrow + \infty}{\longrightarrow} -1$.
\end{Exemple}

\begin{ApplicationDirecte} Pour tout entier $n \geq 1$, on pose :
$$H_n = \sum_{k=1}^n \dfrac{1}{k} $$
Montrer que $(H_n)_{n \geq 1}$ diverge (\emph{on pourra minorer, pour} $n \geq 1$, $H_{2n}-H_n$).
\end{ApplicationDirecte}




\begin{Proposition}{} Soient $(u_n)_{n \geq 0}$ une suite et $\ell \in \overline{\mathbb{R}}$. Les assertions suivantes sont équivalentes :

\begin{enumerate}
\item $u_n$ tend vers $\ell$ quand $n$ tend vers $+ \infty$.
\item $u_{2n}$ et $u_{2n+1}$ tendent vers $\ell$ quand $n$ tend vers $+ \infty$.
\end{enumerate}
\end{Proposition}

\begin{Exemple} Soient $\alpha \in \mathbb{R}_+^*$ et $(u_n)_{n \geq 1}$ la suite définie pour tout entier $n \geq 1$ par :
$$ u_n = \sum_{k=1}^n \frac{(-1)^k}{k^{\alpha}}$$

\medskip

Montrer que $(u_{2n})_{n \geq 0}$ et $(u_{2n+1})_{n \geq 0}$ sont adjacentes. Qu'en déduit-on?
%
%\begin{itemize}
%\item Pour tout $n \geq 1$,
%\begin{align*}
%u_{2n+2}-u_{2n} & = \frac{(-1)^{2n+2}}{(2n+2)^{\alpha}} + \frac{(-1)^{2n+1}}{(2n+1)^{\alpha}} \\
%& = \frac{1}{(2n+2)^{\alpha}} - \frac{1}{(2n+1)^{\alpha}} \leq 0 \\
%\end{align*}
%car $2n+2 \geq 2n+1$. Ainsi $(u_{2n})_{n \geq 0}$ est décroissante.
%\item Pour tout $n \geq 1$,
%\begin{align*}
%u_{2n+3}-u_{2n+1} & = \frac{(-1)^{2n+3}}{(2n+3)^{\alpha}} + \frac{(-1)^{2n+2}}{(2n+2)^{\alpha}}\\
%& = -\frac{1}{(2n+3)^{\alpha}} + \frac{1}{(2n+2)^{\alpha}} \geq 0 \\
%\end{align*}
%car $2n+3 \geq 2n+2$. Ainsi $(u_{2n+1})_{n \geq 0}$ est croissante.
%\item On a : 
%$$ u_{2n+1}-u_{2n} = \frac{(-1)^{n+1}}{(n+1)^{\alpha}} \underset{n \rightarrow + \infty}{\longrightarrow} 0$$
%\end{itemize}
%
%\medskip
%
%Ainsi $(u_{2n})_{n \geq 0}$ et $(u_{2n+1})_{n \geq 0}$ sont adjacentes et convergent donc vers la même limite. D'après la proposition précédente, la suite $(u_n)_{n \geq 0}$ est donc convergente.
\newpage
\end{Exemple}



\subsection{Suites récurrentes}

Soient $I$ un intervalle de $\mathbb{R}$ et $f : I \rightarrow \mathbb{R}$ une fonction. On pose :
$$ \left\lbrace \begin{array}{l}
u_0 \in I \\
\forall n \geq 0, \, u_{n+1} = f(u_n) \\
\end{array}\right.$$

Plus questions se posent :

\begin{enumerate}
\item La suite $(u_n)_{n \geq 0}$ est-elle bien définie ?
\item Si oui, que peut-on dire de sa convergence ?
\item Si celle-ci converge, quelle est sa limite ?
\end{enumerate}

\medskip

\begin{Proposition}{} Si $f(I) \subset I$, la suite est bien définie.
\end{Proposition}

%\begin{metho} La première étape pour étudier ce type de suite est de déterminer des intervalles stables par la fonction $f$ (à l'aide d'une étude de fonction). Bien entendu, si $f$ est définie sur $\mathbb{R}$, il n'y a rien à vérifier.
%\end{metho}

\begin{Proposition}{} Supposons que $f(I) \subset I$.

\begin{enumerate}
\item Si pour tout $x \in I$, $f(x) \geq x$, alors $(u_n)_{n \geq 0}$ est croissante.
\item Si pour tout $x \in I$, $f(x) \leq x$, alors $(u_n)_{n \geq 0}$ est décroissante.
\item Si $f$ est croissante sur $I$, $(u_n)_{n \geq 0}$ est monotone.
\item Si $f$ est décroissante sur $I$, $(u_{2n})_{n \geq 0}$ et $(u_{2n+1})_{n \geq 0}$ sont monotones de variations contraires.
\end{enumerate}
\end{Proposition}

\emph{Idée de la preuve.}

\vspace*{10cm}
%\begin{enumerate}
%\item Pour tout $n \in \mathbb{N}$, $u_n \in I$ donc $f(u_n) \geq u_n$ et ainsi $u_{n+1} \geq u_n$.
%\item Même raisonnement.
%\item Simple récurrence. Les variations se déduisent de l'ordre des deux premiers termes : si $u_0 \leq u_1$, la suite est croissante et si $u_0 \geq u_1$, la suite est décroissante.
%\item Si $f$ est décroissante alors $g= f \circ f$ est croissante sur $I$. Le résultat découle du résultat précédent appliqué aux suites $(u_{2n})_{n \geq 0}$ et $(u_{2n+1})_{n \geq 0}$.
%\end{enumerate}

\newpage
\begin{Proposition}{} Supposons que $f$ est continue sur $I$ et que $f(I) \subset I$. Si la suite $(u_n)_{n \geq 0}$ converge vers un élément $\ell$ de $I$ alors $f(\ell)= \ell$.
\end{Proposition}

\begin{Remarque}{} Ainsi, en cas de convergence, la suite converge vers un point fixe de $f$ \emph{ou} vers une extrémité de l'intervalle (ne pas oublier ce cas).
\end{Remarque}

\begin{Theoreme}{Inégalité des accroissements finis}
Soit $f$ une fonction dérivable sur un intervalle $I$. \\
On suppose qu'il existe deux réels $m$ et $M$ tels que pour tout $x \in I$, $m \leq f'(x) \leq M$.\\
Alors pour tout $(a,b) \in I^2$ avec $a \leq b$, on a :
 $$ \phantom{m(b-a) \leq f(b)-f(a) \leq M (b-a)}$$
\end{Theoreme}

\begin{Remarque}{} On peut juste supposer l'inégalité \og  de droite (ou de gauche) \fg et on obtient en conclusion uniquement  l'inégalité \og  de droite (ou de gauche) \fg .
\end{Remarque}

\begin{Corollaire}{}
Soit $f$ une fonction dérivable sur un intervalle $I$.\\
On suppose qu'il existe un réel positif $M$ tel que  pour tout $x \in I$, $ |f'(x)| \leq M$. \\ 
Alors pour tout $(a,b) \in I^2$, on a :
$$\phantom{ |f(b)-f(a)| \leq M |b-a|}$$
\end{Corollaire}


\newpage
\begin{Methode}{Plan d'attaque d'une suite récurrente}
\begin{enumerate}
\item On justifie que la suite est bien définie :
\begin{itemize}
\item Si $f$ est définie sur $\mathbb{R}$, il n'y a rien à faire.
\item Si $u_0$ est connu, on cherche un intervalle stable par $f$ contenant $u_0$ (étude de fonction).
\item Si $u_0$ est un élément d'un intervalle donné dans l'énoncé, on regarde si celui-ci est stable par $f$.
\end{itemize}
\item On étudie la croissance de la suite : pour tout $n \geq 0$, 
$$ u_{n+1}-u_n = f(u_n)-u_n$$
\begin{itemize}
\item Si le signe de cette expression est évident : il n'y a rien à faire.
\item Si $f$ est croissante sur $I$ (ou sur un sous-intervalle $J$ stable par $f$ contenant $u_0$), la suite est monotone et les variations sont données par l'ordre des deux premiers termes.
\item Si $f$ est décroissante sur $I$ (ou sur un sous-intervalle $J$ stable par $f$ contenant $u_0$), la suite $(u_{2n})_{n \geq 0}$ et $(u_{2n+1})_{n \geq 0}$ sont monotones de variations contraires : on peut essayer de montrer que ces suites sont adjacentes.
\item On peut résoudre l'équation $f(x) \geq x$ (ou $f(x) \leq x$) d'inconnue $x \in I$. Si on obtient un intervalle solution $J$, stable par $f$ et contenant $u_0$, la suite est croissante (ou décroissante).
\end{itemize}
\item Si la suite est monotone : celle-ci est convergente ou diverge vers une limite infinie. Si $f$ est continue sur $I$, les limites finies éventuelles sont les points fixes de $f$ (que l'on obtient en résolvant $f(x)=x$ d'inconnue $x \in I$) ou les extrémités de $I$. Un raisonnement par l'absurde ici peut être efficace (représenter graphiquement les premiers termes de la suite a ici un gros intérêt).
\item Si $f$ est dérivable sur $I$ et si $\vert f' \vert$ est majorée par un réel positif $q<1$ alors si $a$ est un point fixe de $f$ sur $I$, on a pour tout entier $n \geq 0$ d'après l'inégalité des accroissements finis :
$$ \vert u_{n+1}- a \vert = \vert f(u_n) - f(a) \vert \leq q \vert u_n -a \vert$$
Et on montre par récurrence que pour tout $n \geq 0$,
$$ 0 \leq \vert u_n - a \vert \leq q^n \vert u_0- a \vert $$
En utilisant le théorème d'encadrement, on montre que $u_n \underset{n \rightarrow + \infty}{\longrightarrow} a$.
\end{enumerate}
\end{Methode} 
 
\begin{Exemple}
Étudions la suite $(u_n)_{n \geq 0}$ définie par $u_0 \in \mathbb{R}$ et pour tout $n \in \mathbb{N}$ par $u_{n+1}=e^{u_n}-1$.

%\medskip
%
%$\rhd$ La fonction $x \mapsto e^x-1$ est définie sur $\mathbb{R}$ donc la suite est bien définie.
%
%\medskip
%
%$\rhd$ La fonction $x \mapsto e^x-1$ est croissante sur $\mathbb{R}$ donc la suite $(u_n)_{n \geq 0}$ est monotone (démonstration par récurrence) et on a plus précisément :
%
%\begin{itemize}
%\item La suite est croissante si $u_1 > u_0$ ce qui est équivalent à $e^{u_0}-1-u_0 > 0$.
%\item Si $u_1=u_0$, la suite est constante.
%\item La suite est décroissante si $u_1 < u_0$ ce qui est équivalent à $e^{u_0}-1-u_0 <  0$.
%\end{itemize}
%
%On étudie alors le signe de $x \mapsto e^x-1-x$ : cette fonction est dérivable sur $\mathbb{R}$, de dérivée $x \mapsto e^x-1$ donc elle est décroissante sur $\mathbb{R}_-$ et croissante sur $\mathbb{R}_+$ donc attention un minimum en $x=0$ qui vaut $0$. Cette fonction est donc positive sur $\mathbb{R}$ et s'annule uniquement en $0$. On a alors : 
%
%\begin{itemize}
%\item Si $u_0=0$, la suite est constante et donc convergente.
%\item Si $u_0 \neq 0$, la suite est croissante.
%\end{itemize}
%
%On représente graphiquement la situation pour essayer de conjecturer la limite de la suite.
%
%\begin{center}
%\emph{Si $u_0 < 0$}
%
%\includegraphics[scale=0.4]{graph1}
%\end{center}
%
%\medskip
%
%\begin{center}
%\emph{Si $u_0 > 0$}
%
%\includegraphics[scale=0.4]{graph2}
%\end{center}
%
%\medskip
%
%$\rhd$ Si $u_0<0$ : on remarque que $f(\mathbb{R}_-) \subset \mathbb{R}_-$ donc, par récurrence, on montre que pour tout $n \geq 0$, $u_n \leq 0$. Ainsi la suite est croissante et majorée donc elle converge. Le seul point fixe de $f$ sur $\mathbb{R}_+$ est $0$ et $f$ étant continue, la suite converge donc vers $0$.
%
%\medskip
%
%$\rhd$ Si $u_0>0$ : supposons par l'absurde que la suite converge vers un réel $\ell$. La fonction $f$ étant continue sur $\mathbb{R}$, $\ell$ est un point fixe de $\mathbb{R}$ et donc $\ell =0$. C'est impossible car la suite est croissante et $u_0>0$. Ainsi la suite diverge vers $+ \infty$.
\end{Exemple}

\vspace*{\stretch{1}}

\newpage

\vspace*{10cm}

\begin{ApplicationDirecte}
Soit $(u_n)_{n \geq 0}$ la suite définie par $u_0 \in \mathbb{R}$ et pour tout entier $n \geq 0$ par $u_{n+1}=u_n^2+1$. Étudier la convergence de cette suite.
\end{ApplicationDirecte}


\section{Suites usuelles}
\subsection{Suites arithmético-géométriques}

\begin{Definition}{}
On dit qu'une suite $(u_n)_{n \geq 0}$ est \emph{arithmético-géométrique} lorsqu'il existe deux nombres complexes $a$ et $b$ tels que pour tout $n \in \mathbb{N}$, $u_{n+1} = a u_n + b$.
\end{Definition}

\begin{Theoreme}{Expression d'une suite arithmético-géométrique}
Soit $(u_n)_{n \geq 0}$ une suite arithmético-géométrique définie comme ci-dessus.
\begin{itemize}
\item Si $a=1$ alors la suite $(u_n)_{n \geq 0}$ est arithmétique de raison $b$.
\item Si $a \neq 1$ et $x$ est le complexe vérifiant $x=ax+b$, alors la suite $(u_n-x)_{n \geq 0}$ est géométrique de raison $a$.
\end{itemize}
\end{Theoreme}

\begin{Exemple} Déterminons le terme général de la suite $(u_n)_{n \geq 0}$ définie par $u_0=4$ et pour tout $n \in \mathbb{N}$ par $u_{n+1}=3u_n-2$.

\newpage
%\medskip 
%
%On résout $x = 3x-2$ d'inconnue $x \in \mathbb{R}$ : l'unique solution est $x=1$. Pour tout $n \geq 0$, on a :
%$$ \left\lbrace \begin{array}{ccl}
%u_{n+1} & = & 3 u_n - 2 \\
%1 & = & 3 \times 1 - 2 \\
%\end{array}\right.$$
%On soustrait les inégalités : $u_{n+1}-1 = 3 (u_n - 1)$. Ainsi la suite $(u_n-1)_{n \geq 0}$ est géométrique de raison $2$ et donc pour tout $n \geq 0$,
%$$ u_n - 1 = 3^n (u_0-1) = 3\times 2^n $$
%ou encore :
%$$ u_n = 3^{n+1} + 1 $$

\vspace{4cm}
\end{Exemple}

\begin{ApplicationDirecte} Donner le terme général de la suite $(u_n)_{n \geq 0}$ définie par $u_0=1$ et pour tout $n \geq 0$ par $u_{n+1} =  2u_n - 4$. \end{ApplicationDirecte}

\subsection{Suites récurrentes linéaires d'ordre deux}

\begin{Definition}{}
On dit qu'une suite $(u_n)_{n \geq 0}$ est \emph{récurrente linéaire d'ordre 2} lorsqu'il existe deux nombres complexes $a$ et $b$ tels que pour tout $n \in \mathbb{N}, \, u_{n+2} = a u_{n+1}+b u_n$.
\end{Definition}

\begin{Remarque}{} Pour calculer un terme d'une suite récurrente linéaire d'ordre 2, on a besoin des deux termes précédents. 
\end{Remarque} 

\begin{Definition}{}
Soit $(u_n)_{n \geq 0}$ une suite récurrente linéaire d'ordre 2 définie comme ci-dessus.\\
On appelle \emph{polynôme caractéristique} de cette suite le polynôme $X^2-aX-b$.
\end{Definition}

\begin{Theoreme}{Cas complexe : $a,b \in \mathbb{C}$}
Soit $(u_n)_{n \geq 0}$ une suite récurrente linéaire d'ordre 2 définie comme ci-dessus avec $b \neq 0.$
\begin{itemize}
\item Si le le polynôme caractéristique admet deux racines distinctes $r_1$ et $r_2$ alors le terme général de la suite $(u_n)_{n \geq 0}$ est donné par $u_n=\lambda (r_1)^n+\mu (r_2)^n$ pour tout $n \geq 0$ avec $(\lambda,\mu) \in \mathbb{C}^2$.
%est l'unique couple solution du système 
%$$\l\{\begin{array}{lllll}
%\lambda  &+ & \mu & = & u_0\\
%\lambda r_1 & + &\mu r_2 & = & u_1
%\end{array}\r.$$
\item Si le polynôme caractéristique admet une unique racine $r_0$ alors le terme général de la suite $(u_n)_{n \geq 0}$ est donné par $u_n= (\lambda\mu n)(r_0)^n$ pour tout $n \geq 0$ avec $(\lambda,\mu) \in \mathbb{C}^2$.
% est l'unique couple solution du système
%$$\l\{\begin{array}{lll}
%\lambda   & = & u_0\\
%(\lambda  + \mu ) r_0 & = & u_1
%\end{array}\r.$$
\end{itemize}
\end{Theoreme}

\begin{Theoreme}{Cas réel : $a,b \in \mathbb{R}$}
Soit $(u_n)_{n \geq 0}$ une suite récurrente linéaire d'ordre 2 définie comme ci-dessus avec $b \neq 0.$
\begin{itemize}
\item Si le le polynôme caractéristique admet deux racines réelles distinctes $r_1$ et $r_2$ alors le terme général de la suite $(u_n)_{n \geq 0}$ est donné par $u_n=\lambda (r_1)^n+\mu (r_2)^n$ pour tout $n \geq 0$ avec $(\lambda,\mu) \in \mathbb{R}^2$.
%est l'unique couple solution du système 
%$$\l\{\begin{array}{lllll}
%\lambda  &+ & \mu & = & u_0\\
%\lambda r_1 & + &\mu r_2 & = & u_1
%\end{array}\r.$$
\item Si le polynôme caractéristique admet une unique racine $r_0$ alors le terme général de la suite $(u_n)_{n \geq 0}$ est donné par $u_n= (\lambda\mu n)(r_0)^n$ pour tout $n \geq 0$ avec $(\lambda,\mu) \in \mathbb{R}^2$.
\item Si le polynôme caractéristique admet deux racines conjugués : $r e^{i \theta}$ et $r e^{- i \theta}$ alors le terme général de la suite $(u_n)_{n \geq 0}$ est donné par $u_n=r^n (\lambda \cos(n \theta)+ \mu \sin(n \theta))$ pour tout $n \geq 0$ avec $(\lambda,\mu) \in \mathbb{R}^2$.
% est l'unique couple solution du système
%$$\l\{\begin{array}{lll}
%\lambda   & = & u_0\\
%(\lambda  + \mu ) r_0 & = & u_1
%\end{array}\r.$$
\end{itemize}
\end{Theoreme}

\begin{Remarque}{} On détermine $\lambda$ et $\mu$ à l'aide de deux termes de la suite.
\end{Remarque}

\begin{Exemple} Soit $(u_n)_{n \geq 1}$ la suite définie par $u_1 = 1$, $u_2 = 0$ et pour tout $n \geq 1$ par $u_{n+2}= u_{n+1} + 2 u_n$. Déterminer son terme général.

%\medskip
%
%La suite $(a_n)_{n \geq 1}$ est récurrente linéaire d'ordre 2. Le discriminant du polynôme caractéristique $X^2-X-2$ vaut 9 et il admet donc deux racines distinctes $-1$ et $2$. Il existe donc deux réels $\lambda$ et $\mu$ tels que pour tout $n \geq 1$,
%\[ a_n = \lambda (-1)^n + \mu 2^n \]
%Pour déterminer $\lambda$ et $\mu$ on utilise les deux premières valeurs de la suite : $a_1=1$ et \linebreak $a_2=2b_1=0$. Pour $n=1$, on a alors
%\[ a_1 = \lambda (-1)^1 + \mu 2^1 = -\lambda + 2 \mu \]
%On a ainsi $- \lambda + 2 \mu = 1$.
%Pour $n=2$, on a 
%\[ a_2 = \lambda (-1)^2 + \mu 2^2 = \lambda + 4 \mu \]
%On a ainsi $ \lambda + 4 \mu = 0$.
%
%\medskip
%
%En ajoutant les deux équations on obtient : $6 \mu = 1$ ou encore $\mu = \frac{1}{6} \cdot$ Or $\lambda = - 4 \mu = - \frac{4}{6} = - \frac{2}{3}$ d'après la deuxième équation.
%
%\medskip
%
%Ainsi, pour tout $n \geq 1$, $a_n = - \frac{2}{3} (-1)^n + \frac{1}{6} \times 2^n.$

\vspace{11cm}
\end{Exemple}


$\phantom{test}$

\vspace{4cm}

\begin{ApplicationDirecte} Donner le terme général de la suite $(u_n)_{n \geq 0}$ définie par $u_0=0$, $u_1=1$ et pour tout $n \geq 0$ par :  $$u_{n+2} = 2 u_{n+1}-u_n$$
\end{ApplicationDirecte}

\begin{ApplicationDirecte} Donner le terme général de la suite $(u_n)_{n \geq 0}$ définie par $u_0=0$, $u_1=1$ et pour tout $n \geq 0$ par : $$ u_{n+2} =  u_{n+1} - u_n$$ \end{ApplicationDirecte}

\pagebreak
\section{Relations de comparaison}
\subsection{Généralités}
Considérons $(u_n)_{n \geq 0}$ et $(v_n)_{n \geq 0}$ deux suites.

\begin{Definition}{}

\begin{itemize}
\item On dit que la suite $(u_n)_{n \geq 0}$ est \emph{dominée} par la suite $(v_n)_{n \geq 0}$ si il existe une suite bornée $(w_n)_{n \geq 0}$ et un entier $N \in \mathbb{N}$ tel que pour tout entier $n \geq N$,
$$ u_n = w_n v_n$$
On note $u_n  \underset{+ \infty}{=} O(v_n)$.
\item On dit que la suite $(u_n)_{n \geq 0}$ est \emph{négligeable} devant la suite $(v_n)_{n \geq 0}$ si il existe une suite  $(w_n)_{n \geq 0}$ convergeant vers $0$ et un entier $N \in \mathbb{N}$ tel que pour tout entier $n \geq N$,
$$ u_n = w_n v_n$$
On note $u_n \underset{ + \infty}{=} o(v_n)$.
\item On dit que la suite $(u_n)_{n \geq 0}$ est \emph{équivalente} à la suite $(v_n)_{n \geq 0}$ si il existe une suite  $(w_n)_{n \geq 0}$ convergeant vers $1$ et un entier $N \in \mathbb{N}$ tel que pour tout entier $n \geq N$,
$$ u_n = w_n v_n$$
On note $u_n  \underset{ + \infty}{\sim} v_n$.
\end{itemize}
\end{Definition}

\medskip

Dans la pratique, si les termes de $(v_n)_{n \geq 0}$ sont non nuls à partir d'un certain rang $N$, on a :
%
%\begin{itemize}
%\item $u_n = \underset{n \rightarrow + \infty}{O}(v_n)$ si et seulement si la suite $\left( \dfrac{u_n}{v_n} \right)_{n \geq N}$ est bornée.
%\item $u_n = \underset{n \rightarrow + \infty}{o}(v_n)$ si et seulement si $\lim_{n \rightarrow + \infty} \frac{u_n}{v_n} = 0$.
%\item $u_n  \underset{n \rightarrow + \infty}{\sim} v_n$ si et seulement si $\lim_{n \rightarrow + \infty} \frac{u_n}{v_n} = 1$.
%\end{itemize}
%
%\medskip

\newpage

\begin{Remarques}{}
\begin{itemize}
\item $u_n  \underset{+ \infty}{=} O(1)$ signifie que \phantom{$(u_n)_{n \geq 0}$ est bornée.}
\item $u_n  \underset{+ \infty}{=} o(1)$ signifie que \phantom{$(u_n)_{n \geq 0}$ converge vers $0$.}
\item Les seules suites équivalentes à $0$ sont les suites nulles à partir d'un certain rang.
\end{itemize}
\end{Remarques}

\begin{ApplicationDirecte} Donner un équivalent des expressions suivantes quand $n$ tend vers $+ \infty$ :
$$ 3n+\ln(n)+5, \;  3n+\ln(n^2)+5, \; \dfrac{1}{n} + \dfrac{1}{n^2} \; \hbox{ et } \; \dfrac{3n^3+n^2+1}{n^3+n^2}$$
\end{ApplicationDirecte}

\subsection{Propriétés}

 Considérons $(u_n)_{n \geq 0}$, $(v_n)_{n \geq 0}$ et $(w_n)_{n \geq 0}$ trois suites et $(\lambda, \mu) \in \mathbb{C}^2$.

Voici une liste des propriétés à retenir :

\begin{enumerate}
\item $u_n  \underset{+ \infty}{\sim} v_n $ si et seulement si $u_n \underset{+ \infty}{=} v_n + o(v_n)$.
\item \emph{Transitivité} : 
\begin{itemize}
\item Si $u_n   \underset{+ \infty}{=}O(v_n)$ et $v_n   \underset{+ \infty}{=}O(w_n)$ alors $u_n   \underset{+ \infty}{=}O(w_n)$.
\item Si $u_n   \underset{+ \infty}{=}o(v_n)$ et $v_n   \underset{+ \infty}{=}o(w_n)$ alors $u_n   \underset{+ \infty}{=}o(w_n)$.
\end{itemize}
\item \emph{Linéarité} : 
\begin{itemize}
\item Si $u_n  \underset{+ \infty}{=}O(w_n)$ et $v_n \underset{+ \infty}{=}O(w_n)$ alors $\lambda u_n + \mu v_n  \underset{+ \infty}{=}O(w_n)$.
\item Si $u_n  \underset{+ \infty}{=}o(w_n)$ et $v_n \underset{+ \infty}{=}o(w_n)$ alors $\lambda u_n + \mu v_n  \underset{+ \infty}{=}o(w_n)$.
\end{itemize}
\item On peut multiplier et diviser des équivalents. On ne peut pas composer des équivalents par une fonction ou ajouter des équivalents.
\item Pour des suites \emph{réelles}, si $u_n  \underset{+ \infty}{\sim} v_n $ alors à partir d'un certain rang, les termes $u_n$ et $v_n$ sont de même signe. 

\begin{Demonstration}{}

\vspace{7cm}
\end{Demonstration}
\item Soit $\ell \neq 0$. Alors $\lim_{n \rightarrow + \infty} u_n = \ell$ si et seulement si $u_n \underset{+ \infty}{\sim} \ell$.
\item Si deux suites sont équivalentes, elles ont le même comportement asymptotique. Autrement dit, si l'une des deux suites tend vers une limite (finie ou infinie), il en est de même pour l'autre.
\end{enumerate}




\section{Développements limités}

Soient $n \in \mathbb{N}$ et $\alpha \in \mathbb{R}$. 

Voici les développements limités usuels en $0$ qu'il faut connaître parfaitement\footnote{A chaque erreur sur un développement limité, un chaton meurt de votre faute quelque part dans le monde.}.

\begin{center}
\fbox{\begin{minipage}{14cm}
$\e^x \egal_0 1+x+\frac{x^2}2+\dots+\frac{x^n}{n!}+o(x^n)\egal_0 \sum_{k=0}^n \frac{x^k}{k!}~+\,o(x^n)$ \label{DLexp}

$(1+x)^\alpha\egal_0 1+\alpha x+\frac{\alpha(\alpha-1)}{2}\,x^2+\dots+\frac{\alpha(\alpha-1)\cdots(\alpha-n+1)}{n!}\,x^n+o(x^n)$\label{DL1+xAl}

$\frac{1}{1-x}\egal_0 1+x+x^2+\dots+x^n+o(x^n)\egal_0 \sum_{k=0}^n x^k~+\,o(x^n)$\label{DL11-x}

$\frac{1}{1+x}\egal_0 1-x+x^2+\dots+(-1)^nx^n+o(x^n)\egal_0 \sum_{k=0}^n (-1)^kx^k~+\,o(x^n)$\label{DL11+x}

$\ln(1+x)\egal_0 x-\frac{x^2}{2}+\frac{x^3}{3}-\dots+(-1)^{n-1}\frac{x^n}n +o(x^n)\egal_0 \sum_{k=1}^n(-1)^{k-1}\frac{x^k}{k}~+\,o(x^n)$\label{DLln1+x}

$\arctan(x)\egal_0 x-\frac{x^3}{3}+\dots+(-1)^n\frac{x^{2n+1}}{2n+1}+o(x^{2n+2})\egal_0 \sum_{k=0}^n (-1)^k\frac{x^{2k+1}}{2k+1}+o(x^{2n+2})$\label{DLArctan}

$\tan(x)\egal_0 x+\frac{x^3}{3}+\frac{2x^5}{15}+o(x^6)$ \label{DLtan}

$\cos(x)\egal_0 1-\frac{x^2}{2}+\dots+\frac{(-1)^nx^{2n}}{(2n)!}+o(x^{2n+1})\egal_0 \sum_{k=0}^n\frac{(-1)^kx^{2k}}{(2k)!}~+\,o(x^{2n+1})$\label{DLcos}

$\sin(x)\egal_0 x-\frac{x^3}{6}+\dots+\frac{(-1)^nx^{2n+1}}{(2n+1)!}+o(x^{2n+2})\egal_0 \sum_{k=0}^n\frac{(-1)^kx^{2k+1}}{(2k+1)!}~+\,o(x^{2n+2})$\label{DLsin}

$\ch(x)\egal_0 1+\frac{x^2}{2}+\dots+\frac{x^{2n}}{(2n)!}+o(x^{2n+1})\egal_0 \sum_{k=0}^n\frac{x^{2k}}{(2k)!}~+\,o(x^{2n+1})$\label{DLch}

$\sh(x)\egal_0 x+\frac{x^3}{6}+\dots+\frac{x^{2n+1}}{(2n+1)!}+o(x^{2n+2})\egal_0 \sum_{k=0}^n\frac{x^{2k+1}}{(2k+1)!}~+\,o(x^{2n+2})$\label{DLsh}
% $\th(x)~=~x-\frac{x^3}{3}+\frac{2x^5}{15}-\frac{17x^7}{315}+o(x^8)$ \label{DLth}
%
%	\(
%	\arcsin(x)~=~x+\dots+\frac{1\times 3\times\dots\times(2n-1)}{2\times 4\times\dots\times(2n)}\,\frac{x^{2n+1}}{2n+1} + o(x^{2n+2})\\
%	\phantom{\arcsin(x)~}=~\sum_{k=0}^n \frac{(2k)!}{2^{2k}(k!)^2}\,\frac{x^{2k+1}}{2k+1}~+o(x^{2n+2})
%	\)\label{DLArcsin}
 \end{minipage}}
\end{center}

\medskip

\begin{Remarques}{}
\begin{itemize}
\item Les développements limités de l'exponentielle et de $(1+x)^{\alpha}$ s'obtiennent avec la formule de Taylor-Young. Les développements de sinus, cos, sinus hyperbolique et cosinus hyperbolique s'obtiennent avec le développement de l'exponentielle (ou à l'aide de la formule de Taylor-Young).  
\item Les développements limités de $\dfrac{1}{1+x}$ et $\dfrac{1}{1-x}$ proviennent de la formule donnant l'expression d'une somme géométrique:
\[
\forall n\in\N,\forall x\in\R \setminus \lbrace 1 \rbrace, \quad \sum_{k=0}^n x^k=\frac{1-x^{n+1}}{1-x}
\]
\item Le développement limité de $\ln(1+x)$ s'obtient par intégration du développement limité de $\dfrac{1}{1+x}\cdot$
\item Le développement limité de arctan s'obtient par intégration du développement limité de $\dfrac{1}{1+x^2} \cdot$
%\item  Le développement limité de arcsin s'obtient par intégration du développement limité de $(1+x)^{\alpha}$ avec $\alpha=-\frac12$ et $x$ substitué par $-x^2$. 
\end{itemize}
\end{Remarques}

\begin{ApplicationDirecte} Donner le développement limité en $0$ à l'ordre $3$ de $\dfrac{e^x}{1-x} \cdot$
\end{ApplicationDirecte}

\begin{ApplicationDirecte} Donner le développement limité en $0$ à l'ordre $3$ de $\ln(1+\sin(x)).$
\end{ApplicationDirecte}

%\exo Retrouver le développement limité à l'ordre $5$ de la fonction tangente en $0$ (on pourra l'écrire comme un quotient...)


\section{Équivalents et limites usuels}
\subsection{Équivalents usuels de fonctions}

Soit $\alpha$ un réel non nul.

\bigskip
\fbox{\parbox{\linewidth-7pt}{
\begin{align*}
\e^x-1&\equi_{0} x
&\sin(x)&\equi_{0} x
&\arctan(x)&\equi_{0} x\\
\ln(1+x)&\equi_{0} x
&1-\cos(x)&\equi_{0} \frac{x^2}{2}
&\sh(x)&\equi_{0} x\\
\ln(u)&\equi_{1}u-1
&\tan(x)&\equi_{0} x
&\th(x) &\equi_{0} x\\
(1+x)^\alpha-1&\equi_{0}\alpha x
&\arcsin(x) &\equi_{0} x
&\ch(x)-1 &\equi_{0} \frac{x^2}{2}\phantom{1+x}
\end{align*}}}
 
 \bigskip
 
 Les équivalents usuels redonnent en particulier des limites usuelles à connaître. 

\medskip
 \begin{ApplicationDirecte} Donner un équivalent de $(n+1) \sin \left(\dfrac{1}{\sqrt{n^2+1}} \right)$ en $+ \infty$.
 \end{ApplicationDirecte}

\begin{ApplicationDirecte} Donner un équivalent de $\ln \left( \frac{n+1}{n-1} \right)$ en $+ \infty$.
\end{ApplicationDirecte}
 
 \subsection{Limites}
 

\begin{Theoreme}{croissances comparées}
Soient $\alpha$ et $\beta$ deux réels strictement positifs et $n \in \N^*$.
\begin{multicols}{2}
\begin{itemize}
\item $\lim_{x \rightarrow +\infty} \frac{e^{\beta x}}{x^{\alpha}}= + \infty$
\item $\lim_{x \rightarrow -\infty} |x|^{\beta}e^{\alpha x}= 0$

$\phantom{}$
\vspace{0.1cm}

\columnbreak
\item $\lim_{x \rightarrow +\infty} \frac{\ln^{\beta}(x)}{x^{\alpha}}=0$
\item $\lim_{x\to 0} x^{\alpha} |\ln(x)|^{\beta} =0 $

$\phantom{}$
\vspace{0.1cm}

\end{itemize}
\end{multicols}
\end{Theoreme}

\section{Des inégalités}

Il est très important en analyse de savoir manipuler des inégalités. Voici quelques inégalités usuelles (qu'il faut connaître et savoir redémontrer) : 

\begin{multicols}{2}
\begin{itemize}
\item $\forall x \in \mathbb{R}$, $e^x \geq 1+x$.
\item $\forall x > -1$, $\ln(1+x) \leq x$.
\item $\forall x \in \mathbb{R}$, $\vert \sin(x) \vert \leq \vert x \vert$.
\columnbreak
\item $\forall (x,y) \in \mathbb{R}^2$, $\vert x+y \vert \leq \vert x \vert + \vert y \vert$.
\item $\forall (x,y) \in \mathbb{R}^2$, $\vert \vert x \vert - \vert y \vert \vert \leq \vert x-y \vert$.
\item $\forall (x,y) \in \mathbb{R}^2$, $xy \leq \dfrac{x^2+y^2}{2}\cdot$
\end{itemize}
\end{multicols}

\end{document}
