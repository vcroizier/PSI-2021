\documentclass[a4paper,10pt]{report}
\usepackage{cours}
\usepackage{pifont}

\begin{document}
\everymath{\displaystyle}
\begin{center}
\textit{{ {\huge TD 4 : Applications linéaires}}}
\end{center}

\bigskip


\noindent Dans tout le TD, $n$ sera un entier naturel non nul, $\mathbb{K}$ désignera $\mathbb{R}$ ou $\mathbb{C}$ et $I$ sera un intervalle de $\mathbb{R}$ contenant au moins deux points.

\medskip

\begin{center}
\textit{{ {\large Noyaux, images}}}
\end{center}


\begin{Exercice}{} Montrer que les applications suivantes sont linéaires, déterminer leur noyau et leur image (et le rang, si cela a un sens). On précisera si c'est un endomorphisme, isomorphisme ou un automorphisme et on vérifiera si l'application est bien définie.

\begin{enumerate}
\item $f : \mathbb{R}^3 \rightarrow \mathbb{R}^3$ définie par :
$$ \forall (x,y,z) \in \mathbb{R}^3, \; f((x,y,z))= (x+2y+z,y-x,2x+4y+z)$$
\item $f : \mathbb{R}^3 \rightarrow \mathbb{R}^3$ définie par :
$$ \forall (x,y,z) \in \mathbb{R}^3, \; f((x,y,z))= (x+z,x+z,x+y)$$
\item $f : \mathbb{R}_2[X] \rightarrow \mathbb{R}_2[X]$ définie par :
$$ \forall P \in \mathbb{R}_2[X], \; f(P)=P-(X+1)P'+X^2 P''$$
\item $f : \mathcal{M}_2(\mathbb{R}) \rightarrow \mathcal{M}_2(\mathbb{R})$ définie par :
$$ \forall M \in \mathcal{M}_2(\mathbb{R}), \; f(M)=AM \; \hbox{ où } \; A= \begin{pmatrix}
1 & 2 \\
2 & 4 \\
\end{pmatrix} $$
\item $\varphi : \mathcal{C}^1(I, \mathbb{R}) \rightarrow \mathcal{F}(I, \mathbb{R})$ définie par :
$$ \forall f \in \mathcal{C}(I, \mathbb{R}), \; \varphi(f)=f'$$
\end{enumerate}
\end{Exercice}

\begin{Exercice}{} Soit $\varphi : \mathbb{K}_{n+1}[X]\rightarrow \mathbb{K}_{n}[X]$ définie par $\varphi(P) = (n + 1)P - XP'$.
    \begin{enumerate}
      \item
        Justifier que $\varphi$ est bien définie et que c'est une application linéaire.
      \item
        Déterminer le noyau de $\varphi$.
      \item
        $\varphi$ est-elle surjective?
    \end{enumerate}
\end{Exercice}


\begin{Exercice}{} Soient $n \geq 1$ et $a_{0} ,a_{1} , \ldots ,a_{n}$ des éléments deux à deux distincts de $\mathbb{C}$. Montrer que l'application $\varphi \colon \mathbb{C}_n[X] \rightarrow \mathbb{C}^{n+1}$ définie par :
  \[
  \varphi(P) = (P(a_{0}),P(a_{1}), \ldots ,P(a_{n}))
  \]
  est un isomorphisme.
\end{Exercice}

\begin{Exercice}{\ding{80}] Pour tout $P \in \mathbb{R}_3[X]$, on pose $\varphi(P)$ le reste de la division euclidienne de $(X^4-1)P$ par $X^4-X$. Montrer que $\varphi$ définit ainsi une endomorphisme de $\mathbb{R}_3[X}$ puis déterminer son noyau et son image.
\end{Exercice} 



\begin{Exercice}{}
Soit $u$ un endomorphisme d'un espace vectoriel $E$ de dimension $n \geq 1$.
\begin{enumerate}
\item On suppose $u$ injectif. Déterminer pour tout $m \geq 1$, $K_m = \textrm{Ker}(u^m)$ et $I_m = \textrm{Im}(u^m)$.
\item Montrer que pour tout $m \geq 0$, $K_m \subset K_{m+1}$ et $I_{m+1} \subset I_m$.
\item On suppose $u$ non injectif. Montrer l'existence d'un entier $p \in \Interv{1}{n}$ tel que $K_p=K_{p+1}$ et $I_p = I_{p+1}$.

\noindent Montrer alors que pour tout $q \in \mathbb{N}$, $K_p = K_{p+q}$ et $I_p = I_{p+q}$ puis que $E = K_p \oplus I_p$.
\end{enumerate}
\end{Exercice} 


\begin{Exercice}{} Soit $\Delta : \mathbb{C}[X] \rightarrow \mathbb{C}[X]$ l'application définie par :
  \[
  \Delta (P ) = P( X + 1 ) - P(X )
  \]
  \begin{enumerate}
  \item
    Montrer que $\Delta$ est un endomorphisme de $\mathbb{C}[X]$ et que pour tout polynôme $P$ non constant, $\textrm{deg} ( \Delta(P)) = \textrm{deg}(P) - 1$.
  \item
    Déterminer $\textrm{Ker}(\Delta)$ et $\textrm{Im}(\Delta)$.
  \item Soient $P \in \mathbb{C}[X]$ et $n \in \N$. Montrer que :
    \[
    \Delta^{n}(P) = ( - 1)^{n} \sum_{k = 0}^{n} ( - 1)^{k} \binom{n}{k}P(X + k)
    \]
  \item En déduire que, si $\textrm{deg}(P) < n$, alors :
    \[
    \sum_{k = 0}^{n} \binom{n}{k}( - 1)^{k} P(k) = 0 
    \]
  \end{enumerate}
\end{Exercice} 

\begin{Exercice}{} Soient $u$, $v$ deux endomorphismes d'un $\mathbb{K}$-espace vectoriel $E$. On suppose que $u$ et $v$ commutent. Montrer que $\textrm{Ker}(u)$ et $\textrm{Im}(u)$ sont stables par $v$.
\end{Exercice} 

\medskip

\begin{center}
\textit{{ {\large Rang d'une application linéaire}}}
\end{center}

\medskip


\begin{Exercice}{} Soient $E$ un $\mathbb{K}$-espace vectoriel de dimension finie et $f \in \mathcal{L}(E)$. Montrer que :
$$ \textrm{rg}(f^2 ) =  \textrm{rg}( f ) \, \Longleftrightarrow E = \textrm{Ker}(f) \oplus \textrm{Im}(f) $$
\end{Exercice}


\begin{Exercice}{\ding{80}} Soient $f,g \in \mathcal{L}(E)$ où $E$ est un $\mathbb{K}$-espace vectoriel de dimension finie. Montrer que :
    \[
    \vert \textrm{rg}(f) - \textrm{rg}(g) \vert \leq \textrm{rg}(f + g) \leq \textrm{rg}(f) + \textrm{rg}(g)
    \]
\end{Exercice}

\medskip

\begin{center}
\textit{{ {\large Projections, symétries}}}
\end{center}

\medskip


\begin{Exercice}{} Soient $E= \mathbb{R}^3$ et $F$, $G$ les ensembles définis par :
$$ F = \lbrace (x,y,z) \in \mathbb{R}^3 \, \vert \, x-y+z=0 \rbrace \; \hbox{ et }  \; G = \textrm{Vect}((1,1,1)) $$

\begin{enumerate}
\item Donner une base de $F$ et préciser sa dimension.
\item Montrer que $F$ et $G$ sont supplémentaires dans $\mathbb{R}^3$.
\item Soit $p$ la projection de $E$ sur $F$ parallèlement à $G$. Donner l'expression de $p$.
\item Soit $q$ la projection de $E$ sur $G$ parallèlement à $F$. Donner l'expression de $q$.
\end{enumerate}
\end{Exercice}



\begin{Exercice}{} Donner l'expression de la projection de $\mathcal{M}_n(\mathbb{R})$ sur $\mathcal{S}_n(\mathbb{R})$ parallèlement à $\mathcal{A}_n(\mathbb{R})$ et celle de la symétrie par rapport à $\mathcal{S}_n(\mathbb{R})$ parallèlement à $\mathcal{A}_n(\mathbb{R})$.
\end{Exercice}



\begin{Exercice}{} Donner l'expression de la projection de $\mathcal{F}(\mathbb{R}, \mathbb{R})$ sur le sous-espace vectoriel $P$ des fonctions paires parallèlement au sous-espace vectoriel $I$ des fonctions impaires.
\end{Exercice}


\begin{Exercice}{\ding{80}} Soient $p,q$ deux endomorphismes d'un espace vectoriel $E$. Montrer que les assertions suivantes sont équivalentes :
    \begin{enumerate}
\item $p \circ q = p$ et $q \circ p = q$.
\item $p$ et $q$ sont des projecteurs de même noyau.
    \end{enumerate}
\end{Exercice}




\medskip

\begin{center}
\textit{{ {\large Formes linéaires et hyperplans}}}
\end{center}

\medskip

\begin{Exercice}{\ding{80}] Soient $E= \mathbb{R}_n[X}$ et $L \in \mathcal{L}(E, \mathbb{R})$ définie par :
$$ \forall P \in E, \;  L(P) = \int_{-1}^1 P(t) \dt$$
\begin{enumerate}
\item Déterminer l'image de $P= \sum_{k=0}^n a_k X^k$ par $L$.
\item Déterminer la dimension puis une base du noyau de $L$.
\item Soit $(x_0, x_1, \ldots, x_n) \in \mathbb{R}^n$ tels que $-1 \leq x_0 < x_1 < \cdots < x_n \leq 1$. Montrer que pour tout $i \in \Interv{0}{n}$, il existe un unique polynôme $P_i$ de $E$ tel que pour tout $j \in \Interv{0}{n}$, $P_i(x_j)= \delta_{i,j}$.
\item Montrer que $(P_0, \ldots, P_n)$ est une base de $E$.
\item Montrer qu'il existe $(\lambda_0, \ldots, \lambda_n) \in \mathbb{R}^{n+1}$ tel que :
$$ \forall P \in E, \; \int_{-1}^1 P(t) \dt = \lambda_0 P(x_0) + \cdots + \lambda_n P(x_n)$$
\end{enumerate}
\end{Exercice}

\begin{Exercice}{} Soient $f$ une forme linéaire sur un $\mathbb{R}$-espace vectoriel $E$, de dimension $n$, et $a\in E$ tel que $f(a)\neq 0$.
\begin{enumerate}
\item Montrer que $E=\textrm{Ker}(f)\oplus\textrm{Vect}(a)$.
\item On suppose $f(a)=1$. Montrer que $p : E \rightarrow E$ définie par :
$$ \forall x \in E, \; p(x) = f(x) a $$
est un projecteur et donner les sous-espaces $F$ et $G$ tels que $p$ soit le projecteur sur $F$ parallèlement à $G$.
\end{enumerate}
\end{Exercice}


\begin{Exercice}{} 
\begin{enumerate}
\item Montrer que $H$ défini par :
$$ H = \lbrace (x_1, \ldots, x_n)\in \mathbb{R}^n \; \vert \;    x_1 + \cdots + x_n = 0 \rbrace$$
est un sous-espace vectoriel de $\mathbb{R}^n$ et donner sa dimension.
\item Déterminer l'expression de la symétrie par rapport à $H$ parallèlement à $G$ défini par :
$$ G = \textrm{Vect}((1,1, \ldots, 1))$$
\end{enumerate}
\end{Exercice}



\begin{Exercice}{} Soit $E$ un $\mathbb{K}$-espace vectoriel de dimension finie supérieure ou égale à 2. Soient $H_1$ et $H_2$ deux hyperplans de $E$ distincts. Déterminer la dimension de $H_1 \cap H_2$.
\end{Exercice}

\newpage

\begin{center}
\textit{{ {\large Divers}}}
\end{center}

\medskip

\begin{Exercice}{} \begin{enumerate}
 \item Montrer que l'ensemble $V$ des suites complexes défini par :
 $$ V = \lbrace (v_n)_{n \geq 0} \, \vert \, \forall n \geq 0, \; v_{n+3}=v_{n+2}+v_n \rbrace$$
est un $\C$-espace vectoriel.
 \item Montrer que $P(X)=X^3-X^2-1$ admet une unique racine réelle et deux racines complexes conjuguées.
 \item Montrer que $\phi : V \rightarrow \mathbb{C}^3$ définie par :
 $$ \forall v=(v_n)_{n \geq 0} \in V, \;\phi(v)=(v_0,v_1,v_2)$$
  définit un isomorphisme de $V$ dans $\C^3$ et en déduire la dimension de $V$.
 \item Trouver les suites géométriques de $V$ et en déduire une base de $V$.
\end{enumerate}
\end{Exercice}


\begin{Exercice}{} Soient $E$ un $\mathbb{K}$-espace vectoriel et $f \in \mathcal{L}(E)$ tel que :
    \[
    f^2 - 3f + 2 \textrm{Id}_E = \theta
    \]
où $\theta$ est l'endomorphisme nul de $E$.
    \begin{enumerate}
      \item Montrer que $f$ est bijective et donner sa bijection réciproque.
      \item Montrer que $\textrm{Ker}(f - \textrm{Id}_E)$ et $\textrm{Ker}(f - 2\textrm{Id}_E)$ sont des sous-espaces vectoriels supplémentaires de $E$.
    \end{enumerate}
\end{Exercice}



\begin{Exercice}{\ding{80}} Soient $E$ un $\mathbb{K}$-espace vectoriel et $f \in \mathcal{L}(E)$.

\begin{enumerate}
\item On suppose que pour tout $x \in E$, $(x, f(x))$ est liée.
\begin{enumerate}
\item Montrer que pour tout $x \in E$, il existe $\lambda_x \in \mathbb{K}$ tel que $f(x)= \lambda_x x$.
\item Montrer que pour tous $x,y \in E$, non nuls, on a $\lambda_x = \lambda_y$. \textit{On pourra commencer par le cas où $(x,y)$ est une famille liée.}
\item En déduire que $f$ est une homothétie vectorielle.
\end{enumerate}
\item Déterminer les endomorphismes $f$ de $\mathcal{L}(E)$ commutant avec tous les endomorphismes.
\end{enumerate}
\end{Exercice} 








%\noindent Dans la suite, $\mathbb{K}= \mathbb{R}$ ou $\mathbb{C}$.
%
%\medskip
%
%\begin{center}
%{\large \textit{\underline{Noyaux, images}}}
%\end{center}
%
%\bigskip
%%
%%\exo Soit $f$ l'application de $\mathbb{R}^4$ dans $\mathbb{R}^4$ définie par 
%%\[ f(x,y,z,t) = (x+2y+t,2x+z,-5x+6y-4z+3t,4x+4y+z+2t) \]
%%
%%\begin{enumerate}
%%\item Montrer que $f$ est une application linéaire.
%%\item Déterminer le noyau de $f$.
%%\item Que vaut le rang de $f$?
%%\end{enumerate}
%%
%%\medskip
%
%
%\exo Soit $n \in \mathbb{N}^*$. On pose pour tout $P \in \mathbb{R}_n[X]$, $\Delta(P)= P-P'$.
%
%\begin{enumerate}
%\item Montrer que $\Delta$ induit un endomorphisme de $\mathbb{R}_n[X]$.
%\item Déterminer le noyau et l'image de $\Delta$.
%\end{enumerate}
%
%\medskip
%

%    
%\medskip
%

%\medskip
%
%\exo Soit $f$ un endomorphisme d'un $\mathbb{K}$-espace vectoriel $E$ vérifiant $f^3 = \textrm{Id}_E$. Montrer que :
%    \[
%    \textrm{Ker}(f -  \textrm{Id}_E) \oplus  \textrm{Im}(f - \textrm{Id}_{E}) = E
%    \]
%

%
%
%\exo Soient $A = \begin{pmatrix}
%1 & 1 \\
%0 & 1 \\
%\end{pmatrix}$ et $\varphi : \mathcal{M}_2(\mathbb{R}) \rightarrow \mathcal{M}_2(\mathbb{R})$ définie par $\varphi(M)= AM-MA$.
%
%\begin{enumerate}
%\item Montrer que $\varphi$ est un endomorphisme de $\mathcal{M}_2(\mathbb{R})$.
%\item Donner une base $\textrm{Ker}{(\varphi)}$ et une base de $\textrm{Im}{(\varphi)}$.
%\end{enumerate}
%
%
%
%\newpage
%
%
%\begin{center}
%{\large \textit{\underline{Projections, symétries}}}
%\end{center}
%
%\medskip
%
%\exo Soient $E$ un $\mathbb{C}$-espace vectoriel de dimension finie et $u \in \mathcal{L}(E)$. On suppose qu'il existe un projecteur $p$ de $E$ tel que $u = p \circ u - u \circ p$.
%\begin{enumerate}
%\item Montrer que $u(\textrm{Ker} (p)) \subset \textrm{Im}(p)$ et $\textrm{Im}(p) \subset \textrm{Ker}(u)$.
%\item En déduire $u^2 = \theta$ (endomorphisme nul).
%\item La réciproque est-elle vraie ?
%    \end{enumerate}
%\bigskip
%

%\exo Soient $n \in \mathbb{N}^*$ et $(e_1, \ldots, e_n)$ une base d'un $\mathbb{K}$-espace vectoriel $E$. Posons pour tout $i \in \Interv{1}{n}$,
%$$ F_i = \lbrace u \in \mathcal{L}(E), \, \textrm{Im}(u) \subset \textrm{Vect}(e_i) \rbrace$$
%Montrer que $\mathcal{L}(E) = F_1 \oplus F_2 \oplus \cdots \oplus F_n$.
%
%\medskip
%
%\exo Trouver toutes les formes linéaires $\Phi$ de $\mathcal{M}_n(\mathbb{R})$ vérifiant :
%$$ \forall (A,B) \in \mathcal{M}_n(\mathbb{R})^2, \, \Phi(AB)= \Phi(BA)$$
%
%\medskip
%
%\exo Soit $f \in \mathcal{L}(\mathbb{R}^6)$ tel que $\textrm{rg}(f^2) = 3$. Quels sont les rangs possibles pour $f$?



\end{document}