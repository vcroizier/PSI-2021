\documentclass[a4paper,french,11pt,twoside]{VcCours}

\newcommand{\dt}{\text{d}t}
\newcommand{\dx}{\text{d}x}

\begin{document}
\Titre{PSI}{Promotion 2021--2022}{Mathématiques}{Devoir non surveillé n°5}
\begin{center}
\large\bf 
Pour le vendredi 12 novembre 2021
\end{center}
\separationTitre
\vspace{-1.5em}
\begin{itemize}
   \item Alistair, Mathis et Barnabé travailleront sur le sujet Centrale-Supélec Mathématiques 2 PC 2018 et feront l'exercices 2 de ce DNS. Ils peuvent, s'ils veulent, traiter aussi les exercices 1 et 3.
   \item Anta, Marouane et Adama peuvent faire les questions 1 à 15 du sujet Centrale ci-dessus. Ils doivent rendre ce DNS en entier.
   \item Tous les autres élèves doivent rendre ce DNS en entier.
\end{itemize}
\separationTitre
   
\section*{Exercice 1}   
   Soit $n \in \mathbb{N}$. On définit la fonction $f_n : \mathbb{R} \rightarrow \mathbb{R}$ par :
   $$ \forall x \in \mathbb{R}, \; f_n(x) = \dfrac{x^n}{\sqrt{1+x}}$$
   
   \begin{enumerate}
   \item Étudier les variations de $f_n$ sur $[0,1]$.
   \item Représenter sur le même graphique les graphes des fonctions $f_0$, $f_1$ et $f_2$.
   \item Démontrer que $(f_n)_{n \geq 0}$ converge simplement sur $[0,1]$ vers une fonction $f$ que l'on explicitera.
   \item Montrer, de deux manières différentes, que $(f_n)_{n \geq 0}$ ne converge pas uniformément sur $[0,1]$ vers $f$.
   \item On définit la suite $(u_n)_{n \geq 0}$ par :
   $$ \forall n \in \mathbb{N}, \; u_n = \int_0^1 f_n(x) \dx$$
   \begin{enumerate}
   \item Montrer que $(u_n)_{n \geq 0}$ est monotone puis qu'elle converge vers $0$.
   \item Montrer que pour tout entier $n \geq 0$,
   $$ (n+1) u_n = \dfrac{1}{\sqrt{2}} + \dfrac{1}{2} \int_0^1 \dfrac{x^{n+1}}{(\sqrt{1+x})^3} \dx$$
   \item En déduire un équivalent de $u_n$ quand $n$ tend vers $+ \infty$.
   \item En suivant la même démarche que précédemment, déterminer des réels $\alpha$, $\beta$ tels que :
   $$ u_n \underset{+ \infty}{=} \dfrac{\alpha}{n} + \dfrac{\beta}{n^2} + o \left( \dfrac{1}{n^2} \right)$$
   \end{enumerate}
   \end{enumerate}
   
\section*{Exercice 2}   
   Soit $S$ définie par :
   $$ S(x) = \sum_{n=1}^{+ \infty} \dfrac{(-1)^n}{x+n}$$
   \begin{enumerate}
   \item Montrer que $S$ est définie sur $I= ]-1, + \infty[$.
   \item Montrer que $S$ est continue sur $I$.
   \item Montrer que $S$ est de classe $\mathcal{C}^1$ sur $I$ et déterminer $S'$. En déduire que $S$ est croissante sur $I$.
   \item Déterminer la limite de $S$ en $+ \infty$.
   \item Déterminer la limite de $S$ en $-1^+$.
   \end{enumerate}
   
\section*{Exercice 3}   
   Soit $(a_n)_{n \geq 0}$ une série à termes positifs telle que la série de terme général $a_n$ converge. Montrer que que la fonction $f$ définie par :
   $$ f(x) = \sum_{n=0}^{+ \infty} a_n \cos (nx)$$
   est bien définie et continue sur $\mathbb{R}$. Donner une condition simple pour qu'elle soit de classe $\mathcal{C}^1$ sur $\mathbb{R}$.


\end{document}