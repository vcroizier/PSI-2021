\documentclass[a4paper,french,11pt,twoside]{VcCours}

\newcommand{\dt}{\text{d}t}
\newcommand{\dx}{\text{d}x}

\begin{document}
\Titre{PSI}{Promotion 2021--2022}{Mathématiques}{Devoir non surveillé n°2}
\begin{center}
\large\bf 
Pour le vendredi 17 septembre 2021
\end{center}
\separationTitre


\section*{Problème -- Analyse}


\subsection*{Partie 1}


\medskip

\noindent On considère la fonction $f$ définie par :
$$f\left(  x\right)=\dfrac{\ln\left(  1+x\right)}{x}$$

\medskip

\begin{enumerate}
\item Déterminer l'ensemble de définition, notée $\mathcal{D}$, de $f$.
\item Donner le développement limité de $\ln\left(  1+x\right)  $ au voisinage
de $0$ à l'ordre $2$.

\noindent Montrer que $f$ admet en $0$ un prolongement par continuité. On précisera par
quelle valeur $f$ est alors prolongée et on continuera à appeler $f$ le prolongement ainsi obtenu. On appellera $\mathcal{D}'$ le nouvel ensemble de définition de $f$.

\item $f$ est-elle dérivable en $0$? Si oui, préciser $f'(0)$.

\noindent Calculer $f'(x)$ pour $x$ appartenant à $\mathcal{D}$ puis prouver que $f$ est de
classe $C^{1}$ sur $\mathcal{D}^{\prime}$.

\item Étudier les variations de $f$. On dressera son tableau de variations.

\noindent On pourra utiliser la fonction auxiliaire $k$ définie par : 
$$k\left(x\right)  =x-\left(  1+x\right)  \ln\left(  1+x\right)$$
\end{enumerate}

\medskip

\subsection*{Partie 2}

\medskip

\noindent Dans la suite, on s'intéressera à l'intégrale suivante : $%
%TCIMACRO{\dint _{0}^{1}}%
%BeginExpansion
{\int_{0}^{1}}
%EndExpansion
f\left(  t\right)  \dt.$

\medskip

\noindent On notera $L$ la valeur de cette intégrale mais on ne cherchera pas à calculer
cette valeur.

\medskip

\noindent Pour tout entier naturel $n$ non nul, on définit les polyn\^omes $P_n$ et $Q_n$ par :

$$ \left\lbrace \begin{array}{ll}
P_{n}\left(  X\right)  & =X-\dfrac{X^{2}}{2}+\dfrac{X^{3}}{3}-\dfrac{X^{4}}
{4}+\cdots+\left(  -1\right)  ^{n-1}\dfrac{X^{n}}{n} \\[0.3cm]
Q_{n}\left(  X\right)  &=X-\dfrac{X^{2}}{2^{2}}+\dfrac{X^{3}}{3^{2}}
-\dfrac{X^{4}}{4^{2}}+\cdots+\left(  -1\right)  ^{n-1}\dfrac{X^{n}}{n^{2}}\\
\end{array}\right.$$

\medskip

\begin{enumerate}
\item Préciser pourquoi l'intégrale précédente est bien définie.

\item Calculer, pour tout $t \in [0,1]$, 
$$1-t+t^{2}-t^{3}+\cdots+\left(  -1\right)  ^{n-1}t^{n-1}$$

\item En déduire que pour tout $x \in [0,1]$,
$$P_{n}\left(
x\right)  =\ln\left(  1+x\right)  -%
%TCIMACRO{\dint _{0}^{x}}%
%BeginExpansion
{\int_{0}^{x}}
%EndExpansion
\dfrac{\left(  -t\right)  ^{n}}{1+t}\dt$$

\medskip

\noindent Dans toute la suite on notera pour tout $n\in\mathbb{N}$ et tout $x\in\left[  0,1\right]$,
$$ R_{n}\left(  x\right)  =%
%TCIMACRO{\dint _{0}^{x}}%
%BeginExpansion
{\int_{0}^{x}}
%EndExpansion
\dfrac{\left(  -t\right)  ^{n}}{1+t}\dt$$

\item Établir que pour tout entier $n \geq 0$ et tout $x \in [0,1]$,
$$ \left\vert R_{n}\left(  x\right)  \right\vert \leq
\dfrac{x^{n+1}}{n+1}$$

\item Comparer pour tout $x\in\left]  0,1\right]  :Q_{n}' \left(
x\right)  $ et $\dfrac{P_{n}\left(  x\right)  }{x} \cdot$

\item Soit $n \geq 0$. On définit l'application $g_n : ]0,1] \rightarrow \mathbb{R}$ par :
$$ g_n\left(  x\right)  = \left\lbrace \begin{array}{ll}
\dfrac{P_{n}\left(  x\right)  }%
{x}-\dfrac{\ln\left(  1+x\right)  }{x} & \hbox{ si } x \in ]0,1] \\
 0 & \hbox{ si } x = 0 \\
 \end{array}\right.$$
Montrer que :

\[
\left\vert Q_{n}\left(  1\right)  -L\right\vert \leq\int_{0}^{1}\left\vert
g_{n}\left(  x\right)  \right\vert \dx\leq\dfrac{1}{\left(  n+1\right)  ^{2}}%
\]
En déduire $\lim\limits_{n\rightarrow+\infty}Q_{n}\left(  1\right).$

\item Déterminer un entier naturel $N$ tel que $Q_{N}\left(  1\right)  $ donne
une valeur approchée de $L$ à $10^{-4}$ près.
\item Écrire une fonction {\tt Appr} en python prenant en argument un réel $\varepsilon>0$ et renvoyant une valeur approchée de $L$ à $\varepsilon$ près.
\end{enumerate}

\medskip

\subsection*{Partie 3}

\medskip

\noindent On s'intéresse à présent aux dérivées successives de $f$ que l'on note
$f^{\left(  n\right)  }$, $n\in\mathbb{N}^*$.

\begin{enumerate}
\item Montrer que $f$ est indéfiniment dérivable $\left]  0,+\infty\right[  $

\item Déterminer $f''$.

\item Montrer que pour tout entier naturel $n$ non nul il existe un polyn\^ome
$T_{n}$ à coefficients réels et un réel $a_{n}$ tels que :%
\[
\forall x\in\mathbb{R}_{+}^{*},\quad f^{\left(  n\right)  }\left(
x\right)  =\dfrac{T_{n}\left(  x\right)  }{\left(  1+x\right)  ^{n}x^{n}%
}+a_{n}\dfrac{\ln\left(  1+x\right)  }{x^{n+1}}%
\]


\item Montrer que tous les coefficients de $T_{n}$ sont des entiers.

\item En utilisant la formule de Leibniz déterminer $f^{\left(  n\right)
}\left(  x\right)  $ et en déduire l'expression de $T_{n}$.

On ne cherchera pas à expliciter une expression de chacun des coefficients de
$x^{k}$ $\left(  k\in\mathbb{N}\right)  $ de ce polyn\^ome.

Vérifier  cette expression pour $n=2$.
\end{enumerate}






\end{document}