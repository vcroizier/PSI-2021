\documentclass[a4paper,french,11pt,twoside]{VcCours}

\begin{document}
\Titre{PSI}{Promotion 2021--2022}{Mathématiques}{Devoir non surveillé n°1}
\begin{center}
\large\bf 
Pour le lundi 13 septembre 2021
\end{center}
\separationTitre


\section*{Exercice 1}

\emph{Les questions sont indépendantes.}

\begin{enumerate}
\item Résoudre sur $\mathbb{R}$ l'inéquation suivante :
$$ \cos(x) > \cos(x-\pi/6)$$
%\item Déterminer la forme algébrique de $z^{10}$ où :
%$$z= \dfrac{1+i \sqrt{3}}{1-i}  $$
\item Déterminer l'ensemble des entiers relatifs $n$ tels que $(1+i)^n$ 
soit réel.
\item Soit $\theta \in \mathbb{R}$. Linéariser $\sin^4(\theta)$ puis calculer 
la quantité suivante :
$$ S = \sin^4(\pi/8)+ \sin^4(3\pi/8) + \sin^4(5\pi/8) + \sin^4(7\pi/8)$$
\item Résoudre dans $\mathbb{C}$ l'équation suivante :
$$ (E) \quad i z^2+ \sqrt{3}z-1=0$$
\item Déterminer les racines troisièmes de $Z=-8i$.
\end{enumerate}


\section*{Exercice 2}
On considère la suite $(u_n)_{n \geq 0}$ définie par $u_0=0$ et pour tout 
$n \in \mathbb{N}^*$ par :
$$ u_n=\sqrt[4]{n+\sqrt[4]{(n-1)+\sqrt[4]{\cdots+\sqrt[4]{1}}}}$$

\begin{enumerate}
\item Établir une relation entre $u_n, u_{n-1}$ et $n,$ pour tout
$n\in\N^*.$
\item\begin{enumerate}
\item Montrer par récurrence que, pour tout $n\in\N, \ u_n\leq
\sqrt{n}.$
\item En déduire que $u_n\egal_{+\infty}o(n)$ puis que : 
$$u_n \equi_{+\infty}\sqrt[4]{n}$$
\end{enumerate}
\item On considère alors la suite $(\alpha_n)_{n \geq 0}$
définie pour tout $n \in \mathbb{N}$ par :
$$ \alpha_n=u_n-\sqrt[4]{n}$$
Déduire de la première question un équivalent de $\alpha_n$ puis montrer que :
$$u_n \egal_{+\infty}n^{1/4}+\frac{1}{4\sqrt{n}}+o\left(\frac{1}{\sqrt{n}}\right) $$
\end{enumerate}


\section*{Exercice 3}
Soit $f : \mathbb{R} \rightarrow \mathbb{R}$ une fonction continue en $0$ telle que :
$$ \forall (x,y) \in \mathbb{R}^2, \; f(x+y)=f(x)+f(y) $$
On pose $a=f(1)$.

\begin{enumerate}
\item Déterminer $f(0)$.
\item Montrer que pour tout réel $x$, $f(-x)=-f(x)$. En déduire que $f$ est 
continue sur $\mathbb{R}$.
\item Montrer que pour tout entier $n \geq 0$ et tout réel $x$, $f(nx)=nf(x)$ 
puis que le résultat reste vrai pour $n \in \mathbb{Z}$.
\item Montrer que pour tout $p \in \mathbb{Z}$ et tout $q \in \mathbb{N}^*$, 
$f(p/q) = \dfrac{ap}{q}\cdot$
\item Soit $x \in \mathbb{R}$. On pose pour tout entier $n \geq 1$,
$$ r_n = \dfrac{\lfloor nx \rfloor}{n}$$
Montrer que $(r_n)_{n \geq 1}$ converge vers $x$.
\item En déduire l'expression de $f(x)$ pour tout réel $x$. 
\end{enumerate}


\end{document}