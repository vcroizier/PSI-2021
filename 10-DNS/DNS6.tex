\documentclass[a4paper,french,11pt,twoside]{VcCours}

\newcommand{\dt}{\text{d}t}
\newcommand{\dx}{\text{d}x}

\begin{document}
\Titre{PSI}{Promotion 2021--2022}{Mathématiques}{Devoir non surveillé n°6}
\begin{center}
\large\bf 
Pour le vendredi 03 décembre 2021
\end{center}
\separationTitre

   
\section*{Exercice 1}   
Considérons l'arc paramétré du plan défini par :
$$ x(t) =\cos(t)^3, \; y(t) = \sin(t)^3 $$

\begin{enumerate}
\item Calculer pour tout $t \in \R$, $x(\pi-t)$, $y(\pi-t)$, $x(\pi/2-t)$ et $y(\pi/2-t)$.
\item Justifier que l'étude sur $[0,\pi/4]$ suffit à tracer entièrement la courbe.
\item Étudier l'arc paramétré.
\end{enumerate}
   
\section*{Exercice 2}   
Soient $\varphi : \mathbb{R} \rightarrow \mathbb{R}$ définie par :
$$ \varphi(t) = \left\lbrace \begin{array}{ccl}
\dfrac{sh(t)}{t} & \hbox{ si } t \neq 0 \\[0.3cm]
1 & \hbox{ si } t = 0 \\
\end{array}\right.$$
et $f : \mathbb{R} \rightarrow \mathbb{R}$ définie par :
$$ f(x) = \int_x^{2x} \varphi(t) \dt$$
\begin{enumerate}
\item Montrer que $f$ est bien définie sur $\mathbb{R}$.
\item Étudier la parité de $f$.
\item Montrer que $f$ est de classe $\mathcal{C}^1$ sur $\mathbb{R}$ et déterminer $f'$. En déduire les variations de $f$.
\item Déterminer les limites de $f$ en $+ \infty$ et $- \infty$.
\end{enumerate}
   
\section*{Problème : Intégrales de Wallis et formule de Stirling}
\begin{enumerate}
   \item Pour tout entier naturel $n$, on pose $W_n = \int_{0}^{\frac{\pi}{2}} \cos(t)^n \dt$.
   
   \begin{enumerate}
   \item Calculer $W_0$ et $W_1$.
   \item Montrer que pour tout entier $n \geq 0$, $W_n= \int_{0}^{\frac{\pi}{2}} \sin(t)^n \dt$.
   \item Montrer que la suite $(W_n)_{n\geq 0}$ est strictement décroissante.
   \item A l'aide d'une intégration par parties, montrer que pour tout entier $n \geq 0$,
   $$ W_{n+2} = \left( \frac{n+1}{n+2} \right) W_n $$
   \item En déduire que pour tout entier $p \geq 0$,
   $$ W_{2p} = \frac{(2p)!}{2^{2p} (p!)^2} \frac{\pi}{2} \quad \hbox{ et }  \quad W_{2p+1} = \frac{2^{2p} (p!)^2}{(2p+1)!}$$
   \item Montrer que pour tout entier $n  \geq 0$,
   $$ W_n W_{n+1} =  \frac{\pi}{2(n+1)}$$
   \item Montrer que pour tout entier $n \geq 0$,
   $$1- \frac{1}{n+2}< \frac{W_{n+1}}{W_n}<1$$
   et en déduire que $W_n \underset{+ \infty}{\sim} W_{n+1}$.
   \item Montrer que $W_n \underset{+ \infty}{\sim} \sqrt{\frac{\pi}{2n}} \cdot$
   \end{enumerate}
   
   \bigskip
   
   \item Soit $(u_n)_{n\geq1}$ la suite définie pour tout $n \geq 1$ par 
   $$u_n = \dfrac{n! e^n}{n^n \sqrt{n}}$$
   et $(v_n)_{n \geq 2}$ définie pour tout $n \geq 2$ par $ v_n = \ln(u_n) - \ln(u_{n-1})$.
   
   \medskip
   
   \begin{enumerate}
   \item Exprimer simplement $v_n$ en fonction de $n \geq 2$ et donner un développement limité à l'ordre $2$ en $\dfrac{1}{n}$ de $v_n$.
   \item En déduire que $\sum_{n \geq 2} v_n$ converge. 
   
   \noindent Montrer alors que les suites $(\ln(u_n))_{n \geq 1}$ puis $(u_n)_{n \geq 1}$ convergent et donc qu'il existe un réel $K>0$ tel que :
   $$ n!  \underset{+ \infty}{\sim} K \left( \frac{n}{e} \right)^n \sqrt{n}$$
   \item En utilisant cet équivalent, donner un équivalent de $W_{2p}$ quand $p$ tend vers $+ \infty$ et en déduire la valeur de $K$ pour retrouver la formule de Stirling.
   \end{enumerate}
   \end{enumerate}


\section*{Problème facultatif type Centrale}

Il s'agit de calculer par plusieurs méthodes les intégrales $I_n$ 
définies dans la partie 2.

\subsection*{Partie 1.}

 \begin{enumerate}
\item Soit $f$ une application de classe $\mathcal{C}^1$ sur $\left[ a, b \right]$ et {\`a} valeurs dans $\R$. Montrer que :
\[\lim_{x \rightarrow +\infty} \int_a^b f(t) \sin(xt) \, \dt = 0 \]
On admettra que le résultat reste vrai en remplaçant la fonction $\sin$ par la fonction $\cos$.
\item On note pour tout $n \in \mathbb{N}$, $J_n = \int_0^{\frac{\pi}{2}} \frac{\sin(nt)}{\sin(t)} \, \dt$.
\begin{enumerate}
\item Justifier l'existence de $J_n$ pour tout $n \geq 0$.
\item Calculer $J_0, J_1, J_2$ et $J_3$.
\item Montrer que pour tout $n \geq 2$, 
$$ J_n-J_{n-2} = \left\lbrace \begin{array}{ll} 
0 & \hbox{ si } n  \hbox{ est impair } \\
\dfrac{2}{n-1}(-1)^{\frac{n}{2}-1} & \hbox{ si } n \hbox{ est pair } \\
\end{array}\right.$$
\item En déduire une expression de $J_n$ en fonction de $n$.
\item Montrer que $\lim_{n \rightarrow +\infty} \left( J_n - J_{n-1} \right) = 0$ et en déduire que : 
$$\lim_{n \rightarrow +\infty} J_n = \frac{\pi}{2}$$
\item Déduire des résultats précédents l'égalité : $\pi = 4 \sum_{n=0}^{+\infty} \frac{(-1)^n}{2n+1}\cdot$
\end{enumerate}
\item
\begin{enumerate}
\item Soit $a$ un réel strictement positif. Justifier l'existence de l'intégrale $\int_0^a \frac{\sin(nt)}{\sin(t)} \ \dt$ pour $n \in \N$.
\item Soit $a \in \left] 0, \pi \right[$. On admet que l'application $f$ telle que 
$$f(x) = \frac{1}{x} - \frac{1}{\sin(x)}$$
pour $x \neq 0$ et $f(0) = 0$ est de classe $\mathcal{C}^\infty$ sur $\left[ 0, a \right]$.

\medskip

Déterminer $\lim_{n \rightarrow +\infty} \left( \int_0^a \frac{\sin(nt)}{t} \, \dt - \int_0^a \frac{\sin(nt)}{\sin(t)} \, \dt \right)\cdot$
\item En déduire la valeur de : $\lim_{n \rightarrow +\infty} \int_0^a \frac{\sin(nt)}{t} \, \dt$ lorsque $a = \frac{\pi}{2}$, $a < \frac{\pi}{2}$ et $a > \frac{\pi}{2}\cdot$
\end{enumerate}
\item En utilisant les résultats précédents et l'intégrale $\int_0^{n \pi} \frac{\sin(t)}{t} \, \dt$, montrer que la fonction :
$$F : X \mapsto \int_0^X \frac{\sin(t)}{t} \, \dt$$
admet $\frac{\pi}{2}$ pour limite lorsque $X$ tend vers $+\infty$.

\medskip

On pose dorénavant $I_1 = \int_0^{+\infty} \frac{\sin(t)}{t} \, \dt = \frac{\pi}{2}\cdot$
\item En démontrant que pour tout $n \geq 0$, 
$$\int_{n \pi}^{(n+1) \pi} \left\vert \frac{\sin(t)}{t} \right\vert \, \dt \geq \frac{1}{(n+1) \pi} \int_{0}^{\pi} \sin(u) du,$$
montrer que l'application 
$ t \mapsto \frac{\sin(t)}{t} $ n'est pas intégrable sur $\left]0, +\infty \right[$.
\end{enumerate}

\bigskip

\subsection*{Partie 2.}

\bigskip

\begin{enumerate}
\item Montrer que, pour tout $n \geq 2$, l'application 
$$t \mapsto \left( \frac{\sin(t)}{t} \right)^n $$
est intégrable sur $\left]0, +\infty \right[$.

\medskip

On pose alors pour tout $n \geq 2$, $I_n = \int_0^{+\infty} \left( \frac{\sin(t)}{t} \right)^n  \, \dt$.
\item Montrer que : $I_1 = I_2$ ($I_1$ a été définie dans la partie 1).
\item Montrer que $\lim_{n \rightarrow +\infty} I_n = 0$.
\item Montrer que : $I_n > 0$ pour $n \geq 1$. Dans le cas où $n$ est impair, on commencera par prouver les égalités suivantes :
$$ I_n = \sum_{k=0}^{+ \infty} \int_{k \pi}^{(k+1)\pi} \left( \frac{\sin(t)}{t} \right)^n \dt =  \sum_{k=0}^{+ \infty} \int_{0}^{\pi} (-1)^{nk} \left( \frac{\sin(u)}{u+k \pi} \right)^n du = \sum_{k=0}^{+ \infty} (-1)^{k} \int_{0}^{\pi}  \left( \frac{\sin(u)}{u+k \pi} \right)^n du$$
puis on utilisera le critère spécial des séries alternées.
\end{enumerate}

\bigskip

\subsection*{Partie 3.}

\bigskip

Pour $n \in \N$ et $k \in \left\{ 0, \ldots, n-1 \right\}$, on consid{\`e}re les applications  $g_n : \left]0, +\infty \right[ \rightarrow \R$ définies par 
$$g_n(t) = (\sin(t))^n$$
et $h_{n,k} : \left]0, +\infty \right[ \rightarrow \R$ définies par 
$$h_{n,k}(t) = \frac{1}{t^{n-k}}g_n^{(k)}(t)$$
o{\`u} $g_n^{(k)}$ désigne la dérivée d'ordre $k$ de $g_n$.

\medskip

\begin{enumerate}
\item Déterminer pour $n \geq 2$, $g_n^{(0)}$, $g_n^{(1)}$ et $g_n^{(2)}$.
\item Montrer par récurrence sur $k \in \lbrace 0, \ldots, n-1 \rbrace$ l'existence d'un polynôme $P_k$ tel que pour tout $t \in \mathbb{R}_+^{*}$,
$$ g_n^{(k)}(t) =  \sin(t)^{n-k} P_k(\cos(t))$$
\item Montrer que, pour tout $k \in \left\{ 0, \ldots, n-2 \right\}$, $h_{n,k}$ est intégrable sur $\left]0, +\infty \right[$.
\item Montrer que, pour $n \geq 2$ et $k \in \left\{ 0, \ldots, n-2 \right\}$, la valeur de 
$$(n-k-1)! \int_0^{+\infty} h_{n,k}(t) \, \dt$$
 ne dépend pas de $k$. On pourra utiliser une intégration par parties pour $n>2$.
\item Pour $n \geq 2$, prouver que la fonction $G : X \mapsto \int_0^X h_{n,k}(t) \, \dt$ admet en $+\infty$ une limite finie, notée $\int_0^{+\infty} h_{n,k}(t) \, \dt$ et que, pour tout $k \in \left\{ 0, \ldots, n-1 \right\}$ :
\[ \int_0^{+\infty} h_{n,n-1}(t) \, \dt = (n-k-1)! \int_0^{+\infty} h_{n,k}(t) \, \dt \]
\item En déduire que pour $n \geq 2$, on a :
$$I_n= \frac{1}{(n-1)!} \int_0^{+\infty} \frac{1}{t} g_n^{(n-1)}(t) \, \dt$$
\item 
\begin{enumerate}
\item établir pour tout $p \in \N^*$ et tout $t \in \R$ les résultats suivants :
\[ 4^p (\sin t)^{2p} = \binom{2p}{p} + 2 \sum_{k=1}^p (-1)^k \binom{2p}{p-k} \cos(2kt) \qquad (1) \]
 \[ 4^p (\sin t)^{2p+1} = \sum_{k=0}^p (-1)^k \binom{2p+1}{p-k} \sin((2k+1)t) \qquad (2) \]
\item En déduire, en distinguant les cas $n = 2p$ et $n = 2p+1$, une expression de $I_n$ du type $q_n \pi$ o{\`u} $q_n$ est une somme de nombres rationnels (on pourra faire intervenir $I_1$ dans les calculs). On pourra commencer par déterminer toutes les dérivées des fonctions $x \mapsto \cos(\alpha x)$ et $x \mapsto \sin(\alpha x)$ où $\alpha \in \mathbb{R}$.
\item Retrouver la valeur de $I_2$, puis calculer $I_3$ et $I_4$.
\end{enumerate}
\end{enumerate}

\end{document}