\documentclass[twoside,a4paper,french,10pt]{VcCours}
\newcommand{\dt}{\text{d}t}
\newcommand{\dx}{\text{d}x}
\begin{document}

\Titre{PSI}{Promotion 2021--2022}{Mathématiques}{\vspace{-1.5em}}

\begin{center}
\large\bf
Programme de colles\hspace{\stretch{1}}Semaine n\degres22 du 28/03 au 01/04/2022
\end{center}
\separationTitre

\vspace*{-1em}
Pour cette khôlle, chaque étudiant aura :
\begin{itemize}
\item Une ou deux \textbf{question de cours}.
\item Une démonstration \textbf{(D)} ou un exercice à retravailler.
\end{itemize} 
  

\section*{Chapitre 16 : Variables aléatoires discrètes}
  \begin{enumerate}
  % \item Généralités : Variable aléatoire discrète (VAD). Support. Notations $(X=a)$, $(X \in U)$...
  % \item Loi d'une VAD.
  % \begin{itemize}
  % \item Loi. On commence par déterminer le support puis $P(X=k)$ pour $k$ dans le support.
  % \item Système complet d'évènements associé à une VAD.
  % \item \textbf{Fonction de répartition}. Propriétés. \textbf{(D)} : Croissance et limite en $+ \infty$.
  % \item Variable $Y=g(X)$.
  % \end{itemize}
  % \item Lois usuelles.
  % \begin{itemize}
  % \item \textbf{Connaître parfaitement les lois usuelles : uniforme, Bernoulli, Binomiale, Géométrique, Poisson.} (Bien préciser le support à chaque fois).
  % \item Savoir justifier qu'une variable suit une loi binomiale.
  % \item Savoir justifier qu'une variable suit une loi géométrique.
  % \item Caractérisation des lois géométrique, loi sans mémoire.
  % \item \textbf{Approximation de la loi binomiale par la loi de Poisson% + (D)
  % }.
  % \end{itemize}
  % \item Famille de variables aléatoires.
  % \begin{itemize}
  % \item Couple de variables aléatoires discrètes. Loi conjointe, loi marginale.
  % \item Savoir déterminer les lois marginales à l'aide de la loi conjointe. L'autre sens n'est pas possible.
  % \item Loi conditionnelle.
  % \item \textbf{Indépendance de deux variables aléatoires discrètes}. Généralisation. Quelques propriétés.
  % \item Si $X$ et $Y$ sont indépendantes, $f(X)$ et $g(Y)$ aussi.
  % \end{itemize}
  \setcounter{enumi}{4}
  \item Moments d'une variable aléatoire.
  \begin{itemize}
  \item \textbf{Définition} de l'espérance (cas fini et cas infini).
  \item Linéarité, positivité, croissance.
  \item Proposition 19 : soit $X$ une V.A.D à valeurs dans $\mathbb{N}$. Alors $X$ admet une espérance ssi la série :
  
  $\sum_{n \geq 1} \P(X \geq n)$
  converge et dans ce cas, l'espérance de $X$ est égale à la somme de cette série.
  \item \textbf{Espérances associées lois usuelles} + \textbf{(D) pour loi géométrique}.
  \item Espérance d'un produit de deux variables aléatoires indépendantes.
  \item \textbf{Théorème de transfert : cas fini et infini}. 
  \item Si $X^2$ admet une espérance alors $X$ admet une espérance : \textbf{(D)}.
  \item \textbf{Définition de la variance + propriétés + formule de Kœnig-Huygens}.
  \item \textbf{Variances associées aux lois usuelles}.
  \item Inégalité de Cauchy-Schwarz.
  \item \textbf{Définition : covariance et coefficient de corrélation}.
  \item Si $X$ et $Y$ sont deux variables aléatoires indépendantes admettant une variance alors $\textrm{Cov}(X,Y)=0$. La réciproque est fausse.
  $\rho(X,Y) \in [-1,1]$.
  \item \textbf{Variance d'une somme de variables aléatoires discrètes}.
  \end{itemize}
  \item
  \begin{itemize}
  \item \textbf{Fonction génératrice}.
  \item La loi d'une variable aléatoire est déterminée par sa fonction génératrice.
  \item Fonctions génératrices associées aux lois usuelles. \textbf{(D) pour la loi de Poisson}.
  \item Fonction génératrice : lien avec espérance et variance.
  \item Fonction génératrice d'une somme de variables aléatoires indépendantes.
  \item Somme de deux variables aléatoires indépendantes suivant une loi de Poisson : \textbf{(D)}
  \end{itemize}
  \item
  \begin{itemize}
  \item \textbf{Inégalité de Markov et inégalité de Bienaymé-Tchebychev}.
  \item \textbf{Loi faible des grands nombres.}
  \end{itemize}
  \end{enumerate}

\section*{Chapitre 17 : Équations différentielles}
\begin{enumerate}
\item Équations différentielles linéaires scalaires d'ordre $1$.
\begin{itemize}
\item Définition. Équation homogène, coefficient et second membre.
\item Problème de Cauchy et théorème de Cauchy-Lipschitz linéaire.
\item Résolution de l'équation homogène.
\item Résolution générale.
\item Obtention d'une solution particulière : si $a$ est constante, variation de la constante, principe de superposition.
\item Problème de raccord.
\end{itemize}
\item Système linéaire d'équations différentielles.
\begin{itemize}
\item Système différentiel linéaire d'ordre $1$. Système homogène, coefficients.
\item Problème de Cauchy et théorème de Cauchy-Lipschitz linéaire.
\item Ensemble des solutions d'un système homogène : \textbf{connaître la dimension de $\mathcal{S}_H$ quand $A$ est continue}.
\item Résolution générale avec une solution particulière.
\end{itemize}
\item Systèmes différentiels à coefficients constants.
\begin{itemize}
\item Cas où $A$ est diagonalisable.
\item Cas ou $A$ est trigonalisable.
\end{itemize}
\item Équations différentielles linéaires scalaires d'ordre deux.
\begin{itemize}
\item Système différentiel associé.
\item Problème de Cauchy, unicité d'une solution si $a$, $b$ et $c$ sont continues sur $I$. $\mathcal{S}_H$ est de dimension $2$.
\item Équations à coefficients constants. \textbf{Connaître parfaitement les différents cas suivant les racines de l'équation caractéristiques}.
\item Solution particulière dans le cas d'un second membre de la forme $t \mapsto C e^{\alpha t}$, $t \mapsto \cos(t)$ ou $t \mapsto \sin(t)$ (par complexification).
\item Que faire quand le second membre n'est pas simple ? Se laisser guider par l'énoncé ...
\end{itemize}
\item Utilisation des séries entières. 
\end{enumerate}
    
  
\begin{Exercice}
  Soient $X$ et $Y$ deux variables aléatoires à valeurs dans $\N$. On suppose que la loi conjointe de $X$ et $Y$ vérifie :
  \[
  \forall (j,k) \in \mathbb{N}^2,  \P( X = j,Y = k ) = \frac{a}{j!k!}\quad\text{où $a \in \mathbb{R}$.}
  \]
  \begin{enumerate}
  \item
    Déterminer la valeur de $a$.
  \item
    Retrouvons les lois marginales de $X$ et $Y$.
  \item
    Les variables $X$ et $Y$ sont elles indépendantes?
  \end{enumerate}
\end{Exercice}


\begin{Exercice}{} 
  Soient $\lambda >0$ et $X$ une variable aléatoire suivant la loi de Poisson de paramètre $\lambda$. Montrer que la variable aléatoire $Y$ définie par $Y = \frac{1}{X+1}$
  admet une espérance et donner la.
\end{Exercice}
  
\begin{Exercice}
  Résoudre sur $\R$ l'équation différentielle :
  \[(E):\ y''+2y'+y=\cos(t)\]
\end{Exercice}

\begin{Exercice}
  Considérons le problème de Cauchy suivant :
$ \sys{y''+xy'+y=1\\y(0)=y'(0)=0}$.

Ce problème de Cauchy a une unique solution d'après le théorème de Cauchy-Lipschitz linéaire. Déterminons cette unique solution.
\end{Exercice}

\end{document}  