\documentclass[twoside,a4paper,french,10pt]{VcCours}

\begin{document}

\Titre{PSI}{Promotion 2021--2022}{Mathématiques}{\vspace{-1.5em}}

\begin{center}
\large\bf
Programme de colles\hspace{\stretch{1}}Semaine n\degres10 du 29/11 au 03/12/2021
\end{center}
\separationTitre


Pour cette khôlle, chaque étudiant aura :
\begin{itemize}
\item Une \textbf{question de cours}.
\item Une démonstration \textbf{(D)} ou un exercice à retravailler.
\end{itemize}


\section*{Chapitre 7 : Fonctions vectorielles et arcs paramétrés}
\emph{On préférera tester le cours au travers d'un exercice d'étude (complète ou partielle) d'arc paramétrique.}

\begin{itemize}
    \item Fonction à valeurs dans $\R^n$: fonctions coordonnées, continuité, dérivabilité, classe $C^k$.
    \item Norme (euclidienne) sur $\R^n$.
    \item Arcs paramétrés: définition, support
    \item Étude locale d’un arc paramétré du plan : point régulier, point 
    singulier/stationnaire, tangente. Point ordinaire, d'inflexion, de 
    rebroussement de première et deuxième espèces. Branches infinies.
    \item Longueur d’un arc.
\end{itemize} 

\section*{Chapitre 8 : Rappels sur l'intégration}
  
  \textbf{En exercice :} révisions de Sup : primitives, théorème fondamental de l'analyse, sommes de Riemann : propriété de convergence, propriétés classiques (linéarité, relation de Chasles, positivité), \textbf{intégration par parties}, changement de variable, \textbf{primitives usuelles}, formule de Taylor reste-intégrale, formule de Taylor-Young. Quelques méthodes de calcul (en particulier : linéarisation, passage par une fonction complexe).

  Refaire les exercices du cours (correction sur moodle).
  
  \section*{Chapitre 9 : Intégrales généralisées}

  \begin{enumerate}
  \item Intégrale sur un segment d'une fonction continue par morceaux. 
  \begin{itemize}
  \item Fonction continue par morceaux.
  \item Intégrale d'une fonction continue par morceaux.
  \item Propriétés (attention : stricte positivité fausse en général dans ce cadre).
  \item Méthodes de calcul. Changement de variable. Fractions rationnelles (décomposition en éléments simples et application au calcul de primitive).
  \end{itemize}
  \item Intégrales généralisées sur $[a, + \infty[$.
  \item Intégrales généralisées sur un intervalle quelconque. \textbf{Savoir définir la convergence dans les différents cas ($[a,b[$, $]a,b]$, $]a,b[$).}
  \item Intégrales de références.
  \begin{itemize}
  \item Intégrales de Riemann : \textbf{(D)}
  \item Intégrale de $\ln$ sur $]0,1]$.
  \item Critère de convergence pour $\int_{0}^{+ \infty} e^{-\lambda t} dt$ où $\lambda \in \mathbb{R}$ : \textbf{(D)}
  \item Lien avec l'intégrale usuelle. \textbf{Intégrale faussement impropre} (\textbf{important dans la pratique}).
  \item Propriétés de base : linéarité, positivité, croissance, 
  fonction à valeurs complexes.
  \end{itemize}
  \end{enumerate}
  
  \begin{Exercice}
    Étude complète et tracé de l'arc paramétré du plan de composantes:
    \[x:t\longmapsto\frac{1}{1-t^2} \et y:t\longmapsto\frac{t^3}{1-t^2}\]  
  \end{Exercice} 
  
\begin{Exercice}
Savoir déterminer des primitives de fonctions de la forme :
  $$ x \mapsto \frac{1}{x^2+px+q} \quad \hbox{ ou } x \mapsto \frac{ax+b}{x^2+px+q}$$
  où $(a,b,p,q) \in \mathbb{R}^4$ et $p^2-4q<0$.
\end{Exercice} 

\begin{Exercice}
Calculer $I= \int_{0}^1 \sqrt{1-t^2} \text{d}t$.
Les deux derniers exercices proviennent du chapitre de rappel (correction sur moodle) : 
\end{Exercice}   

\begin{Exercice}
Justifier que la fonction $H$ définie par $H(x) = \int_{-x}^{x^2} \ln(1+t^4) \text{d}t$ est de classe $\mathcal{C}^1$ sur $\mathbb{R}$ et donner l'expression de sa dérivée.
\end{Exercice} 

\begin{Exercice}
Calculer $I = \int_{0}^{1} \cos(t) e^t \text{d}t$ en utilisant fonction complexe.
\end{Exercice} 

\end{document}