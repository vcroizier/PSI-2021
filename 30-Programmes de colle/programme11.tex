\documentclass[twoside,a4paper,french,1pt]{VcCours}

\begin{document}

\Titre{PSI}{Promotion 2021--2022}{Mathématiques}{\vspace{-1.5em}}

\begin{center}
\large\bf
Programme de colles\hspace{\stretch{1}}Semaine n\degres10 du 06/12 au 10/12/2021
\end{center}
\separationTitre


Pour cette khôlle, chaque étudiant aura :
\begin{itemize}
\item Une ou deux \textbf{question de cours}.
\item Une démonstration \textbf{(D)} ou un exercice à retravailler.
\end{itemize}


\section*{Chapitre 8 : Rappels sur l'intégration}
  
  \textbf{En exercice :} révisions de Sup : primitives, théorème fondamental de l'analyse, sommes de Riemann : propriété de convergence, propriétés classiques (linéarité, relation de Chasles, positivité), \textbf{intégration par parties}, changement de variable, \textbf{primitives usuelles}, formule de Taylor reste-intégrale, formule de Taylor-Young. Quelques méthodes de calcul (en particulier : linéarisation, passage par une fonction complexe).

  Refaire les exercices du cours (correction sur moodle).
  
  \section*{Chapitre 9 : Intégrales généralisées}
  Tout le début du chapitre au programme en exercice (Intégrales généralisées sur un intervalle, connaître parfaitement les intégrales de références, fausse impropreté).
  \begin{itemize}
    \item Méthodes de calcul d'intégrales généralisées.
    \begin{itemize}
    \item Avec une primitive.
    \item Avec une intégration par parties (je déconseille le théorème du programme : passer par une intégrale sur un segment).
    \item \textbf{Changement de variable.}
    \end{itemize}
    \item \textbf{Intégrale absolument convergente.} La convergence absolue implique la convergence. Réciproque fausse.
    \item \textbf{Fonction intégrable.} Quelques propriétés.
    \item Critères de comparaison pour des fonctions intégrables.
    \item Critères de comparaison pour des intégrales.
    \item Exemples (il faut s'entraîner beaucoup!).
    \item \textbf{Théorème de convergence dominée}. 
    \item \textbf{Théorème d'intégration terme à terme.}
    \item \textbf{Théorème de comparaison série-intégrale}.
  \end{itemize}  

  \begin{Exercice}
    Justifier que la fonction $H$ définie par $H(x) = \int_{-x}^{x^2} \ln(1+t^4) \text{d}t$ est de classe $\mathcal{C}^1$ sur $\mathbb{R}$ et donner l'expression de sa dérivée.
    \end{Exercice} 

    \begin{Exercice} Nature des intégrales suivantes : 
    $$  \int_{0}^{+ \infty} \frac{2x}{x^2+x+1} \dx, \; \int_{1}^{+ \infty} \frac{\ln(t)}{\sqrt{t}}, \; \int_{0}^{+ \infty} \frac{\ln(t)}{t^3+t^2+1} \dt$$
  \end{Exercice} 
    
    \begin{Exercice} Étudier l'intégrabilité sur $[1, + \infty[$ de $f_1$ définie par :
    $$ f_1(x)=\frac{\sqrt{x^2+x+1}-\sqrt{x^2-x+1}}{x}$$
  \end{Exercice} 
    
    \begin{Exercice} Justifier l'existence puis donner la valeur de :
      \[
      I = \int_{0}^{ + \infty} \frac{\dt}{(1 + t^{2})^{2}}
      \]
    On pourra utiliser le changement de variable $u = \dfrac{1}{t} \cdot$
  \end{Exercice} 
    
    \begin{Exercice} Déterminer $ \lim_{n \rightarrow + \infty} \int_{0}^{\frac{\pi}{4}} \tan(t)^n \dt$.
    \end{Exercice} 
    
  %     \begin{Exercice} Pour tout entier naturel $n \geq 1$, on définit la fonction $f_n : \mathbb{R}_+ \rightarrow \mathbb{R}$ par :
  %   $$ \forall x \in \mathbb{R}_+, \; f_n(x)=xe^{-nx}$$
  %   Utiliser le théorème d'intégration terme à terme pour obtenir une jolie égalité.
  % \end{Exercice} 
    
  



\end{document}