\documentclass[twoside,a4paper,french,10pt]{VcCours}
\newcommand{\dt}{\text{d}t}
\newcommand{\dx}{\text{d}x}
\begin{document}

\Titre{PSI}{Promotion 2021--2022}{Mathématiques}{\vspace{-1.5em}}

\begin{center}
\large\bf
Programme de colles\hspace{\stretch{1}}Semaine n\degres14 du 10/01 au 14/01/2022
\end{center}
\separationTitre


Pour cette khôlle, chaque étudiant aura :
\begin{itemize}
\item Une ou deux \textbf{question de cours}.
\item Une démonstration \textbf{(D)} ou un exercice à retravailler.
\end{itemize} 

\section*{Chapitre 10 : Réduction}
Tout le début du chapitre au programme en exercice.
\begin{enumerate}
% \item Compléments d'algèbre linéaire.
% \begin{itemize}
% \item Sous-espace vectoriel stable, endomorphisme induit.
% \item Si $u$ et $v$ sont deux endomorphismes d'un $\mathbb{K}$-espace vectoriel $E$ qui commutent, le noyau et l'image de $u$ sont stables par $v$ : \textbf{(D)}
% \end{itemize}
% \item Élément propre d'un endomorphisme.
% \begin{itemize}
% \item \textbf{Définitions} : valeurs propres et vecteurs propres d'un endomorphisme.
% \item Attention : un vecteur propre est non nul ! 
% \item Spectre d'un endomorphisme.
% \item Proposition (3) ($\lambda$ est valeur propre si et seulement si le noyau $u- \lambda Id_E$ est différent de $\lbrace 0_E \rbrace$). \textbf{définition d'un sous-espace propre}.
% \item Un sous-espace propre est un sous-espace vectoriel.
% \item Cas particulier de la valeur propre nulle.
% \item Tout sous-espace propre de $u$ est stable par $u$. 
% \item Si $u$ et $v$ sont deux endomorphismes d'un $\mathbb{K}$-espace vectoriel $E$ qui commutent, tout sous-espace propre $u$ est stable par $v$.
% \item Proposition (8) : la somme des sous-espaces propres est directe.%: \textbf{(D)}.
% \item Corollaire : une famille de vecteurs propres associés à des valeurs propres deux à deux distinctes d'un endomorphisme d'un $\mathbb{K}$-espace vectoriel est une famille libre.
% \end{itemize}
% \item Élément propre en dimension finie.
% \begin{itemize}
% \item \textbf{Définitions} : valeur propre, vecteur propre, sous-espace propre d'une matrice (retenir le \og autrement dit \fg après la définition).
% \item Lien entre éléments propres d'un endomorphisme et d'une matrice.
% \item \textbf{Définition} : Polynôme caractéristique d'une matrice. C'est un polynôme unitaire, de degré $n$, son coefficient constant est $(-1)^n \textrm{det}(A)$.
% \item Deux matrices semblables ont le même polynôme caractéristique : \textbf{(D)}. Polynôme caractéristique d'un endomorphisme.
% \item Les racines du polynôme caractéristique de $u$ ou de $A$ sont les valeurs propres. 
% \item Une matrice carrée admet au plus $n$ valeurs propres. Une matrice complexe admet au moins une valeur propre. Résultats semblables pour un endomorphisme. 
% \end{itemize}
% \item Diagonalisation.
% \begin{itemize}
% \item \textbf{Définition : endomorphisme diagonalisable, matrice diagonalisable.}
% \item \textbf{Théorème 21} : 4 assertions équivalentes ($u$ diagonalisable / base de vecteurs propres / $E$ égal à la somme directe des sous-espaces propres / somme des dimensions des sous-espaces propres égale à la dimension de $E$).
% \item \textbf{Corollaire 22} (un endomorphisme est diagonalisable si et seulement si son polynôme caractéristique est scindé sur $\mathbb{K}$ et si la multiplicité de chaque valeur propre est égale à la dimension du sous-espace propre associé).
% \item Corollaire : si un endomorphisme d'un espace vectoriel de dimension $n$ admet $n$ valeurs propres distinctes alors il est diagonalisable.
% \end{itemize}
\setcounter{enumi}{4}
\item Polynômes d'endomorphismes et de matrices.
\begin{itemize}
\item Définition et quelques propriétés.
\item Proposition 26 (hors-programme) : Soient $u \in \mathcal{L}(E)$ et $P \in \mathbb{K}[X]$ un polynôme annulateur de $u$. Alors toute valeur propre de $u$ est racine de $P$. \textbf{(D)}
\item \textbf{Théorème de Cayley-Hamilton}.
\item \textbf{Théorème 28 : 3 assertions équivalentes}
($u$ diagonalisable/existence d'un polynôme annulateur scindé à racines simples/
le polynôme $\prod(X- \lambda)$ est annulateur de $u$).
\item Exemple pour une projection, une symétrie.
\item %cSpell:ignore Diagonalisabilité
\item Diagonalisabilité d'un endomorphisme induit.
\end{itemize}
\item Trigonalisation.
\begin{itemize}
\item \textbf{Définition : endomorphisme trigonalisable, matrice trigonalisable.}
\item \textbf{Théorème 31} (un endomorphisme est trigonalisable si et seulement si son polynôme caractéristique est scindé). Corollaire : toute matrice est trigonalisable sur $\mathbb{C}$.
\item Si le polynôme caractéristique est scindé, expression du déterminant et de la trace avec les valeurs propres. 
\end{itemize}
\end{enumerate}

\section*{Chapitre 11 : Espaces probabilisés}
  \begin{enumerate}
  \item Rappels sur les ensembles finis.
    \begin{itemize}
    \item Définition et propriétés.
    \item Cardinal d'un produit cartésien de deux ensembles finis, 
    cardinal de $\mathcal{P}(E)$ quand $E$ est fini \textbf{(D)} et cardinal de $\mathcal{F}(E,F)$ quand $E$ et $F$ sont finis.
    \item Dénombrement : listes avec ou sans répétitions, combinaisons. 
    \item \textbf{Ensembles dénombrables}, au plus dénombrables.
    \item $\mathbb{Z}$ est dénombrable.% \textbf{(D)}.
    \item Un produit cartésien d'ensembles dénombrables est dénombrable.
    \end{itemize}
  \item Probabilité
  \begin{itemize}
  \item Intersection et union dénombrable.
  \item Tribu.
  \item Stabilité par intersection au plus dénombrable dans une tribu.
  \item Système complet d'évènements.
  \item \textbf{Définition} d'une probabilité.
  \item Cas particulier si l'univers est dénombrable.
  \item Évènements presque-sur, presque-impossible.
  \item Propriétés usuelles (croissance, union de deux ou $n$ évènements deux à deux incompatibles, formule du crible).
  \item \textbf{Théorème de continuité croissante et décroissante}.
  \item Application (\textbf{à savoir refaire}) pour calculer la probabilité d'une union ou intersection dénombrable \textit{quelconque} d'évènements.
  \item Sous-additivité.
  \end{itemize}
  \item Cas de l'équiprobabilité. Refaire des exercices du type de l'AD5 (calcul de probabilité en déterminant le cardinal de l'univers, cardinal de l'évènement en comprenant si les tirages sont simultanés, successifs, avec ou sans remise...).
  \item Probabilités conditionnelles.
  \begin{itemize}
  \item Probabilité de $A$ sachant $B$.
  \item $P_B$ est une probabilité. Conséquence : corollaire 22.
  \item \textbf{Formule des probabilités composées}.
  \item \textbf{Formule des probabilités totales : cas fini et infini}. \textbf{Interdiction d'oublier de préciser le système complet d'évènements et interdiction de calculer directement la probabilité d'un évènement sans passer par la formule théorique}.
  \item Formule de Bayes.
  \end{itemize}
  \item Indépendance de deux évènements. 
  \item Évènements mutuellement indépendants (pour une famille finie ou dénombrable).
  \end{enumerate}
  
\begin{Exercice} 
  En France la probabilité de gagner au loto est $p=1/19 068 840$. 
  Un joueur immortel décide de jouer tous les jours au loto. 
  Pour tout $n \geq 1$, on note $A_n$ l'évènement
  \og Le joueur a perdu tous les jours jusqu'au jour $n$ \fg
  Que représente l'évènement $\bigcap_{k=0}^{+\infty}A_k$? Calculer sa probabilité.
\end{Exercice} 

\begin{Exercice} 
  Une urne contient 3 boules blanches et 7 boules noires. On tire successivement et sans remise 3 boules dans cette urne. Quelle est la probabilité d'obtenir la première boule blanche au 3$^{\text{ème}}$ tirage ?
\end{Exercice}   

\begin{Exercice} 
  On considère une infinité d'urnes. On en choisit une au hasard de telle sorte que la probabilité de choisir l'urne $n$ soit égale à $\frac{1}{2^n}$ (avec $n \in \mathbb{N}^*$), on note $A_n$ l'évènement \og on choisit l'urne $n$ \fg .
  \begin{enumerate}
    \item Vérifier que $P \bigg{(} \bigcup_{n \in \N^*} A_n \bigg{)} = 1$.
    \item Pour tout entier $n \geq 1$, l'urne $n$ contient $2^n$ boules dont une seule est blanche. On choisit une urne au hasard puis on tire une boule et on note $B$ l'évènement \og on tire une boule blanche \fg . Déterminer $P(B)$. 
  \end{enumerate}
\end{Exercice} 

\begin{Exercice}
  Calculer les sommes suivantes ($n \in \mathbb{N}$) : 
  $$\sum_{k=0}^n k \binom{n}{k}, \; \sum_{k=0}^n\frac{1}{k+1} \binom{n}{k}, \; \sum_{k=0}^n \binom{2n+1}{k}$$
\end{Exercice}   
    
\end{document}