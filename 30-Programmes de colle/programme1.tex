\documentclass[twoside,a4paper,french,10pt]{VcCours}
\newcommand{\Sum}[2]{\ensuremath{\textstyle{\sum\limits_{#1}^{#2}}}}
\newcommand{\Int}[2]{\ensuremath{\mathchoice%
	{{playstyle\int_{#1}^{#2}}}
	{{playstyle\int_{#1}^{#2}}}
	{\int_{#1}^{#2}}
	{\int_{#1}^{#2}}
	}}

\begin{document}

\Titre{PSI}{Promotion 2021--2022}{Mathématiques}{\vspace{-1.5em}}

\begin{center}
\large\bf
Programme de colles\hspace{\stretch{1}}Semaine n\degres1 : du 13/09 au 17/09/2021
\end{center}
\separationTitre


Pour cette khôlle, chaque étudiant aura :
\begin{itemize}
\item Un ou deux développements limités à donner.
\item Une question de cours.
\item Une démonstration \textbf{(D)} ou un exercice à retravailler.
\end{itemize}


\medskip 

\section*{Chapitre 1 : Rappels d'analyse.}


\begin{enumerate}
\item Suites.
\begin{itemize}
\item \textbf{Convergence d'une suite}. Unicité de la limite \textbf{(D)}. Divergence. Divergence d'une suite réelle vers $\pm \infty$. Toute suite convergente est bornée. Limites et inégalités larges. 
\item Théorèmes de convergence pour les suites réelles (comparaison, encadrement, suites monotones, suites adjacentes).
\item Convergence d'une suite complexe.
\item Suites extraites.
\item Suites récurrentes. Quelques propriétés. \textbf{Inégalité des accroissements finis et corollaire}. Plan d'attaque d'une suite récurrente.
\end{itemize}
\item Suites usuelles.
\begin{itemize}
\item Suites arithmético-géométriques.
\item Suites récurrentes linéaires d'ordre deux.
\end{itemize} 
\item Relations de comparaisons.
\begin{itemize}
\item \textbf{Domination, négligeabilité et équivalences} (pour les suites).
\item Quelques propriétés : transitivité, linéarité, multiplication et division d'équivalents.
\item Pour des suites réelles : si $u_n  \underset{n \rightarrow + \infty}{\sim} v_n $ alors à partir d'un certain rang, les termes $u_n$ et $v_n$ sont de même signe \textbf{(D)}.
\end{itemize}
\item\textbf{Développements limités usuels en 0}.
\item Équivalents et limites usuelles.
\begin{itemize}
\item \textbf{Équivalents usuels.}
\item Théorème des croissances comparées.
\end{itemize}
\item Quelques inégalités usuelles.
\end{enumerate}


\begin{Exercice}{}
  Étudier la suite $(u_n)_{n \geq 0}$ définie par $u_0 \in \mathbb{R}$ et pour tout $n \in \mathbb{N}$ par $u_{n+1}=e^{u_n}-1$.
\end{Exercice}

\begin{Exercice}{}
  Déterminer le terme général d'une suite arithmético-géométrique ou d'une suite récurrente linéaire d'ordre deux.
\end{Exercice}

\begin{Exercice}{}
  On note $E$ la fonction partie entière. Soit $x \in \mathbb{R}$. On pose pour tout $n \geq 1$, 
  $$ u_n = \frac{1}{n^2} \sum_{k=1}^n E(kx)$$
  Étudier la convergence de la suite $(u_n)_{n \geq 1}$.
\end{Exercice}

\begin{Exercice}{}
  Soient $n$ un entier naturel et $E_n $ l'équation $x + \tan x = n$ d'inconnue $x \in \left] { - \pi  / 2,\pi  / 2} \right[$.
  \begin{enumerate}
  \item Montrer que l'équation $E_n$ possède une solution unique notée $x_n $.
  \item Montrer que la suite $(x_n)$ converge et déterminer sa limite.
  \end{enumerate}
\end{Exercice}


\end{document}