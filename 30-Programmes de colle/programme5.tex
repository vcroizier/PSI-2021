\documentclass[twoside,a4paper,french,10pt]{VcCours}

\begin{document}

\Titre{PSI}{Promotion 2021--2022}{Mathématiques}{\vspace{-1.5em}}

\begin{center}
\large\bf
Programme de colles\hspace{\stretch{1}}Semaine n\degres5: du 11/10 au 15/10/2021
\end{center}
\separationTitre


Pour cette khôlle, chaque étudiant aura :
\begin{itemize}
\item Une \textbf{question de cours}.
\item Une démonstration \textbf{(D)} ou un exercice à retravailler.
\end{itemize}



\section*{Chapitre 3 : Espaces vectoriels}
\begin{enumerate}
    \item Espaces et sous-espaces vectoriels.
    \begin{itemize}
    \item Généralités : $\mathbb{K}$-espace vectoriel. Combinaison linéaire. Méthodes des trois points. Intersection de sous-espaces vectoriels. Espace vectoriel engendré. \textbf{(D)} : c'est un sous-espace vectoriel de l'espace dans lequel il est inclus.
    \item Produit de sous-espaces vectoriels.
    \item Somme de sous-espaces vectoriels. \textbf{Somme directe de sous-espaces}, proposition faisant le lien avec l'unicité dans la décomposition.
    \item \textbf{(D)} : Soient $E$ un $\mathbb{K}$-espace vectoriel et $E_1$, $E_2$ deux sous-espaces vectoriels de $E$. Alors :
    $$ E_1 + E_2 \hbox{ est directe si et seulement si } E_1 \cap E_2 = \lbrace 0_E \rbrace $$
    \item \textbf{Espaces supplémentaires}. Cas particulier pour deux sous-espaces.
    \end{itemize}
    \item  Familles libres, génératrices, bases.
    \begin{itemize}
    \item Famille libre, famille liée, famille génératrice, base. Famille libre dans le cas d'une famille constituée d'un vecteur, de deux vecteurs (vecteurs non colinéaires).
    \item \textbf{(D) :} Une famille échelonné en degré de polynômes de $\mathbb{R}[X]$ (et ne contenant pas le polynôme nul) est une famille libre de $\mathbb{R}[X]$.
    \item Coordonnées (ou composantes) d'un vecteur dans une base.
    \end{itemize}
    \item Espaces vectoriels de dimension finie.
    \begin{itemize}
    \item Espace vectoriel de dimension finie. Théorème de la base incomplète. Dimension d'un espace vectoriel de dimension finie. Bases canoniques des espaces vectoriels classiques.
    \item Une famille de $n$ vecteurs dans un espace vectoriel $E$ de dimension $n \geq 1$ est une base si et seulement si elle est libre si et seulement si elle est génératrice de $E$.
    \item Dimension d'un sous espace, dimension d'un produit d'espaces vectoriels.
    \item Fractionnement d'une base : \textit{utile pour montrer rapidement que des sous-espaces sont supplémentaires}. Base adaptée associée. Existence d'un supplémentaire en dimension finie. Un supplémentaire n'est pas unique. 
    \item Dimension d'une somme directe. Formule de Grassmann. Caractérisation de $E= F \oplus G$ à l'aide de la dimension.
    \item \textbf{Rang d'une famille de vecteurs}.
    \end{itemize}
\end{enumerate}

\section*{Chapitre 4 : Applications linéaires}

\begin{enumerate}
\item Généralités.
\begin{itemize}
\item Définition d'une application linéaire (trois équivalentes). Endomorphisme, isomorphisme, automorphisme, forme linéaire. Quelques propriétés.
\item Image directe d'un sous-espace et image réciproque d'un sous-espace.
\item \textbf{Image et noyau d'une application linéaire.} 
\item \textbf{(D)} : Soient $E$, $F$ deux $\mathbb{K}$-espaces vectoriels et $f \in \mathcal{L}(E,F)$. L'application $f$ est injective si et seulement si $\textrm{Ker}(f) = \lbrace 0_E \rbrace$.
\item Une application linéaire est déterminée par l'image des vecteurs d'une base de $E$. 
\item $\textrm{Im}(f) = \textrm{Vect}(f(e_1), \ldots, f(e_n))$ : \textit{très utile dans la pratique}. Lien entre injectivité, surjectivité et bijectivité et image d'une base.
\item Rang d'une application linéaire. \textbf{Théorème du rang}. Soient $E, F$ deux $\mathbb{K}$-espaces vectoriels de même dimension $n$ et $f \in \mathcal{L}(E,F)$. Alors $f$ est un isomorphisme ssi $f$ est injective ssi $f$ est surjective ssi le rang de $f$ est égal à $n$.
\item \textbf{Définition d'une projection vectorielle}. 
\item \textbf{(D)} $p$ est un endomorphisme de $E$ vérifiant $p \circ p = p$, $\textrm{Im}(p) = \textrm{Ker}(p-\textrm{Id})=F$ et $\textrm{Ker}(p)=G$.
\item Définition d'un projecteur. Un endomorphisme $p$ de $E$ est une projection si et seulement si $p$ est un projecteur. Dans ce cas, $p$ est une projection sur $\textrm{Im}(p)$ parallèlement à $\textrm{Ker}(p)$.
\item \textbf{Définition d'une symétrie vectorielle}. Quelques propriétés : $s$ est un endomorphisme de $E$ vérifiant $s \circ s = \textrm{Id}$, $F = \textrm{Ker}(s-\textrm{Id})$ et $G= \textrm{Ker}(s+\textrm{Id})$.
\item Définition d'un endomorphisme involutif. Un endomorphisme $s$ de $E$ est une symétrie si et seulement si $s$ est un endomorphisme involutif.
\item \textbf{Il est important de savoir déterminer l'expression d'une projection ou d'une symétrie en raisonnant par analyse-synthèse.}
\end{itemize}
\item Hyperplan.
\begin{itemize}
\item Trois définitions équivalentes d'un hyperplan (existence d'une droite vectorielle comme supplémentaire, noyau d'une forme linéaire non nulle, dimension égal à $\textrm{dim}(E)-1$).
\item Formes linéaires avec le même noyau.
\item Équation d'un hyperplan dans une base.
\end{itemize}
\end{enumerate}



\medskip

\begin{Exercice}
    Montrer que $G = \lbrace P \in \mathbb{R}_3[X] \, \vert \, P(1)=P'(1)=0\rbrace$ est un sous-espace vectoriel et donner en une base.
\end{Exercice} 

\begin{Exercice}
    Soient $F= \lbrace (x+y,y-2x, x) \, \vert \,  (x,y) \in \mathbb{R}^2 \rbrace$ et $G = \textrm{Vect}((1,2,3))$. Montrer que $F$ et $G$ sont supplémentaires dans $\mathbb{R}^3$.
\end{Exercice} 

\begin{Exercice}
    Soient $F = \ensemble{f \in \CC([-1,1],\R)}{\int_{-1}^{1} f(t)\text{d}t = 0}$
    et $G = \ensemble{f\in\CC([-1,1],\R)}{f\text{ est constante}}.$

    Montrer que $F$ et $G$ sont des sous-espaces vectoriels supplémentaires de 
    $\CC([-1,1], \R)$.
\end{Exercice} 

\begin{Exercice}
    Soit $f$ l'application définie sur $\mathbb{R}_2[X]$ par $f(P)=P-(X+1)P'+X^2 P''$. Montrer que $f$ est un endomorphisme de $\mathbb{R}_2[X]$. Déterminer son noyau, son image et son rang.
\end{Exercice} 


\end{document}