\documentclass[twoside,a4paper,french,10pt]{VcCours}

\begin{document}

\Titre{PSI}{Promotion 2021--2022}{Mathématiques}{\vspace{-1.5em}}

\begin{center}
\large\bf
Programme de colles\hspace{\stretch{1}}Semaine n\degres7: du 08/11 au 12/11/2021
\end{center}
\separationTitre


Pour cette khôlle, chaque étudiant aura :
\begin{itemize}
\item Une \textbf{question de cours}.
\item Une démonstration \textbf{(D)} ou un exercice à retravailler.
\end{itemize}


\section*{Chapitre 5 : Matrices et applications linéaires}

% cSpell:ignore Sarrus
\begin{enumerate}
\item Calcul matriciel.
\begin{itemize}
\item Espace vectoriel $\mathcal{M}_{n,p}(\mathbb{K)}$. Produit de matrices. Matrices particulières. 
\item Puissances de matrices : par conjecture, par cyclicité, si la matrice est diagonale, d'après la formule du binôme de Newton.
\item Transposée d'une matrice.
\item Inversibilité d'une matrice. Quelques propriétés. 
\end{itemize}
\item Matrices, vecteurs et applications linéaires.
\begin{itemize}
\item Matrice d'un vecteur dans une base. Matrice d'une famille de vecteurs dans une base.
\item Matrice d'une application linéaire dans des bases données. Image d'un vecteur. Matrices d'une composition d'une application linéaire. En particulier $\textrm{Mat}_{\mathcal{B}}(u^n)$. 
\item \textbf{Matrice de passage.}
\item \textbf{Formule de changement de base (théorème 14 et corollaire 15).}
\item Matrices semblables.
\end{itemize}
\item Image, noyau et rang d'une matrice.
\begin{itemize}
\item Noyau, image et rang d'une matrice.
\item Méthodes pratique d'obtention du noyau et de l'image. 
\end{itemize}
\item Trace.
\begin{itemize}
\item Trace d'une matrice carrée. Trace d'un produit de matrice. Trace d'un endomorphisme. 
\item \textbf{(D)} Deux matrices semblables ont la même trace.
\end{itemize}
\item Déterminant.
\begin{itemize}
\item Définition. Quelques propriétés. Cas d'une matrice carrée d'ordre deux, d'ordre trois (Sarrus). 
\item Opérations élémentaires.
\item Propriétés élémentaires : déterminant d'un produit, lien avec l'inversibilité, deux matrices semblables ont le même déterminant, matrice d'une transposée.
\item Déterminant d'une famille de vecteurs : \textit{pratique} pour justifier qu'une famille est une base.
\item Déterminant d'un endomorphisme.
\item Méthodes de calcul. Matrice triangulaire. Matrices définies par blocs. Développement par rapport à une ligne ou par rapport à une colonne.
\item Déterminant de Vandermonde (\textbf{définition et valeur}). \textbf{(D)} : relation de récurrence par une méthode (choisie par l'étudiant).
\end{itemize} 
\end{enumerate}

\section*{Chapitre 6 : Suites et séries de fonctions}

\begin{enumerate}
\item Introduction avec un exemple.
\item Modes de convergences pour des suites de fonctions. 
\begin{itemize}
\item \textbf{Définition de convergence simple}. Limite simple.
\item \textbf{Définition de convergence uniforme}. 
    \textbf{Les quantificateurs ont un sens ($\forall$ est différent de $\exists$) et l'ordre est important !!}
\item convergence uniforme sur un intervalle implique la 
    convergence simple sur cet intervalle. Limite uniforme.
\item Convergence uniforme et norme infini. 
    \textbf{Définition de la norme infini}.
\item \textbf{(D)} : La norme infini est une norme sur 
    $\mathcal{B}(I, \mathbb{K})$.
\item Lien entre convergence uniforme et norme infini.
\item \textbf{Méthode 2 (étude de la convergence uniforme) : 
    tout étudiant doit être capable de détailler cette méthode 
    à l'oral !}
\item \textbf{Définition : convergence simple et uniforme d'une série de fonctions}.
\item Une série de fonctions converge uniformément sur $I$ si et seulement si elle converge simplement et si la suite des restes converge uniformément vers $0$.
\item \textbf{Définition : convergence normale d'une série de fonctions}.
\item \textbf{(D)} : Théorème 5 (convergence normale implique convergence absolue en tout point et convergence uniforme).
\item \textbf{Théorème de continuité pour des suites de fonctions et corollaire}. Bien comprendre que travailler sur tout segment peut ici être utile.
\item \textbf{Théorème de continuité pour des séries de fonctions}.
\item Théorème de la double limite.
\end{itemize}
\end{enumerate}



% \begin{Exercice}
%     Soient $F= \lbrace (x,y,z) \in \mathbb{R}^3 \, \vert \, x+y+z =0 \rbrace$ et $G= \textrm{Vect}((1,1,1))$. Justifier l'existence de la projection sur $F$ parallèlement à $G$ et donner son expression. De même avec la symétrie par rapport à $F$ parallèlement à $G$. 
% \end{Exercice} 

\begin{Exercice}
    Posons :
    $$ \application{f}{\mathbb{R}_2[X]}{\mathbb{R}_2[X]}{P}{(X+1)P'}$$

    \begin{enumerate}
        \item Montrer que $f$ est un endomorphisme de $ \mathbb{R}_2[X]$.
        \item Déterminer la matrice de $f$ dans la base canonique $\mathcal{B}$ de $\mathbb{R}_2[X]$.
        \item Soit $\mathcal{B}'= (1, (X+1),(X+1)^2)$. Montrer que $\mathcal{B}'$ est une base de $\mathbb{R}_2[X]$ et déterminer la matrice de $f$ dans cette base.
        \item Déterminer le lien entre $\textrm{Mat}_{\mathcal{B}'}(f)$ et $\textrm{Mat}_{\mathcal{B}}(f)$.
    \end{enumerate}
\end{Exercice} 


% \begin{Exercice}
%     Étudier la convergence simple  sur $\mathbb{R}_+$ de la suite $(f_n)_{n \geq 0}$ définie par :
%     $$ \forall x \in \mathbb{R}_+, \quad f_n(x) = \frac{x^n}{x^n+1}$$
% \end{Exercice} 

\begin{Exercice}
    On pose pour tout $n \geq 0$ et tout $x \in \mathbb{R}_+$, $f_n(x)= nx^2 e^{-nx}$. Étudier la convergence uniforme de $(f_n)_{n \geq 0}$ sur $\mathbb{R}_+$. 
\end{Exercice} 

\begin{Exercice}
    On pose pour tout $n \geq 0$ et tout $x \in \mathbb{R}_+$, $f_n(x)= e^{-nx} \sin(nx)$.
    \begin{enumerate}
        \item Étudier la convergence simple de $(f_n)_{n \geq 0}$ sur $\mathbb{R}_+$. 
        \item Étudier la convergence uniforme $(f_n)_{n \geq 0}$ sur $[a, + \infty[$ où $a>0$. 
    \end{enumerate}
\end{Exercice} 

\begin{Exercice}
    Déterminer pour tout entier $n \geq 1$, le déterminant $D_n$ de taille $n$ suivant :
    \[
        D_n =
        \begin{vmatrix}
            2 & 1 & {} & {(0)} \\
            1 & \ddots & \ddots & {} \\
            {} & \ddots & \ddots & 1 \\
            {(0)} & {} & 1 & 2 \\
        \end{vmatrix}_{[n]}
    \]
\end{Exercice} 

\end{document}