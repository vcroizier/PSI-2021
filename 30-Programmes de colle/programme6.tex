\documentclass[twoside,a4paper,french,10pt]{VcCours}

\begin{document}

\Titre{PSI}{Promotion 2021--2022}{Mathématiques}{\vspace{-1.5em}}

\begin{center}
\large\bf
Programme de colles\hspace{\stretch{1}}Semaine n\degres6: du 18/10 au 22/10/2021
\end{center}
\separationTitre


Pour cette khôlle, chaque étudiant aura :
\begin{itemize}
\item Une \textbf{question de cours}.
\item Une démonstration \textbf{(D)} ou un exercice à retravailler.
\end{itemize}


\section*{Chapitre 4 : Applications linéaires}

\begin{enumerate}
\item Généralités.
\begin{itemize}
\item Définition d'une application linéaire (trois équivalentes). Endomorphisme, isomorphisme, automorphisme, forme linéaire. Quelques propriétés.
\item Image directe d'un sous-espace et image réciproque d'un sous-espace.
\item \textbf{Image et noyau d'une application linéaire.} 
\item \textbf{(D)} : Soient $E$, $F$ deux $\mathbb{K}$-espaces vectoriels et $f \in \mathcal{L}(E,F)$. L'application $f$ est injective si et seulement si $\textrm{Ker}(f) = \lbrace 0_E \rbrace$.
\item Une application linéaire est déterminée par l'image des vecteurs d'une base de $E$. 
\item $\textrm{Im}(f) = \textrm{Vect}(f(e_1), \ldots, f(e_n))$ : \textit{très utile dans la pratique}. Lien entre injectivité, surjectivité et bijectivité et image d'une base.
\item Rang d'une application linéaire. \textbf{Théorème du rang}. Soient $E, F$ deux $\mathbb{K}$-espaces vectoriels de même dimension $n$ et $f \in \mathcal{L}(E,F)$. Alors $f$ est un isomorphisme ssi $f$ est injective ssi $f$ est surjective ssi le rang de $f$ est égal à $n$.
\item \textbf{Définition d'une projection vectorielle}. 
\item \textbf{(D)} $p$ est un endomorphisme de $E$ vérifiant $p \circ p = p$, $\textrm{Im}(p) = \textrm{Ker}(p-\textrm{Id})=F$ et $\textrm{Ker}(p)=G$.
\item Définition d'un projecteur. Un endomorphisme $p$ de $E$ est une projection si et seulement si $p$ est un projecteur. Dans ce cas, $p$ est une projection sur $\textrm{Im}(p)$ parallèlement à $\textrm{Ker}(p)$.
\item \textbf{Définition d'une symétrie vectorielle}. Quelques propriétés : $s$ est un endomorphisme de $E$ vérifiant $s \circ s = \textrm{Id}$, $F = \textrm{Ker}(s-\textrm{Id})$ et $G= \textrm{Ker}(s+\textrm{Id})$.
\item Définition d'un endomorphisme involutif. Un endomorphisme $s$ de $E$ est une symétrie si et seulement si $s$ est un endomorphisme involutif.
\item \textbf{Il est important de savoir déterminer l'expression d'une projection ou d'une symétrie en raisonnant par analyse-synthèse.}
\end{itemize}
\item Hyperplan.
\begin{itemize}
\item Trois définitions équivalentes d'un hyperplan (existence d'une droite vectorielle comme supplémentaire, noyau d'une forme linéaire non nulle, dimension égal à $\textrm{dim}(E)-1$).
\item Formes linéaires avec le même noyau.
\item Équation d'un hyperplan dans une base.
\end{itemize}
\end{enumerate}

\newpage
\section*{Chapitre 5 : Matrices et applications linéaires}

\begin{enumerate}
\item Calcul matriciel.
\begin{itemize}
\item Espace vectoriel $\mathcal{M}_{n,p}(\mathbb{K)}$. Produit de matrices. Matrices particulières. 
\item Puissances de matrices : par conjecture, par cyclicité, si la matrice est diagonale, d'après la formule du binôme de Newton.
\item Transposée d'une matrice.
\item Inversibilité d'une matrice. Quelques propriétés. 
\end{itemize}
\item Matrices, vecteurs et applications linéaires.
\begin{itemize}
\item Matrice d'un vecteur dans une base. Matrice d'une famille de vecteurs dans une base.
\item Matrice d'une application linéaire dans des bases données. Image d'un vecteur. Matrices d'une composition d'une application linéaire. En particulier $\textrm{Mat}_{\mathcal{B}}(u^n)$. 
\item \textbf{Matrice de passage.}
\item \textbf{Formule de changement de base (théorème 14 et corollaire 15).}
\item Matrices semblables.
\end{itemize}
\item Image, noyau et rang d'une matrice.
\begin{itemize}
\item Noyau, image et rang d'une matrice.
\item Méthodes pratique d'obtention du noyau et de l'image. 
\end{itemize}
\item Trace.
\begin{itemize}
\item Trace d'une matrice carrée. Trace d'un produit de matrice. Trace d'un endomorphisme. 
\item \textbf{(D)} Deux matrices semblables ont la même trace.
\end{itemize}
\end{enumerate}

\begin{Exercice}
    Soit $f$ l'application définie sur $\mathbb{R}_2[X]$ par $f(P)=P-(X+1)P'+X^2 P''$. Montrer que $f$ est un endomorphisme de $\mathbb{R}_2[X]$. Déterminer son noyau, son image et son rang.
\end{Exercice} 

\begin{Exercice}
    Soient $F= \lbrace (x,y,z) \in \mathbb{R}^3 \, \vert \, x+y+z =0 \rbrace$ et $G= \textrm{Vect}((1,1,1))$. Justifier l'existence de la projection sur $F$ parallèlement à $G$ et donner son expression. De même avec la symétrie par rapport à $F$ parallèlement à $G$. 
\end{Exercice} 

\begin{Exercice}
    Soient $n \geq 1$ et $J$ la matrice carrée d'ordre $n$ dont tous les coefficients valent $1$. Posons $M=J+ I_n$. Déterminer une relation entre $M^2$, $M$ et $I_n$ et en déduire que $M$ est inversible.
\end{Exercice} 

\begin{Exercice}
    Posons :
    $$ \application{f}{\mathbb{R}_2[X]}{\mathbb{R}_2[X]}{P}{(X+1)P'}$$

    \begin{enumerate}
    \item Montrer que $f$ est un endomorphisme de $ \mathbb{R}_2[X]$.
    \item Déterminer la matrice de $f$ dans la base canonique $\mathcal{B}$ de $\mathbb{R}_2[X]$.
    \item Soit $\mathcal{B}'= (1, (X+1),(X+1)^2)$. Montrer que $\mathcal{B}'$ est une base de $\mathbb{R}_2[X]$ et déterminer la matrice de $f$ dans cette base.
    \item Déterminer le lien entre $\textrm{Mat}_{\mathcal{B}'}(f)$ et $\textrm{Mat}_{\mathcal{B}}(f)$.
    \end{enumerate}
\end{Exercice} 


\end{document}