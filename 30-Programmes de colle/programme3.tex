\documentclass[twoside,a4paper,french,10pt]{VcCours}

\begin{document}

\Titre{PSI}{Promotion 2021--2022}{Mathématiques}{\vspace{-1.5em}}

\begin{center}
\large\bf
Programme de colles\hspace{\stretch{1}}Semaine n\degres3 : du 27/09 au 01/10/2021
\end{center}
\separationTitre


Pour cette khôlle, chaque étudiant aura :
\begin{itemize}
\item Une question de cours.
\item Une démonstration \textbf{(D)} ou un exercice à retravailler.
\end{itemize}


\medskip 
\section*{Chapitre 2 : Séries.}

\begin{enumerate}
\item Généralités.
\begin{itemize}
\item Séries, sommes partielles, convergence/divergence d'une série et somme d'une série en cas de convergence.  Nature d'une série. Il est important de comprendre la différence entre les notations :
$$ \sum u_n, \,  \sum_{k=0}^n u_k, \, \sum_{k=0}^{+ \infty} u_k $$
\item Condition nécessaire de convergence : si une série converge, son terme général tend vers $0$ (la réciproque est fausse : connaître un contre-exemple). Série grossièrement divergente.
\item \textbf{Reste d'ordre $m$ d'une série} (convergente). La suite des restes converge vers $0$.
\item \textbf{Séries géométriques} + Théorème lié à la convergence des séries géométriques et expressions des restes \textbf{(D)}.
\item Divergence de la série harmonique.
\item Convergence de la série harmonique alternée.
\item Proposition concernant la convergence d'une série télescopique.
\end{itemize}
\item Quelques propriétés.
\begin{itemize}
\item L'ensemble des séries convergentes est un $\mathbb{K}$-espace vectoriel.
\item Convergence d'une série à termes complexes.
\end{itemize}
\item Comparaison série-intégrale.
\begin{itemize}
\item Principe.
\item Quelques exemples.
\end{itemize}
\item Séries à termes positifs.
\begin{itemize}
\item Une série à termes positifs est convergente si et seulement si sa suite des sommes partielles est majorée (car celle-ci est croissante).
\item \textbf{Critère de comparaison de séries à termes positifs}. 
\item \textbf{Séries de Riemann. Critère de convergence d'une série de Riemann.}
\item \textbf{Règle de D'Alembert}. 
\item Convergence absolue d'une série. \textbf{(D)} : la convergence absolue implique la convergence. \textit{Uniquement la preuve dans le cas réel.}
\end{itemize} 
\item Critères du type $u_n = O(v_n)$ et $u_n= o(v_n)$.
\item Séries alternées.
\begin{itemize}
\item Définition d'une série alternée.
\item \textbf{Critère spécial des séries alternées}.
\end{itemize}
\item Produit de Cauchy.
\begin{itemize}
\item \textbf{Définition du produit de Cauchy}.
\item Si deux séries sont absolument convergentes alors le produit de Cauchy de ces deux séries est absolument convergent. De plus, le produit des sommes est égal à la somme du produit de Cauchy.
\item Application : la série exponentielle. \textbf{(D)} : la fonction exponentielle est bien définie sur $\mathbb{C}$, la série associée est absolument convergente et on a pour tout $(z,z') \in \mathbb{C}^2$, $\exp(z)\exp(z')= \exp(z+z')$.
\end{itemize}
\item Compléments.
\begin{itemize}
\item \textbf{Formule de Stirling}.
\item Cas de convergence des séries de Bertrand (hors-programme) : \textit{important} de retenir les différentes méthodes dans la pratique.
\end{itemize}
\end{enumerate}

\medskip

\begin{Exercice}
Montrer que $\sum_{k=1}^n \frac{1}{k} \underset{+ \infty}{\sim} \ln(n)$.
\end{Exercice}

\begin{Exercice}
    Montrer que $\sum_{n \geq 0} \dfrac{\cos(n \pi/3)}{2^n}$ converge et donner sa somme.
\end{Exercice}

\begin{Exercice}
    Montrer que $\sum_{n \geq 0} {\dfrac{( - 1)^n 8^n}{(2n)!}}$ est convergente et que sa somme est négative.
\end{Exercice}

\begin{Exercice}
    Savoir étudier rapidement une série de Bertrand dans le \og cas simple \fg{} ($\alpha<1$ ou $\alpha>1$).
\end{Exercice}                        

\end{document}