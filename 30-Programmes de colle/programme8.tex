\documentclass[twoside,a4paper,french,10pt]{VcCours}

\begin{document}

\Titre{PSI}{Promotion 2021--2022}{Mathématiques}{\vspace{-1.5em}}

\begin{center}
\large\bf
Programme de colles\hspace{\stretch{1}}Semaine n\degres8: du 15/11 au 19/11/2021
\end{center}
\separationTitre


Pour cette khôlle, chaque étudiant aura :
\begin{itemize}
\item Une \textbf{question de cours}.
\item Une démonstration \textbf{(D)} ou un exercice à retravailler.
\end{itemize}


\section*{Chapitre 5 : Matrices et applications linéaires}

\emph{Le début du chapitre est bien sûr utilisable en exercice.}
\begin{enumerate}\setcounter{enumi}{4}
\item Déterminant.
\begin{itemize}%cSpell:ignore Sarrus
\item Définition. Quelques propriétés. Cas d'une matrice carrée d'ordre deux, d'ordre trois (Sarrus). 
\item Opérations élémentaires.
\item Propriétés élémentaires : déterminant d'un produit, lien avec l'inversibilité, deux matrices semblables ont le même déterminant, matrice d'une transposée.
\item Déterminant d'une famille de vecteurs : \textit{pratique} pour justifier qu'une famille est une base.
\item Déterminant d'un endomorphisme.
\item Méthodes de calcul. Matrice triangulaire. Matrices définies par blocs. Développement par rapport à une ligne ou par rapport à une colonne.
\item Déterminant de Vandermonde (\textbf{définition et valeur}). \textbf{(D)} : relation de récurrence par une méthode (choisie par l'étudiant).
\end{itemize} 
\end{enumerate}

\section*{Chapitre 6 : Suites et séries de fonctions}

\begin{itemize}
\item \textbf{Définition de convergence simple}. Limite simple.
\item \textbf{Définition de convergence uniforme}. 
    \textbf{Les quantificateurs ont un sens ($\forall$ est différent de $\exists$) et l'ordre est important !!}
\item convergence uniforme sur un intervalle implique la 
    convergence simple sur cet intervalle. Limite uniforme.
\item Convergence uniforme et norme infini. 
    \textbf{Définition de la norme infini}.
\item \textbf{(D)} : La norme infini est une norme sur 
    $\mathcal{B}(I, \mathbb{K})$.
\item Lien entre convergence uniforme et norme infini.
\item \textbf{Méthode 2 (étude de la convergence uniforme) : 
    tout étudiant doit être capable de détailler cette méthode 
    à l'oral !}
\item \textbf{Définition : convergence simple et uniforme d'une série de fonctions}.
\item Une série de fonctions converge uniformément sur $I$ si et seulement si elle converge simplement et si la suite des restes converge uniformément vers $0$.
\item \textbf{Définition : convergence normale d'une série de fonctions}.
\item \textbf{(D)} : Théorème 5 (convergence normale implique convergence absolue en tout point et convergence uniforme).
\item \textbf{Théorème de continuité pour des suites de fonctions et corollaire}. Bien comprendre que travailler sur tout segment peut ici être utile.
\item \textbf{Théorème de continuité pour des séries de fonctions}.
\item Théorème de la double limite.
\item \textbf{Théorème d'interversion limite-somme}.
\item \textbf{Théorème d'interversion limite-intégrale pour des suites de fonctions} +\textbf{(D)}.
\item \textbf{Théorème : Interversion somme-intégrale}.
\item Théorème : Dérivation et suites de fonctions.
\item \textbf{Théorème : Dérivation terme à terme et séries de fonctions}. Application : fonction exponentielle.
\item Compléments pour des fonctions de classe $\mathcal{C}^k$.
\end{itemize}


\begin{Exercice}
    Déterminer pour tout entier $n \geq 1$, le déterminant $D_n$ de taille $n$ suivant :
    \[
        D_n =
        \begin{vmatrix}
            2 & 1 & {} & {(0)} \\
            1 & \ddots & \ddots & {} \\
            {} & \ddots & \ddots & 1 \\
            {(0)} & {} & 1 & 2 \\
        \end{vmatrix}_{[n]}
    \]
\end{Exercice} 

\begin{Exercice}
Pour tout entier $n \geq 1$ et tout $x \in \mathbb{R}$, on pose :
$$ f_n(x) = \dfrac{\left(-1\right)^{n}e^{-nx}}{n} $$

\begin{enumerate}
\item Étudier la convergence simple sur $\mathbb{R}$  de la série de fonctions $\sum_{n\geq 1} f_n.$
\item Étudier la convergence uniforme sur $\left[ 0,+\infty\right[ $  de la série de fonctions $\sum_{n\geq 1} f_n.$
\end{enumerate}
\end{Exercice} 

\begin{Exercice}
Soit $\sum_{n \geq 1} u_n$ la série de fonctions d'une variable réelle de terme général $u_n$ défini pour tout $n \in \N^*$ par : 
$$ \forall x \in \R ,\ u_n(x)=\dfrac{2x}{x^2+n^2\pi^2}$$
\begin{enumerate}
\item Montrer que $\sum_{n \geq 1} u_n$ converge simplement sur $\R$. On note $U$ la somme de cette  série de fonctions.
\item Montrer que, pour tout $a > 0$ , $\sum_{n \geq 1} u_n$ converge normalement sur $[-a,a]$.
\item La série $\sum_{n \geq 1} u_n$ converge-t-elle normalement sur $\mathbb{R}$ ?
\item Montrer que $U$ est continue sur $\R$.
\end{enumerate}
\end{Exercice} 

\begin{Exercice}
\begin{enumerate}
    \item Déterminer l'ensemble $I$ des réels $x$ pour lesquels la série $\sum_{n\geq 1} \dfrac{1}{1+(nx)^2}$ converge.\\
    On définit alors la fonction $f$ de $I$ dans $\R$ en posant $f(x)=\sum_{n=1}^{+\infty}\frac{1}{1+(nx)^2}\cdot$
    \item Déterminer la parité de $f$ puis son sens de variation de $f$. 
    \item Prouver que $f$ est de classe $C^1$ sur $I$.
    \item Calculer $\lim_{x\to +\infty}f(x)$.
    \end{enumerate}
\end{Exercice} 

\end{document}