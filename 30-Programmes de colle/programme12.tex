\documentclass[twoside,a4paper,french,10pt]{VcCours}
\newcommand{\dt}{\text{d}t}
\newcommand{\dx}{\text{d}x}
\begin{document}

\Titre{PSI}{Promotion 2021--2022}{Mathématiques}{\vspace{-1.5em}}

\begin{center}
\large\bf
Programme de colles\hspace{\stretch{1}}Semaine n\degres12 du 13/12 au 17/12/2021
\end{center}
\separationTitre


Pour cette khôlle, chaque étudiant aura :
\begin{itemize}
\item Une ou deux \textbf{question de cours}.
\item Une démonstration \textbf{(D)} ou un exercice à retravailler.
\end{itemize}


  \section*{Chapitre 9 : Intégrales généralisées}
  Tout le début du chapitre au programme en exercice (Intégrales généralisées sur un intervalle, connaître parfaitement les intégrales de références, fausse impropreté).
  \begin{itemize}
    \item Méthodes de calcul d'intégrales généralisées.
    \begin{itemize}
    \item Avec une primitive.
    \item Avec une intégration par parties (je déconseille le théorème du programme : passer par une intégrale sur un segment).
    \item \textbf{Changement de variable.}
    \end{itemize}
    \item \textbf{Intégrale absolument convergente.} La convergence absolue implique la convergence. Réciproque fausse.
    \item \textbf{Fonction intégrable.} Quelques propriétés.
    \item Critères de comparaison pour des fonctions intégrables.
    \item Critères de comparaison pour des intégrales.
    \item Exemples (il faut s'entraîner beaucoup!).
    \item \textbf{Théorème de convergence dominée}. 
    \item \textbf{Théorème d'intégration terme à terme.}
    \item \textbf{Théorème de comparaison série-intégrale}.
  \end{itemize}  

  \section*{Chapitre 10 : Réduction}
\begin{enumerate}
\item Compléments d'algèbre linéaire.
\begin{itemize}
\item Sous-espace vectoriel stable, endomorphisme induit.
\item Si $u$ et $v$ sont deux endomorphismes d'un $\mathbb{K}$-espace vectoriel $E$ qui commutent, le noyau et l'image de $u$ sont stables par $v$ : \textbf{(D)}
\end{itemize}
\item Élément propre d'un endomorphisme.
\begin{itemize}
\item \textbf{Définitions} : valeurs propres et vecteurs propres d'un endomorphisme.
\item Attention : un vecteur propre est non nul ! 
\item Spectre d'un endomorphisme.
\item Proposition (3) ($\lambda$ est valeur propre si et seulement si le noyau $u- \lambda Id_E$ est différent de $\lbrace 0_E \rbrace$). \textbf{définition d'un sous-espace propre}.
\item Un sous-espace propre est un sous-espace vectoriel.
\item Cas particulier de la valeur propre nulle.
\item Tout sous-espace propre de $u$ est stable par $u$. 
\item Si $u$ et $v$ sont deux endomorphismes d'un $\mathbb{K}$-espace vectoriel $E$ qui commutent, tout sous-espace propre $u$ est stable par $v$.
\item Proposition (8) : la somme des sous-espaces propres est directe : \textbf{(D)}.
\item Corollaire : une famille de vecteurs propres associés à des valeurs propres deux à deux distinctes d'un endomorphisme d'un $\mathbb{K}$-espace vectoriel est une famille libre.
\end{itemize}
\item Élément propre en dimension finie.
\begin{itemize}
\item \textbf{Définitions} : valeur propre, vecteur propre, sous-espace propre d'une matrice (retenir le \og autrement dit \fg après la définition).
\item Lien entre éléments propres d'un endomorphisme et d'une matrice.
\item \textbf{Définition} : Polynôme caractéristique d'une matrice. C'est un polynôme unitaire, de degré $n$, son coefficient constant est $(-1)^n \textrm{det}(A)$.
\item Deux matrices semblables ont le même polynôme caractéristique : \textbf{(D)}. Polynôme caractéristique d'un endomorphisme.
\item Les racines du polynôme caractéristique de $u$ ou de $A$ sont les valeurs propres. 
\item Une matrice carrée admet au plus $n$ valeurs propres. Une matrice complexe admet au moins une valeur propre. Résultats semblables pour un endomorphisme. 
\end{itemize}
\item Diagonalisation.
\begin{itemize}
\item \textbf{Définition : endomorphisme diagonalisable, matrice diagonalisable.}
\item \textbf{Théorème 21} : 4 assertions équivalentes ($u$ diagonalisable / base de vecteurs propres / $E$ égal à la somme directe des sous-espaces propres / somme des dimensions des sous-espaces propres égale à la dimension de $E$).
\item \textbf{Corollaire 22} (un endomorphisme est diagonalisable si et seulement si son polynôme caractéristique est scindé sur $\mathbb{K}$ et si la multiplicité de chaque valeur propre est égale à la dimension du sous-espace propre associé).
\item Corollaire : si un endomorphisme d'un espace vectoriel de dimension $n$ admet $n$ valeurs propres distinctes alors il est diagonalisable.
\end{itemize}
% \item Polynômes d'endomorphismes et de matrices.
% \begin{itemize}
% \item Définition et quelques propriétés.
% \item Proposition 5.6 (hors-programme) : Soient $u \in \mathcal{L}(E)$ et $P \in \mathbb{K}[X]$ un polynôme annulateur de $u$. Alors toute valeur propre de $u$ est racine de $P$. \textbf{(D)}
% \item \textbf{Théorème de Cayley-Hamilton}.
% \item \textbf{Théorème 5.12 : 3 assertions équivalentes} ($u$ diagonalisable/existence d'un polynôme annulateur scindé à racines simples/ le polynôme $\dis \prod X- \lambda$ est annulateur de $u$).
% \item Exemple pour une projection, une symétrie.
% \item Diagonalisabilité d'un endomorphisme induit.
% \end{itemize}
% \item Trigonalisation.
% \begin{itemize}
% \item \textbf{Définition : endomorphisme trigonalisable, matrice trigonalisable.}
% \item \textbf{Théorème 6.4} (un endomorphisme est trigonalisable si et seulement si son polynôme caractéristique est scindé). Corollaire : toute matrice est trigonalisable sur $\mathbb{C}$.
% \item Si le polynôme caractéristique est scindé, expression du déterminant et de la trace avec les valeurs propres. 
% \end{itemize}
\end{enumerate}


    \begin{Exercice} Nature des intégrales suivantes : 
    $$  \int_{0}^{+ \infty} \frac{2x}{x^2+x+1} \dx, \; \int_{1}^{+ \infty} \frac{\ln(t)}{\sqrt{t}}, \; \int_{0}^{+ \infty} \frac{\ln(t)}{t^3+t^2+1} \dt$$
  \end{Exercice} 
    
    \begin{Exercice} Étudier l'intégrabilité sur $[1, + \infty[$ de $f_1$ définie par :
    $$ f_1(x)=\frac{\sqrt{x^2+x+1}-\sqrt{x^2-x+1}}{x}$$
  \end{Exercice} 
    
    \begin{Exercice} Justifier l'existence puis donner la valeur de :
      \[
      I = \int_{0}^{ + \infty} \frac{\dt}{(1 + t^{2})^{2}}
      \]
    On pourra utiliser le changement de variable $u = \dfrac{1}{t} \cdot$
  \end{Exercice} 
    
    \begin{Exercice} Déterminer $ \lim_{n \rightarrow + \infty} \int_{0}^{\frac{\pi}{4}} \tan(t)^n \dt$.
    \end{Exercice} 
    
\begin{Exercice} 
  On pourra poser une diagonalisation de matrice comme exercice.
\end{Exercice} 
    
  



\end{document}