\documentclass[twoside,a4paper,french,10pt]{VcCours}
\newcommand{\dt}{\text{d}t}
\newcommand{\dx}{\text{d}x}
\begin{document}

\Titre{PSI}{Promotion 2021--2022}{Mathématiques}{\vspace{-1.5em}}

\begin{center}
\large\bf
Programme de colles\hspace{\stretch{1}}Semaine n\degres15 du 17/01 au 21/01/2022
\end{center}
\separationTitre


Pour cette khôlle, chaque étudiant aura :
\begin{itemize}
\item Une ou deux \textbf{question de cours}.
\item Une démonstration \textbf{(D)} ou un exercice à retravailler.
\end{itemize} 

\section*{Chapitre 11 : Espaces probabilisés}
  \begin{enumerate}
  \item Rappels sur les ensembles finis.
    \begin{itemize}
    \item Définition et propriétés.
    \item Cardinal d'un produit cartésien de deux ensembles finis, 
    cardinal de $\mathcal{P}(E)$ quand $E$ est fini \textbf{(D)} et cardinal de $\mathcal{F}(E,F)$ quand $E$ et $F$ sont finis.
    \item Dénombrement : listes avec ou sans répétitions, combinaisons. 
    \item \textbf{Ensembles dénombrables}, au plus dénombrables.
    \item $\mathbb{Z}$ est dénombrable.% \textbf{(D)}.
    \item Un produit cartésien d'ensembles dénombrables est dénombrable.
    \end{itemize}
  \item Probabilité
  \begin{itemize}
  \item Intersection et union dénombrable.
  \item Tribu.
  \item Stabilité par intersection au plus dénombrable dans une tribu.
  \item Système complet d'évènements.
  \item \textbf{Définition} d'une probabilité.
  \item Cas particulier si l'univers est dénombrable.
  \item Évènements presque-sur, presque-impossible.
  \item Propriétés usuelles (croissance, union de deux ou $n$ évènements deux à deux incompatibles, formule du crible).
  \item \textbf{Théorème de continuité croissante et décroissante}.
  \item Application (\textbf{à savoir refaire}) pour calculer la probabilité d'une union ou intersection dénombrable \textit{quelconque} d'évènements.
  \item Sous-additivité.
  \end{itemize}
  \item Cas de l'équiprobabilité. Refaire des exercices du type de l'AD5 (calcul de probabilité en déterminant le cardinal de l'univers, cardinal de l'évènement en comprenant si les tirages sont simultanés, successifs, avec ou sans remise...).
  \item Probabilités conditionnelles.
  \begin{itemize}
  \item Probabilité de $A$ sachant $B$.
  \item $P_B$ est une probabilité. Conséquence : corollaire 22.
  \item \textbf{Formule des probabilités composées}.
  \item \textbf{Formule des probabilités totales : cas fini et infini}. \textbf{Interdiction d'oublier de préciser le système complet d'évènements et interdiction de calculer directement la probabilité d'un évènement sans passer par la formule théorique}.
  \item Formule de Bayes.
  \end{itemize}
  \item Indépendance de deux évènements. 
  \item Évènements mutuellement indépendants (pour une famille finie ou dénombrable).
  \end{enumerate}
  

\section*{Chapitre 12 : Espaces vectoriels normés}
    \begin{enumerate}
    \item Espaces vectoriels normés.
    \begin{itemize}
    \item Définition d'une norme.
    \item Exemples : norme sur un espace préhilbertien réel, normes usuelles sur $\mathbb{K}^n$, norme de la convergence uniforme.
    \item Une inégalité triangulaire.
    \item Distance associée à une norme. 
    \end{itemize}
    \item Rudiments de topologie.
    \begin{itemize}
    \item Boule ouverte et fermée, sphère, boule unité, sphère unité
    \item Suites d'éléments dans un espace vectoriel.
    \item Normes équivalentes.
    \item Parties, suites et applications bornées.
    \item Définition : \textbf{ensemble convexe}. \textbf{(D)} : Les boules sont convexes. 
    \end{itemize}
    % \item Convergence de suites.
    % \begin{itemize}
    % \item Convergence de suite. Propriétés usuelles.
    % \item Lien entre convergence et suites coordonnées.
    % \end{itemize}
    % \item Topologie d'un espace vectoriel normé.
    % \begin{itemize}
    % \item Point intérieur d'une partie $A$, intérieur d'une partie $A$.
    % \item \textbf{Partie ouverte}.
    % \item \textbf{(D)} : une boule ouverte est un ouvert.
    % \item Point adhérent d'une partie $B$, adhérence d'une partie $B$.
    % \item \textbf{Partie fermée}. Critère lié aux suites (\textbf{utile dans la pratique !}).
    % \item Une partie $A$ d'un espace vectoriel normé ($E$, $\Vert \cdot \Vert$) est fermée si et seulement si son complémentaire est une partie ouverte.
    % \item Corollaire : une boule fermée est un fermé.
    % \item Frontière d'une partie.
    % \end{itemize}
    % \item Fonctions entres espaces vectoriels normés.
    % \begin{itemize}
    % \item Différents cas de limites. Continuité en un point, sur une partie.
    % \item Caractérisation séquentielle.
    % \item Limite et composantes d'une fonction.
    % \item Opérations usuelles.
    % \item Une application polynomiale sur $\mathbb{K}^n$ à valeurs dans $\mathbb{K}$ est continue.
    % \item Si $f : E \rightarrow \mathbb{R}$ est continue, $\lbrace x \in E, \, f(x)>0 \rbrace$ est un ouvert de $E$, $\lbrace x \in E, \, f(x) \geq 0 \rbrace$ est un fermé de $E$ et $\lbrace x \in E, \, f(x)=0 \rbrace$ est un fermé de $E$. \textbf{(IMPORTANT)} 
    % \item Soit $K$ une partie fermée bornée sur un espace $E$ de dimension finie (ce qu'on appelle un compact). Alors toute application $f : K \rightarrow \mathbb{R}$ est bornée et atteint ses bornes sur $K$. 
    % \end{itemize}
    % \item Applications lipschitziennes, linéaires et multilinéaires.
    % \begin{itemize}
    % \item \textbf{Définition application lipschitzienne}.
    % \item Toute fonction lipschitzienne est continue.
    % \item Si $u \in \mathcal{L}(E,F)$ alors $u$ est lipschitzienne (les espaces sont de dimension finies).
    % \item Continuité des applications multilinéaires.
    % \end{itemize}
    \end{enumerate}
    
    % \exo Pour tout $A = (a_{i,j})_{1 \leq i,j \leq n} \in \mathcal{M}_{n}(\mathbb{C})$, on pose :
    %   \[
    %   \Vert A \Vert = \max_{1 \leq i \leq n} \sum_{j = 1}^{n}  \vert a_{i,j} \vert
    %   \]
    %   \begin{enumerate}
    %   \item Montrer que $A \mapsto \Vert A \Vert$ définit une norme sur $\mathcal{M}_{n}(\mathbb{C})$.
    %   \item Montrer cette norme est une \textit{norme d'algèbre}, c'est-à-dire vérifiant :
    %     \[
    %     \forall A,B \in \mathcal{M}_{n}(\mathbb{C}),  \; \Vert AB \Vert \leq \Vert A \Vert \, \Vert B \Vert
    %     \]
    %   \end{enumerate}
      
    %  \exo Montrer que $\ln$ est $1$-lipschitzienne sur $[1, + \infty[$.
     
    %  \exo Montrer que les normes $1$, $2$ et $\infty$ sont équivalentes sur $\mathbb{R}^n$.


\begin{Exercice} 
  En France la probabilité de gagner au loto est $p=1/19 068 840$. 
  Un joueur immortel décide de jouer tous les jours au loto. 
  Pour tout $n \geq 1$, on note $A_n$ l'évènement
  \og Le joueur a perdu tous les jours jusqu'au jour $n$ \fg
  Que représente l'évènement $\bigcap_{k=0}^{+\infty}A_k$? Calculer sa probabilité.
\end{Exercice} 

\begin{Exercice} 
  Une urne contient 3 boules blanches et 7 boules noires. On tire successivement et sans remise 3 boules dans cette urne. Quelle est la probabilité d'obtenir la première boule blanche au 3$^{\text{ème}}$ tirage ?
\end{Exercice}   

\begin{Exercice} 
  On considère une infinité d'urnes. On en choisit une au hasard de telle sorte que la probabilité de choisir l'urne $n$ soit égale à $\frac{1}{2^n}$ (avec $n \in \mathbb{N}^*$), on note $A_n$ l'évènement \og on choisit l'urne $n$ \fg .
  \begin{enumerate}
    \item Vérifier que $P \bigg{(} \bigcup_{n \in \N^*} A_n \bigg{)} = 1$.
    \item Pour tout entier $n \geq 1$, l'urne $n$ contient $2^n$ boules dont une seule est blanche. On choisit une urne au hasard puis on tire une boule et on note $B$ l'évènement \og on tire une boule blanche \fg . Déterminer $P(B)$. 
  \end{enumerate}
\end{Exercice} 

\begin{Exercice}
  Calculer les sommes suivantes ($n \in \mathbb{N}$) : 
  $$\sum_{k=0}^n k \binom{n}{k}, \; \sum_{k=0}^n\frac{1}{k+1} \binom{n}{k}, \; \sum_{k=0}^n \binom{2n+1}{k}$$
\end{Exercice}   
    
\end{document}