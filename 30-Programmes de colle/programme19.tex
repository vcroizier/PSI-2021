\documentclass[twoside,a4paper,french,10pt]{VcCours}
\newcommand{\dt}{\text{d}t}
\newcommand{\dx}{\text{d}x}
\begin{document}

\Titre{PSI}{Promotion 2021--2022}{Mathématiques}{\vspace{-1.5em}}

\begin{center}
\large\bf
Programme de colles\hspace{\stretch{1}}Semaine n\degres19 du 07/03 au 11/03/2022
\end{center}
\separationTitre


Pour cette khôlle, chaque étudiant aura :
\begin{itemize}
\item Une ou deux \textbf{question de cours}.
\item Une démonstration \textbf{(D)} ou un exercice à retravailler.
\end{itemize} 
  

\section*{Chapitre 16 : Espaces préhilbertiens réels, espaces euclidiens}
  \begin{enumerate}
  \item Produit scalaire.
  \begin{itemize}
  \item Définition d'un produit scalaire. Espace préhilbertien réel, espace euclidien.
  \item Connaître parfaitement les produits scalaires usuels sur \fbox{$\mathbb{R}^n$}, \fbox{$\mathcal{M}_{n,1}(\mathbb{R})$}, \fbox{$\mathcal{M}_n(\mathbb{R})$ + \textbf{(D)}}, \\\fbox{$\mathcal{C}([a,b], \mathbb{R})$} (avec un poids $\omega$).
  \item \textbf{Inégalité de Cauchy-Schwarz} + \textbf{(D) (avec le cas d'égalité)}
  \item Norme euclidienne, Distance euclidienne.
  \item \textbf{Cas d'égalité dans l'inégalité triangulaire}.
  \item Savoir calculer rapidement $\Vert x \pm y \Vert^2$. \textbf{Identités de polarisation et du parallélogramme (à savoir retrouver rapidement)}.
  \end{itemize}
  \item Orthogonalité.
  \begin{itemize}
  \item Vecteur unitaires. Vecteurs orthogonaux. 
  \item Famille normée, orthogonale, orthonormale.
  \item Une famille orthogonale finie de vecteurs non nuls de $H$ est libre + \textbf{(D)}.
  \item Théorème de Pythagore.
  \item Bases orthonormées.
  \item Calcul dans une base orthonormée.
  \item Matrice d'un endomorphisme dans une base orthonormée.
  \item Sous-espaces orthogonaux.
  \item Des sous-espaces deux à deux orthogonaux sont en somme directe (orthogonale).
  \item \textbf{Orthogonal d'un sous-espace vectoriel} + \textbf{(D)} : c'est un sous-espace vectoriel de $H$.
  \item Proposition 2.23 (importante pour déterminer $F^{\perp}$) : se ramener à une famille génératrice.
  \item \textbf{Théorème : supplémentaire orthogonal d'un sous-espace vectoriel de dimension finie}. 
  \item \textbf{Théorème : projection orthogonal}. Deux méthodes pour déterminer l'expression d'une projection orthogonale.
  \item Décomposition d'un vecteur dans une base orthonormée.
  \item Si $H$ est euclidien et $F$ un S.E.V, $ \textrm{dim}(H) = \textrm{dim}(F) + \textrm{dim}(F^{\perp})$.
  \item Distance.
  \begin{itemize}
  \item \textbf{Théorème 3.1} (distance à un sous-espace).
  \item Inégalité de Bessel.
  \item Théorème 4.1 : une forme linéaire sur un espace euclidien est de la forme $x \mapsto <x,a>$.
  \item Hyperplan, vecteur normal.
  \item \textbf{Théorème : distance d'un vecteur à un hyperplan, à un sous-espace.}
  \end{itemize}
  \item Espaces fonctionnels.
  \begin{itemize}
  \item Espaces $L^1$ et $L^2$.
  \item Le produit de deux éléments de $L^2$ est un élément de $L^1$.
  \end{itemize}
  \end{itemize}
  \end{enumerate}
  
  Deux méthodes générales comptant comme questions de cours :
  \begin{itemize}
  \item Savoir appliquer l'algorithme de Gram-Schmidt.
  \item Savoir déterminer rapidement la projection orthogonale sur un plan ou sur une droite dans $\mathbb{R}^3$.
  \end{itemize}

\section*{Chapitre 15 : Séries entières}
\begin{enumerate}
\item Convergence d'une série entière.
\begin{itemize}
\item Définition d'une série entière.
\item Somme d'une série entière.
\item \textbf{Lemme D'Abel + (D)}
\item \textbf{Définition du rayon de convergence}.
\item Proposition 1.10 : si $\vert z \vert < R$ et si $\vert z \vert >R$...
\item Intervalle (ou disque) ouvert de convergence. Bien comprendre qu'il peut tout se passer sur la frontière ($-R$ et $R$ dans le cas réel et le cercle $C(0,R)$ dans le cas complexe).
\item Connaître quelques méthodes pour obtenir des inégalités sur le rayon de convergence (avant l'exemple 2). \textbf{Important dans la pratique}.
\item \textbf{Critère de comparaison $a_n = O(b_n)$ et $a_n \sim b_n$ + (D)}
\item Utilisation du critère de d'Alembert.
\item Différentes convergences sur l'intervalle ouvert de convergence.
\end{itemize}
\item Opérations sur les séries entières.
\begin{itemize}
\item Produit par un réel. \textbf{Somme}. 
\item Produit de Cauchy.
\end{itemize}
\item Propriétés de la somme d'une série entière.
\begin{itemize}
\item Continuité de la somme sur $]-R,R[$. Cas complexe.
% \item \textbf{Petit résultat utile}.
% \item \textbf{Primitivation terme à terme.}
% \item \textbf{Dérivation terme à terme.}
\end{itemize}
\end{enumerate}

\begin{Exercice}{} Considérons $\R^4$ muni de son produit scalaire usuel. Soient $e_1=(1;0;1;0)$ et $e_2=(1;-1;1;-1)$ et $F=\textrm{Vect}(e_1,e_2)$.
  \begin{enumerate}
      \item  Déterminer une base orthonormale de $F.$
      \item  Déterminer la matrice dans la base canonique de $\R^4$ du projecteur orthogonal
  sur $F.$
      \item  Déterminer la distance du vecteur $(1;1;1;3)$ au sous-espace vectoriel  $F$.
  \end{enumerate}
  \end{Exercice}

%\exo Déterminer le rayon de convergence de $\sum_{n\geq 0} \sin \left( \dfrac{n \pi}{3} \right) x^n$ et déterminer sa somme.

\begin{Exercice}
  Déterminer le produit de Cauchy de $\sum_{n \geq 0} z^n$ avec $\sum_{n \geq 0} (n+1)z^n$.
\end{Exercice}

%\exo Déterminer le rayon de convergence puis la somme de $\sum_{n \geq 0} n^2x^n$.
\end{document}  