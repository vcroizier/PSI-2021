\documentclass[twoside,a4paper,french,10pt]{VcCours}

\begin{document}

\Titre{PSI}{Promotion 2021--2022}{Mathématiques}{\vspace{-1.5em}}

\begin{center}
\large\bf
Programme de colles\hspace{\stretch{1}}Semaine n\degres4 : du 04/10 au 08/10/2021
\end{center}
\separationTitre


Pour cette khôlle, chaque étudiant aura :
\begin{itemize}
\item Une question de cours.
\item Une démonstration \textbf{(D)} ou un exercice à retravailler.
\end{itemize}


\medskip 
\section*{Chapitre 2 : Séries.}
\emph{Bien entendu, le début du chapitre est utilisable pour les exercices.}
\begin{enumerate}
% \item Généralités.
% \begin{itemize}
% \item Séries, sommes partielles, convergence/divergence d'une série et somme d'une série en cas de convergence.  Nature d'une série. Il est important de comprendre la différence entre les notations :
% $$ \sum u_n, \,  \sum_{k=0}^n u_k, \, \sum_{k=0}^{+ \infty} u_k $$
% \item Condition nécessaire de convergence : si une série converge, son terme général tend vers $0$ (la réciproque est fausse : connaître un contre-exemple). Série grossièrement divergente.
% \item \textbf{Reste d'ordre $m$ d'une série} (convergente). La suite des restes converge vers $0$.
% \item \textbf{Séries géométriques} + Théorème lié à la convergence des séries géométriques et expressions des restes \textbf{(D)}.
% \item Divergence de la série harmonique.
% \item Convergence de la série harmonique alternée.
% \item Proposition concernant la convergence d'une série télescopique.
% \end{itemize}
% \item Quelques propriétés.
% \begin{itemize}
% \item L'ensemble des séries convergentes est un $\mathbb{K}$-espace vectoriel.
% \item Convergence d'une série à termes complexes.
% \end{itemize}
% \item Comparaison série-intégrale.
% \begin{itemize}
% \item Principe.
% \item Quelques exemples.
% \end{itemize}
% \item Séries à termes positifs.
% \begin{itemize}
% \item Une série à termes positifs est convergente si et seulement si sa suite des sommes partielles est majorée (car celle-ci est croissante).
% \item \textbf{Critère de comparaison de séries à termes positifs}. 
% \item \textbf{Séries de Riemann. Critère de convergence d'une série de Riemann.}
% \item \textbf{Règle de D'Alembert}. 
% \item Convergence absolue d'une série. \textbf{(D)} : la convergence absolue implique la convergence. \textit{Uniquement la preuve dans le cas réel.}
% \end{itemize} 
\setcounter{enumi}{4}
\item Critères du type $u_n = O(v_n)$ et $u_n= o(v_n)$.
\item Séries alternées.
\begin{itemize}
\item Définition d'une série alternée.
\item \textbf{Critère spécial des séries alternées}.
\end{itemize}
\item Produit de Cauchy.
\begin{itemize}
\item \textbf{Définition du produit de Cauchy}.
\item Si deux séries sont absolument convergentes alors le produit de Cauchy de ces deux séries est absolument convergent. De plus, le produit des sommes est égal à la somme du produit de Cauchy.
\item Application : la série exponentielle. \textbf{(D)} : la fonction exponentielle est bien définie sur $\mathbb{C}$, la série associée est absolument convergente et on a pour tout $(z,z') \in \mathbb{C}^2$, $\exp(z)\exp(z')= \exp(z+z')$.
\end{itemize}
\item Compléments.
\begin{itemize}
\item \textbf{Formule de Stirling}.
\item Cas de convergence des séries de Bertrand (hors-programme) : \textit{important} de retenir les différentes méthodes dans la pratique.
\end{itemize}
\end{enumerate}

\section*{Chapitre 3 : Espaces vectoriels}
\begin{enumerate}
    \item Espaces et sous-espaces vectoriels.
    \begin{itemize}
    \item Généralités : $\mathbb{K}$-espace vectoriel. Combinaison linéaire. Méthodes des trois points. Intersection de sous-espaces vectoriels. Espace vectoriel engendré. \textbf{(D)} : c'est un sous-espace vectoriel de l'espace dans lequel il est inclus.
    \item Produit de sous-espaces vectoriels.
    \item Somme de sous-espaces vectoriels. \textbf{Somme directe de sous-espaces}, proposition faisant le lien avec l'unicité dans la décomposition.
    \item \textbf{(D)} : Soient $E$ un $\mathbb{K}$-espace vectoriel et $E_1$, $E_2$ deux sous-espaces vectoriels de $E$. Alors :
    $$ E_1 + E_2 \hbox{ est directe si et seulement si } E_1 \cap E_2 = \lbrace 0_E \rbrace $$
    \item \textbf{Espaces supplémentaires}. Cas particulier pour deux sous-espaces.
    \end{itemize}
    \item  Familles libres, génératrices, bases.
    \begin{itemize}
    \item Famille libre, famille liée, famille génératrice, base. Famille libre dans le cas d'une famille constituée d'un vecteur, de deux vecteurs (vecteurs non colinéaires).
    \item \textbf{(D) :} Une famille échelonné en degré de polynômes de $\mathbb{R}[X]$ (et ne contenant pas le polynôme nul) est une famille libre de $\mathbb{R}[X]$.
    \item Coordonnées (ou composantes) d'un vecteur dans une base.
    \end{itemize}
    \item Espaces vectoriels de dimension finie.
    \begin{itemize}
    \item Espace vectoriel de dimension finie. Théorème de la base incomplète. Dimension d'un espace vectoriel de dimension finie. Bases canoniques des espaces vectoriels classiques.
    \item Une famille de $n$ vecteurs dans un espace vectoriel $E$ de dimension $n \geq 1$ est une base si et seulement si elle est libre si et seulement si elle est génératrice de $E$.
    \item Dimension d'un sous espace, dimension d'un produit d'espaces vectoriels.
    \item Fractionnement d'une base : \textit{utile pour montrer rapidement que des sous-espaces sont supplémentaires}. Base adaptée associée. Existence d'un supplémentaire en dimension finie. Un supplémentaire n'est pas unique. 
    \item Dimension d'une somme directe. Formule de Grassmann. Caractérisation de $E= F \oplus G$ à l'aide de la dimension.
    \item \textbf{Rang d'une famille de vecteurs}.
    \end{itemize}
\end{enumerate}
\medskip

\begin{Exercice}
    Montrer que $\sum_{n \geq 0} {\dfrac{( - 1)^n 8^n}{(2n)!}}$ est convergente et que sa somme est négative.
\end{Exercice}

\begin{Exercice}
    Savoir étudier rapidement une série de Bertrand dans le \og cas simple \fg{} ($\alpha<1$ ou $\alpha>1$).
\end{Exercice}                        

\begin{Exercice}
    Soit $F = \lbrace (x,y,z) \in \mathbb{R}^3 \,  \vert \, x+y+z= 0 \rbrace.$ Montrer que $F$ est un sous-espace vectoriel engendré de $\mathbb{R}^3$ et déterminer-en une base.
\end{Exercice} 

\begin{Exercice}
    Soient $F = \ensemble{f \in \CC([-1,1],\R)}{\int_{-1}^{1} f(t)\text{d}t = 0}$
    et $G = \ensemble{f\in\CC([-1,1],\R)}{f\text{ est constante}}.$

    Montrer que $F$ et $G$ sont des sous-espaces vectoriels supplémentaires de 
    $\CC([-1,1], \R)$.
\end{Exercice} 

\end{document}